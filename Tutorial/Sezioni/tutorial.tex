\section{Tutorial}
\subsection{Tabelle}
\subsubsection{Esempio tabella}
Codice che genera una tabella a 2 colonne
\begin{lstlisting}
\renewcommand{\arraystretch}{1}
	\begin{table}[H]
		\begin{center}
			\setlength{\aboverulesep}{0pt}
			\setlength{\belowrulesep}{0pt}
			\setlength{\extrarowheight}{.75ex}
			\rowcolors{2}{AzzuroGruppo!10}{white}
			\begin{tabular}{ c c }
				\rowcolor{AzzuroGruppo!30} 
				\textbf{Prima colonna} & \textbf{Seconda colonna}  \\
				\toprule
				Prima colonna della prima riga & Seconda colonna della prima riga \\
				Prima colonna della seconda riga & Seconda colonna della seconda riga \\
				\bottomrule
			\end{tabular}
			\caption{Tabella di esempio}
		\end{center}
	\end{table}
\end{lstlisting}
Codice compilato
\renewcommand{\arraystretch}{1}
	\begin{table}[H]
		\begin{center}
			\setlength{\aboverulesep}{0pt}
			\setlength{\belowrulesep}{0pt}
			\setlength{\extrarowheight}{.75ex}
			\rowcolors{2}{AzzurroGruppo!10}{white}
			\begin{tabular}{ c c }
				\rowcolor{AzzurroGruppo!30} 
				\textbf{Prima colonna} & \textbf{Seconda colonna}  \\
				\toprule
				Prima colonna della prima riga & Seconda colonna della prima riga \\
				Prima colonna della seconda riga & Seconda colonna della seconda riga \\
				\bottomrule
			\end{tabular}
			\caption{Tabella di esempio}
		\end{center}
	\end{table}

\subsubsection{Spiegazione tabella}
L'espressione seguente inizializza la tabella
\begin{lstlisting}
\begin{tabular}{ c c }
\end{lstlisting}
\begin{itemize}
\item Il numero di c presenti all'interno delle parentesi graffe determina il numero di colonne presenti nella tabella.
\item La lettera c determina l'allineamento del testo, in questo caso al centro.
\item Ci sono altre opzioni di allineamento

\begin{table}[H]
		\begin{center}
			\setlength{\aboverulesep}{0pt}
			\setlength{\belowrulesep}{0pt}
			\setlength{\extrarowheight}{.75ex}
			\rowcolors{2}{AzzurroGruppo!10}{white}
			\begin{tabular}{ r l c }
				\rowcolor{AzzurroGruppo!30} 
				\textbf{Allineamento a destra} & \textbf{Allineamento a sinistra} & \textbf{Allineamento al centro}  \\
				\toprule
				r & l & c \\
				\bottomrule
			\end{tabular}
			\caption{Opzioni allineamento del testo}
		\end{center}
	\end{table}

\end{itemize}
L'espressione seguente determina il testo presente nella prima riga
\begin{lstlisting}
\textbf{Prima colonna} & \textbf{Seconda colonna}  \\
\end{lstlisting}
\begin{itemize}
\item Come norma usiamo il comando \lstinline[columns=fixed]{textbf} per evidenziare il testo della prima riga
\item Per separare il contenuto di una colonna dall'altra si usa il carattere \lstinline[columns=fixed]{&}
\item Dopo ogni riga bisogna aggiungere \lstinline[columns=fixed]{\\} per andare a capo
\end{itemize}

\subsection{Immagini}
\subsubsection{Esempio di inclusione di un immagine}
Codice per includere un immagine
\begin{lstlisting}
\begin{figure}[H]
    \centering
    \includegraphics[scale = 0.25]{../Stile/Logo.png}
    \caption{Il nostro logo}
    \label{fig:logo}
\end{figure}
\end{lstlisting}
Codice compilato
\begin{figure}[H]
    \centering
    \includegraphics[scale = 0.5]{../Stile/Logo.png}
    \caption{Il nostro logo}
    \label{fig:logo}
\end{figure}
\subsubsection{Spiegazione dell' inclusione dell'immagine}
\begin{itemize}
\item L'attributo \lstinline[columns=fixed]{scale} determina la grandezza dell'immagine
\item Il comando \lstinline[columns=fixed]{caption} definisce la didascalia
\item Il comando \lstinline[columns=fixed]{label} definisce un etichetta attraverso la quale si potrà fare riferimento all'immagine con il comando \lstinline[columns=fixed]{ref} \\
\qquad Ecco un riferimento all'immagine \ref{fig:logo}
\end{itemize}

\subsection{Codice}
\subsubsection{Esempio di inclusione di codice}
Codice per includere del codice
\begin{lstlisting}
begin{lstlisting}
Testo di esempio
end{lstlisting}
\end{lstlisting}
Codice compilato
\begin{lstlisting}
Testo di esempio
\end{lstlisting}

\subsection{Url}
\subsubsection{Esempio di inclusione di un url}
Codice per includere un url
\begin{lstlisting}
\url{https://www.unipd.it/}
\end{lstlisting}
Codice compilato\\
\url{https://www.unipd.it/}
