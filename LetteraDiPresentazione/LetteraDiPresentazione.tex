%ricordarsi di compilare più di una volta
\documentclass[a4paper,12pt]{letteracdp}
\usepackage{../Stile/mercuryseven}
\usepackage[demo]{graphicx}
\usepackage{titlesec}

\address{
\begin{figure}
\centering
\begin{minipage}{.5\textwidth}
  \centering
  \includegraphics[width=.5\linewidth]{../Stile/Unipd.png}
  \label{fig:test1}
\end{minipage}%
\begin{minipage}{.5\textwidth}
  \centering
  \includegraphics[width=.8\linewidth]{../Stile/Logo.png}
  \label{fig:test2}
\end{minipage}
\end{figure}

	
    \vspace{0.2cm}
	Ingegneria del Software \\
	Gruppo \textit{\gruppo} \\
}

\signature{
	\Responsabile{} \\
	\RdP{}\\
	\vspace{0.2cm}
	\includegraphics[scale=0.25]{firme_membri/firma-dg.png}\par
}

\date{11 gennaio 2021}

\pagenumbering{gobble}
\begin{document}

\begin{letter}{
	Prof. \Tullio{} \\
	Prof. \Riccardo{} \\
	Dipartimento di Matematica \\
	Università degli Studi di Padova \\
	Via Trieste, 63 \\
	35121 Padova}
	
\opening{Egregio Professore Tullio Vardanega, \newline
Egregio Professore Riccardo Cardin,}

\begin{quotation}
	\noindent 
	Con la presente il gruppo \gruppo{} intende comunicarVi ufficialmente l'impegno alla realizzazione del prodotto da Voi commissionato, denominato \progetto{}, proposto dall’azienda \proponente.\newline
	Si allegano alla presente lettera i seguenti documenti:
	\begin{itemize}
		\item \docSdF{}\versSdF;
		\item \docAdR{}\versAdR;
		\item \docG{}\versGlo;
		\item \docNdP{}\versNdP;
		\item \docPdP{}\versPdP;
		\item \docPdQ{}\versPdQ;
		\item \docVI{}2020-11-09;
		\item \docVI{}2020-11-16;
		\item \docVI{}2020-11-23;
		\item \docVI{}2020-11-30;
		\item \docVI{}2020-11-07;
		\item \docVI{}2020-12-18;
		\item \docVI{}2020-12-23;
		\item \docVI{}2021-01-04;
		\item \docVI{}2021-01-06;
		\item \docVE{}2020-12-28;
		\item \docVE{}2020-12-31;
		
	\end{itemize}
Come descritto nel documento \docPdP{}\versPdP{}, il gruppo \gruppo{} ha preso l'impegno di consegnare il prodotto richiesto entro la prima data fissata per la Revisione di Accettazione, preventivando costi per un importo totale di \textbf{\EUR{13058}}. \newline
Di seguito si riportano i nomi dei componenti del gruppo e i rispettivi numeri di matricola:
	
	
\renewcommand{\arraystretch}{1}
	\begin{table}
		\begin{center}
			\setlength{\aboverulesep}{0pt}
			\setlength{\belowrulesep}{0pt}
			\setlength{\extrarowheight}{.75ex}
			\rowcolors{2}{AzzurroGruppo!10}{white}
			\begin{tabular}{ c c c }
				\rowcolor{AzzurroGruppo!30} 
				\textbf{Nominativo} & \textbf{Matricola}\\
				\toprule
				
				
		\Daniele{} & \DanieleM{} \\
		\Davide{} & \DavideM{} \\
		\Francesco{} & \FrancescoM{} \\
		\Giosue{} & \GiosueM{} \\
		\Lucrezia{} & \LucreziaM{} \\
		\Matteo{} & \MatteoM{} \\
		\Tommaso{} & \TommasoM{} \\	
				
				\bottomrule
			\end{tabular}
		\end{center}
	\end{table}
	
	Rimaniamo a Vostra completa disposizione per qualsiasi chiarimento.

\vspace{0.5cm}
\closing{ Cordiali Saluti,}

\end{quotation}

\end{letter}

\end{document}
