\documentclass[a4paper, oneside]{article}
\usepackage{../../Stile/mercuryseven}
\newcommand{\Titolo}{Piano Di Progetto}

\newcommand{\VersioneDocumento}{\versPdP}

\newcommand{\ACapoRedazione}{\textit{\Davide{}} \newline \textit{\Francesco{}} \newline{} \textit{\Matteo{}}}

\newcommand{\Verifica}{\textit{\Daniele{}} \newline{} \textit{\Lucrezia{}} \newline{} \textit{\Giosue{}}}

\newcommand{\Responsabile}{\textit{\Tommaso{}}}

\newcommand{\Distribuzione}{\textit{Prof. \Tullio{}} \newline \textit{Prof. \Riccardo{}} \newline Gruppo \textit{\Gruppo{}}}

\newcommand{\Uso}{Esterno}

\newcommand{\DescrizioneDoc}{Descrizione della pianificazione realizzata da \textit{\Gruppo{}}, l'analisi dei rischi che potrebbero verificarsi, il prospetto economico e l'organigramma}

\setcounter{secnumdepth}{0}
\begin{document}
\copertina{}

\section*{\centerline{Registro delle modifiche}}
{
	\newlength{\freewidth}
	\setlength{\freewidth}{\dimexpr\textwidth-10\tabcolsep}
	\renewcommand{\arraystretch}{1.5}
	\centering
	\setlength{\aboverulesep}{0pt}
	\setlength{\belowrulesep}{0pt}
	\rowcolors{2}{AzzurroGruppo!10}{white}
	\begin{longtable}{C{.131777\freewidth} C{.134036\freewidth} C{.263554\freewidth} C{.188254\freewidth} C{.282379\freewidth}}
		\toprule 
		\rowcolor{AzzurroGruppo!30}
		\textbf{Versione} & \textbf{Data} & \textbf{Autore} & \textbf{Ruolo} & \textbf{Descrizione}\\
		\toprule
		\endhead

		v1.0.0-0.10 & 2021-04-15 & \Matteo{} & \RdP{} & Approvazione del documento. \\
		v0.4.0-0.10 & 2021-04-14 & \Daniele{}, \newline{} \Lucrezia{} & \ver{}, \newline{} \prog{} & Correzione e verifica segnalazioni Product Baseline.\\
		v0.3.0-0.9 & 2021-04-05 & \Daniele{}, \newline{} \Giosue{} & \ver{}, \newline{} \progr{} & Stesura e verifica glossario. \\
		v0.2.1-0.9 & 2021-04-03 & \Davide{}, \newline{} \Lucrezia{} & \ver{}, \newline{} \prog{} & Aggiunta e verifica immagini. \\
		v0.2.0-0.9 & 2021-04-02 & \Davide{}, \newline{} \Giosue{} & \ver{}, \newline{} \progr{} & Stesura e verifica test e archittettura. \\
		v0.1.0-0.9 & 2021-03-30 & \Daniele{}, \newline{} \Giosue{} & \ver{}, \newline{} \progr{} & Stesura e verifica tecnologie e librerie. \\
		v0.0.2-0.9 & 2021-03-26 & \Davide{}, \newline{} \Lucrezia{} & \ver{},\newline{} \prog{} & Stesura e verifica introduzione. \\
		v0.0.1-0.8 & 2021-03-23 & \Davide{},\newline{} \Lucrezia{} & \ver{},\newline{} \prog{} & Stesura e verifica scheletro documento. \\
		\bottomrule
		\hiderowcolors
	\end{longtable}
}
\newpage
%%%%%%%%%%%%%%%%%%%%%%%%%%%%%%%%%%%%%%%%%%%%%%%%%%%%%%%%%%%%%%%%%%%%%%%%%%%%%%%%%%%%%%%%%%%%%%%%%%%%%%%
%SOMMARIO
\newpage
\tableofcontents

\newpage

% QUESTA SEZIONE È GENERATA AUTOMATICAMENTE. NON MODIFICARLA !!!! MAI !!!! DIRETTAMENTE O PERDERETE LE MODIFICHE EFFETTUATE
% --------------------------------------------------- INIZIO_SEZIONE_GENERATA_AUTOMATICAMENTE ---------------------------------------------------
\section{A}
\begin{description}
  \item[API] Acronimo di Application Programming Interface, sono set di definizioni e protocolli con i quali vengono realizzati e integrati software applicativi. Consentono ai prodotti o servizi di comunicare con altri prodotti o servizi senza sapere come vengono implementati, semplificando così lo sviluppo delle app e consentendo un netto risparmio di tempo e denaro.
  \item[AWS] Acronimo per Amazon Web Service. É una collezione di servizi di cloud computing on demand che costituiscono la piattaforma di cloud computing offerta da \glo{Amazon}.
  \item[Agile] Nell'ingegneria del software, l'espressione metodologia agile si riferisce ad un insieme di metodi di sviluppo del software emersi a partire dai primi anni 2000 e fondati su un insieme di principi comuni, derivati dal "Manifesto per lo sviluppo agile del software", pubblicato nel 2001 da Kent Beck, Robert C. Martin, Martin Fowler e altri. I metodi agili si contrappongono al modello a cascata e altri modelli di sviluppo tradizionali, proponendo un approccio meno strutturato e focalizzato sull'obiettivo di consegnare al cliente, in tempi brevi e frequentemente, software funzionante e di qualità.
  \item[Amazon] Azienda di commercio elettronico statunitense, con sede a Seattle nello stato di Washington; è la più grande Internet company al mondo. Dal 2002, Amazon fornisce commercialmente una suite di servizi web e di cloud computing, chiamata \glo{AWS}.
\end{description}
\newpage
\section{B}
\begin{description}
  \item[Branch] Traducibile con ramo o ramificazione, vengono utilizzati in \glo{Git} per l'implementazione di funzionalità tra loro isolate, cioè sviluppate in modo indipendente l'una dall'altra, ma a partire dalla medesima radice.
\end{description}
\newpage
\section{C}
\begin{description}
  \item[CSV] Acronimo di Comma Separated Values, è un formato di file basato su file di testo utilizzato per l'importazione ed esportazione di una tabella di dati. Non esiste uno standard formale che lo definisca, ma solo alcune prassi più o meno consolidate.
  \item[Casi d'uso] Insieme di scenari che hanno in comune un obiettivo per un utente. Inglese: use case.
  \item[Code base] L'intera collezione di codice sorgente usata per costruire una particolare applicazione o un particolare componente. Tipicamente, il codebase include solo file di codice sorgente scritto a mano, e non ad esempio file di codice sorgente generati da altri tool o file binari di libreria.
\end{description}
\newpage
\section{F}
\begin{description}
  \item[Framework Qt] É un framework per lo sviluppo di interfacce grafiche multipiattaforma, basato su C++.
\end{description}
\newpage
\section{G}
\begin{description}
  \item[GNU aspell] É un software libero per il controllo ortografico. Sviluppato dal Progetto GNU e pubblicato sotto GNU LGPL.
  \item[Git] É un software libero di controllo di versione distribuito creato da Linus Torvalds nel 2005. Consente, tra le tante cose, il controllo delle versioni e di far collaborare più developer su uno stesso progetto contemporaneamente.É un software libero di controllo di versione distribuito creato da Linus Torvalds nel 2005. Consente, tra le tante cose, il controllo delle versioni e di far collaborare più developer su uno stesso progetto contemporaneamente.
  \item[GitHub] GitHub è un servizio di hosting per progetti software. Il nome deriva dal fatto che "GitHub" è una implementazione di Git.
  \item[GraphQL] É un linguaggio di interrogazione lato server per API, in grado di fornire ai client unicamente i dati di cui hanno bisogno.
\end{description}
\newpage
\section{I}
\begin{description}
  \item[Issue] É costituita da un problema che impedisce o rallenta l'avanzamento di un progetto e che necessita di un lavoro aggiuntivo da parte del \RdP{}, del team di progetto e di eventuali esperti esterni per trovare una soluzione che consenta di superarlo o risolverlo.
\end{description}
\newpage
\section{L}
\begin{description}
  \item[Lower Camel Case] É una convenzione di denominazione in cui un nome è formato da più parole unite insieme, la prima lettera di ciascuna delle parole (eccetto la prima) è maiuscola.
\end{description}
\newpage
\section{M}
\begin{description}
  \item[MQTT] Il Message Queue Telemetry Transport è un protocollo ISO standard (ISO/IEC PRF 20922) di messaggistica leggero di tipo \glo{publish/subscriber} posizionato in cima a TCP/IP. È stato progettato per le situazioni in cui è richiesto un basso impatto e dove la banda è limitata. Il pattern \glo{publish/subscriber} richiede un message broker, che è il responsabile della distribuzione dei messaggi ai client destinatari. Questo protocollo è molto utilizzato nei dispositivi IoT (Internet of Things).
  \item[MVC] Acronimo di Model View Controller, in informatica, è un pattern architetturale molto diffuso nello sviluppo di sistemi software, in particolare nell'ambito della programmazione orientata agli oggetti e in applicazioni web, in grado di separare la logica di presentazione dei dati dalla logica di business.
  \item[Merge] {scrivere o ignorare questa definizione}
  \item[Metriche] {scrivere o ignorare questa definizione}
\end{description}
\newpage
\section{P}
\begin{description}
  \item[Pascal Case] É una convenzione di denominazione in cui un nome è formato da più parole unite insieme, la prima lettera di ciascuna delle parole è maiuscola.
  \item[Product Backlog] É la lista ordinata di tutti gli elementi che servono nel prodotto. Elenca caratteristiche, funzioni, requisiti, miglioramenti e correzioni che costituiscono le modifiche da apportare alle versioni future del prodotto.
  \item[Proponente] {scrivere o ignorare questa definizione}
  \item[Publisher/Subscriber] {scrivere o ignorare questa definizione}
\end{description}
\newpage
\section{R}
\begin{description}
  \item[RFID] Acronimo di Radio Frequency IDentification, in telecomunicazioni ed elettronica, si intende una tecnologia per l'identificazione e/o memorizzazione automatica di informazioni inerenti a oggetti, animali o persone basata su particolari etichette elettroniche chiamate tag, capaci di memorizzare tali dati e di rispondere all'interrogazione a distanza da parte di appositi apparati fissi o portatili chiamati reader.
  \item[Repository] Archivio digitale dove dati e informazioni sono raccolti, valorizzati e archiviati sulla base di metadati che ne permettono la rapida individuazione. Grazie alla sua peculiare architettura, un repository consente di gestire in modo ottimale anche grandi volumi di dati.
  \item[Requisiti] Funzionalità che il nuovo prodotto (o il prodotto modificato) deve offrire. Possono essere: \begin{itemize} \item obbligatori: requisiti alla quale gli stakeholder non sono disposti a rinunciare; \item desiderabili: requisiti non strettamente necessari, ma che forniscono un valore aggiunto al prodotto; \item opzionali: requisiti relativamente utili, oppure contrattabili in fasi di progetto più avanzate. \end{itemize}
\end{description}
\newpage
\section{S}
\begin{description}
  \item[Slack] É un software che rientra nella categoria degli strumenti di collaborazione aziendale che mette a disposizione svariate funzionalità come la messaggistica istantanea, la condivisione di file, la creazione di canali testuali separati e molto altro.
  \item[Software] L'insieme delle componenti immateriali (strato logico/intangibile) di un sistema elettronico di elaborazione; è contrapposto all'hardware, cioè la parte materiale (strato fisico/tangibile) dello stesso sistema
  \item[Stakeholder] Chiunque sia portatore di interessi nei confronti di un’attività o di un progetto economico: chi cioè possa direttamente o indirettamente godere di benefici o subire danni da quell’attività.
\end{description}
\newpage
\section{T}
\begin{description}
  \item[Team] I membri del gruppo \gruppo{}.
  \item[Test di accettazione] {scrivere o ignorare questa definizione}
  \item[Test di integrazione] {scrivere o ignorare questa definizione}
  \item[Test di regressione] {scrivere o ignorare questa definizione}
  \item[Test di sistema] {scrivere o ignorare questa definizione}
  \item[Test di unità] Sono test eseguiti su isolate porzioni di codice, che ne valutano il comportamento e la logica allo scopo di verificare che funzioni correttamente.
  \item[TexMaker] É un editor LaTeX open source multipiattaforma con un visualizzatore PDF integrato. Integra molti strumenti necessari per sviluppare documenti con LaTeX.
\end{description}
\newpage
\section{U}
\begin{description}
  \item[User Story] Brevi e semplici descrizioni di una funzionalità raccontata dal punto di vista dell’utente o del cliente del sistema.
\end{description}
\newpage
\section{V}
\begin{description}
  \item[Verifica] Serve a stabilire che il software rispetti i requisiti e le specifiche richiesti. Si definisce statica se effettuata in fase di progetto, dinamica se effettuata durante lo sviluppo attraverso test creati ad hoc.
\end{description}
\newpage
\section{W}
\begin{description}
  \item[Workspace] {scrivere o ignorare questa definizione}
\end{description}
\newpage
\section{Z}
\begin{description}
  \item[Zextras Drive] {scrivere o ignorare questa definizione}
  \item[Zimbra] {scrivere o ignorare questa definizione}
  \item[Zoom] Zoom è un programma software di videotelefonia sviluppato da Zoom Video Communications. Fornisce un servizio di chat video utilizzabile per meeting online.
\end{description}
\newpage
% --------------------------------------------------- FINE_SEZIONE_GENERATA_AUTOMATICAMENTE ---------------------------------------------------
\end{document}