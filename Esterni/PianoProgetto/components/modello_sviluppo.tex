\section{Modello di sviluppo }

Come \glo{team} di sviluppo dobbiamo poter garantire in ogni momento la qualità del prodotto \glo{software} e la sua conformità rispetto ai \glo{requisiti}. \newline
Il modello \glo{iterativo} permette un ciclo evolutivo più flessibile e favorisce il dialogo con gli \glo{stackholder}, adattandosi più facilmente alle esigenze. \newline 
Per tali motivi abbiamo deciso di adottare questo tipo di modello, riducendo i rischi e producendo nuovo valore ad ogni iterazione.

%info
%https://medium.com/geekandjob-blog/scrum-cos%C3%A8-e-come-funziona-metodologia-agile-7c8988feec01
\subsection{Modello Agile}
Il modello di sviluppo \glo{agile} permette di progredire tramite cicli iterativi e incrementali, detti sprint.
É un modello altamente dinamico basato su di un manifesto concepito per evitare un'eccessiva rigidità e valorizzare le risorse disponibili. \newline
Il manifesto contiene questi quattro principi fondanti:

\begin{enumerate}
    \item \textbf{Gli individui e le interazioni più che i processi e gli strumenti}: le persone che fanno parte del progetto producono valore e sono una grande risorsa diponibile;
    \item \textbf{Il software funzionante più che la documentazione esaustiva}: la documentazione va snellita e resa più flessibile senza però perderne in qualità, allocando più risorse al \glo{software}
    \item \textbf{La collaborazione col cliente più che la negoziazione dei contratti}: un contratto inflessibile può essere una potenziale perdità di valore per gli \glo{attori} coinvolti. Va quindi incentivata l’interazione con gli \glo{stakeholder} durante lo sviluppo del progetto;
    \item \textbf{Rispondere al cambiamento più che seguire un piano}: i cambiamenti sono probabili durante lo sviluppo di grandi progetti per questo pianificare tutto fin dall'inizio potrebbe essere uno spreco di risorse.
\end{enumerate}

% FIXME compila con rientro sfasato
L’idea di base per implementare tali principi è l’utilizzo del \textbf{\glo{Product Backlog}}, che cresce man mano che il prodotto viene costruito. \newline
Il lavoro viene quindi suddiviso in piccole funzionalità chiamate \textbf{\glo{User Story}} le quali poi vengono selezionate per essere portate a ternime negli \textbf{sprint}, venendo, di fatto, sviluppate indipendentemente in una sequenza continua dall’analisi all’integrazione. \newline

Gli obiettivi strategici sono quindi:
\begin{itemize}
    \item poter costantemente e in ogni momento dimostrare al cliente ciò che è stato fatto;
    \item verificare l’avanzamento tramite il progresso reale dello sviluppo;
    \item soddisfare e motivare gli sviluppatori con risultati immediati;
    \item assicurare e dimostrare una buona \glo{verifica} e integrazione dell’intero prodotto \glo{software}.
\end{itemize}


\begin{figure}[H]
    \centering
    \includegraphics[scale = 0.25]{components/img/agile.png}
    \caption{Rappresentazione del modello agile}
    \label{fig:Rappresentazione del modello agile}
\end{figure}