\section{Modello di sviluppo}

Dobbiamo poter garantire in ogni momento la qualità del prodotto e la sua conformità rispetto ai requisiti richiesti.
Per questo motivo abbiamo scelto il modello di sviluppo incrementale in quanto ad ogni iterazione 
riduce il rischio di fallimento e produce nuovo valore.

\subsection{Modello incrementale}

Il modello di sviluppo incrementale permette di progredire tramite cicli di incremento, 
ripetuti fino a quando il prodotto non soddisferà i requisiti richiesti dal cliente.
Il ciclo di incremento risulta suddiviso nei seguenti passi:
\begin{enumerate}
    \item pianificazione;
    \item analisi dei requisiti;
    \item progettazione;
    \item implementazione;
    \item test;
    \item valutazione.
\end{enumerate}
Ogni ciclo di incremento permette lo sviluppo di una funzionalità aggiuntiva,
preventivamente discussa con il cliente durante la fase iniziale del ciclo stesso.
la ciclicità prevista dal modello incrementale facilita anche il versionamento del sistema,
tracciando modifiche nette al software.

\begin{figure}[H]
    \centering
    \includegraphics[scale = 0.5]{components/img/incrementale.png}
    \caption{Rappresentazione del modello incrementale}
    \label{fig:logo}
\end{figure}

l'utilizzo di questo modello permette di ottenere importanti vantaggi, quali:
\begin{itemize}
    \item assegnare una maggior priorità alle funzionalità primarie, in modo tale da poter essere sottoposte al cliente nel minor tempo possibile;
    \item l'utilizzo dello sviluppo per incrementi successivi limita la modifica e le correzioni degli errori al singolo incremento, risultano meno onerose in termini di tempo e, di conseguenza, di costi;
    \item le verifiche e i test sono circoscritti al singolo incremento, cioè alle nuove funzionalità;
    \item possibilità di un maggior numero di feedback da parte del cliente, aumentando l'efficienza.
\end{itemize}