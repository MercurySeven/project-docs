\section{Analisi dei Rischi}
Durante lo sviluppo di un progetto la probabilità che si verifichino delle criticità è alta, 
soprattutto quando si tratta di progetti complessi e/o di grandi dimensioni, per questo motivo
un'attenta analisi dei rischi può aiutare ad evitare o quantomeno a gestire al meglio tali criticità.
Il piano di gestione dei rischi si suddivide in 4 attività:
\begin{itemize}
    \item \textbf{Individuazione dei rischi}: \textit{TODO;}
    \item \textbf{Analisi dei rischi}: \textit{TODO;}
    \item \textbf{Pianificazione di controllo}: \textit{TODO;}
    \item \textbf{Monitoraggio dei rischi}: \textit{TODO.}
\end{itemize}
Abbiamo suddiviso i pricipali fattori di rischio nelle seguenti categorie:
\begin{itemize}
    \item Rischi legati alle tecnologie
    \item Rischi legati all’organizzazione
    \item Rischi legati alle persone
\end{itemize}

\subsection{Rischi legati alle tecnologie}

\paragraph{TODO Nome rischio}
\renewcommand{\arraystretch}{1}
	\begin{table}[H]
		\begin{center}
			\setlength{\aboverulesep}{0pt}
			\setlength{\belowrulesep}{0pt}
			\setlength{\extrarowheight}{.75ex}
			\rowcolors{2}{AzzurroGruppo!10}{white}
			\begin{tabular}{ c c }
				\rowcolor{AzzurroGruppo!30} 
				%\textbf{Prima colonna} & \textbf{Seconda colonna}  \\
                \toprule
                Descrizione & TODO \\
				Conseguenze & TODO \\
                Occorrenza & TODO \\
                Prericolosità & TODO \\
                Precauzioni & TODO \\
                Piano di contingenza & TODO \\
				\bottomrule
			\end{tabular}
			\caption{Tabella dettaglio TODO nome rischio}
		\end{center}
    \end{table}

\paragraph{TODO Nome rischio}
\renewcommand{\arraystretch}{1}
    \begin{table}[H]
        \begin{center}
            \setlength{\aboverulesep}{0pt}
            \setlength{\belowrulesep}{0pt}
            \setlength{\extrarowheight}{.75ex}
            \rowcolors{2}{AzzurroGruppo!10}{white}
            \begin{tabular}{ c c }
                \rowcolor{AzzurroGruppo!30} 
                %\textbf{Prima colonna} & \textbf{Seconda colonna}  \\
                \toprule
                Descrizione & TODO \\
                Conseguenze & TODO \\
                Occorrenza & TODO \\
                Prericolosità & TODO \\
                Precauzioni & TODO \\
                Piano di contingenza & TODO \\
                \bottomrule
            \end{tabular}
            \caption{Tabella dettaglio TODO nome rischio}
        \end{center}
    \end{table}

\subsection{Rischi legati all’organizzazione}

\subsection{Rischi legati alle persone}