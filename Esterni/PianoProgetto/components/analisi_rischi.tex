\section{Analisi dei Rischi}
Durante lo sviluppo di un progetto la probabilità che si verifichino delle criticità è alta, 
soprattutto quando si tratta di progetti complessi e/o di grandi dimensioni, per questo motivo
un'attenta analisi dei rischi può aiutare ad evitare o quantomeno a gestire al meglio tali criticità.
Il piano di gestione dei rischi si suddivide in 4 attività:
\begin{itemize}
    \item \textbf{Individuazione dei rischi}: \textit{TODO;}
    \item \textbf{Analisi dei rischi}: \textit{TODO;}
    \item \textbf{Pianificazione di controllo}: \textit{TODO;}
    \item \textbf{Monitoraggio dei rischi}: \textit{TODO.}
\end{itemize}
Abbiamo suddiviso i pricipali fattori di rischio nelle seguenti categorie:
\begin{itemize}
    \item Rischi legati alle tecnologie
    \item Rischi legati all’organizzazione
    \item Rischi legati alle persone
\end{itemize}

\subsection{Rischi legati alle tecnologie}

\paragraph{Inesperienza Tecnologica}
\renewcommand{\arraystretch}{1}
	\begin{table}[H]
		\begin{center}
			\setlength{\aboverulesep}{0pt}
			\setlength{\belowrulesep}{0pt}
			\setlength{\extrarowheight}{.75ex}
			\rowcolors{2}{AzzurroGruppo!10}{white}
			\begin{tabular}{ c c }
				\rowcolor{AzzurroGruppo!30} 
				%\textbf{Prima colonna} & \textbf{Seconda colonna}  \\
                \toprule
                Descrizione & Il team presanta un'esperienza eterogenea e superficiale nell'utilizzo di molte tecnologie richieste in questo progetto. \\
				Conseguenze & Potrebbe comportare l'insorgere di problemi operativi o ritardi dovuti ai tempi di apprendimento soggettivi. \\
                Occorrenza & Media \\
                Prericolosità & Media \\
                Precauzioni & Il \textit(Responsabile di progetto) si occuperà di suddividere il carico di lavoro in modo bilanciato, tenendo conto delle capacità del singolo individuo. Ogni membro del team dovrà quindi preoccuparsi di segnalare le sue conoscenze preliminari al \textit(Responsabile di progetto). \\
                Piano di contingenza & Ogni membro del team avrà cura di prendere dimestichezza con le tecnologie impiegate in modo tale da diventare il più possibile autonomo nel proprio lavoro. I compiti più complessi saranno affidati a più membri in modo tale da velocizzare i tempi di sviluppo e favorire l'aiuto reciproco. \\
				\bottomrule
			\end{tabular}
			\caption{Tabella dettaglio rischio di Inesperienza Tecnologica}
		\end{center}
    \end{table}

\paragraph{Problematiche Hardware}
\renewcommand{\arraystretch}{1}
    \begin{table}[H]
        \begin{center}
            \setlength{\aboverulesep}{0pt}
            \setlength{\belowrulesep}{0pt}
            \setlength{\extrarowheight}{.75ex}
            \rowcolors{2}{AzzurroGruppo!10}{white}
            \begin{tabular}{ c c }
                \rowcolor{AzzurroGruppo!30} 
                %\textbf{Prima colonna} & \textbf{Seconda colonna}  \\
                \toprule
                Descrizione & Ogni membro del team lavora allo sviluppo utilizzando il proprio computer personale, quest'ultimo può essere soggetto a malfunzionamenti. \\
                Conseguenze & Il guasto di uno o più computer porterebbe ad una parziale perdita di dati causando ritardi. Sebbene remota esiste anche la possibilità di una totale perdita di dati qualora l'avaria avvenisse in tutti i computer contemporaneamnte. \\
                Occorrenza & Bassa \\
                Prericolosità & Alta \\
                Precauzioni & Ogni membro del team si adopererà per eseguire backup frequenti durante il processo di sviluppo e mantenere operativo il proprio hardware. \\
                Piano di contingenza & Nel caso in cui si verifichi un problema hardware bisognerà procedere a riparare il danno il più rapidamente possibile. A seconda delle norme sulla mobilità vigenti, sarà possibile utilizzare i computer messi a diposizione nei laboratori informatici dell'Ateneo. \\
                \bottomrule
            \end{tabular}
            \caption{Tabella dettaglio rischio di Problematiche Hardware}
        \end{center}
    \end{table}

\paragraph{Problematiche Software}
\renewcommand{\arraystretch}{1}
    \begin{table}[H]
        \begin{center}
            \setlength{\aboverulesep}{0pt}
            \setlength{\belowrulesep}{0pt}
            \setlength{\extrarowheight}{.75ex}
            \rowcolors{2}{AzzurroGruppo!10}{white}
            \begin{tabular}{ c c }
                \rowcolor{AzzurroGruppo!30} 
                %\textbf{Prima colonna} & \textbf{Seconda colonna}  \\
                \toprule
                Descrizione & il team si avvale di servizi e software di terze parti per la gestione dei documenti e della \glossary{codebase}. La buona gestione di questi ultimi non è competenza del team. \\
                Conseguenze & Eventuali disservizi potrebbero causare gravi perdite di dati e ritardi. \\
                Occorrenza & Bassa \\
                Prericolosità & Alta \\
                Precauzioni & Ogni membro del team si impegnerà a tenere aggiornato un backup locale di tutti i \glossary{repository} remoti. \\
                Piano di contingenza & Il \textit{Responsabile di progetto} avrà cura di cercare altri servizi simili ai precedenti e più affidabili. \\
                \bottomrule
            \end{tabular}
            \caption{Tabella dettaglio rischio di Problematiche Software}
        \end{center}
    \end{table}

\subsection{Rischi legati all’organizzazione}

\paragraph{TODO}
\renewcommand{\arraystretch}{1}
    \begin{table}[H]
        \begin{center}
            \setlength{\aboverulesep}{0pt}
            \setlength{\belowrulesep}{0pt}
            \setlength{\extrarowheight}{.75ex}
            \rowcolors{2}{AzzurroGruppo!10}{white}
            \begin{tabular}{ c c }
                \rowcolor{AzzurroGruppo!30} 
                %\textbf{Prima colonna} & \textbf{Seconda colonna}  \\
                \toprule
                Descrizione & TODO \\
                Conseguenze & TODO \\
                Occorrenza & TODO \\
                Prericolosità & TODO \\
                Precauzioni & TODO \\
                Piano di contingenza & TODO \\
                \bottomrule
            \end{tabular}
            \caption{Tabella dettaglio rischio di TODO}
        \end{center}
    \end{table}

\subsection{Rischi legati alle persone}

\paragraph{TODO}
\renewcommand{\arraystretch}{1}
    \begin{table}[H]
        \begin{center}
            \setlength{\aboverulesep}{0pt}
            \setlength{\belowrulesep}{0pt}
            \setlength{\extrarowheight}{.75ex}
            \rowcolors{2}{AzzurroGruppo!10}{white}
            \begin{tabular}{ c c }
                \rowcolor{AzzurroGruppo!30} 
                %\textbf{Prima colonna} & \textbf{Seconda colonna}  \\
                \toprule
                Descrizione & TODO \\
                Conseguenze & TODO \\
                Occorrenza & TODO \\
                Prericolosità & TODO \\
                Precauzioni & TODO \\
                Piano di contingenza & TODO \\
                \bottomrule
            \end{tabular}
            \caption{Tabella dettaglio rischio di TODO}
        \end{center}
    \end{table}