\section{Test}

In questa sezione verrà trattato il testing del codice, in particolare i framework utilizzati e l'automatizzazione dello svolgimento dei test.

\subsection{unittest — Unit testing framework}

Il framework unittest viene utilizzato per effettuare i test di unità, per eseguire i test basterà eseguire il comando 

\centerline{ \textbf{python -m unittest discover -v -s "./tests/" -p "*\_test.py"}}

Verranno quindi elencati tutti i test eseguiti ed il loro risultato.


\subsection{Analisi statica}

\subsubsection{pre-commit}

Per essere sicuri che codice non conforme agli standard di qualità finisca nella repository si è utilizzato il plugin pre-commit. Esso dovrà essere attivato con il comando

\centerline{\textbf{pre-commit install}}

Dopo aver eseguito questo comando ad ogni commit il codice verrà formattato automaticamente tramite autopep8 e non sarà possibile eseguire con successo il commit se non supera i test statici di Flake8

\subsubsection{Flake8}

I test di analisi statica del codice vengono effettuati tramite lo strumento Flake8. Per effettuare i test sul codice è necessario eseguire il comando

\centerline{\textbf{Run flake8}}
In caso di errori verrà indicato quali sono ed in che file. 

\subsection{Codecov}

Per svolgere l'attività di \glo{code coverage} è stato scelto di utilizzare Codecov. Per far eseguire le revisioni automatiche del codice bisogna attivare la repository di GitHub dal sito di Codecov.

\subsection{GitHub Actions}
Il servizio di CI utilizzato è GitHub Actions, con dei workflow personalizzati presenti nella cartella .github. Queste Actions vanno a controllare staticamente il codice tramite Flake8, ad eseguire il codice ed i test su MacOS, Windows ed Ubuntu ed infine mandare una notifica della buona o cattiva riuscita sul canale predisposto di Slack.