\section{Tecnologie e Librerie utilizzate}
\subsection{Tecnologie}
{
	\setlength{\freewidth}{\dimexpr\textwidth-1\tabcolsep}
	\renewcommand{\arraystretch}{1.5}
	\setlength{\aboverulesep}{0pt}
	\setlength{\belowrulesep}{0pt}
	\rowcolors{2}{AzzurroGruppo!10}{white}
	\begin{longtable}{L{.15\freewidth} L{.20\freewidth} L{.60\freewidth}}
		\rowcolor{AzzurroGruppo!30}
		\textbf{Tecnologia} & \textbf{Versione minima} & \textbf{Link}\\
		\toprule
		\endhead	
		Python & 3.9.4 & \url{https://www.python.org/downloads/}\\
		Zextras Drive & Zimbra 8.8.12 &  \url{https://docs.zextras.com/zextras-suite-documentation/latest/drive.html}\\	
		Qt & 6 & \url{https://www.qt.io/download}\\
		GraphQL & - & \url{https://graphql.org/} \\
		INI & - & -\\
		\bottomrule
		\hiderowcolors
		\caption{Tecnologie utilizzate e la loro versione minima supportata}
	\end{longtable}
\subsubsection{Python}
\gloman{Python} è un linguaggio di programmazione interpretato ad alto livello. È stato scelto per la sua semplicità e per la grande quantità di librerie presenti per questo linguaggio. Era inoltre uno dei pochi linguaggi per i quali sono presenti librerie di \gloman{Qt} e librerie per effettuare chiamate \gloman{GraphQL}.
\subsubsection{Zextras Drive}
\gloman{Zextras Drive} è un sistema di archiviazione e collaborazione in cloud per \gloman{Zimbra}. È utilizzato dall'applicazione per caricare e scaricare i file sincronizzati.
\subsubsection{Qt}
Qt è una libreria per lo sviluppo di applicazioni multipiattaforma. Tramite le sue librerie ed il sistema di segnali e slot ha permesso di creare la \gloman{GUI} multipiattaforma rispettando il pattern architetturale \gloman{MVC} con estrema semplicità.
\subsubsection{GraphQL}
GraphQL è un linguaggio di query. È stato utilizzato per comunicare con Zextras Drive grazie alla sua semplicità sia nell'utilizzo e sia grazie alla facilità di espansione delle query e quindi di futura manutenzione.
\subsubsection{INI} \label{sec:ini}
\textit{INI} è il file di inizializzazione di formato testuale utilizzato per memorizzare le opzioni di funzionamento del programma. Questo formato è stato scelto perché risulta facilmente interpretabile dall'uomo e facilmente leggibile dalla macchina. Queste caratteristiche lo rendono un ottimo formato per file di configurazione non complessi.
\subsection{Librerie}

{
	\setlength{\freewidth}{\dimexpr\textwidth-1\tabcolsep}
	\renewcommand{\arraystretch}{1.5}
	\setlength{\aboverulesep}{0pt}
	\setlength{\belowrulesep}{0pt}
	\rowcolors{2}{AzzurroGruppo!10}{white}
	\begin{longtable}{L{.15\freewidth} L{.15\freewidth} L{.70\freewidth}}
		\rowcolor{AzzurroGruppo!30}
		\textbf{Libreria} & \textbf{Versione minima} & \textbf{Descrizione ed utilizzo libreria}\\
		\toprule
		\endhead		
		PySide6 & 6.0.2 & \gloman{PySide} è un binding di Qt per Python, è sotto licenza \gloman{LGPL}. Viene utilizzato PySide perchè supporta la versione 6 di Qt.\\
		gql&3.0.0a5 & Client GraphQL per python, utilizzata per realizzare le chiamate GraphQL permettendo di comunicare con Zextras Drive.\\
		watchdog&2.0.2 & Libreria che permette di monitorare eventi del file system. Viene utilizzata per aggiornare l'elenco dei file nella GUI.\\
		aiohttp&3.7.4 & Libreria che permette di realizzare chiamate HTTP. Viene utilizzata per realizzare l'autenticazione con il servizio Zextras Drive.\\
		configparser&5.0.2 & Libreria utilizzata per la lettura dei file INI (descritti a \S{}\ref{sec:ini}).\\
		requests&2.25.1	& Libreria che permette di mandare richieste HTTP/1.1. Viene utilizzata per effettuare il download e l'upload dei file.\\
		pip&20.2.3 & Strumento utilizzato per scaricare e gestire tutte le librerie aggiuntive.\\
				
		\bottomrule
		\hiderowcolors
		\caption{Librerie utilizzate, la loro versione minima supportata ed il loro utilizzo}
	\end{longtable}	
Per installare le librerie elencate fare riferimento a \S{}\ref{sec:Librerie} 