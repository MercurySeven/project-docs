\section{Tecnologie e Librerie utilizzate}

\subsection{Tecnologie}
{
	\setlength{\freewidth}{\dimexpr\textwidth-1\tabcolsep}
	\renewcommand{\arraystretch}{1.5}
	\setlength{\aboverulesep}{0pt}
	\setlength{\belowrulesep}{0pt}
	\rowcolors{2}{AzzurroGruppo!10}{white}
	\begin{longtable}{L{.15\freewidth} L{.20\freewidth} L{.60\freewidth}}
		\rowcolor{AzzurroGruppo!30}
		\textbf{Tecnologia} & \textbf{Versione minima} & \textbf{Link}\\
		\toprule
		\endhead	
		Python & 3.9.4 & \url{https://www.python.org/downloads/}\\
		Zextras Drive & Zimbra 8.8.12 &  \url{https://docs.zextras.com/zextras-suite-documentation/latest/drive.html}\\	
		PySide6 & 6.0.2 & \url{https://pypi.org/project/PySide6/}\\
		GraphQL & - & \url{https://graphql.org/} \\
		INI & - & -\\
		\bottomrule
		\hiderowcolors
		\caption{Tecnologie utilizzate e la loro versione minima supportata}
	\end{longtable}
\subsubsection{Python}
\textit{Python} è un linguaggio di programmazione interpretato ad alto livello. È stato scelto per la sua semplicità e per la grande quantità di librerie presenti per questo linguaggio. Era inoltre uno dei pochi linguaggi per i quali sono presenti librerie di Qt e librerie per effettuare chiamate GraphQL.
\subsubsection{Zextras Drive}

\subsubsection{PySide6}

\subsubsection{GraphQL}

\subsubsection{INI}
\textit{INI} è il file di inizializzazione di formato testuale utilizzato per memorizzare le opzioni di funzionamento del programma. Questo formato è stato scelto perchè risulta facilmente interpretabile dall'uomo e facilmente leggibile dalla macchina. Queste caratteristiche lo rendono un ottimo formato per file di configurazione non complessi.

\subsection{Librerie}

{
	\setlength{\freewidth}{\dimexpr\textwidth-1\tabcolsep}
	\renewcommand{\arraystretch}{1.5}
	\setlength{\aboverulesep}{0pt}
	\setlength{\belowrulesep}{0pt}
	\rowcolors{2}{AzzurroGruppo!10}{white}
	\begin{longtable}{L{.210\freewidth} L{.68\freewidth}}
		\rowcolor{AzzurroGruppo!30}
		\textbf{Libreria} & \textbf{Versione minima} \\
		\toprule
		\endhead		
		PySide6 & 6.0.2 \\
		gql&3.0.0a5 \\
		watchdog&2.0.2 \\
		aiohttp&3.7.4 \\
		configparser&5.0.2 \\
		requests&2.25.1	 \\
		pip&20.2.3\\
				
		\bottomrule
		\hiderowcolors
		\caption{Librerie utilizzate e la loro versione minima supportata}
	\end{longtable}







