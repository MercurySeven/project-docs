\section{Introduzione}
\subsection{Scopo del documento}
Lo scopo di questo documento è illustrare tutte le funzionalità del \gloman{sync engine} richiesto dal capitolato \textit{C7}. L'utente finale in questo modo avrà dunque a disposizione tutte le indicazioni per il corretto uso del software.

\subsection{Scopo del prodotto}
Lo scopo del capitolato \textit{C7} è la creazione di un algoritmo \gloman{solido} ed efficiente in grado di garantire il salvataggio e la sincronizzazione dei cambiamenti presenti in \gloman{cloud}. È inoltre richiesta la creazione di una interfaccia grafica multipiattaforma. Il progetto deve dunque funzionare sui principali sistemi operativi desktop quali Windows, macOS e Linux senza richiedere l'installazione manuale di ulteriori prodotti per il corretto funzionamento. 
\subsection{Obiettivo del prodotto}
L'obiettivo del progetto è la realizzazione di un modulo per la piattaforma \gloman{Zimbra}, che abbia le caratteristiche sopra descritte, e che si appoggi per il suo funzionamento al \gloman{cloud} \gloman{Zextras Drive}.
Nello specifico, lo scopo finale del prodotto è quello di fornire agli utenti un servizio che automatizzi il salvataggio e la sincronizzazione con il \gloman{cloud}.

\subsection{Glossario}
Nel documento sono presenti termini che possono avere significati ambigui o poco chiari a seconda del contesto. Per evitare incomprensioni il gruppo fornisce un glossario individuabile nell'appendice \S{}\ref{sec:Glossario} contenente i termini e la loro spiegazione.\newline{}
Nel documento corrente si è deciso di segnare tutte le parole presenti nel \G{} con una "G" a pedice di ogni parola.
\subsection{Riferimenti}
\subsubsection{Riferimenti normativi}
\begin{itemize}
\item Capitolato d'appalto C7: \newline{} \url{https://www.math.unipd.it/~tullio/IS-1/2020/Progetto/C7.pdf}
\end{itemize}

\subsubsection{Riferimenti informativi}

\begin{itemize}
    \item \url{https://www.zextras.com/}
    \item \textbf{Model-View-Controller - Materiale didattico del corso di Ingegneria del Software}: \newline{}
    \url{https://www.math.unipd.it/~rcardin/pdf/Design\%20Pattern\%20Architetturali\%20-\%20Model\%20View\%20Controller_4x4.pdf}
    \item \textbf{SOLID Principles - Materiale didattico del corso di Ingegneria del Software}: \newline{}
    \url{https://www.math.unipd.it/\%7Ercardin/swea/2021/SOLID\%20Principles\%20of\%20Object-Oriented\%20Design_4x4.pdf}
    \item \textbf{Diagrammi delle classi - Materiale didattico del corso di Ingegneria del Software}: \newline{}
    \url{https://www.math.unipd.it/\%7Ercardin/swea/2021/Diagrammi\%20delle\%20Classi_4x4.pdf}
    \item \textbf{Diagrammi dei package - Materiale didattico del corso di Ingegneria del Software}: \newline{}
    \url{https://www.math.unipd.it/\%7Ercardin/swea/2021/Diagrammi\%20dei\%20Package_4x4.pdf}
    \item \textbf{Diagrammi dei sequenza - Materiale didattico del corso di Ingegneria del Software}: \newline{}
    \url{https://www.math.unipd.it/\%7Ercardin/swea/2021/Diagrammi\%20di\%20Sequenza_4x4.pdf}
    \item \textbf{Design Pattern Comportamentali - Materiale didattico del corso di Ingegneria del Software}: \newline{} \url{https://www.math.unipd.it/\%7Ercardin/swea/2021/Design\%20Pattern\%20Comportamentali_4x4.pdf}
    \item \textbf{Design Pattern Strutturali - Materiale didattico del corso di Ingegneria del Software}: \newline{}
    \url{https://www.math.unipd.it/\%7Ercardin/swea/2021/Design\%20Pattern\%20Strutturali_4x4.pdf}
\end{itemize}

\subsubsection{Riferimenti legali}

\begin{itemize}
    \item \textbf{Licenza GPL-3.0}:\newline{}
    \url{https://opensource.org/licenses/GPL-3.0}\newline{}
    \url{https://github.com/MercurySeven/project-SSD/blob/main/LICENSE.MD}
\end{itemize}
