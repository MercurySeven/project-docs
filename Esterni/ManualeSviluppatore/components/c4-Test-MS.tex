\section{Test}

In questa sezione verrà trattato il testing del codice, in particolare i framework utilizzati e l'automatizzazione dello svolgimento dei test.

\subsection{Unittest - Unit testing framework}
\label{sec:unittest}
Il framework unittest viene utilizzato per effettuare i test di unità, per eseguire i test basterà eseguire il comando: \newline{}
\centerline{ \textbf{python -m unittest discover -v -s "./tests/" -p "*\_test.py"}}\newline{}
Verranno quindi elencati tutti i test eseguiti ed il loro risultato.
La documentazione completa della libreria unittest è disponibile al seguente indirizzo:
\newline{}\centerline{\url{https://docs.python.org/3/library/unittest.html}}

\subsection{Analisi statica}
\subsubsection{Pre-commit}
Per essere sicuri che codice non conforme agli standard di qualità finisca nel \gloman{repository} si è utilizzato il plugin pre-commit. Esso dovrà essere attivato con il comando: \newline{}
\centerline{\textbf{pre-commit install}}\newline{}
Dopo aver eseguito questo comando ad ogni commit il codice verrà formattato automaticamente tramite autopep8 e non sarà possibile eseguire con successo il commit se non supera i test statici di Flake8.
La documentazione completa di pre-commit è disponibile al seguente indirizzo:
\newline{}\centerline{\url{https://pre-commit.com}}

\subsubsection{Flake8}
I test di analisi statica del codice vengono effettuati tramite lo strumento Flake8. Per effettuare i test sul codice è necessario eseguire il comando\newline{}
\centerline{\textbf{Run flake8}}\newline{}
In caso di errori verrà indicato quali sono ed in che file.
La documentazione completa di Flake8 è disponibile al seguente indirizzo:
\newline{}\centerline{\url{https://flake8.pycqa.org/en/latest/}}

\subsection{Codecov}
Per svolgere l'attività di \gloman{code coverage} è stato scelto di utilizzare Codecov.
Per far eseguire le revisioni automatiche del codice bisogna attivare il \gloman{repository} di GitHub dal sito di Codecov.
La documentazione completa di Codecov è disponibile al seguente indirizzo:
\newline{}\centerline{\url{https://docs.codecov.io/docs}}

\subsection{GitHub Actions}
Il servizio di CI utilizzato è GitHub Actions, con dei workflow personalizzati presenti nella cartella .github.
Queste Actions vanno a controllare staticamente il codice tramite Flake8, ad eseguire il codice ed i test su macOS, Windows ed Ubuntu ed infine mandare una notifica della buona o cattiva riuscita sul canale predisposto di Slack.
La documentazione completa di GitHub Actions è disponibile al seguente indirizzo:
\newline{}\centerline{\url{https://docs.github.com/en/actions}}
