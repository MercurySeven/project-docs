\section{Setup}
\subsection{Requisiti minimi di sistema}
\subsubsection{Requisiti Hardware}
Non sono presenti requisiti minimi per l'\gloman{hardware,} il carico di lavoro aumenta a seconda della quantità di file da sincronizzare.
\subsubsection{Requisiti Software}
Il programma è multipiattaforma, in seguito sono elencati i sistemi operativi in cui il funzionamento del \gloman{Software} è stato testato e la loro versione. Altri sistemi operativi potrebbero comunque funzionare.

{
	\setlength{\freewidth}{\dimexpr\textwidth-1\tabcolsep}
	\renewcommand{\arraystretch}{1.5}
	\setlength{\aboverulesep}{0pt}
	\setlength{\belowrulesep}{0pt}
	\rowcolors{2}{AzzurroGruppo!10}{white}
	\begin{longtable}{L{.210\freewidth} C{.68\freewidth}}
		\rowcolor{AzzurroGruppo!30}
		\textbf{Sistema operativo} & \textbf{Versione minima testata} \\
		\toprule
		\endhead		
		Windows & 10 \\ 
		MacOS & 11.0 \\ 
		Ubuntu & 20.04 \\	
				
		\bottomrule
		\hiderowcolors
		\caption{Sistemi operativi testati}
	\end{longtable}
}
\subsection{Installazione}

\subsubsection{Ambiente di lavoro}

Per il setup degli IDE è consigliato l'utilizzo di PyCharm o Visual Studio Code, le guide per la loro installazione sono presenti sui corrispettivi siti ed anche nel nostro \gloman{Repository} al seguente link

\centerline{\url{https://github.com/MercurySeven/project-SSD/tree/main/docs}}
\subsubsection{Librerie}
\label{sec:Librerie}
Per installare tutte le librerie si può utilizzare il seguente comando all'interno della cartella principale del progetto:\newline{}
\centerline{\textbf{pip install -r requirements.txt}}
	



