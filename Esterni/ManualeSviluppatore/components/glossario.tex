\appendix

\section{Glossario}

\subsubsection{C}
\begin{itemize}
	\item \textbf{Cloud}:
	\item \textbf{Code coverage}:
\end{itemize}

\subsubsection{G}
\begin{itemize}
	\item \textbf{GitHub}: è un servizio di hosting per progetti software. Il nome "GitHub" deriva dal fatto che
GitHub è una implementazione dello strumento di controllo versione distribuito Git.
\item \textbf{GraphQL}:
\item \textbf{GUI}:
\end{itemize}

\subsubsection{H}
\begin{itemize}
	\item \textbf{Hardware}:
\end{itemize}

\subsubsection{L}
\begin{itemize}
\item \textbf{LGPI}:
\end{itemize}

\subsubsection{M}
\begin{itemize}
\item \textbf{MVC}:
\end{itemize}

\subsubsection{P}
\begin{itemize}
	\item \textbf{Python}:
\end{itemize}

\subsubsection{Q}
\begin{itemize}
\item \textbf{Qt}:
\end{itemize}

\subsubsection{R}
\begin{itemize}
	\item \textbf{Repository}:
	\item \textbf{Requisiti minimi di sistema}: i requisiti sono le capacità minime sia hardware che software che il sistema dovrà avere per svolgere le funzioni.
\end{itemize}

\subsubsection{S}
\begin{itemize}
	\item \textbf{Software}:
	\item \textbf{Solido}:
	\item \textbf{Sync engine}:
\end{itemize}

\subsubsection{Z}
\begin{itemize}
	\item \textbf{Zextras Drive}:
	\item \textbf{Zimbra}:
\end{itemize}