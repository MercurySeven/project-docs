\appendix

\section{Glossario}
\label{sec:Glossario}

\subsubsection*{C}
\begin{itemize}
    \item \textbf{Cloud}: Indica un paradigma di erogazione di servizi offerti su richiesta da un fornitore a un cliente finale attraverso la rete internet, a partire da un insieme di risorse preesistenti, configurabili e disponibili in remoto sotto forma di architettura distribuita. 
    \item \textbf{Code coverage}: La percentuale di codice attraversato dei test rispetto al totale della code base
\end{itemize}

\subsubsection*{G}
\begin{itemize}
    \item \textbf{GitHub}: È un servizio di hosting per progetti software. Il nome "GitHub" deriva dal fatto che GitHub è una implementazione dello strumento di controllo versione distribuito Git.
    \item \textbf{GraphQL}: É un linguaggio di interrogazione lato server per API, in grado di fornire ai client unicamente i dati di cui hanno bisogno.
    \item \textbf{GUI}: Un tipo di interfaccia utente che consente l'interazione uomo-macchina in modo visuale. 
\end{itemize}

\subsubsection*{H}
\begin{itemize}
    \item \textbf{Hardware}: Parte materiale di un sistema elettronico di elaborazione.
\end{itemize}

\subsubsection*{L}
\begin{itemize}
    \item \textbf{Lazy initialization}: Nella programmazione, la lazy initialization consiste nel ritardare la creazione dell'oggetto, il calcolo del valore o l'esecuzione di un processo impegnativo fino al primo momento per cui è necessario.
    \item \textbf{LGPL}: Licenza di software libero di tipo copyleft che non richiede che eventuale software "linkato" al programma sia pubblicato sotto la medesima licenza.
\end{itemize}

\subsubsection*{M}
\begin{itemize}
    \item \textbf{MVC}: Acronimo di Model View Controller, in informatica, è un pattern architetturale molto diffuso nello sviluppo di sistemi software, in particolare nell'ambito della programmazione orientata agli oggetti e in applicazioni web, in grado di separare la logica di presentazione dei dati dalla logica di business.
\end{itemize}

\subsubsection*{P}
\begin{itemize}
    \item \textbf{Python}: Linguaggio di programmazione orientato agli oggetti dinamico ad alto livello.
\end{itemize}

\subsubsection*{Q}
\begin{itemize}
    \item \textbf{Qt}: É un framework per lo sviluppo di interfacce grafiche multipiattaforma, basato su C++.
\end{itemize}

\subsubsection*{R}
\begin{itemize}
    \item \textbf{Repository}: Archivio digitale dove dati e informazioni sono raccolti, valorizzati e archiviati sulla base di metadati che ne permettono la rapida individuazione. Grazie alla sua peculiare architettura, un repository consente di gestire in modo ottimale anche grandi volumi di dati.
    \item \textbf{Requisiti minimi di sistema}: I requisiti sono le capacità minime sia hardware che software che il sistema dovrà avere per svolgere le funzioni.
\end{itemize}

\subsubsection*{S}
\begin{itemize}
    \item \textbf{Software}: Insieme delle componenti immateriali di un sistema elettronico di elaborazione
    \item \textbf{Solido}: In informatica, e in particolare in programmazione, l'acronimo SOLID si riferisce ai "primi cinque principi" dello sviluppo del software orientato agli oggetti descritti da Robert C. Martin in diverse pubblicazioni. I SOLID principles (Single responsibility, Open-closed, Liskov substitution, Interface segregation, Dependency inversion) sono intesi come linee guida per lo sviluppo di software leggibile, estendibile e mantenibile, in particolare nel contesto di pratiche di sviluppo agili.
    \item \textbf{Sync engine}: È il software che opera nel computer, che confronta i file presenti nel computer e nel drive. Si occupa inoltre di decidere che operazioni fare con il server come il download o l'upload dei file.
\end{itemize}

\subsubsection*{Z}
\begin{itemize}
    \item \textbf{Zextras Drive}: Un nuovo componente di Zimbra che offre un sistema di archiviazione integrato con il WebClient di Zimbra. 
    \item \textbf{Zimbra}: Una suite di software collaborativi che include sia un server email che un web client.
\end{itemize}