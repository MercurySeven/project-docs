\appendix

\section{Attualizzazione dei rischi}

In questa sezione vengono mostrate, sotto forma di tabella, tutte le contromisure adottate dal gruppo \gruppo{} riguardanti i rischi effettivamente riscontrati durante lo svolgimento del lavoro.  Per riuscire ad avere riscontro sull'efficacia delle contromisure prese, il gruppo ha deciso di adottare uno strumento di valutazione basato su tre livelli:

\setlength{\freewidth}{\dimexpr\textwidth-0\tabcolsep}
	\renewcommand{\arraystretch}{1.5}
	\setlength{\aboverulesep}{0pt}
	\setlength{\belowrulesep}{0pt}
	\rowcolors{2}{AzzurroGruppo!10}{white}
	\begin{longtable}{L{.210\freewidth} L{.6\freewidth} }
		\toprule 
		\rowcolor{AzzurroGruppo!30}
		\textbf{Nome} & \textbf{Descrizione} \\
		\hline
		\textbf{Bassa} & La contromisura non si è rivelata sufficiente e non è riuscita a mitigare al meglio il rischio riscontrato \\
		\textbf{Media} & La contromisura si è rivelata sufficiente a mitigare alcuni aspetti del rischio riscontrato. \\
		\textbf{Alta} & La contromisura si è rivelata sufficiente a mitigare molti o tutti gli aspetti del rischio riscontrato.\\
		\endhead		
		\hiderowcolors
		\caption{Descrizione livelli di efficacia contromisura }
	\end{longtable}

\subsection{Progettazione architetturale}

\setlength{\freewidth}{\dimexpr\textwidth-0\tabcolsep}
	\renewcommand{\arraystretch}{1.5}
	\setlength{\aboverulesep}{0pt}
	\setlength{\belowrulesep}{0pt}
	\rowcolors{2}{AzzurroGruppo!10}{white}
	\begin{longtable}{L{.210\freewidth} L{.6\freewidth} L{.1\freewidth}}
		\toprule 
		\rowcolor{AzzurroGruppo!30}
		\multicolumn{3}{c}{Rischi legati alle tecnologie}\\
		\toprule
		\textbf{Rischio} & \textbf{Mitigazione} & \textbf{Efficacia} \\
		\hline
		Inesperienza tecnologica & Durante la fase di apprendimento delle tecnologie necessarie allo sviluppo del \glo{Proof of Concept} sono stati riscontrati dei problemi. Questi si sono risolti grazie a continue comunicazioni col \glo{proponente} e al lavoro di gruppo, affiancato a materiale aggiuntivo da studiare singolarmente. & Alta\\
		\endhead		
		\hiderowcolors
		\caption{Attualizzazione per rischi legati alle tecnologie nel periodo di progettazione architetturale }
	\end{longtable}
	


\setlength{\freewidth}{\dimexpr\textwidth-0\tabcolsep}
	\renewcommand{\arraystretch}{1.5}
	\setlength{\aboverulesep}{0pt}
	\setlength{\belowrulesep}{0pt}
	\rowcolors{2}{AzzurroGruppo!10}{white}
	\begin{longtable}{L{.210\freewidth} L{.6\freewidth} L{.1\freewidth}}
		\toprule 
		\rowcolor{AzzurroGruppo!30}
		\multicolumn{3}{c}{Rischi legati all'organizzazione}\\
		\toprule
		\textbf{Rischio} & \textbf{Mitigazione} & \textbf{Efficacia}\\
		\hline
		Inesperienza gestionale & Si è riscontrato una differenza tra l'ammontare di ore stabilite per il lavoro e quelle effettivamente dovuto alle modifiche dei vari documenti e alla preparazione del \glo{Proof of Concept}. Questo ha spinto il gruppo alla ridistribuzione del carico di lavoro in modo da non rischiare ritardi.  & Bassa\\
		Interpretazione errata o cambio dei \glo{requisiti} & In seguito alle osservazioni fatte dopo la \textit{Revisione dei \ignore{Requisiti}}, il gruppo ha riscontrato gli errori svolti nell'\textit{Analisi dei \ignore{Requisiti}} e ha provveduto ad apportare le opportune modifiche e riempire le mancanze segnalate con priorità elevata. & Media\\
		
		\endhead		
		\hiderowcolors
		\caption{Attualizzazione per rischi legati all'organizzazione nel periodo di progettazione architetturale }
	\end{longtable}
	

\setlength{\freewidth}{\dimexpr\textwidth-0\tabcolsep}
	\renewcommand{\arraystretch}{1.5}
	\setlength{\aboverulesep}{0pt}
	\setlength{\belowrulesep}{0pt}
	\rowcolors{2}{AzzurroGruppo!10}{white}
	\begin{longtable}{L{.210\freewidth} L{.6\freewidth} L{.1\freewidth}}
		\toprule 
		\rowcolor{AzzurroGruppo!30}
		\multicolumn{3}{c}{Rischi legati alle persone}\\
		\toprule
		\textbf{Rischio} & \textbf{Mitigazione} & \textbf{Efficacia} \\
		\hline
		Impegni personali & Durante il periodo i membri del gruppo si sono trovati a svolgere la sessione d'esame. Per assicurare la continuità del progetto il carico di lavoro è stato ridistribuito tra i membri con più disponibilità.  & Alta\\
		\endhead		
		\hiderowcolors
		\caption{Attualizzazione per rischi legati alle persone nel periodo di progettazione architetturale }
	\end{longtable}
	

\subsection{Progettazione di dettaglio e codifica}


	
	\setlength{\freewidth}{\dimexpr\textwidth-0\tabcolsep}
	\renewcommand{\arraystretch}{1.5}
	\setlength{\aboverulesep}{0pt}
	\setlength{\belowrulesep}{0pt}
	\rowcolors{2}{AzzurroGruppo!10}{white}
	\begin{longtable}{L{.210\freewidth} L{.6\freewidth} L{.1\freewidth}}
		\toprule 
		\rowcolor{AzzurroGruppo!30}
		\multicolumn{3}{c}{Rischi legati all'organizzazione}\\
		\toprule
		\textbf{Rischio} & \textbf{Mitigazione} & \textbf{Efficacia}\\
		\hline
		Inesperienza gestionale & La mitigazione del periodo precedente, cioè la ridistribuzione del carico di lavoro, non si è dimostrata efficace a contenere i ritardi. Per questo si è deciso di redigere una \glo{timeline} in modo da avere una più chiara visione d'insieme di ciò che va fatto e di chi è disponibile in quel determinato lasso di tempo. Grazie all'introduzione della suddetta \glo{timeline} e agli obiettivi di \glo{sprint} i ritardi sono stati efficacemente contenuti e si è riuscito a ottenere un buon livello di organizzazione. & Alta\\
		
		\endhead		
		\hiderowcolors
		\caption{Attualizzazione per rischi legati all'organizzazione nel periodo di progettazione di dettaglio e codifica }
	\end{longtable}
	
\subsection{Validazione e collaudo}
	
	\setlength{\freewidth}{\dimexpr\textwidth-0\tabcolsep}
	\renewcommand{\arraystretch}{1.5}
	\setlength{\aboverulesep}{0pt}
	\setlength{\belowrulesep}{0pt}
	\rowcolors{2}{AzzurroGruppo!10}{white}
	\begin{longtable}{L{.210\freewidth} L{.6\freewidth} L{.1\freewidth}}
		\toprule 
		\rowcolor{AzzurroGruppo!30}
		\multicolumn{3}{c}{Rischi legati alle tecnologie}\\
		\toprule
		\textbf{Rischio} & \textbf{Mitigazione} & \textbf{Efficacia}\\
		\hline
		Problematiche Hardware & Sono stati riscontrati malfunzionamenti sull'unico computer avente come sistema operativo macOS. Questo poteva intaccare lo sviluppo e testing in locale su quel sistema operativo. Fortunatamente le problematiche erano note al gruppo da tempo e si è riusciti a procurare un computer di ricambio poco dopo la rottura del primo.  & Alta\\
		Problematiche Software & Per rilasciare l'eseguibile dell'applicativo per i vari sistemi operativi ci si è imbattuti in diverse problematiche con i programmi utilizzati per effettuare il rilascio. Si sono presentati diversi problemi nati dall'utilizzo di Qt6 ed il gruppo ha dovuto dedicare diverse ore per studiare metodi alternativi per il rilascio, riuscendo infine ad effettuarlo gestendo manualmente le librerie da utilizzare.  & Alta\\
		\endhead		
		\hiderowcolors
		\caption{Attualizzazione per rischi legati alle tecnologie nel periodo di validazione e collaudo }
	\end{longtable}
	
	
		\setlength{\freewidth}{\dimexpr\textwidth-0\tabcolsep}
	\renewcommand{\arraystretch}{1.5}
	\setlength{\aboverulesep}{0pt}
	\setlength{\belowrulesep}{0pt}
	\rowcolors{2}{AzzurroGruppo!10}{white}
	\begin{longtable}{L{.210\freewidth} L{.6\freewidth} L{.1\freewidth}}
		\toprule 
		\rowcolor{AzzurroGruppo!30}
		\multicolumn{3}{c}{Rischi legati alle persone}\\
		\toprule
		\textbf{Rischio} & \textbf{Mitigazione} & \textbf{Efficacia}\\
		\hline
		Impegni personali & Durante questo periodo alcuni membri del gruppo, per via di impegni personali e lavorativi, non sono riusciti a dedicare abbastanza tempo al progetto; questo ha fatto si che il lavoro si accumulasse. Per risolvere è stato ridistribuito il lavoro tra i membri con più disponibilità.& Alta\\
		
		\endhead		
		\hiderowcolors
		\caption{Attualizzazione per rischi legati alle persone nel periodo di validazione e collaudo }
	\end{longtable}
	