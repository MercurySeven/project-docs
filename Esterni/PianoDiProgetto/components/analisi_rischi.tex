\section{Analisi dei rischi}
Durante lo sviluppo di un progetto la probabilità che si verifichino delle criticità è alta, 
soprattutto quando si tratta di progetti complessi e/o di grandi dimensioni. Per questo motivo,
un'attenta analisi dei rischi può aiutare ad evitare o, quantomeno, a gestire al meglio tali criticità.
Il piano di gestione dei rischi si suddivide in 4 attività:
\begin{itemize}
    \item \textbf{Individuazione dei rischi}: attività di identificazione dei possibili fattori che potrebbero compromettere il corretto proseguimento del progetto;
    \item \textbf{Analisi dei rischi}: attività di analisi dei fattori individuati, con lo scopo di individuarne la gravità e la probabilità che essi si verifichino;
    \item \textbf{Pianificazione di controllo}: attività di pianificazione delle strategie da adottare per evitare che si verifichino i rischi individuati, o per gestirli al meglio qualora si verifichino;
    \item \textbf{Monitoraggio dei rischi}: attività continua, eseguita con l'obiettivo di monitorare l'insorgenza di problematiche legate ai fattori di rischio e permettere l'applicazione tempestiva delle strategie di contenimento individuate.
\end{itemize}
Abbiamo suddiviso i principali fattori di rischio nelle seguenti categorie:
\begin{itemize}
    \item rischi legati alle tecnologie;
    \item rischi legati all'organizzazione;
    \item rischi legati alle persone.
\end{itemize}

\subsection{Rischi legati alle tecnologie}


\renewcommand{\arraystretch}{1}
	\begin{table}[H]
		\begin{center}
			\setlength{\aboverulesep}{0pt}
			\setlength{\belowrulesep}{0pt}
			\setlength{\extrarowheight}{.75ex}
			\rowcolors{2}{AzzurroGruppo!10}{white}
			\begin{tabular}{ c p{10cm} }
				\toprule 
		\rowcolor{AzzurroGruppo!30}
		\multicolumn{2}{c}{\textbf{Inesperienza tecnologica}}\\
		\toprule
                \textbf{Descrizione} & Il \glo{team} presenta un'esperienza eterogenea e superficiale nell'utilizzo di molte tecnologie richieste in questo progetto. \\
				\textbf{Conseguenze} & Potrebbe comportare l'insorgere di problemi di natura tecnica o ritardi dovuti ai tempi di apprendimento soggettivi. \\
               \textbf{Occorrenza}& Media. \\
                \textbf{Pericolosità} & Media. \\
                \textbf{Precauzioni} & Il \RdP{} si occuperà di suddividere il carico di lavoro in modo bilanciato, tenendo conto delle capacità del singolo individuo. Ogni membro del \glo{team} dovrà quindi preoccuparsi di segnalare le sue conoscenze preliminari al \RdP{}. \\
                \textbf{Piano di contingenza} & Ogni membro del \glo{team} avrà cura di prendere dimestichezza con le tecnologie impiegate in modo tale da diventare il più possibile autonomo nel proprio lavoro. I compiti più complessi saranno affidati a più membri in modo tale da velocizzare i tempi di sviluppo e favorire l'aiuto reciproco. \\
				\bottomrule
			\end{tabular}
			\caption{Tabella dettaglio rischio di inesperienza tecnologica}
		\end{center}
    \end{table}


\renewcommand{\arraystretch}{1}
    \begin{table}[H]
        \begin{center}
            \setlength{\aboverulesep}{0pt}
            \setlength{\belowrulesep}{0pt}
            \setlength{\extrarowheight}{.75ex}
            \rowcolors{2}{AzzurroGruppo!10}{white}
            \begin{tabular}{ c p{10cm} }
                		\toprule 
		\rowcolor{AzzurroGruppo!30}
		\multicolumn{2}{c}{\textbf{Problematiche hardware}}\\
                \toprule
                \textbf{Descrizione} & Ogni membro del \glo{team} lavora allo sviluppo utilizzando il proprio computer personale, quest'ultimo può essere soggetto a malfunzionamenti. \\
                \textbf{Conseguenze} & Il guasto di uno o più computer porterebbe ad una parziale perdita di dati causando ritardi. Sebbene remota, esiste anche la possibilità di una totale perdita di dati qualora l'avaria avvenisse in tutti i computer contemporaneamente. \\
                 \textbf{Occorrenza} & Bassa. \\
                \textbf{Pericolosità}  & Alta. \\
                \textbf{Precauzioni} & Ogni membro del \glo{team} si adopererà per eseguire backup frequenti durante il processo di sviluppo e mantenere operativo il proprio hardware. \\
                 \textbf{Piano di contingenza} & Nel caso in cui si verifichi un problema hardware bisognerà procedere a riparare il danno il più rapidamente possibile. A seconda delle norme sulla mobilità vigenti, sarà possibile utilizzare i computer messi a disposizione nei laboratori informatici dell'Ateneo. \\
                \bottomrule
            \end{tabular}
            \caption{Tabella dettaglio rischio di problematiche hardware}
        \end{center}
    \end{table}


\renewcommand{\arraystretch}{1}
    \begin{table}[H]
        \begin{center}
            \setlength{\aboverulesep}{0pt}
            \setlength{\belowrulesep}{0pt}
            \setlength{\extrarowheight}{.75ex}
            \rowcolors{2}{AzzurroGruppo!10}{white}
            \begin{tabular}{ c p{10cm} }
                		\toprule 
		\rowcolor{AzzurroGruppo!30}
		\multicolumn{2}{c}{\textbf{Problematiche \ignore{software}}}\\
                \toprule
                \textbf{Descrizione} & Il \glo{team} si avvale di servizi e \glo{software} di terze parti per la gestione dei documenti e della \glo{code base}. La buona gestione di questi ultimi non è competenza del \glo{team}. \\
                \textbf{Conseguenze} & Eventuali disservizi potrebbero causare gravi perdite di dati e ritardi. \\
                 \textbf{Occorrenza} & Bassa. \\
                \textbf{Pericolosità}  & Alta. \\
                \textbf{Precauzioni} & Ogni membro del \glo{team} si impegnerà a tenere aggiornato un backup locale di tutti i \glo{repository} remoti. \\
                 \textbf{Piano di contingenza} & Il \RdP{} avrà cura di cercare altri servizi simili ai precedenti e più affidabili. \\
                \bottomrule
            \end{tabular}
            \caption{Tabella dettaglio rischio di problematiche software}
        \end{center}
    \end{table}


\subsection{Rischi legati all'organizzazione}


\renewcommand{\arraystretch}{1}
\begin{table}[H]
    \begin{center}
        \setlength{\aboverulesep}{0pt}
        \setlength{\belowrulesep}{0pt}
        \setlength{\extrarowheight}{.75ex}
        \rowcolors{2}{AzzurroGruppo!10}{white}
        \begin{tabular}{ c p{10cm} }
            		\toprule 
		\rowcolor{AzzurroGruppo!30}
		\multicolumn{2}{c}{\textbf{Inesperienza gestionale}}\\
            \toprule
            \textbf{Descrizione} & Il \glo{team} non ha mai affrontato un progetto di tali dimensioni, né a livello di carico di lavoro né a livello di ampiezza del gruppo da gestire.\\
            \textbf{Conseguenze} & È presente la mancanza di abitudine, da parte del singolo membro del \glo{team}, a relazionarsi con un gruppo di persone. I membri del \glo{team} non hanno familiarità con i ruoli che devono intraprendere e con i compiti da svolgere, questo potrebbe comportare un metodo di sviluppo poco \glo{solido}. \\
             \textbf{Occorrenza} & Alta. \\
            \textbf{Pericolosità}  & Alta. \\
            \textbf{Precauzioni} & Una frequente comunicazione interna di eventuali difficoltà al \RdP{} aiuterà a mitigare l'insorgere di problemi. \\
             \textbf{Piano di contingenza} & Sarà cura del \RdP{} gestire il \glo{team}, eventualmente riassegnando i membri con difficoltà, in task più adatte alle loro competenze personali, ridistribuendo il carico di lavoro. \\
            \bottomrule
        \end{tabular}
        \caption{Tabella dettaglio rischio di inesperienza gestionale}
    \end{center}
\end{table}


\renewcommand{\arraystretch}{1}
    \begin{table}[H]
        \begin{center}
            \setlength{\aboverulesep}{0pt}
            \setlength{\belowrulesep}{0pt}
            \setlength{\extrarowheight}{.75ex}
            \rowcolors{2}{AzzurroGruppo!10}{white}
            \begin{tabular}{ c p{10cm} }
                		\toprule 
		\rowcolor{AzzurroGruppo!30}
		\multicolumn{2}{c}{\textbf{Cattiva gestione della documentazione}}\\
                \toprule
                \textbf{Descrizione} & La documentazione rappresenta una parte fondamentale dei progetti, soprattutto se di grandi dimensioni. \newline In questo progetto il \glo{team} deve creare ed aggiornare costantemente una grande mole di documenti, e data la poca esperienza dei membri, potrebbe risultare complicato, creando difficoltà. \\
                \textbf{Conseguenze} & La non ottimale gestione della documentazione può portare a confusione all'interno del \glo{repository} e generare ritardi. \\
                 \textbf{Occorrenza} & Bassa. \\
                \textbf{Pericolosità}  & Alta. \\
                \textbf{Precauzioni} & Il \RdP{} deve predisporre un \glo{repository} dedicato alla documentazione, in cui ogni membro avrà cura di caricare il proprio lavoro senza modificare quello di altri. \\
                 \textbf{Piano di contingenza} & In caso il \glo{repository} presenti errori, il \RdP{} avrà cura di informare tutti i membri del \glo{team} e successivamente risolverli nel minor tempo possibile, al fine di bloccarne la propagazione. \\
                \bottomrule
            \end{tabular}
            \caption{Tabella dettaglio rischio di cattiva gestione della documentazione}
        \end{center}
    \end{table}


\renewcommand{\arraystretch}{1}
    \begin{table}[H]
        \begin{center}
            \setlength{\aboverulesep}{0pt}
            \setlength{\belowrulesep}{0pt}
            \setlength{\extrarowheight}{.75ex}
            \rowcolors{2}{AzzurroGruppo!10}{white}
            \begin{tabular}{ c p{10cm} }
                		\toprule 
		\rowcolor{AzzurroGruppo!30}
		\multicolumn{2}{c}{\textbf{Interpretazione errata o cambio dei \ignore{requisiti}}}\\
                \toprule
                \textbf{Descrizione} & Durante la fase di analisi alcuni \glo{requisiti} potrebbero venire equivocati e durante lo sviluppo potrebbe sorgere il bisogno di modificarne uno già fissato o aggiungerne uno non identificato in precedenza. \\
                \textbf{Conseguenze} & La mal interpretazione di uno o più \glo{requisiti} oppure l'aggiunta in corso d'opera potrebbe causare ritardi. \\
                 \textbf{Occorrenza} & Bassa. \\
                \textbf{Pericolosità}  & Alta. \\
                \textbf{Precauzioni} & Il \glo{team} si impegna ad avere una costante comunicazione con l'azienda \proponente{} al fine di discutere e chiarire ogni dubbio. \newline É utile, in ciascuna milestone, controllare che i \glo{requisiti} siano sempre coerenti con le linee guida date dall'azienda.  \\
                 \textbf{Piano di contingenza} & L'aggiunta o la rimozione di alcuni \glo{requisiti} non è una cosa rara, e al verificarsi di questa  \textbf{Occorrenza} il requisito andrà analizzato per allocargli le giuste risorse e la giusta priorità all'interno dello sviluppo. \\
                \bottomrule
            \end{tabular}
            \caption{Tabella dettaglio rischio di interpretazione errata o cambio dei requisiti}
        \end{center}
    \end{table}


\renewcommand{\arraystretch}{1}
    \begin{table}[H]
        \begin{center}
            \setlength{\aboverulesep}{0pt}
            \setlength{\belowrulesep}{0pt}
            \setlength{\extrarowheight}{.75ex}
            \rowcolors{2}{AzzurroGruppo!10}{white}
            \begin{tabular}{ c p{10cm} }
                		\toprule 
		\rowcolor{AzzurroGruppo!30}
		\multicolumn{2}{c}{\textbf{Stima errata dei costi}}\\
                \toprule
                \textbf{Descrizione} & Considerata l'inesperienza del \glo{team} potrebbero venire commessi errori di stima (sottostima o sovrastima) di natura economica sul valore delle varie attività. \\
                \textbf{Conseguenze} & Questi errori potrebbero portare a dei ritardi e/o a spreco di risorse. \\
                 \textbf{Occorrenza} & Media. \\
                \textbf{Pericolosità}  & Alta. \\
                \textbf{Precauzioni} & Ad ogni fase del progetto, il gruppo svolgerà delle considerazioni sulle attività svolte e da svolgere per chiarire se la stima di tempistiche/costi rientra in quella preventivata. \\
                 \textbf{Piano di contingenza} & All'insorgere di rilevanti variazioni orarie rispetto al preventivo iniziale, queste verranno comunicate tempestivamente a \proponente{}. Se \proponente{} lo riterrà necessario sarà opportuno ridiscutere le proposte formulate dal gruppo, specialmente quelle considerate più lunghe da implementare. \\
                \bottomrule
            \end{tabular}
            \caption{Tabella dettaglio rischio di stima errata dei costi}
        \end{center}
    \end{table}

\subsection{Rischi legati alle persone}


\renewcommand{\arraystretch}{1}
    \begin{table}[H]
        \begin{center}
            \setlength{\aboverulesep}{0pt}
            \setlength{\belowrulesep}{0pt}
            \setlength{\extrarowheight}{.75ex}
            \rowcolors{2}{AzzurroGruppo!10}{white}
            \begin{tabular}{ c p{10cm} }
                		\toprule 
		\rowcolor{AzzurroGruppo!30}
		\multicolumn{2}{c}{\textbf{Intesa parziale del \ignore{team}}}\\
                \toprule
                \textbf{Descrizione} & Precedentemente all'inizio del progetto didattico i membri del gruppo non si conoscevano. \\
                \textbf{Conseguenze} & La mancata conoscenza delle abilità reciproche potrebbe portare a confusione, difficoltà e una gestione poco efficiente del lavoro. \\
                 \textbf{Occorrenza} & Bassa. \\
                \textbf{Pericolosità}  & Media. \\
                \textbf{Precauzioni} & I membri del \glo{team} tramite strumenti di comunicazione come \glo{Zoom} e \glo{Slack} acquisiranno una conoscenza reciproca graduale, e si confronteranno con i metodi di lavoro più conformi per ognuno. \\
                 \textbf{Piano di contingenza} & Nel caso di dibattiti in cui sia difficile stabilire una decisione unanime, ci si affiderà al voto, con una decisione di maggioranza. \\
                \bottomrule
            \end{tabular}
            \caption{Tabella dettaglio rischio di intesa parziale del team}
        \end{center}
    \end{table}


\renewcommand{\arraystretch}{1}
    \begin{table}[H]
        \begin{center}
            \setlength{\aboverulesep}{0pt}
            \setlength{\belowrulesep}{0pt}
            \setlength{\extrarowheight}{.75ex}
            \rowcolors{2}{AzzurroGruppo!10}{white}
            \begin{tabular}{ c p{10cm} }
                		\toprule 
		\rowcolor{AzzurroGruppo!30}
		\multicolumn{2}{c}{\textbf{Approvazione errata dei documenti}}\\
                \toprule
                \textbf{Descrizione} & Potrebbero verificarsi degli errori da parte del \RdP{} nella fase di approvazione dei documenti. \\
                \textbf{Conseguenze} & La consegna di documenti errati o di bassa qualità, oltre a dare un'impressione negativa a \proponente{}, causerebbe uno spreco di risorse per scovare e risolvere tali errori. \\
                 \textbf{Occorrenza} & Media. \\
                \textbf{Pericolosità}  & Alta. \\
                \textbf{Precauzioni} & Il \RdP{} deve poter controllare il lavoro in maniera costante e organizzata evitando così moli di lavoro troppo onerose e a rischio di errore. \\
                 \textbf{Piano di contingenza} & Il \RdP{} deve lavorare in maniera incrociata con i \vers{}, così da poter minimizzare l' \textbf{Occorrenza} di sviste ed errori generici. \\
                \bottomrule
            \end{tabular}
            \caption{Tabella dettaglio rischio di approvazione errata dei documenti}
        \end{center}
    \end{table}


\renewcommand{\arraystretch}{1}
    \begin{table}[H]
        \begin{center}
            \setlength{\aboverulesep}{0pt}
            \setlength{\belowrulesep}{0pt}
            \setlength{\extrarowheight}{.75ex}
            \rowcolors{2}{AzzurroGruppo!10}{white}
            \begin{tabular}{ c p{10cm} }
                		\toprule 
		\rowcolor{AzzurroGruppo!30}
		\multicolumn{2}{c}{\textbf{Impegni personali}}\\
                \toprule
                \textbf{Descrizione} & Uno o più membri del \glo{team} potrebbero dover far fronte a impegni di natura personale o di salute, compromettendo il lavoro di sviluppo per un periodo di tempo variabile. \\
                \textbf{Conseguenze} & Questa diminuzione di risorse umane potrebbe portare a ritardi nelle attività individuali o collaborative, in cui la persona era assegnata. \\
                 \textbf{Occorrenza} & Bassa. \\
                \textbf{Pericolosità}  & Media. \\
                \textbf{Precauzioni} & Ciascun componente si impegna a segnalare i propri impegni per tempo, in modo da permettere al \glo{team} di riorganizzarsi al meglio. In caso di impegni imprevisti il \RdP{} dovrà essere avvisato il prima possibile. \\
                 \textbf{Piano di contingenza} & Nel caso in cui il periodo di assenza si estenda per un periodo prolungato, il \RdP{} dovrà ridistribuire gli incarichi scoperti ai restati membri del \glo{team}. \\
                \bottomrule
            \end{tabular}
            \caption{Tabella dettaglio rischio di impegni personali}
        \end{center}
    \end{table}