\section{Analisi dei Rischi}
Durante lo sviluppo di un progetto la probabilità che si verifichino delle criticità è alta, 
soprattutto quando si tratta di progetti complessi e/o di grandi dimensioni. Per questo motivo,
un'attenta analisi dei rischi può aiutare a evitare o, quantomeno, a gestire al meglio tali criticità.
Il piano di gestione dei rischi si suddivide in 4 attività:
\begin{itemize}
    \item \textbf{Individuazione dei rischi}: attività di identificazione dei possibili fattori che potrebbero compromettere il corretto proseguimento del progetto;
    \item \textbf{Analisi dei rischi}: attività di analisi di tali fattori, con lo scopo di individuarne la gravità e la probabilità che essi si verifichino;
    \item \textbf{Pianificazione di controllo}: attività di pianificazione delle strategie da adottare per evitare che si verifichino i rischi individuati o per gestirli al meglio qualora si verifichino;
    \item \textbf{Monitoraggio dei rischi}: attività continua eseguita con l'obiettivo di monitorare l'insorgenza di problematiche legate ai fattori di rischio e permettere l'applicazione tempestiva delle strategie di contenimento individuate.
\end{itemize}
Abbiamo suddiviso i principali fattori di rischio nelle seguenti categorie:
\begin{itemize}
    \item rischi legati alle tecnologie;
    \item rischi legati all’organizzazione;
    \item rischi legati alle persone.
\end{itemize}

\subsection{Rischi legati alle tecnologie}

\paragraph*{Inesperienza Tecnologica}
\renewcommand{\arraystretch}{1}
	\begin{table}[H]
		\begin{center}
			\setlength{\aboverulesep}{0pt}
			\setlength{\belowrulesep}{0pt}
			\setlength{\extrarowheight}{.75ex}
			\rowcolors{2}{AzzurroGruppo!10}{white}
			\begin{tabular}{ c p{10cm} }
				\rowcolor{AzzurroGruppo!30} 
				%\textbf{Prima colonna} & \textbf{Seconda colonna}  \\
                \toprule
                Descrizione & Il \glo{team} presenta un'esperienza eterogenea e superficiale nell'utilizzo di molte tecnologie richieste in questo progetto. \\
				Conseguenze & Potrebbe comportare l'insorgere di problemi di natura tecnica o ritardi dovuti ai tempi di apprendimento soggettivi. \\
                Occorrenza & Media \\
                Pericolosità & Media \\
                Precauzioni & Il \textit{Responsabile di progetto} si occuperà di suddividere il carico di lavoro in modo bilanciato, tenendo conto delle capacità del singolo individuo. Ogni membro del \glo{team} dovrà quindi preoccuparsi di segnalare le sue conoscenze preliminari al \textit(Responsabile di progetto). \\
                Piano di contingenza & Ogni membro del \glo{team} avrà cura di prendere dimestichezza con le tecnologie impiegate in modo tale da diventare il più possibile autonomo nel proprio lavoro. I compiti più complessi saranno affidati a più membri in modo tale da velocizzare i tempi di sviluppo e favorire l'aiuto reciproco. \\
				\bottomrule
			\end{tabular}
			\caption{Tabella dettaglio rischio di Inesperienza Tecnologica}
		\end{center}
    \end{table}

\paragraph*{Problematiche Hardware}
\renewcommand{\arraystretch}{1}
    \begin{table}[H]
        \begin{center}
            \setlength{\aboverulesep}{0pt}
            \setlength{\belowrulesep}{0pt}
            \setlength{\extrarowheight}{.75ex}
            \rowcolors{2}{AzzurroGruppo!10}{white}
            \begin{tabular}{ c p{10cm} }
                \rowcolor{AzzurroGruppo!30} 
                %\textbf{Prima colonna} & \textbf{Seconda colonna}  \\
                \toprule
                Descrizione & Ogni membro del \glo{team} lavora allo sviluppo utilizzando il proprio computer personale, quest'ultimo può essere soggetto a malfunzionamenti. \\
                Conseguenze & Il guasto di uno o più computer porterebbe ad una parziale perdita di dati causando ritardi. Sebbene remota, esiste anche la possibilità di una totale perdita di dati qualora l'avaria avvenisse in tutti i computer contemporaneamente. \\
                Occorrenza & Bassa \\
                Pericolosità & Alta \\
                Precauzioni & Ogni membro del \glo{team} si adopererà per eseguire backup frequenti durante il processo di sviluppo e mantenere operativo il proprio hardware. \\
                Piano di contingenza & Nel caso in cui si verifichi un problema hardware bisognerà procedere a riparare il danno il più rapidamente possibile. A seconda delle norme sulla mobilità vigenti, sarà possibile utilizzare i computer messi a disposizione nei laboratori informatici dell'Ateneo. \\
                \bottomrule
            \end{tabular}
            \caption{Tabella dettaglio rischio di Problematiche Hardware}
        \end{center}
    \end{table}

\paragraph*{Problematiche Software}
\renewcommand{\arraystretch}{1}
    \begin{table}[H]
        \begin{center}
            \setlength{\aboverulesep}{0pt}
            \setlength{\belowrulesep}{0pt}
            \setlength{\extrarowheight}{.75ex}
            \rowcolors{2}{AzzurroGruppo!10}{white}
            \begin{tabular}{ c p{10cm} }
                \rowcolor{AzzurroGruppo!30} 
                %\textbf{Prima colonna} & \textbf{Seconda colonna}  \\
                \toprule
                Descrizione & il \glo{team} si avvale di servizi e software di terze parti per la gestione dei documenti e della \glo{code base}. La buona gestione di questi ultimi non è competenza del \glo{team}. \\
                Conseguenze & Eventuali disservizi potrebbero causare gravi perdite di dati e ritardi. \\
                Occorrenza & Bassa \\
                Pericolosità & Alta \\
                Precauzioni & Ogni membro del \glo{team} si impegnerà a tenere aggiornato un backup locale di tutti i \glo{repository} remoti. \\
                Piano di contingenza & Il \textit{Responsabile di progetto} avrà cura di cercare altri servizi simili ai precedenti e più affidabili. \\
                \bottomrule
            \end{tabular}
            \caption{Tabella dettaglio rischio di Problematiche Software}
        \end{center}
    \end{table}


\subsection{Rischi legati all’organizzazione}

\paragraph*{Inesperienza Gestionale}
\renewcommand{\arraystretch}{1}
\begin{table}[H]
    \begin{center}
        \setlength{\aboverulesep}{0pt}
        \setlength{\belowrulesep}{0pt}
        \setlength{\extrarowheight}{.75ex}
        \rowcolors{2}{AzzurroGruppo!10}{white}
        \begin{tabular}{ c p{10cm} }
            \rowcolor{AzzurroGruppo!30} 
            %\textbf{Prima colonna} & \textbf{Seconda colonna}  \\
            \toprule
            Descrizione & Il \glo{team} non ha mai affrontato un progetto di tali dimensioni ne a livello di carico di lavoro ne a livello di ampiezza del gruppo da gestire.\\
            Conseguenze & È presente la mancanza di abitudine, da parte del singolo membro del team, a relazionarsi con un gruppo di persone. I membri del team non hanno familiarità con i ruoli che devono intraprendere e con i compiti da svolgere che potrebbe comportare un metodo di sviluppo poco solido. \\
            Occorrenza & Alta \\
            Pericolosità & Alta \\
            Precauzioni & Una frequente comunicazione interna di eventuali difficoltà al \textit{Responsabile di progetto} aiuterà a mitigare l'insorgere di problemi \\
            Piano di contingenza & Sarà cura del \textit{Responsabile di progetto} gestire il team, eventualmente riassegnando i membri con difficoltà in task più adatti alle loro competenze personali, ridistribuendo il carico di lavoro. \\
            \bottomrule
        \end{tabular}
        \caption{Tabella dettaglio rischio di Inesperienza Gestionale}
    \end{center}
\end{table}

\paragraph*{Cattiva Gestione Della Documentazione}
\renewcommand{\arraystretch}{1}
    \begin{table}[H]
        \begin{center}
            \setlength{\aboverulesep}{0pt}
            \setlength{\belowrulesep}{0pt}
            \setlength{\extrarowheight}{.75ex}
            \rowcolors{2}{AzzurroGruppo!10}{white}
            \begin{tabular}{ c p{10cm} }
                \rowcolor{AzzurroGruppo!30} 
                %\textbf{Prima colonna} & \textbf{Seconda colonna}  \\
                \toprule
                Descrizione & La Documentazione rappresenta una parte fondamentale dei progetti soprattutto se di grandi dimensioni. \newline In questo progetto il team deve creare ed aggiornare costantemente una grande mole di documentazione e questo, data la poca esperienza dei membri, potrebbe risultare complicato e creare delle difficoltà. \\
                Conseguenze & La non ottimale gestione della documentazione può portare a confusione all'interno del \glo{repository} e generare ritardi. \\
                Occorrenza & Bassa \\
                Pericolosità & Alta \\
                Precauzioni & Il \RdP deve predisporre un \glo{repository} dedicato alla documentazione in cui ogni membro avrà cura di caricare il proprio lavoro senza modificare quello di altri. \\
                Piano di contingenza & In caso il \glo{repository} presenti degli errori, il \RdP avrà cura di informare tutti i membri del team e successivamente risolverli nel minor tempo possibile al fine di bloccarne la propagazione. \\
                \bottomrule
            \end{tabular}
            \caption{Tabella dettaglio rischio di Cattiva Gestione Della Documentazione}
        \end{center}
    \end{table}

\paragraph*{Interpretazione Errata O Cambio Dei Requisiti}
\renewcommand{\arraystretch}{1}
    \begin{table}[H]
        \begin{center}
            \setlength{\aboverulesep}{0pt}
            \setlength{\belowrulesep}{0pt}
            \setlength{\extrarowheight}{.75ex}
            \rowcolors{2}{AzzurroGruppo!10}{white}
            \begin{tabular}{ c p{10cm} }
                \rowcolor{AzzurroGruppo!30} 
                %\textbf{Prima colonna} & \textbf{Seconda colonna}  \\
                \toprule
                Descrizione & Durante la fase di analisi alcuni requisiti potrebbero venire equivocati e durante lo sviluppo potrebbe sorgere il bisogno di modificarne uno già fissato o aggiungerne uno non identificato in precedenza. \\
                Conseguenze & La mal interpretazione di uno o più requisiti oppure l'aggiunta in corso d'opera potrebbe causare ritardi \\
                Occorrenza & Bassa \\
                Pericolosità & Alta \\
                Precauzioni & Il \glo{team} si impegna ad avere una costante comunicazione con l'azienda \proponente{} al fine di discutere e chiarire ogni dubbio. \newline É utile, in ciascuna milestone, controllare che i requisiti siano sempre coerenti con le linee guida date dall'azienda  \\
                Piano di contingenza & L'aggiunta o la rimozione di alcuni requisiti non è una cosa rara e al verificarsi di questa occorrenza il requisito andrà analizzato per allocargli le giuste risorse e la giusta priorità all'interno dello sviluppo. \\
                \bottomrule
            \end{tabular}
            \caption{Tabella dettaglio rischio di Interpretazione Errata O Cambio Dei Requisiti}
        \end{center}
    \end{table}

\paragraph*{Stima Errata Dei Costi}
\renewcommand{\arraystretch}{1}
    \begin{table}[H]
        \begin{center}
            \setlength{\aboverulesep}{0pt}
            \setlength{\belowrulesep}{0pt}
            \setlength{\extrarowheight}{.75ex}
            \rowcolors{2}{AzzurroGruppo!10}{white}
            \begin{tabular}{ c p{10cm} }
                \rowcolor{AzzurroGruppo!30} 
                %\textbf{Prima colonna} & \textbf{Seconda colonna}  \\
                \toprule
                Descrizione & Considerata l'inesperienza del team potrebbero venire commessi errori di stima (sottostima o sovrastima) di natura economica sul valore delle varie attività. \\
                Conseguenze & questi errori potrebbero portare a dei ritardi e/o a spreco di risorse. \\
                Occorrenza & Media \\
                Pericolosità & Alta \\
                Precauzioni & ad ogni fase del progetto, il gruppo svolgerà delle considerazioni sulle attività svolte e da svolgere per chiarire se la stima di tempistiche/costi rientra in quella preventivata. \\
                Piano di contingenza & All’insorgere di rilevanti variazioni orarie rispetto al preventivo iniziale, queste verranno comunicate tempestivamente al \proponente{}. \newline Se il \proponente{} lo riterrà necessario sarà opportuno ridiscutere le proposte formulate dal gruppo, specialmente quelle considerate più lunghe da implementare. \\
                \bottomrule
            \end{tabular}
            \caption{Tabella dettaglio rischio di Stima Errata Dei Costi}
        \end{center}
    \end{table}

\subsection{Rischi legati alle persone}

\paragraph*{Intesa Parziale Del Team}
\renewcommand{\arraystretch}{1}
    \begin{table}[H]
        \begin{center}
            \setlength{\aboverulesep}{0pt}
            \setlength{\belowrulesep}{0pt}
            \setlength{\extrarowheight}{.75ex}
            \rowcolors{2}{AzzurroGruppo!10}{white}
            \begin{tabular}{ c p{10cm} }
                \rowcolor{AzzurroGruppo!30} 
                %\textbf{Prima colonna} & \textbf{Seconda colonna}  \\
                \toprule
                Descrizione & Precedentemente alla creazione del gruppo i membri non si conoscevano. \\
                Conseguenze & Questa mancanza di conoscenza delle abilità altrui potrebbe creare confusione all'interno del team e portare ad una gestione del lavoro poco efficace. \\
                Occorrenza & Bassa \\
                Pericolosità & Media \\
                Precauzioni & I membri del team tramite strumenti di comunicazione come \glo{Zoom} e \glo{Slack} acquisiranno una conoscenza reciproca graduale e si confronteranno con i metodi di lavoro più conformi per ognuno. \\
                Piano di contingenza & Nel caso di dibattiti in cui sia difficile stabilire una decisione unanime ci si affiderà al voto, con una decisione di maggioranza. \\
                \bottomrule
            \end{tabular}
            \caption{Tabella dettaglio rischio di Intesa Parziale Del Team}
        \end{center}
    \end{table}

\paragraph*{Approvazione Errata Dei Documenti}
\renewcommand{\arraystretch}{1}
    \begin{table}[H]
        \begin{center}
            \setlength{\aboverulesep}{0pt}
            \setlength{\belowrulesep}{0pt}
            \setlength{\extrarowheight}{.75ex}
            \rowcolors{2}{AzzurroGruppo!10}{white}
            \begin{tabular}{ c p{10cm} }
                \rowcolor{AzzurroGruppo!30} 
                %\textbf{Prima colonna} & \textbf{Seconda colonna}  \\
                \toprule
                Descrizione & Potrebbero verificarsi degli errori da parte del \RdP{} nella fase di approvazione dei documenti. \\
                Conseguenze & La consegna di documenti errati o di bassa qualità oltre, a dare un'impressione negativa al \proponente{}, causerebbe uno spreco di risorse per scovare e risolvere tali errori \\
                Occorrenza & Media \\
                Pericolosità & Alta \\
                Precauzioni & Il \RdP{} deve poter controllare il lavoro in maniera costante e organizzata evitando cosi moli di lavoro troppo onerose e a rischio di errore \\
                Piano di contingenza & L'\RdP{} deve lavorare in maniera incrociata con i \vers{} così da poter minimizzare l'occorrenza di sviste ed errori generici. \\
                \bottomrule
            \end{tabular}
            \caption{Tabella dettaglio rischio di Approvazione Errata Dei Documenti}
        \end{center}
    \end{table}

\paragraph*{Impegni Personali}
\renewcommand{\arraystretch}{1}
    \begin{table}[H]
        \begin{center}
            \setlength{\aboverulesep}{0pt}
            \setlength{\belowrulesep}{0pt}
            \setlength{\extrarowheight}{.75ex}
            \rowcolors{2}{AzzurroGruppo!10}{white}
            \begin{tabular}{ c p{10cm} }
                \rowcolor{AzzurroGruppo!30} 
                %\textbf{Prima colonna} & \textbf{Seconda colonna}  \\
                \toprule
                Descrizione & Uno o più membri del \glo{team} potrebbero dover far fronte a impegni di natura personale o di salute compromettendo il lavoro di sviluppo per un periodo di tempo variabile. \\
                Conseguenze & Questa diminuzione di risorse umane potrebbe portare a ritardi nelle attivati individuali o collaborative in cui la risorsa era assegnata. \\
                Occorrenza & Bassa \\
                Pericolosità & Media \\
                Precauzioni & Ciascun componente si impegna a segnalare i propri impegni per tempo in modo da permettere al \glo{team} di riorganizzarsi al meglio. In caso di impegni imprevisti il \RdP{} dovrà essere avvisato il prima possibile. \\
                Piano di contingenza & Nel caso in cui il periodo di assenza si estenda per un periodo prolungato il \RdP{} dovrà ridistribuire gli incarichi scoperti ai restati membri del \glo{team}. \\
                \bottomrule
            \end{tabular}
            \caption{Tabella dettaglio rischio di Impegni Personali}
        \end{center}
    \end{table}