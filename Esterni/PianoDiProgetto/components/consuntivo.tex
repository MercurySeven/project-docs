\section{Consuntivo}
Di seguito verranno elencate le spese effettivamente sostenute per ogni ruolo. Il bilancio potrà essere:
\begin{itemize}
\item \textbf{Positivo:} se il preventivo supera il consuntivo;
\item \textbf{Pari:} se il consuntivo e il preventivo sono pari;
\item \textbf{Negativo:} se il consuntivo supera il preventivo.
\end{itemize}

\subsection{Periodo di analisi}
Le ore di lavoro sostenute in questo periodo sono considerate come ore di investimento, motivo per il quale esse non vengono rendicontate.
\begin{table}[H]
		\begin{center}
			\setlength{\aboverulesep}{0pt}
			\setlength{\belowrulesep}{0pt}
			\setlength{\extrarowheight}{.75ex}
			\rowcolors{2}{AzzurroGruppo!10}{white}
			\begin{tabular}{ c c c }
				\rowcolor{AzzurroGruppo!30} 
				\textbf{Ruolo} & \textbf{Ore} & \textbf{Costo}  \\
				\toprule
				Responsabile & 29 & 870 \euro \\
				Amministratore & 35(+3) & 700\euro(+60\euro) \\
				Analista & 94(+10) & 2350\euro(+250\euro) \\
				Progettista & - & - \\
				Programmatore & - & - \\
				Verificatore & 70(+5) & 1050\euro(+75\euro) \\
				\textbf{Totale Preventivo} & \textbf{210} & \textbf{4585 \euro} \\
				\textbf{Totale Consuntivo} & \textbf{228} & \textbf{4970 \euro} \\
				\textbf{Differenza} & \textbf{18} & \textbf{385 \euro} \\
				\bottomrule
			\end{tabular}
			\caption{Consuntivo della fase di analisi}
		\end{center}
\end{table}

\subsubsection{Conclusioni}
Come emerge dai dati riportati nella tabella soprastante, è stato necessario investire più tempo nei ruoli di \adm{}, \ana{} e \ver{}. Il bilancio risultante è negativo e di seguito sono spiegati i motivi:
\begin{itemize}
\item \textbf{Amministratore:} sono state aggiunte ed aggiornate alcune sezioni nelle \NdP{}, per chiarire alcune problematiche riscontrare durante la stesura dei documenti;
\item \textbf{Analista:} alcuni \glo{requisiti} si sono rilevati di non facile comprensione, dunque sono state impiegate un maggior numero di ore lavorative per la discussione interna;
\item \textbf{Verificatore:} l'aggiunta di nuove sezioni nelle \NdP{} e il numero di \glo{requisiti} trovati hanno richiesto più ore di lavoro anche per questo ruolo.
\end{itemize}


\subsection{Consuntivo di consolidamento dei requisiti}
La seguente tabella contiene i dati del consuntivo per il periodo di consolidamento dei \glo{requisiti}:
\begin{table}[H]
		\begin{center}
			\setlength{\aboverulesep}{0pt}
			\setlength{\belowrulesep}{0pt}
			\setlength{\extrarowheight}{.75ex}
			\rowcolors{2}{AzzurroGruppo!10}{white}
			\begin{tabular}{ c c c c c c c c }
				\rowcolor{AzzurroGruppo!30} 
				\textbf{Ruolo} & \textbf{Ore Preventivate} & \textbf{Ore Effettive} & \textbf{Costi Preventivati} & \textbf{Costi Effettivi}\\
				\toprule
				Responsabile   & 3 & 3 & 90 \euro{}  & 90 \euro{} \\
				Amministratore & 3 & 3 & 60 \euro{}  & 60 \euro{} \\
				Analista       & 16 & 16 & 400 \euro{}  & 400 \euro{} \\
				Progettista    & - & - & - & - \\
				Programmatore  & - & - & - & - \\
				Verificatore   & 13 & 13 & 195 \euro{}  & 195 \euro{} \\
				\bottomrule
			\end{tabular}
			\caption{Consuntivo del periodo di consolidamento dei requisiti}
		\end{center}
	\end{table}
	
Resoconto:

\begin{itemize}
	\item \textbf{Differenza oraria:} 0 ore (non sono servite ore in più del previsto), per un investimento totale di 35 ore;
	\item \textbf{Differenza economica:} 0 \euro{} (non è stato speso più del previsto), per un investimento totale di 745 \euro{}.
\end{itemize}
	
\subsubsection{Considerazioni e preventivo a finire}
Ore e costi non sono variati rispetto a quanto pianificato. \newline{}
In questa fase non c'è stato alcun cambiamento orario rispetto alla suddivisione preventivata: pertanto il preventivo a finire resta invariato.



\subsection{Consuntivo di progettazione architetturale}
La seguente tabella contiene i dati del consuntivo per il periodo di progettazione architetturale:
\begin{table}[H]
		\begin{center}
			\setlength{\aboverulesep}{0pt}
			\setlength{\belowrulesep}{0pt}
			\setlength{\extrarowheight}{.75ex}
			\rowcolors{2}{AzzurroGruppo!10}{white}
			\begin{tabular}{ c c c c c c c c }
				\rowcolor{AzzurroGruppo!30} 
				\textbf{Ruolo} & \textbf{Ore Preventivate} & \textbf{Ore Effettive} & \textbf{Costi Preventivati} & \textbf{Costi Effettivi}\\
				\toprule
				Responsabile   & 18 & 20\textcolor{green}{(+2)} & 540 \euro{}  & 600 \euro{} \textcolor{green}{(+60 \euro{})}\\
				Amministratore & 20 & 20 & 400 \euro{}  & 400 \euro{} \\
				Analista       & 38 & 35\textcolor{red}{(-3)} & 950 \euro{}  & 875 \euro{} \textcolor{red}{(-75 \euro{})} \\
				Progettista    & 61 & 61 & 1342 \euro{} & 1342 \euro{} \\
				Programmatore  & 21 & 23\textcolor{green}{(+2)} & 315 \euro{}  & 345 \euro{}\textcolor{green}{(+30 \euro{})} \\
				Verificatore   & 52 & 52 & 780 \euro{}  & 780 \euro{} \\
				\bottomrule
			\end{tabular}
			\caption{Consuntivo del periodo di progettazione architetturale}
		\end{center}
	\end{table}
	
Resoconto:
\begin{itemize}
	\item \textbf{Differenza oraria:} +1 ora (non sono servite ore in meno del previsto), per un investimento totale di 211 ore;
	\item \textbf{Differenza economica:} +15 \euro{} (sono stati spesi più soldi del previsto), per un investimento totale di 4342 \euro{}.
\end{itemize}
	


\subsubsection{Considerazioni e preventivo a finire}
Il ruolo di \textit{Analista} è stato svolto più velocemente di ciò che era stato preventivato richiedendo tre ore in meno, mentre il ruolo di \textit{Programmatore} e di \textit{Responsabile di Progetto} hanno richiesto entrambe due ore aggiuntive. Queste ore aggiuntive hanno comportato un ritardo e la richiesta di un ricevimento a sportello.\newline{}
Il preventivo a finire risulta quindi essere maggiore di 15 \euro{}, portando a 4342 \euro{} l'investimento per questo incremento.

\subsection{Consuntivo di progettazione di dettaglio e codifica}

\subsubsection{Consuntivo sprint 8}

La seguente tabella contiene i dati del consuntivo per lo sprint 8:
\begin{table}[H]
		\begin{center}
			\setlength{\aboverulesep}{0pt}
			\setlength{\belowrulesep}{0pt}
			\setlength{\extrarowheight}{.75ex}
			\rowcolors{2}{AzzurroGruppo!10}{white}
			\begin{tabular}{ c c c c c c c c }
				\rowcolor{AzzurroGruppo!30} 
				\textbf{Ruolo} & \textbf{Ore Preventivate} & \textbf{Ore Effettive} & \textbf{Costi Preventivati} & \textbf{Costi Effettivi}\\
				\toprule
				Responsabile   & 8 & 8 & 240 \euro{}  & 240 \euro{}\\
				Amministratore & 10 & 10 & 200 \euro{}  & 200 \euro{} \\
				Analista       & - & - & -  & - \\
				Progettista    & 44 & 44 & 968 \euro{} & 968 \euro{} \\
				Programmatore  & 71 & 68\textcolor{red}{(-3)} & 1065 \euro{}  & 1020 \euro{} \textcolor{red}{(-45 \euro{})} \\
				Verificatore   & 42 & 42 & 630 \euro{}  & 630 \euro{} \\
				\bottomrule
			\end{tabular}
			\caption{Consuntivo del periodo di progettazione architetturale}
		\end{center}
	\end{table}
	
\begin{itemize}
	\item \textbf{Differenza oraria:} -3 ore (sono servite ore in meno del previsto), per un investimento totale di 172 ore;
	\item \textbf{Differenza economica:} -45 \euro{} (sono stati spesi meno soldi del previsto), per un investimento totale di 3058 \euro{}.
\end{itemize}

\subsubsection{Consuntivo sprint 9}

La seguente tabella contiene i dati del consuntivo per lo sprint 9:
\begin{table}[H]
		\begin{center}
			\setlength{\aboverulesep}{0pt}
			\setlength{\belowrulesep}{0pt}
			\setlength{\extrarowheight}{.75ex}
			\rowcolors{2}{AzzurroGruppo!10}{white}
			\begin{tabular}{ c c c c c c c c }
				\rowcolor{AzzurroGruppo!30} 
				\textbf{Ruolo} & \textbf{Ore Preventivate} & \textbf{Ore Effettive} & \textbf{Costi Preventivati} & \textbf{Costi Effettivi}\\
				\toprule
				Responsabile   & 8 & 8 & 240 \euro{}  & 240 \euro{}\\
				Amministratore & 8 & 8 & 160 \euro{}  & 160 \euro{} \\
				Analista       & 2 & 2 & 50 \euro{}  & 50 \euro{} \\
				Progettista    & 45 & 45 & 990 \euro{} & 990 \euro{} \\
				Programmatore  & 63 & 63 & 945 \euro{}  & 945 \euro{} \\
				Verificatore   & 49 & 49 & 735 \euro{}  & 735 \euro{} \\
				\bottomrule
			\end{tabular}
			\caption{Consuntivo del periodo di progettazione architetturale}
		\end{center}
	\end{table}
	
\begin{itemize}
	\item \textbf{Differenza oraria:} 0 ore (non sono servite ore in più del previsto), per un investimento totale di 175 ore;
	\item \textbf{Differenza economica:} 0 \euro{} (non è stato speso più del previsto), per un investimento totale di 3120 \euro{}.
\end{itemize}
	
\subsubsection{Consuntivo di periodo}

La seguente tabella contiene i dati del consuntivo per il periodo di progettazione di dettaglio e codifica, riassumendo le tabelle di consuntivo dello sprint 8 e 9:
\begin{table}[H]
		\begin{center}
			\setlength{\aboverulesep}{0pt}
			\setlength{\belowrulesep}{0pt}
			\setlength{\extrarowheight}{.75ex}
			\rowcolors{2}{AzzurroGruppo!10}{white}
			\begin{tabular}{ c c c c c c c c }
				\rowcolor{AzzurroGruppo!30} 
				\textbf{Ruolo} & \textbf{Ore Preventivate} & \textbf{Ore Effettive} & \textbf{Costi Preventivati} & \textbf{Costi Effettivi}\\
				\toprule
				Responsabile   & 16 & 16 & 480 \euro{}  & 480 \euro{}\\
				Amministratore & 8 & 8 & 360 \euro{}  & 360 \euro{} \\
				Analista       & 2 & 2 & 50 \euro{}  & 50 \euro{} \\
				Progettista    & 89 & 89 & 1958 \euro{} & 1958 \euro{} \\
				Programmatore  & 134 & 131\textcolor{red} {(-3)} & 2010 \euro{}  & 1965 \euro{}\textcolor{red}{(-45 \euro{})} \\
				Verificatore   & 91 & 91 & 1365 \euro{}  & 1365 \euro{} \\
				\bottomrule
			\end{tabular}
			\caption{Consuntivo del periodo di progettazione architetturale}
		\end{center}
	\end{table}
	
\begin{itemize}
	\item \textbf{Differenza oraria:} -3 ore (sono servite ore in meno del previsto), per un investimento totale di 347 ore;
	\item \textbf{Differenza economica:} -45 \euro{} (sono stati spesi meno soldi del previsto), per un investimento totale di 6178 \euro{}.
\end{itemize}

\subsubsection{Considerazioni e preventivo a finire}
Il ruolo di \textit{Programmatore} è stato svolto più velocemente di ciò che era stato preventivato grazie all'approccio ai test adottato richiedendo tre ore in meno. La suddivisione del preventivo nei due sprint che lo compongono ha inoltre permesso più flessibilità e ha portato ad avere il preventivo del secondo sprint del periodo in linea con il consuntivo. \newline{}
Il preventivo a finire risulta quindi essere minore di 45 \euro{}, portando a 6178 \euro{} l'investimento per questo incremento.
