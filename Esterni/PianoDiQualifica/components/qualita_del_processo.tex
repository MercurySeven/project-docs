\section{Qualità del processo}

\subsection{Scopo}
Per assicurarsi dell'efficienza dello sviluppo è importante definire delle \glo{metriche} di qualità relative al processo.
Per il progetto è stato scelto il modello di sviluppo \glo{agile}.
Le \glo{metriche} scelte devono quindi essere generiche per poter essere applicabili a ogni fase del progetto in maniera efficace.
L'obbiettivo di questo approccio è quello di garantire la centralità del prodotto in ogni fase del progetto, subordinando quindi i processi al prodotto.

\subsection{Processi}
\subsubsection{Descrizione}
In questa sezione vengono definiti i macro-processi per il controllo della qualità.
Questa sezione verrà aggiornata durante lo svolgimento del progetto tramite l'aggiunta dei processi di \ignore{verifica} che si riterranno necessari.

\subsubsection{Pianificazione di progetto}
Il macro-processo di pianificazione di progetto ha come scopo quello di definire dei piani di sviluppo per il raggiungimento degli obbiettivi richiesti dal committente, nel tempo disponibile, senza un consumo eccessivo di risorse.

\paragraph{Obiettivi}
Il gruppo \gruppo{} deve quindi strutturare la pianificazione attorno ai seguenti elementi:
\begin{itemize}
\item \textbf{\ignore{Timeline}}: Assicurarsi che il progetto rispetti le scadenze imposte dal committente;
\item \textbf{Budget}: Assicurarsi che gli elementi e i tempi definiti nel piano rispettino le risorse disponibili.
\end{itemize}

\paragraph{Metriche}%
Per tenere traccia di questi elementi vengono utilizzate le seguenti \glo{metriche}:
\begin{itemize}
\item MPR01 Schedule Variance;
\item MPR02 Budget Variance.
\end{itemize}

\subsubsection{Tabella riassuntiva delle metriche}%
Di seguito sono riportate le \glo{metriche} in forma di tabella.
\begin{table}[H]
		\begin{center}
			\setlength{\aboverulesep}{0pt}
			\setlength{\belowrulesep}{0pt}
			\setlength{\extrarowheight}{.75ex}
			\rowcolors{2}{AzzurroGruppo!10}{white}
			\begin{tabular}{ c C{6cm} C{3cm} C{3cm} }
				\rowcolor{AzzurroGruppo!30} 
				\textbf{Metrica} & \textbf{Obiettivo} & \textbf{Valori accettati} & \textbf{Valori ottimali}  \\
				\toprule
				MPR01 & Monitoraggio del rispetto della \glo{timeline} da parte del gruppo \gruppo{} & $ x \leq -7$ giorni & $0$ giorni \\
				MPR02 & Monitoraggio dell'utilizzo eccessivo di risorse fuori budget & $ x \leq 10 \%$ & $0 \%$ \\
				\bottomrule
			\end{tabular}
			\caption{Tabella delle metriche e degli obiettivi relativi al processo di pianificazione}
		\end{center}
	\end{table}



