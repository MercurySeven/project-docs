\section{Qualità del processo}

\subsection{Scopo}
Per assicurarsi dell'efficienza dello sviluppo è importante definire delle \glo{metriche} di qualità relative al processo.
Per il progetto è stato scelto il modello di sviluppo \glo{agile}.
Le \glo{metriche} scelte devono quindi essere generiche per poter essere applicabili a ogni fase del progetto in maniera efficace.
L'obbiettivo di questo approccio è quello di garantire la centralità del prodotto in ogni fase del progetto, subordinando quindi i processi al prodotto.

\subsection{Processi}
\subsubsection{Descrizione}
In questa sezione vengono definiti i macro-processi per il controllo della qualità.
Questa sezione verrà aggiornata durante lo svolgimento del progetto tramite l'aggiunta dei processi di \ignore{verifica} che si riterranno necessari.

\subsubsection{Sviluppo}
\paragraph{Obbiettivi}
\begin{itemize}
\item \textbf{Accoppiamento}: Assicurarsi che l'architettura abbia un basso grado di accoppiamento;
\item \textbf{Annidamento}: Assicurarsi che i metodi abbiano un basso livello di annidamento;
\item \textbf{Parametri per metodo}: Salvaguardare la complessità dei metodi limitando il numero di parametri a 5.
\end{itemize}

\paragraph{Metriche}
Per tenere traccia di questi elementi vengono utilizzate le seguenti \glo{metriche}:
\begin{itemize}
\item MCD10 Coupling Between Objects;
\item MCD11 Numero di Parametri per Metodo;
\item MCD12 Livello di Annidamento.

\end{itemize}

\subsubsection{Pianificazione di progetto}
Il macro-processo di pianificazione di progetto ha come scopo quello di definire dei piani di sviluppo per il raggiungimento degli obbiettivi richiesti dal committente, nel tempo disponibile, senza un consumo eccessivo di risorse.

\paragraph{Obiettivi}
Il gruppo \gruppo{} deve quindi strutturare la pianificazione attorno ai seguenti elementi:
\begin{itemize}
\item \textbf{\ignore{Timeline}}: Assicurarsi che il progetto rispetti le scadenze imposte dal committente;
\item \textbf{Budget}: Assicurarsi che gli elementi e i tempi definiti nel piano rispettino le risorse disponibili.
\end{itemize}

\paragraph{Metriche}%
Per tenere traccia di questi elementi vengono utilizzate le seguenti \glo{metriche}:
\begin{itemize}
\item MPR01 Schedule Variance;
\item MPR02 Budget Variance;
\item MPR05 Actual Cost of Work Performed;
\item MPR07 Budgeted Cost of Work Scheduled;
\item MPR03 Budgeted Cost of Work Performed.

\end{itemize}

\subsubsection{Tabella riassuntiva delle metriche}%
Di seguito sono riportate le \glo{metriche} in forma di tabella.
\begin{table}[H]
		\begin{center}
			\setlength{\aboverulesep}{0pt}
			\setlength{\belowrulesep}{0pt}
			\setlength{\extrarowheight}{.75ex}
			\rowcolors{2}{AzzurroGruppo!10}{white}
			\begin{tabular}{ c C{6cm} C{3cm} C{3cm} }
				\rowcolor{AzzurroGruppo!30} 
				\textbf{Metrica} & \textbf{Obiettivo} & \textbf{Valori accettati} & \textbf{Valori ottimali}  \\
				\toprule
				MCD10 & Monitoraggio dell'accoppiamento tra classi. & $ 0 \leq x \leq 6$ & $ 0 \leq x \leq 1$  \\
				MCD11 & Monitoraggio del numero dei parametri per metodo. & $x \leq 5$ & $x \leq 3 $\\ 
				MCD12 & Monitoraggio del livello di annidamento nei vari metodi. & $1 \leq x \leq 7 $ & $1 \leq x \leq 3$ \\	
				
				MPR01 & Monitoraggio del rispetto della \glo{timeline} da parte del gruppo. \gruppo{} & x = 0 & x > 0 \\
				
				MPR02 & Monitoraggio dell'utilizzo eccessivo di risorse fuori budget. & $ x \leq$ MPR05 & $ x \leq$ Budget Totale \\
				
				MPR05 & Monitoraggio del denaro speso fino al momento del calcolo. & $ x \leq $ MPR06 & $x \leq $Budget Totale  \\
				
				MPR06 & Monitoraggio del costo pianificato fino al momento del calcolo. & $ x \geq 0$ & $x \geq 0$\\
				
				MPR07	& Monitoraggio del lavoro fatto fino al momento del calcolo. & $ x \geq 0 $ & $ x \geq 0 $ \\			
				

				\bottomrule
			\end{tabular}
			\caption{Tabella delle metriche e degli obiettivi relativi al processo di pianificazione}
		\end{center}
	\end{table}



