\section{Resoconto attività di verifica}

\subsection{Revisione dei requisiti}

\subsubsection{ di processo}

Di seguito sono riassunti i risultati delle attività di \glo{verifica} effettuate sui documenti consegnati nelle varie revisioni di progetto.

\subsubsection{ di prodotto}

Le \glo{metriche} stanziate in questa fase sono quelle che riguardano i documenti.

\paragraph{MDC1 Indice di Gulpease}

La metrica MDC1 Indice di Gulpease è stata realizzata grazie all'utilizzo di script automatici. I risultati dell'analisi effettuata sui principali documenti prodotti vengono riassunti nella seguente tabella:

\begin{table}[H]
		\begin{center}
			\setlength{\aboverulesep}{0pt}
			\setlength{\belowrulesep}{0pt}
			\setlength{\extrarowheight}{.75ex}
			\rowcolors{2}{AzzurroGruppo!10}{white}
			\begin{tabular}{ c C{4cm} C{2cm} C{3cm} }
				\rowcolor{AzzurroGruppo!30} 
				\textbf{Nome documento} & \textbf{Indice di Gulpease} & \textbf{Esito} \\
				\toprule
		Studio di Fattibilità & 67 & ottimo\\
		Norme di Progetto & 68 & ottimo\\
		Analisi dei \ignore{Requisiti} & 81 & ottimo\\
		Piano di Progetto & 82 & ottimo\\
		Piano di qualifica & 76 & ottimo\\
			
		\bottomrule
			\end{tabular}
			\caption{Resoconto metrica MDC1 Indice di Gulpease}
		\end{center}
	\end{table}

\paragraph{MDC2 Correttezza ortografica}

La correttezza ortografica è stata valutata tramite un attento controllo da parte dei verificatori e tramite il correttore ortografico aspell.  Ogni documento raggiunge il valore accettabile della metrica MDC2 Correttezza ortografica. 
\newpage{}
\subsection{Revisione di Progettazione}
\subsubsection{Cruscotto indice di Gulpease}

Di seguito riportiamo il cruscotto dei valori dell'indice di Gulpease valutato sui documenti da consegnare.

\begin{figure}[H]
    \centering
    \includegraphics[scale = 0.55]{immagini/cruscottoRP.png}
    \caption{Cruscotto MDC01, RP}
\end{figure}

\newpage

\subsection{Revisione di Qualifica}
\subsubsection{Cruscotto qualità dei documenti}
Di seguito riportiamo i cruscotti dei valori riscontrati delle metriche riguardanti i documenti.
\begin{figure}[H]
    \centering
    \includegraphics[scale = 0.63]{immagini/cruscottoRQ.png}
    \caption{Cruscotto MDC01, RQ}
\end{figure}

\begin{figure}[H]
    \centering
    \includegraphics[scale = 0.63]{immagini/cruscottoFleshRQ.png}
    \caption{Cruscotto MDC02, RQ}
\end{figure}

\subsubsection{Cruscotto  qualità del software}
Di seguito riportiamo i cruscotti dei valori riscontrati delle metriche riguardanti il codice.
\begin{figure}[H] 
    \centering
    \includegraphics[scale = 0.63]{immagini/MCD01.png}
    \caption{Cruscotto MCD01, RQ}
\end{figure}

\begin{figure}[H] 
    \centering
    \includegraphics[scale = 0.63]{immagini/MCD02.png}
    \caption{Cruscotto MCD02, RQ}
\end{figure}

\begin{figure}[H] 
    \centering
    \includegraphics[scale = 0.63]{immagini/MCD03.png}
    \caption{Cruscotto MCD03, RQ}
\end{figure}


\begin{figure}[H] 
    \centering
    \includegraphics[scale = 0.63]{immagini/MCD04.png}
    \caption{Cruscotto MCD04, RQ}
\end{figure}

\begin{figure}[H] 
    \centering
    \includegraphics[scale = 0.5]{immagini/MCD05.png}
    \caption{Cruscotto MCD05, RQ}
\end{figure}

\begin{figure}[H] 
    \centering
    \includegraphics[scale = 0.63]{immagini/MCD06-07.png}
    \caption{Cruscotto MCD06 e MCD07, RQ}
\end{figure}