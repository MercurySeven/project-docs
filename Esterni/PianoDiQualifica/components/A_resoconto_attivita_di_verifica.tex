\section{Resoconto attività di verifica}

\subsection{Revisione dei requisiti}

\subsubsection{ di processo}

Di seguito sono riassunti i risultati delle attività di verifica effettuate sui documenti consegnati nelle varie revisioni di progetto.

\subsubsection{ di prodotto}

Le metriche stanziate in questa fase sono quelle che riguardano i documenti.

\paragraph{MDC1 Indice di Gulpease}

La metrica MDC1 Indice di Gulpease è stata realizzata grazie all'utilizzo di script automatici. I risultati dell'analisi effettuata sui principali documenti prodotti vengono riassunti nella seguente tabella:

\begin{table}[H]
		\begin{center}
			\setlength{\aboverulesep}{0pt}
			\setlength{\belowrulesep}{0pt}
			\setlength{\extrarowheight}{.75ex}
			\rowcolors{2}{AzzurroGruppo!10}{white}
			\begin{tabular}{ c C{4cm} C{2cm} C{3cm} }
				\rowcolor{AzzurroGruppo!30} 
				\textbf{Nome documento} & \textbf{Indice di Gulpease} & \textbf{Esito} \\
				\toprule
		Studio di Fattibilità & 0 & ..\\
		Norme di Progetto & 0 & ..\\
		Analisi dei Requisiti & 0 & ..\\
		Piano di Progetto & 0 & ..\\
		Piano di qualifica & 0 & ..\\
			
		\bottomrule
			\end{tabular}
			\caption{Resoconto metrica MDC1 Indice di Gulpease}
		\end{center}
	\end{table}
	
\paragraph{MDC2 Correttezza ortografica}

La correttezza ortografica è stata valutata tramite un attento controllo da parte dei verificatori e tramite il correttore ortografico aspell.  Ogni documento raggiunge il valore accettabile della metrica MDC2 correttezza ortografica. 
