\section{Introduzione}
\subsection{Scopo del documento}

Lo scopo del  \PdQ{}  è quello di descrivere le strategie adottate dal gruppo \gruppo{} utilizzate nelle fasi di controllo qualità e \glo{validazione} per garantire la qualità del prodotto durante l'intera durata del progetto.  I contenuti iniziali verranno aggiornati o modificati con il procedere della realizzazione del prodotto,  per questo il documento alla forma attuale è da considerarsi incompleto.  

\subsection{Scopo del prodotto}

L'obbiettivo del prodotto \progetto{} proposto da \textit{Zextras} è la creazione di  un algoritmo e un'applicazione Desktop multipiattaforma per la sincronizzazione e il salvataggio su cloud di file,  appoggiandosi al servizio Zextras Drive.  

\subsection{Glossario}

Nel documento sono presenti termini che possono presentare dei significati ambigui o poco chiari a seconda del contesto. Per evitare incomprensioni il gruppo fornisce un glossario individuabile nel file \G{} \versGlo{} contenente i termini e la loro spiegazione. Nel documento corrente si è deciso di segnare tutte le parole presenti nel \G{} con una "G" a pedice di ogni parola

\subsection{Riferimenti}
\subsubsection{Riferimenti normativi}
Si fa riferimento alle \NdP{} \versNdP{} per quanto riguarda la presentazione di metriche d'interesse per il documento e gli strumenti a esse correlati. 
\subsubsection{Riferimenti informativi}
\begin{itemize}

\item Qualità di prodotto: \url{https://www.math.unipd.it/~tullio/IS-1/2019/Dispense/L12.pdf} ;

\item Qualità di processo: \url{https://www.math.unipd.it/~tullio/IS-1/2019/Dispense/L13.pdf} ;

\item Verifica e validazione: \url{https://www.math.unipd.it/~tullio/IS-1/2019/Dispense/L14.pdf} .

\end{itemize}
