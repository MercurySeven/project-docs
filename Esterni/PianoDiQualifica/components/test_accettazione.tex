\subsection{Test di accettazione}
I test di accettazione devono dimostrare che il software sviluppato soddisfi i requisiti del capitolato concordati con il proponente, essi vengono eseguiti durante il collaudo finale.
Si compongono essenzialmente dei test di sistema però eseguiti in sede aziendale.
{
    \setlength{\freewidth}{\dimexpr\textwidth-8\tabcolsep}
    \renewcommand{\arraystretch}{1.5}
    \centering
    \setlength{\aboverulesep}{0pt}
    \setlength{\belowrulesep}{0pt}
    \rowcolors{2}{AzzurroGruppo!10}{white}
    \begin{longtable}{C{.20\freewidth} C{.20\freewidth} C{.50\freewidth} C{.10\freewidth}}
        \toprule 
        \rowcolor{AzzurroGruppo!30}
        \textbf{Codice} & \textbf{Requisito} & \textbf{Descrizione} & \textbf{Stato} \\
        \toprule
        \endhead

        TAO01 & RFO01.0 &L'utente effettua il login all'interno dell'applicazione & NIP \\
        TAO02 & RFO02.0 & L'utente scieglie cosa sincronizzare dai file presenti nella cartella root & NIP \\
        TAO03 & RFD02.1/RFD03.1/RFO12.0 & L'utente mette in pausa la sincronizzazione con il server & NIP \\
        TAO04 & RFD02.2/RFD03.2/RFO12.0 & L'utente riprende la sincronizzazione con il server & NIP \\
        TAO05 & RFD02.3/RFD03.3/RFO12.0 & L'utente annulla la sincronizzazione con il server & NIP \\
        TAO06 & RFO03.0/RFO12.0 & L'utente scieglie cosa sincronizzare dai file presenti nel server & NIP \\
        TAO07 & RFO04.0 & L'utente controlla che lo stesso file è presente sia nel suo computer che nel server & NIP \\
        TAO08 & RFD05.0/RFO12.0 & L'utente modifica la lista dei file che sincronizza con il server, dopo aver fatto almeno una sincronizzazione & NIP \\
        TAO09 & RFF06.0 & L'utente deve ritornare alla versione precedente dei file, dopo aver annullato una sincronizzazione & NIP \\
        TAO10 & RFO07.0 & L'utente utilizza l'applicazione mentre avviene un trasferimento di file (anche pesante) & NIP \\
        TAO11 & RFF08.0 & L'utente riceve una notifica di un file modificato sul server e che è stato salvato nel suo computer & NIP \\
        TAO12 & RFF09.0 & L'utente, tramite menù contestuale, deve poter inviare quel file al server & NIP \\
        TAO13 & RFD10.0 & L'utente nasconde l'applicazione e deve poter vedere la sua icona nel system tray & NIP \\
        TAD14 & RFD11.0 & L'utente visualizza le informazioni dell'applicazione & NIP \\
        TAO15 & RFD13.0 & L'utente riavvia il computer e l'applicazione deve avviarsi in autonomia & NIP \\
        TAO16 & RFD14.0 & L'utente modifica la quota disco & NIP \\
        TAO17 & RFO15.0 & L'utente effettua il logout & NIP \\
        TAO18 & RFF16.0 & L'utente controlla le informazioni dei file contenuti nella cartella root & NIP \\
        TAO19 & RFF16.1 & L'utente richiede un link per poter condividere il file con un'altra persona & NIP \\
        TAO20 & RFO17.0 & L'utente viene informato se una sincronizzazione è fallita per assenza di rete & NIP \\
        TAO21 & RFO18.0 & L'utente viene informato se inserisce delle credenziali sbagliate & NIP \\
        TAO22 & RFD19.0/RFD20.0 & L'utente riceve un messaggio se la cartella root è piena o il server ha finito lo spazio di storage & NIP \\
        TAO23 & RFD21.0 & L'utente riceve un messaggio se il valore di quota disco inserito non è adatto & NIP \\
        TAO24 & RFO22.0 & L'utente scieglie o modifica il percorso delle cartella di root & NIP \\
        TAO25 & RFD23.0 & L'utente modifica la politica di risoluzione dei conflitti tra file & NIP \\
        TAO26 & RFD24.0 & L'utente decide che versione del file tenere, impostando la politica "decisione manuale" & NIP \\
        TAO27 & RFD25.0 & L'utente modifica ogni quanto sincronizzare la cartella root con il server & NIP \\

        \bottomrule
        \hiderowcolors
        \caption{Tabella dei test di accettazione}
    \end{longtable}
}