\section{Qualità del prodotto}
\subsection{Scopo}
Per garantire la qualità dei prodotti realizzati il gruppo \gruppo{} ha deciso di adottare lo standard ISO/IEC 9126.
Le \glo{metriche} e gli obiettivi definiti di seguito dovranno essere integrate nel processo di testing di ogni \glo{sprint}.
\subsection{Prodotti}
\subsubsection{Descrizione}
In questa sezione vengono definite le caratteristiche di cui i membri del gruppo \gruppo{} dovranno tenere conto durante la creazione dei prodotti.
\subsubsection{Qualità dei documenti}
\label{QualitàDoc}
\paragraph{Comprensibilità}
I documenti prodotti devono essere facilmente leggibili e accessibili a tutti i membri del progetto. \\

\textbf{Obiettivi}
\begin{itemize}
\item \textbf{Leggibilità}: i documenti devono essere leggibili e comprensibili;
\item \textbf{Correttezza ortografica}: i documenti non devono presentare errori ortografici.
\end{itemize}

\textbf{\ignore{Metriche}}
\begin{itemize}
\item MDC01 Indice di Gulpease;
\item MDC02 Indice di Flesch-Vacca
\item MDC03 Correttezza ortografica.
\end{itemize}


\subsubsection{Qualità del software}
\label{QualitàSW}
\paragraph{Funzionalità}

Il prodotto deve fornire tutte le funzionalità determinate nel documento di \AdR{} \versAdR{}. \\

\textbf{Obiettivi}
\begin{itemize}
\item \textbf{Accuratezza}: le funzionalità del prodotto soddisfano le richieste del committente;
\item \textbf{Adeguatezza}: le funzionalità del prodotto hanno il comportamento aspettato;
\item \textbf{Sicurezza}: il prodotto garantisce la sicurezza dei dati sensibili gestiti.
\end{itemize}

\textbf{\ignore{Metriche}}
\begin{itemize}
\item MPR03 Soddisfazione \glo{requisiti} obbligatori;
\item MPR04 Soddisfazione \glo{requisiti} opzionali.
\end{itemize}

\paragraph{Affidabilità}

Durante l'utilizzo del prodotto non si devono riscontrare errori durante l'utilizzo delle funzionalità da parte dell'utente. Eventuali errori devono essere adeguatamente gestiti. \\

\textbf{Obiettivi}
\begin{itemize}
\item \textbf{Maturità}: il prodotto non deve avere malfunzionamenti;
\item \textbf{Tolleranza agli errori}: capacità del prodotto di gestire casi di errore.
\end{itemize}

\textbf{\ignore{Metriche}}
\begin{itemize}
\item MVR01 Percentuale di malfunzionamenti del programma in rapporto ai \glo{test} eseguiti;
\item MCD01 Percentuale degli errori gestiti.
\end{itemize}

\paragraph{Usabilità}

Il prodotto deve essere intuitivo e pratico. \\

\textbf{Obiettivi}
\begin{itemize}
\item \textbf{Comprensibilità}: il prodotto deve essere quanto più intuitivo possibile per garantire che l'utente sia in grado di riconoscere e utilizzare le funzionalità offerte con uno sforzo e un investimento di tempo minimo;
\item \textbf{Apprendibilità}: presenza di documentazione per l'utente;
\item \textbf{Operabilità}: le funzionalità offerte dal prodotto sono in linea con le aspettative dell'utente;
\item \textbf{Attrattiva}: il prodotto deve essere utilizzabile facilmente e avere una presentazione attraente per l'utente.
\end{itemize}

\textbf{\ignore{Metriche}}
\begin{itemize}
\item MVR02 Comprensibilità delle funzionalità offerte;
\item MVR03 Facilità di apprendimento e di utilizzo del prodotto.
\end{itemize}

\paragraph{Efficienza}

La quantità di risorse utilizzate deve essere ottimale. \\

\textbf{Obiettivi}
\begin{itemize}
\item \textbf{Complessità delle funzioni}: le funzioni codificate devono rimanere al di sotto di una specifica soglia di complessità.
\item \textbf{Utilizzo delle risorse}: il prodotto deve essere efficiente e consumare la quantità ottimale di risorse.
\end{itemize}

\textbf{\ignore{Metriche}}
\begin{itemize}
\item MCD02 Complessità ciclomatica;
\item MCD05 Code coverage;
\item MCD06 Lunghezza della riga di codice;
\item MCD07 Brevità dei metodi.
\end{itemize}

\paragraph{Manutenibilità}

Il prodotto deve essere facilmente estendibile e adattabile. \\

\textbf{Obiettivi}
\begin{itemize}
\item \textbf{Stabilità}: la codifica del prodotto deve essere chiara e robusta così da facilitare l'aggiunta di funzionalità;
\item \textbf{Testabilità}: il prodotto deve essere nella condizione di poter essere testato in modo facile e veloce;
\item \textbf{Modificabilità}: la codifica del prodotto deve essere facilmente estensibile;
\item \textbf{Analizzabilità}: la struttura del prodotto deve facilitare la ricerca delle cause di eventuali malfunzionamenti.
\end{itemize}

\textbf{\ignore{Metriche}}
\begin{itemize}
\item MCD03 Variabili non utilizzate;
\item MCD04 Rapporto linee di codice e linee commentate.
\end{itemize}

\subsubsection{Tabella riassuntiva delle metriche}
\paragraph{Documenti}
Di seguito sono riportate le \glo{metriche} in forma di tabella.
\begin{table}[H]
		\begin{center}
			\setlength{\aboverulesep}{0pt}
			\setlength{\belowrulesep}{0pt}
			\setlength{\extrarowheight}{.75ex}
			\rowcolors{2}{AzzurroGruppo!10}{white}
			\begin{tabular}{ c C{6cm} C{3cm} C{3cm} }
				\rowcolor{AzzurroGruppo!30} 
				\textbf{Metrica} & \textbf{Obiettivo} & \textbf{Valori accettati} & \textbf{Valori ottimali}  \\
				\toprule
				MDC01 & Indice di Gulpease & $ 50 < x < 100$ & $60 < x < 100$ \\
				MDC02 & Indice di Flesch-Vacca & $ 50 < x < 100$ & $60 < x < 100$\\
				MDC03 & Correttezza ortografica & $ 95 \%$ privi di errori & $ 100\%$ privi di errori \\
				\bottomrule
			\end{tabular}
			\caption{Tabella delle metriche e degli obiettivi relativi alla documentazione}
		\end{center}
	\end{table}

\paragraph{Prodotto}
\begin{table}[H]
		\begin{center}
			\setlength{\aboverulesep}{0pt}
			\setlength{\belowrulesep}{0pt}
			\setlength{\extrarowheight}{.75ex}
			\rowcolors{2}{AzzurroGruppo!10}{white}
			\begin{tabular}{ c C{6cm} C{3cm} C{3cm} }
				\rowcolor{AzzurroGruppo!30} 
				\textbf{Metrica} & \textbf{Obiettivo} & \textbf{Valori accettati} & \textbf{Valori ottimali}  \\
				\toprule
				MPR03 & Soddisfazione \glo{requisiti} obbligatori & $100 \%$ & $100 \%$ \\
				MPR04 & Soddisfazione \glo{requisiti} opzionali & $ 0 \%$ & $ 100\%$ \\
				MVR01 & Percentuale di malfunzionamenti del programma in rapporto ai \glo{test} eseguiti & $ 0\%$  & $0\%$ \\
				MCD01 & Percentuale degli errori gestiti & $ 95\%$ - $100\% $ & $100\%$ \\
				MVR02 & Comprensibilità delle funzionalità offerte & $ 80\%$ - $100\% $ & $100\%$ \\
				MVR03 & Facilità di apprendimento e di utilizzo del prodotto & $10$-$15$ minuti & $5$-$10$ minuti \\
				MCD02 & Complessità ciclomatica & $0$ - $15$ & $0$ - $10$ \\
				MCD03 & Variabili non utilizzate & $0$ & $0$ \\
				MCD04 & Rapporto linee di codice e linee commentate & $5$ - $3$ & $3$ - $1$\\
				MCD05 & Code coverage & $80\%$-$100\%$ & $100\%$ \\
				MCD06 & Lunghezza della riga di codice & $0$-$100$ caratteri & $0$-$80$ caratteri \\
				MCD07 & Brevità dei metodi & $0$-$50$ istruzioni & $0$-$50$ istruzioni \\
				\bottomrule
			\end{tabular}
			\caption{Tabella delle metriche e degli obiettivi relativi al prodotto}
		\end{center}
	\end{table}

\section{Approccio al testing}
Per rendere lo sviluppo più efficace e garantire una applicazione migliore dei principi \glo{agile} è stato deciso di adottare un' approccio \glo{test} driven allo sviluppo (TDD). \\
Questo tipo di approccio allo sviluppo è composto da due fasi alternate:
\begin{itemize}
\item Code driven testing
\item Refactoring
\end{itemize}
\subsection{Code driven testing}
L'obiettivo di questa fase è di assicurarsi che il codice creato soddisfi i \glo{requisiti} qualitativi imposti dalle \glo{metriche}. Viene quindi scritto il \glo{test} assicurandosi che esso non sia soddisfatto dal codice presente al momento della sua creazione. Una volta sviluppato il \glo{test} si procede a scrivere la quantità minima di codice necessaria per soddisfarlo.
\subsection{Refactoring}
L'obiettivo di questa fase è migliorare la struttura del codice per allinearla con gli standard di codifica delineati, rendere più efficace ed efficiente lo svolgimento della funzione implementata e assicurarsi una maggiore manutenibilità. Una volta passato il \glo{test} il codice viene ristrutturato in base alle necessità e agli standard di codifica e testato nuovamente, eventualmente aggiornando i \glo{test} per prevenire regressione, fino al raggiungimento di uno standard qualitativo soddisfacente. \newline{}
\newline{}
Queste due fasi si devono ripetere ciclicamente per ogni nuovo \glo{test} e per raffinare il codice, allo scopo di evitare che scelte sub ottimali effettuate in un primo momento, con lo scopo di superare i \glo{test}, si possano consolidare nella codebase.
\subsection{Test di sistema}
L'obiettivo dei \glo{test di \ignore{sistema}} è dimostrare che i \glo{requisiti} elencati nel documento \AdR{} \versAdR{} vengano rispettati.


{
    \setlength{\freewidth}{\dimexpr\textwidth-8\tabcolsep}
    \renewcommand{\arraystretch}{1.5}
    \centering
    \setlength{\aboverulesep}{0pt}
    \setlength{\belowrulesep}{0pt}
    \rowcolors{2}{AzzurroGruppo!10}{white}
    \begin{longtable}{C{.20\freewidth} C{.20\freewidth} C{.50\freewidth} C{.10\freewidth}}
        \toprule 
        \rowcolor{AzzurroGruppo!30}
        \textbf{Codice} & \textbf{Requisito} & \textbf{Descrizione} & \textbf{Stato} \\
        \toprule
        \endhead

        TSO01 & RFO01.0 & L'utente effettua il login all'interno dell'applicazione & NI \\
        TSO02 & RFO02.0 & L'utente sceglie cosa sincronizzare dai file presenti nella cartella root & NI \\
        TSO03 & RFD02.1/RFD03.1/RFO12.0 & L'utente mette in pausa la sincronizzazione con il server & NI \\
        TSO04 & RFD02.2/RFD03.2/RFO12.0 & L'utente riprende la sincronizzazione con il server & NI \\
        TSO05 & RFD02.3/RFD03.3/RFO12.0 & L'utente annulla la sincronizzazione con il server & NI \\
        TSO06 & RFO03.0/RFO12.0 & L'utente sceglie cosa sincronizzare dai file presenti nel server & NI \\
        TSO07 & RFO04.0 & L'utente controlla che lo stesso file è presente sia nel suo computer che nel server & NI \\
        TSO08 & RFD05.0/RFO12.0 & L'utente modifica la lista dei file che sincronizza con il server, dopo aver fatto almeno una sincronizzazione & NI \\
        TSO09 & RFF06.0 & L'utente deve ritornare alla versione precedente dei file, dopo aver annullato una sincronizzazione & NI \\
        TSO10 & RFO07.0 & L'utente utilizza l'applicazione mentre avviene un trasferimento di file (anche pesante) & NI \\
        TSO11 & RFF08.0 & L'utente riceve una notifica di un file modificato sul server e che è stato salvato nel suo computer & NI \\
        TSO12 & RFF09.0 & L'utente, tramite menù contestuale, deve poter inviare quel file al server & NI \\
        TSO13 & RFD10.0 & L'utente nasconde l'applicazione e deve poter vedere la sua icona nel system tray & NI \\
        TSD14 & RFD11.0 & L'utente visualizza le informazioni dell'applicazione & NI \\
        TSO15 & RFD13.0 & L'utente riavvia il computer e l'applicazione deve avviarsi in autonomia & NI \\
        TSO16 & RFD14.0 & L'utente modifica la quota disco & NI \\
        TSO17 & RFO15.0 & L'utente effettua il logout & NI \\
        TSO18 & RFF16.0 & L'utente controlla le informazioni dei file contenuti nella cartella root & NI \\
        TSO19 & RFF16.1 & L'utente richiede un link per poter condividere il file con un'altra persona & NI \\
        TSO20 & RFO17.0 & L'utente viene informato se una sincronizzazione è fallita per assenza di rete & NI \\
        TSO21 & RFO18.0 & L'utente viene informato se inserisce delle credenziali sbagliate & NI \\
        TSO22 & RFD19.0/RFD20.0 & L'utente riceve un messaggio se la cartella root è piena o il server ha finito lo spazio di storage & NI \\
        TSO23 & RFD21.0 & L'utente riceve un messaggio se il valore di quota disco inserito non è adatto & NI \\
        TSO24 & RFO22.0 & L'utente sceglie o modifica il percorso delle cartella di root & NI \\
        TSO25 & RFD23.0 & L'utente modifica la politica di risoluzione dei conflitti tra file & NI \\
        TSO26 & RFD24.0 & L'utente decide che versione del file tenere, impostando la politica "decisione manuale" & NI \\
        TSO27 & RFD25.0 & L'utente modifica ogni quanto sincronizzare la cartella root con il server & NI \\

        \bottomrule
        \hiderowcolors
        \caption{Tabella dei test di sistema}
    \end{longtable}
}

\subsection{Test di accettazione}
I \glo{test di accettazione} devono dimostrare che il \glo{software} sviluppato soddisfi i \glo{requisiti} del capitolato concordati con il \glo{proponente}, essi vengono eseguiti durante il collaudo finale.
Si compongono essenzialmente dei \glo{test di \ignore{sistema}} però eseguiti in sede aziendale.

\subsection{Test di unità}
Hanno lo scopo di testare le parti più piccole e autonome dell'applicazione software. Servono per far emergere eventuali difetti, in modo che essi possano essere corretti il prima possibile.
I \glo{test} verranno sviluppati seguendo la proprietà ATRIP.