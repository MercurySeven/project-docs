\section{Informazioni generali}

\begin{itemize}

    \item \textbf{Canale di comunicazione:} Slack;

    \item \textbf{Settimana:} dal 2021-05-10 al 2021-05-20;

    \item \textbf{Segretario:} \ACapoRedazione{};

    \item \textbf{Partecipanti}: Due rappresentanti dell'azienda Zextras, \textit{\Alessio{}} e \textit{\Federico{}}, e i membri del gruppo \Gruppo{}:
        \begin{itemize}
            \item \Daniele{};
            \item \Davide{};
            \item \Francesco{};
            \item \Tommaso{};
            \item \Lucrezia{};
            \item \Matteo{};
            \item \Giosue{}.
        \end{itemize}
\end{itemize}

\section{Ordine del giorno}

\begin{itemize}
    \item\textbf{Gestione dell'upload continuo;}
    \item\textbf{Presentazione dei progressi;}
    \item\textbf{Discussione sulla data e sui contenuti della \textit{Revisione di Accettazione}.}
\end{itemize}
\newpage


\section{Resoconto}
\subsection{Gestione dell'upload continuo}
Il gruppo ha esposto lo studio di fattibilità realizzato per la nuova feature proposta da \textit{\Alessio{}}. I problemi maggiori rilevati sono:
\begin{itemize}
	\item l'utilizzo di soluzioni multithread risultano complesse dato l'accesso concorrente dei vari thread di upload sullo snapshot locale, rischiando di avere uno snapshot corrotto;
	\item impossibilità con il sistema attuale di capire se un file è stato aggiunto da file system o se è stato scaricato, la risoluzione porterebbe ad un refactor importante della logica dell'algoritmo.
\end{itemize}
L'azienda con il gruppo rimane d'accordo che questi cambiamenti richiederebbero troppo impegno nella fase finale andando quindi a non attuare queste modifiche.
\subsection{Presentazione dei progressi}
Sono stati esposti i rimanenti progressi richiesti dall'azienda, sono state introdotte icone per migliorare notevolmente l'esperienza di utente. Queste migliorie richieste dall'azienda sono state approvate nello stato nelle quali sono state presentate.

\subsection{Discussione sulla data e sui contenuti della \textit{Revisione di Accettazione}}
Il gruppo ha esposto la struttura della \textit{Revisione di Accettazione} di questo anno. L'azienda è stata dunque informata di dover partecipare e ci si è accordati per stabilire una data.

\newpage

\section{Riepilogo delle decisioni \hfil}
{
    \setlength{\freewidth}{\dimexpr\textwidth-4\tabcolsep}
    \renewcommand{\arraystretch}{1.5}
    \setlength{\aboverulesep}{0pt}
    \setlength{\belowrulesep}{0pt}
    \rowcolors{2}{AzzurroGruppo!10}{white}
    \begin{longtable}{L{.3\freewidth} L{.7\freewidth}}
        \toprule
        \rowcolor{AzzurroGruppo!30}
        \textbf{Codice} & \textbf{Decisione}\\
        \toprule
        \endhead

        VE\_\DataMeeting{}.1 & Non si implementerà l'upload continuo dei file.\\
        VE\_\DataMeeting{}.2 & Il miglioramento della esperienza di utente è stato approvato.\\
        VE\_\DataMeeting{}.3 & Scelta la data per la \textit{Revisione di Accettazione}.\\
        \bottomrule
        \hiderowcolors
    \end{longtable}
}