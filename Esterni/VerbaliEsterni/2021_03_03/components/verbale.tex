\section{Informazioni generali}

\begin{itemize}

    \item \textbf{Canale di comunicazione:} Slack;

    \item \textbf{Settimana:} dal 2021-03-03 al 2021-03-05;

    \item \textbf{Segretario:} \Davide{};

    \item \textbf{Partecipanti}: \Federico{}, rappresentante dell'azienda Zextras e i membri del gruppo \Gruppo{}:
        \begin{itemize}
            \item \Daniele{};
            \item \Davide{};
            \item \Matteo{}.
        \end{itemize}

\end{itemize}
\section{Temi trattati durante la settimana}

\begin{itemize}
    \item\textbf{Consegna del server Zextras Drive;}
    \item\textbf{Problemi riscontrati durante l'utilizzo delle API;}
    \item\textbf{Problemi di sincronizzazione.}
\end{itemize}
\newpage


\section{Resoconto}
Durante la settimana del 2021-03-03 il gruppo \textit{\Gruppo{}} ha avuto contatti giornalmente con l'azienda questo perchè sono sorte molte domande relative all'integrazione delle API di Zextras Drive con l'applicazione sviluppata.
Per avere una comunicazione molto più rapida si è deciso di aggiungere \Alessio{} e \Federico{} al gruppo Slack.

\subsection{Consegna del server Zextras Drive}
Il 2021-03-03, \Federico{} ci ha comunicato l'indirizzo web per usare il loro prodotto Zextras Drive. Ci sono stati forniti due account con cui provare le funzionalità del drive.
Inoltre è stato fornito l'endpoint per effettuare le chiamate API con GraphQL. Al gruppo è stato mostrato inoltre, come effettuare upload e download dei file.

\subsection{Problemi riscontrati durante l'utilizzo delle API}
Sempre il 2021-03-03 il gruppo ha potuto iniziare a testare le varie chiamate API, ma sono iniziati i primi dubbi in quanto il server rispondeva sempre con l'errore 401 (Unauthorized).
Dopo un'ora e mezza di prove per capire dove avessimo sbagliato nell'usare le API, abbiamo chiesto aiuto a \Federico{}, il quale ci ha spiegato che le API hanno bisogno di specificare i cookie di autenticazione nell'header di richiesta.
\Federico{} ci ha comunicato che sta sviluppando un'API apposta per noi per gestire l'autenticazione e che per il momento i cookie dobbiamo recuperarli dal browser.

\subsection{Problemi di sincronizzazione}
Nei giorni seguenti il gruppo ha notato un problema presente nel server Zextras Drive, in quanto la data di ultima modifica del file viene generata nel server e non è possibile specificarla durante l'upload del file.
Questo ha creato un grosso problema con l'algoritmo di sincronizzazione in quanto basandosi sulla data di ultima modifica la sincronizzazione non risultava possibile.
Abbiamo immediatamente fatto presente il problema all'azienda che ci ha comunicato che lavoreranno ad una patch per consentirci di specificare la data durante la fase di upload del file.


\newpage

\section{Riepilogo delle decisioni \hfil}
{
    \setlength{\freewidth}{\dimexpr\textwidth-4\tabcolsep}
    \renewcommand{\arraystretch}{1.5}
    \setlength{\aboverulesep}{0pt}
    \setlength{\belowrulesep}{0pt}
    \rowcolors{2}{AzzurroGruppo!10}{white}
    \begin{longtable}{L{.3\freewidth} L{.7\freewidth}}
        \toprule
        \rowcolor{AzzurroGruppo!30}
        \textbf{Codice} & \textbf{Decisione}\\
        \toprule
        \endhead

        VE\_\DataMeeting{}.1 &  E' stato illustrato al gruppo come usare le API di Zextras Drive.\\
        VE\_\DataMeeting{}.2 &  E' stato chiarito come usare le API utilizzando i cookie di autenticazione. \\
        VE\_\DataMeeting{}.3 &  E' stato fatto notare all'azienda un problema relativo al loro server. \\
        \bottomrule
        \hiderowcolors
    \end{longtable}
