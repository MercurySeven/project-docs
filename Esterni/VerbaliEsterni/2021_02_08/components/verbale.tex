\section{Informazioni generali}

\begin{itemize}

    \item \textbf{Canale di comunicazione:} Zoom;

    \item \textbf{Data:} \DataMeeting{};
    
    \item \textbf{Ora inizio:} 13:00;

    \item \textbf{Ora fine:} 13:30;

    \item \textbf{Segretario:} \Davide{};

    \item \textbf{Partecipanti}: Il Prof. \Riccardo{} e alcuni componenti del gruppo \Gruppo{}:
    
        \begin{itemize}
            \item \Daniele{};
            \item \Davide{};
            \item \Francesco{};
            \item \Giosue{};
            \item \Lucrezia{};
            \item \Matteo{}.
        \end{itemize}

    \item \textbf{Membri assenti}:
        \begin{itemize}
            \item \Tommaso{}.
        \end{itemize}
\end{itemize}
\section{Ordine del giorno}

\begin{itemize}
    \item\textbf{Discussione su UC1;}
    \item\textbf{Discussione su UC2;}
    \item\textbf{Discussione su cosa presentare come POC.}
\end{itemize}

\newpage


\section{Resoconto}
La riunione è stata richiesta dal gruppo \textit{\Gruppo{}} per poter chiarire dubbi sorti durante la lettura dell'esito della prima convocazione RR.
Durante l'incontro è stato possibile discutere di una serie di quesiti riportati in seguito.
\subsection{Discussione su UC1}
Il prof. \Riccardo{}, ci ha fatto notare che creare un sottocaso UC1.1("Inserimento credenziali"), non aggiunge informazioni a UC1 e ci ha consigliato di risolvere creando due sottocasi
specifici per l'inserimento della email e password dell'account di Zextras Drive.
Per quanto riguarda al messaggio di errore UC8, il professore ha detto che va bene collegarlo ai due sottocasi sopra citati.
\subsection{Discussione su UC2}
Il gruppo \textit{\Gruppo{}} ha chiesto un parere su come ha rimodellato UC2, in quanto erano sorti dei dubbi sul fatto di usare la parola "Sincronizzazione" per indicare il processo
di selezione dei file da tenere sincronizzati all'interno del computer. Il professore ha risposto che non c'è nessun problema nell'usare "Sincronizzazione" come parola, ma se il gruppo
prefrisce usare "Selezione" per evitare dubbi futuri può farlo.
\subsection{Discussione su cosa presentare come POC}
Il professore ci ha raccomandato che durante il processo di creazione della POC, l'obiettivo principale è quello di identificare e utilizzare tutte le librerie per realizzare il progetto.

\newpage

\section{Riepilogo delle decisioni \hfil}
{
    \setlength{\freewidth}{\dimexpr\textwidth-4\tabcolsep}
    \renewcommand{\arraystretch}{1.5}
    \setlength{\aboverulesep}{0pt}
    \setlength{\belowrulesep}{0pt}
    \rowcolors{2}{AzzurroGruppo!10}{white}
    \begin{longtable}{L{.3\freewidth} L{.7\freewidth}}
        \toprule 
        \rowcolor{AzzurroGruppo!30}
        \textbf{Codice} & \textbf{Decisione}\\
        \toprule
        \endhead

        VE\_\DataMeeting{}.1 & Creare due sottocasi per email e password. \\
        VE\_\DataMeeting{}.2 & Scelto di usare la parola "Sincronizzazione" per decrivere il caso d'uso.\\
        VE\_\DataMeeting{}.3 & Il gruppo dovrà utilizzare tutte le librerie che verranno usate per la realizzazione del progetto. \\
        \bottomrule
        \hiderowcolors
    \end{longtable}
