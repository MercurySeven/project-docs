\section{Informazioni generali}

\begin{itemize}

	\item \textbf{Canale di comunicazione:} Zoom;
	
	\item \textbf{Data:} \DataMeeting{};
	
	\item \textbf{Ora inizio:} 16:20;
	
	\item \textbf{Ora fine:} 17:00;
	
	\item \textbf{Segretario:} \Lucrezia{};
	
	\item \textbf{Partecipanti}: \textit{Alessio Crestani}, \textit{Federico Rispo} come rappresentanti dell'azienda proponente \proponente{} e alcuni componenti del gruppo \Gruppo{}:
	
		\begin{itemize}
			\item \Daniele{};
			\item \Davide{};
			\item \Francesco{};
			\item \Giosue{};
			\item \Lucrezia{};
			\item \Matteo{}.
		\end{itemize}



	\item \textbf{Membri assenti}:
		\begin{itemize}
			\item \Tommaso{}.
		\end{itemize}
	\end{itemize}
\section{Ordine del giorno}

\begin{itemize}
	\item\textbf{Chiarimenti sulle funzionalità aggiuntive e opzionali;}
	\item\textbf{Discussione sulle API offerte da Zextras Drive;}
	\item\textbf{Discussione sul linguaggio di programmazione consigliato per la realizzazione del progetto;}
	\item\textbf{Progettazione dell'algoritmo di sincronizzazione: modalità da preferire e approccio corretto.}


\end{itemize}

\newpage


\section{Resoconto}
La riunione è stata richiesta dal gruppo \textit{\Gruppo{}} per approfondire la conoscenza del referente dell'azienda proponente e per chiarire dubbi sorti dopo una attenta analisi del capitolato.
Durante l'incontro è stato possibile sottoporre una serie di quesiti, sia sulle funzionalità desiderate dall'azienda e sia su aspetti prettamente tecnici.
Di seguito sono riportati gli argomenti affrontati.
\subsection{Chiarimenti sulle funzionalità aggiuntive e opzionali}
Il proponente ha esposto le caratteristiche fondamentali del prodotto da sviluppare, evidenziando l'importanza che riveste la corretta realizzazione dell'algoritmo di sincronizzazione.
I requisiti fondamentali sono:
\begin{itemize}
	\item Sviluppo dell'algoritmo di sincronizzazione;
	\item Sviluppo di un'interfaccia multipiattaforma;
	\item Integrazione con il prodotto Zextras Drive.
\end{itemize}
Sono inoltre stati discussi i requisiti opzionali.
\subsection{Discussione sulle API offerte da Zextras Drive}
L'azienda ha esposto le API offerte dal prodotto Zextras Drive e discusso la loro documentazione con il gruppo. È stato inoltre chiarito che l'azienda potrà aggiungere API sotto opportuna richiesta da parte del gruppo e se di possibile realizzazione.
\subsection{Discussione sui linguaggio di programmazione utilizzabili}
Il gruppo ha esposto e motivato una lista di possibili linguaggi scelti per la realizzazione del progetto. Essi sono:
\begin{itemize}
	\item Java;
	\item C++;
	\item Python.
\end{itemize}
Dopo una breve discussione con \textit{Federico Rispo} si è scelto all'unanimità Python per la semplicità e il supporto che le diverse librerie offrono. Tutte queste qualità risultano invece mancanti negli altri linguaggi:
\begin{itemize}
\item C++ risulta complesso e scarno di librerie.
\item Java è stato fortemente sconsigliato dall'azienda data la difficile installazione di un prodotto avente la JVM.
\end{itemize}
\subsection{Requisiti dell'algoritmo di sincronizzazione}
\textit{Alessio Crestani} ha provveduto a chiarire l'importanza che riveste la progettazione dell'algoritmo di sincronizzazione. Esso deve realizzare le scelte migliori per l'utente e permettere la perdita del minor numero di dati nei casi limite.


\newpage

\section{Riepilogo delle decisioni \hfil}
{
	\setlength{\freewidth}{\dimexpr\textwidth-4\tabcolsep}
	\renewcommand{\arraystretch}{1.5}
	\setlength{\aboverulesep}{0pt}
	\setlength{\belowrulesep}{0pt}
	\rowcolors{2}{AzzurroGruppo!10}{white}
	\begin{longtable}{L{.3\freewidth} L{.7\freewidth}}
		\toprule 
		\rowcolor{AzzurroGruppo!30}
		\textbf{Codice} & \textbf{Decisione}\\
		\toprule
		\endhead
		
		VE\_\DataMeeting{}.1 & Elencate le funzionalità obbligatorie e opzionali. \\  
		VE\_\DataMeeting{}.2 & Presentato l'elenco delle API disponibili.\\
		VE\_\DataMeeting{}.3 & Scelto \textit{Python} come linguaggio per la realizzazione del progetto. \\ 
		VE\_\DataMeeting{}.4 & Chiarite le priorità nella progettazione dell'algoritmo di sincronizzazione. \\		
		\bottomrule
		\hiderowcolors
	\end{longtable}
