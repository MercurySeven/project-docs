\section{Informazioni generali}

\begin{itemize}

    \item \textbf{Canale di comunicazione:} Zoom;

    \item \textbf{Data:} \DataMeeting{};

    \item \textbf{Ora inizio:} 16:00;

    \item \textbf{Ora fine:} 17:00;

    \item \textbf{Segretario:} \ACapoRedazione{};

    \item \textbf{Partecipanti}: Due rappresentanti dell'azienda Zextras, \textit{\Alessio{}} e \textit{\Federico{}}, e i membri del gruppo \Gruppo{}:
        \begin{itemize}
            \item \Daniele{};
            \item \Davide{};
            \item \Francesco{};
            \item \Giosue{};
            \item \Lucrezia{};
            \item \Matteo{}.
        \end{itemize}
    \item \textbf{Membri assenti}:
        \begin{itemize}
            \item \Tommaso{}.
        \end{itemize}
\end{itemize}

\section{Ordine del giorno}

\begin{itemize}
    \item\textbf{Confronto con i prodotti dei competitor;}
    \item\textbf{Come gestire l'eliminazione dei file;}
    \item\textbf{Selezionare cosa sincronizzare.}
\end{itemize}
\newpage


\section{Resoconto}
\subsection{Confronto con i prodotti dei competitor}
Per poter ragionare più nel dettaglio sulla logica dell'applicazione e le scelte che dovrà effettuare, si è deciso insieme all'azienda di confrontare il prodotto fino ad ora realizzato con i prodotti offerti dai competitor del settore.

\subsection{Come gestire l'eliminazione dei file}
Durante lo sviluppo dell'algoritmo di sincronizzazione il gruppo si è posto la domanda di come gestire l'eliminazione di un file sul client. Attualmente il problema è far capire agli altri client che il file non è presente nel server perché è stato eliminato e non devono effettuare l'upload ma devono eliminarlo. Il gruppo ha suggerito di usare un flag presente nelle API di Zextras ovvero di marcare il file come 
da cancellare. \textit{\Federico{}} ha proposto di non marcare il file ma di eliminarlo direttamente, mentre \textit{\Alessio{}} la pensava diversamente, ovvero di non eliminare il file anche nel server. Basandosi sul confronto avvenuto in precedenza con i prodotti dei competitor, si è deciso di adottare la soluzione di \textit{\Federico{}} con l'intenzione di rielaborarla in una successiva riunione interna.

\subsection{Selezionare cosa sincronizzare}
L'azienda ci ha raccomandato di pensare a come far decidere all'utente quali file o cartelle non scaricare durante il processo di sincronizzazione.
L'utente potrà selezionare quali nodi non verranno sincronizzati con il server e questi nodi non saranno disponibili offline.

\newpage

\section{Riepilogo delle decisioni \hfil}
{
    \setlength{\freewidth}{\dimexpr\textwidth-4\tabcolsep}
    \renewcommand{\arraystretch}{1.5}
    \setlength{\aboverulesep}{0pt}
    \setlength{\belowrulesep}{0pt}
    \rowcolors{2}{AzzurroGruppo!10}{white}
    \begin{longtable}{L{.3\freewidth} L{.7\freewidth}}
        \toprule
        \rowcolor{AzzurroGruppo!30}
        \textbf{Codice} & \textbf{Decisione}\\
        \toprule
        \endhead

        VE\_\DataMeeting{}.1 &  Si è deciso di confrontare il prodotto con quello dei competitor.\\
        VE\_\DataMeeting{}.2 &  Si è deciso di rielaborare in una riunione interna la logica di eliminazione.\\
        VE\_\DataMeeting{}.3 &  Si è deciso che l'utente potrà selezionare che file/cartelle scaricare.\\
        \bottomrule
        \hiderowcolors
    \end{longtable}
}