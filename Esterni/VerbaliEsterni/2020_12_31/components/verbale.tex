\section{Informazioni generali}

\begin{itemize}

	\item \textbf{Canale di comunicazione:} Zoom;
	
	\item \textbf{Data:} \DataMeeting{};
	
	\item \textbf{Ora inizio:} 10:30;
	
	\item \textbf{Ora fine:} 11:00;
	
	\item \textbf{Segretario:} \Tommaso{};
	
	\item \textbf{Partecipanti}:  e alcuni componenti del gruppo \Gruppo{}:
	
		\begin{itemize}
			\item \Daniele{};
			\item \Davide{};
			\item \Francesco{};
			\item \Giosue{};
			\item \Lucrezia{};
			\item \Tommaso{}.
			
		\end{itemize}

	\item \textbf{Membri assenti}:
		\begin{itemize}
			\item \Matteo{}.
		\end{itemize}
	\end{itemize}
\section{Ordine del giorno}

\begin{itemize}
	\item\textbf{Discussione sul corretto utilizzo del glossario;}
	\item\textbf{Discussione sullo snellimento dei singoli documenti;}
	\item\textbf{Discussione sulle informazioni ridondanti nei vari documenti.}


\end{itemize}

\newpage


\section{Resoconto}
La riunione è stata richiesta dal gruppo \textit{\Gruppo{}} per poter chiarire dubbi sorti durante la stesura della documentazione.
Durante l'incontro è stato possibile sottoporre una serie di quesiti di seguito riportati.
\subsection{Discussione sul corretto utilizzo del glossario}
Dopo una esposizione delle problematiche affrontate dal gruppo e una spiegazione da parte del Prof. \Tullio{} si è arrivati alle seguenti conclusioni:
\begin{itemize}
	\item Il Glossario avrà al suo interno solamente le informazioni che possono essere di difficile comprensione per il proponente ed il gruppo.
	\item Il Glossario potrà essere in futuro trasformato in una wiki per una più facile fruizione.
\end{itemize}
\subsection{Discussione sullo snellimento dei singoli documenti}
Ricordando le nozioni comunicate dal Prof. \Tullio{} riguardanti lo snellimento dei documenti tramite la rimozione di informazioni ridondanti, superflue o di facile obsolescenza, il gruppo ha comunicato il processo avvenuto e ricevuto diversi consigli, arrivando alle seguenti conclusioni:
\begin{itemize}
	\item Si dovrà effettuare un processo di revisione aggiuntivo, andando ad eliminare tutte le definizioni che possono rientrare nel Glossario.
	\item Si dovrà cercare di rendere i documenti di più facile ed immediata fruizione per quanto possibile, evitando l'utilizzo di riempitivi o di ripetizioni.
\end{itemize}
\subsection{Discussione sulle informazioni ridondanti in più documenti}
Il punto più critico della chiamata è risultata la rimozione delle informazioni che venivano ripetute in più documenti. Per ovviare più possibile a questo problema il gruppo andrà a dividere le informazioni ridondanti in diverse categorie:
\begin{itemize}
	\item \textbf{Informazioni critiche per la comprensione del documento}: esse non verranno rimosse ma rimarranno una ridondanza accettata.
	\item \textbf{Informazioni contestuali}: esse dovranno essere analizzate in ogni singola occorrenza per capire se possono essere mantenute o meno nel documento corrente.	
\end{itemize}
Qualsiasi altra informazione che non rientra nelle due categorie seguenti verrà eliminata.\newline
Il gruppo ritiene questa questione ancora aperta e da continuare a tenere al centro dello sviluppo della documentazione, le soluzioni qui adottate sono sicuramente da migliorare e da studiare con più cura.
\newpage

\section{Riepilogo delle decisioni \hfil}
{
	\setlength{\freewidth}{\dimexpr\textwidth-4\tabcolsep}
	\renewcommand{\arraystretch}{1.5}
	\setlength{\aboverulesep}{0pt}
	\setlength{\belowrulesep}{0pt}
	\rowcolors{2}{AzzurroGruppo!10}{white}
	\begin{longtable}{L{.3\freewidth} L{.7\freewidth}}
		\toprule 
		\rowcolor{AzzurroGruppo!30}
		\textbf{Codice} & \textbf{Decisione}\\
		\toprule
		\endhead
		
		VE\_\DataMeeting{}.1 & Scelto l'utilizzo da adottare e la granularità del Glossario. \\  
		VE\_\DataMeeting{}.2 & Adottato un processo atto a diminuire le informazioni inutili nel singolo documento.\\
		VE\_\DataMeeting{}.3 & Adottato un processo atto a diminuire le informazioni ripetute in più documenti. \\ 	
		\bottomrule
		\hiderowcolors
	\end{longtable}
