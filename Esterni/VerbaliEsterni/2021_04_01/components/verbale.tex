\section{Informazioni generali}

\begin{itemize}

    \item \textbf{Canale di comunicazione:} Slack;

    \item \textbf{Settimana:} dal 2021-03-23 al 2021-03-31;

    \item \textbf{Segretario:} \ACapoRedazione{};

    \item \textbf{Partecipanti}: Due rappresentanti dell'azienda Zextras, \textit{\Alessio{}} e \textit{\Federico{}}, e i membri del gruppo \Gruppo{}:
        \begin{itemize}
            \item \Daniele{};
            \item \Davide{};
            \item \Francesco{};
            \item \Giosue{};
            \item \Lucrezia{};  
            \item \Matteo{}.
        \end{itemize}
    \item \textbf{Membri assenti}:
        \begin{itemize}
            \item \Tommaso{}.
        \end{itemize}
\end{itemize}

\section{Ordine del giorno}

\begin{itemize}
    \item\textbf{Gestione del login;}
    \item\textbf{Gestione token di sessione;}
    \item\textbf{Gestione dei test.}
\end{itemize}
\newpage


\section{Resoconto}
\subsection{Gestione del login}
Il gruppo ha fatto notare che al momento la gestione del login non avviene usando delle API, ma viene fatto facendo una chiamata POST all'url di login presente su Zextras Drive.
Ciò comporta che ogni volta venga istanziata la classe API, venga effettuato il login e scaricata la pagina web per recuperare il cookie di sessione, che porta a un rallentamento notevole dell'applicazione. Per questo, il gruppo ha chiesto se si possono avere delle API dedicate. \textit{\Federico{}} ci ha risposto che hanno delle API pronte ma devono ancora essere verificate. Ci ha rassicurato che non appena ci saranno novità ci dirà come usarle.

\subsection{Gestione token di sessione}
Il gruppo ha chiesto dei consigli su come gestire la scadenza del token di sessione. \textit{\Alessio{}} ci ha risposto che in azienda ogni risposta ricevuta nel server viene passata ad un "bridge" che controlla:
\begin{itemize}
    \item Se la risposta di una chiamata è 401 Unauthorized, l'applicazione deve rifare il login (usando le credenziali salvate), se il login è valido, viene rifatta la chiamata che aveva portato all'errore.
    \item Se ottengo due 401 Unauthorized consecutivi, vuol dire che l'utente ha cambiato la password e deve essere mostrata la form di login.
\end{itemize}

\subsection{Gestione dei test}
Il gruppo ha esposto all'azienda il proprio modus operandi per la gestione dei test. L'azienda si è mostrata soddisfatta ed ha consigliato delle metodologie e tecnologie aggiuntive per poter correttamente testare la comunicazione con i propri server.

\newpage

\section{Riepilogo delle decisioni \hfil}
{
    \setlength{\freewidth}{\dimexpr\textwidth-4\tabcolsep}
    \renewcommand{\arraystretch}{1.5}
    \setlength{\aboverulesep}{0pt}
    \setlength{\belowrulesep}{0pt}
    \rowcolors{2}{AzzurroGruppo!10}{white}
    \begin{longtable}{L{.3\freewidth} L{.7\freewidth}}
        \toprule
        \rowcolor{AzzurroGruppo!30}
        \textbf{Codice} & \textbf{Decisione}\\
        \toprule
        \endhead

        VE\_\DataMeeting{}.1 &  Si è deciso di continuare ad eseguire il login con il metodo corrente, in attesa delle API.\\
        VE\_\DataMeeting{}.2 &  Si è deciso di sistemare la parte network creando un "bridge" per gestire la scadenza del cookie.\\
        VE\_\DataMeeting{}.3 &  Si è deciso di utilizzare metodologie alternative per testare la comunicazione con il server.\\
        \bottomrule
        \hiderowcolors
    \end{longtable}
}