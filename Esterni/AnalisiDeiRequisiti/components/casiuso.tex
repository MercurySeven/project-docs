\section{Casi d'uso}
\subsection{Introduzione}
In questa sezione vengono descritti i \glo{casi d'uso} individuati dal gruppo \gruppo{}.
I \glo{casi d'uso} analizzati fanno riferimento a tutte le funzionalità che il \glo{software} di sincronizzazione dovrà fornire all'utente finale.\newline
Essendo il progetto finalizzato alla creazione di un applicativo per la sincronizzazione di file, le interazioni da parte dell'utilizzatore sono limitate a pochi e semplici \glo{casi d'uso}.

\subsection{Attori}

\subsubsection{Attori Principali}
\begin{itemize}
\item \textbf{Utente non autenticato:} È un utente che ha avviato per la prima volta l'applicazione, non è in grado di sincronizzare i file con \glo{Zextras Drive} fino a quando non effettuerà il login.
\item \textbf{Utente autenticato:} Utente presente nel \glo{sistema} Zextras, è riconosciuto all'interno dell'applicazione e può usare il programma per sincronizzare i file all'interno di \glo{Zextras Drive}.
\end{itemize}

\subsubsection{Attori Secondari}
\begin{itemize}
\item \textbf{Gestore credenziali Zextras:} Consente di verificare se le credenziali inserite dall'utente sono presenti all'interno del database di Zextras.
\item \textbf{\glo{Zextras Drive}:} È un servizio on-premises per conservare i file online, consente di condividere i file presenti nel proprio spazio utente con altri utenti presenti nel \glo{sistema}. Fornisce delle \glo{API} con cui è possibile caricare/scaricare file, che sono utili all'applicazione per effettuare la sincronizzazione.
\end{itemize}

\newpage

\subsection{UC1 - Autenticazione}
\begin{figure}[H]
    \centering
    \includegraphics[scale = 0.7]{components/img/UC1.png}
    \caption{UC1 - Autenticazione}
\end{figure}
\subsubsection{UC1.1 - Inserimento credenziali}
\begin{itemize}
\item \textbf{Attore Primario:} Utente non autenticato;
\item \textbf{Attore Secondario:} Gestore credenziali Zextras;
\item \textbf{Precondizione:} L'utente non è riconosciuto dal sistema;
\item \textbf{Postcondizione:} L'utente ha effettuato il login usando le credenziali di \glo{Zextras Drive};
\item \textbf{Scenario principale:}
    \begin{enumerate}
    \item L'utente avvia l'applicazione per la prima volta;
    \item L'utente inserisce le credenziali \glo{Zextras Drive};
    \end{enumerate}
\item \textbf{Estensioni:}
\begin{itemize}
\item Visualizzazione messaggio di errore inserimento credenziali (UC1.2 \S{}\ref{UC1.2}).
\end{itemize}
\end{itemize}

\subsubsection{UC1.2 - Visualizzazione messaggio di errore inserimento credenziali}
\label{UC1.2}
\begin{itemize}
\item \textbf{Attore Primario:} Utente non autenticato;
\item \textbf{Precondizione:} Il \glo{sistema} rileva un inserimento di credenziali errate;
\item \textbf{Postcondizione:} L'utente viene informato che le credenziali da lui inserite sono errate;
\item \textbf{Scenario principale:}
    \begin{enumerate}
    \item L'utente ha inserito delle credenziali;
    \item Il \glo{sistema} rileva che le credenziali inserite sono sbagliate.
    \end{enumerate}
\end{itemize}%Autenticazione

\subsection{UC2 - Sincronizzazione file da client a server}
\label{UC2}
\begin{figure}[H]
    \centering
    \includegraphics[scale = 0.7]{components/img/UC2.png}
    \caption{UC2 - Sincronizzazione file da client a server}
\end{figure}
\begin{itemize}
\item \textbf{Attore Primario:} Utente autenticato;
\item \textbf{Attore Secondario:} \glo{Zextras Drive};
\item \textbf{Precondizione:} L'utente ha la necessità di sincronizzare uno o più file, presenti nella cartella di root, dal client al server;
\item \textbf{Postcondizione:} L'utente ha sincronizzato i file presenti nella cartella di root dal client al server;
\item \textbf{Scenario principale:}
\begin{enumerate}
\item L'utente vuole sincronizzare dei file dal client verso il server;
\item Viene scelto dalla cartella di root l'insieme dei file che dovranno essere sincronizzati;
\item La sincronizzazione delle modifiche viene attivata per l'insieme scelto.
\end{enumerate}
\item \textbf{Estensioni:}
    \begin{itemize}
    \item Visualizza messaggio di errore file troppo grande (UC9 \S{}\ref{UC9});
    \item Visualizza messaggio di errore di rete (UC7 \S{}\ref{UC7}).
    \end{itemize}
\end{itemize}
%Seleziona file da sincronizzare dal client al server

\subsection{UC3 - Sincronizzazione file da server a client}
\label{UC3}
\begin{figure}[H]
    \centering
    \includegraphics[scale = 0.5]{components/img/UC3.png}
    \caption{UC3 - Sincronizzazione file da server a client}
\end{figure}
\begin{itemize}
\item \textbf{Attore Primario:} Utente autenticato;
\item \textbf{Attore Secondario:} \glo{Zextras Drive};
\item \textbf{Precondizione:} L'utente ha necessità di sincronizzare uno o più file presenti sul server verso il client;
\item \textbf{Postcondizione:} L'utente ha sincronizzato i file dal server al client;
\item \textbf{Scenario principale:}
    \begin{enumerate}
    \item L'utente vuole sincronizzare dei file dal server al client;
    \item Viene scelto l'insieme dei file che dovrà essere sincronizzato;
    \item La sincronizzazione delle modifiche viene attivata per l'insieme scelto;
    \end{enumerate}
\item \textbf{Estensioni:} Visualizza messaggio di opzione per il file in conflitto (UC15 \S{}\ref{UC15}).
\end{itemize}

\subsubsection{UC3.1 - Selezione file presenti nel server}
\label{UC3.1}
\begin{itemize}
\item \textbf{Attore Primario:} Utente autenticato;
\item \textbf{Attore Secondario:} \glo{Zextras Drive};
\item \textbf{Precondizione:} L'utente ha necessità di sincronizzare uno o più file presenti sul server verso il client;
\item \textbf{Postcondizione:} L'utente ha selezionato l'insieme dei file da sincronizzare;
\item \textbf{Scenario principale:} Viene scelto l'insieme dei file che dovrà essere sincronizzato;
\item \textbf{Estensioni:}
    \begin{itemize}
    \item Visualizza messaggio di errore di rete (UC7 \S{}\ref{UC7});
    \item Visualizza messaggio di errore spazio non disponibile in locale (UC10 \S{}\ref{UC10});
    \end{itemize}
\end{itemize}%Seleziona file da sincronizzare dal server al client

\subsection{UC4 - Visualizzazione messaggio di errore inserimento credenziali}
\label{UC4}
\begin{itemize}
\item \textbf{Attore Primario:} Utente non autenticato;
\item \textbf{Precondizione:} Il sistema rileva un inserimento di credenziali errate;
\item \textbf{Postcondizione:} L'utente viene informato che le credenziali da lui inserite sono errate;
\item \textbf{Scenario principale:}
    \begin{enumerate}
    \item L'utente ha inserito delle credenziali;
    \item Il sistema rileva che le credenziali inserite sono sbagliate.
    \end{enumerate}
\end{itemize}%Seleziona quota disco

\subsection{UC5 - Logout}
\begin{itemize}
\item \textbf{Attore Primario:} Utente autenticato;
\item \textbf{Precondizione:} L'utente vuole effettuare il logout;
\item \textbf{Postcondizione:} L'utente ha effettuato il logout e non è più riconoscibile all'interno dell'applicazione;
\item \textbf{Scenario principale:}
    \begin{enumerate}
    \item L'utente vuole effettuare il logout;
    \item L'utente ha effettuato il logout;
    \item Tutti i file che l'utente in precedenza sincronizzava non vengono più sincronizzati con il server.
    \end{enumerate}
\end{itemize}%Logout

\subsection{UC6 - Visualizza dettagli file}
\begin{figure}[H]
    \centering
    \includegraphics[scale = 0.7]{components/img/UC6.png}
    \caption{UC6 - Visualizza dettagli file}
\end{figure}
\begin{itemize}
\item \textbf{Attore Primario:} Utente autenticato;
\item \textbf{Precondizione:} L'utente ha selezionato la cartella root e sono presenti file al suo interno;
\item \textbf{Postcondizione:} L'utente ha informazioni riguardo al file;
\item \textbf{Scenario principale:} L'utente può visualizzare le informazioni riguardanti i file all'interno della cartella root selezionata.
\end{itemize}%Visualizza dettagli file

\subsection{UC7 - Riprendi trasferimento}
\begin{figure}[H]
    \centering
    \includegraphics[scale = 0.4]{components/img/UC7.png}
    \caption{UC7 - Riprendi trasferimento}
\end{figure}
\begin{itemize}
\item \textbf{Attore Primario:} Utente autenticato;
\item \textbf{Precondizione:} L'utente ha a disposizione la possibilità di riprendere il trasferimento di un file messo precedentemente in pausa;
\item \textbf{Postcondizione:} Viene modificato lo stato del file da trasferimento in pausa a trasferimento in corso;
\item \textbf{Scenario principale:}
    \begin{enumerate}
    \item L'utente può riprendere il trasferimento di un file che si trova nello stato di trasferimento in pausa.
    \end{enumerate}
\end{itemize}%Visualizza messaggio di errore di rete

\subsection{UC8 - Annullamento operazione}
\begin{figure}[H]
    \centering
    \includegraphics[scale = 0.7]{components/img/UC8.png}
    \caption{UC8 - Annullamento operazione}
\end{figure}
\begin{itemize}
\item \textbf{Attore Primario:} Utente autenticato;
\item \textbf{Precondizione:} L'utente ha a disposizione la possibilità di effettuare una selezione;
\item \textbf{Postcondizione:} tutte le selezioni effettuate fino a quel momento vengono cancellate;
\item \textbf{Scenario principale:}
    \begin{enumerate}
    \item L'utente cambia idea sulle selezioni effettuate;
    \item L'utente può cancellare ogni selezione fatta fino a quel momento.
    \end{enumerate}
\end{itemize}%Visualizza messaggio di errore inserimento credenziali

\subsection{UC9 - Visualizza messaggio di errore file troppo grande}
\label{UC9}
\begin{itemize}
\item \textbf{Attore Primario:} Utente autenticato;
\item \textbf{Precondizione:} L'utente ha selezionato i file da sincronizzare con \glo{Zextras Drive};
\item \textbf{Postcondizione:} L'utente viene informato che non è stato possibile completare la sincronizzazione;
\item \textbf{Scenario principale:}
    \begin{enumerate}
    \item L'utente ha selezionato i file da sincronizzare;
    \item L'applicazione si connette a \glo{Zextras Drive}, ma fallisce la sincronizzazione per mancanza di spazio nel cloud.
    \end{enumerate}
\end{itemize}%Visualizza messaggio di errore file troppo grande

\subsection{UC10 - Nascondi applicazione}
\begin{figure}[H]
    \centering
    \includegraphics[scale = 0.7]{components/img/UC10.png}
    \caption{UC10 - Nascondi applicazione}
\end{figure}
\begin{itemize}
\item \textbf{Attore Primario:} Utente autenticato;
\item \textbf{Precondizione:} L'applicazione è mostrata a schermo;
\item \textbf{Postcondizione:} L'applicazione viene mostrata nell'area di notifica;
\item \textbf{Scenario principale:}
    \begin{enumerate}
    \item L'utente visualizza l'applicazione a tutto schermo;
    \item L'utente chiude l'applicazione.
    \end{enumerate}
\end{itemize}%Visualizza messaggio di errore spazio non disponibile in locale

\subsection{UC11- Visualizzazione informazioni sul software}
\begin{figure}[H]
    \centering
    \includegraphics[scale = 0.7]{components/img/UC11.png}
    \caption{UC11 - Visualizzazione informazioni sul software}
\end{figure}
\begin{itemize}
\item \textbf{Attore Primario:} Utente autenticato;
\item \textbf{Precondizione:} Il sistema si trova nella schermata delle impostazioni;
\item \textbf{Postcondizione:} Viene visualizzata una finestra con le informazioni sul software;
\item \textbf{Scenario principale:}
    \begin{enumerate}
    \item L'utente vuole avere informazioni sul software e clicca sull'apposito pulsante;
    \item Il sistema visualizza una finestra apposita con ciò che l'utente ha chiesto.
    \end{enumerate}
\end{itemize}%La quota disco selezionata è minore dei file scaricati

\subsection{UC12 - Visualizzazione messaggio di errore}
\begin{figure}[H]
    \centering
    %\includegraphics[scale = 0.4]{components/img/UC12.png}
    \caption{UC12 - Visualizzazione messaggio di errore}
\end{figure}
\begin{itemize}
\item \textbf{Attore Primario:} Utente;
\item \textbf{Precondizione:} Il sistema rileva una condizione di errore;
\item \textbf{Postcondizione:} Viene mostrato un messaggio di errore opportuno;
\item \textbf{Scenario principale:}
    \begin{enumerate}
    \item Il sistema rileva una condizione di errore e mostra un messaggio di errore relativo al contesto nel quale si è riscontrata l'anomalia.
    \end{enumerate}
\end{itemize}%La quota disco selezionata è maggiore della dimensione libera del disco

\subsection{UC13 - Scelta della cartella root}
\label{UC13}
\begin{itemize}
\item \textbf{Attore Primario:} Utente autenticato;
\item \textbf{Precondizione:} L'utente si è appena autenticato;
\item \textbf{Postcondizione:} L'utente ha scelto una cartella all'interno della quale verranno sincronizzati i file;
\item \textbf{Scenario principale:} L'utente deve scegliere una cartella presente nel suo computer.
\end{itemize}%Scelta della cartella root

\subsection{UC14 - Scelta politica di risoluzione conflitti}
\label{UC14}
\begin{itemize}
\item \textbf{Attore Primario:} Utente autenticato;
\item \textbf{Precondizione:} L'utente vuole cambiare il modo con cui l'applicazione gestisce i conflitti;
\item \textbf{Postcondizione:} I prossimi conflitti individuati dall'applicazione verranno risolti con la politica scelta dall'utente;
\item \textbf{Scenario principale:}
    \begin{enumerate}
    \item All'utente vengono proposte due politiche per risolvere i possibili conflitti:
        \begin{enumerate}
        \item Mantieni le modifiche del client;
        \item Lascia scegliere all'utente cosa tenere e cosa scartare;
        \end{enumerate}
    \item L'utente sceglie la politica da applicare.
    \end{enumerate}
\end{itemize}%Scelta politica di risoluzione conflitti

\subsection{UC15 - Visualizza messaggio di opzione per il file in conflitto}
\label{UC15}
\begin{itemize}
\item \textbf{Attore Primario:} Utente autenticato;
\item \textbf{Precondizione:} L'utente ha scelto di decidere manualmente come vengono risolti i conflitti;
\item \textbf{Postcondizione:} Il conflitto tra due file è stato risolto e il file caricato sul cloud;
\item \textbf{Scenario principale:}
    \begin{enumerate}
    \item All'utente comparirà un avviso dove lo si informa che un file ha bisogno del suo intervento;
    \item L'utente dovrà scegliere se tenere il suo file locale e caricarlo sul cloud o accettare le modifiche provenienti dal cloud e sovrascrivere le sue modifiche.
    \end{enumerate}
\end{itemize}%Visualizza messaggio di opzione per il file in conflitto

\subsection{UC16 - Impostare ogni quanto sincronizzare la cartella root}
\label{UC16}
\begin{itemize}
\item \textbf{Attore Primario:} Utente autenticato;
\item \textbf{Precondizione:} L'utente vuole cambiare quanto spesso la cartella venga sincronizzata con il cloud;
\item \textbf{Postcondizione:} Le modifiche effettuate vengono applicate alla prossima sincronizzazione;
\item \textbf{Scenario principale:} L'utente potrà scegliere un intervallo di tempo compreso da 1 minuto a 24 ore;
\end{itemize}%Impostare ogni quanto sincronizzare la cartella root

\subsection{UC17 - Seleziona file da condividere ad altre persone}
\label{UC17}
\begin{itemize}
\item \textbf{Attore Primario:} Utente autenticato;
\item \textbf{Attore Secondario:} \glo{Zextras Drive};
\item \textbf{Precondizione:} L'utente ha effettuato dei trasferimenti verso il server;
\item \textbf{Postcondizione:} L'utente ottiene un link che gli consente di convdividere dei documenti ad altre persone;
\item \textbf{Scenario principale:}
    \begin{enumerate}
    \item L'utente seleziona il file che vuole condividere ad altre persone;
    \item L'applicazione farà una richiesta tramite \glo{API} a \glo{Zextras Drive} per ottenere un link da dare all'utente.
    \end{enumerate}
\end{itemize}%Seleziona file da condividere ad altre persone