\section{Introduzione}
\subsection{Scopo del documento}
Il seguente documento ha come obbiettivo la descrizione completa e dettagliata dei \glo{casi d'uso} e dei \glo{requisiti} del progetto. Tutte le informazioni presenti nel documento derivano dallo studio del capitolato e dagli incontri avvenuti con l'azienda \glo{proponente}.

\subsection{Scopo del prodotto}
Il capitolato chiede di sviluppare un algoritmo e un'applicazione per la sincronizzazione file tra server e client appoggiandosi sul sevizio \glo{Zextras Drive}.
Il programma client deve essere multipiattaforma e il prodotto deve essere sviluppato aderendo al pattern \glo{MVC}.

\subsection{Glossario}
Nel documento sono presenti termini che possono presentare dei significati ambigui o poco chiari a seconda del contesto.
Per evitare incomprensioni il gruppo fornisce un glossario individuabile nel file \G{} \versGlo{} contenente i termini e la loro spiegazione.\newline{}
Nel documento corrente si è deciso di segnare tutte le parole presenti nel \G{} con una "G" a pedice di ogni parola.

\subsection{Riferimenti}
\subsubsection{Riferimenti normativi}
\begin{itemize}
\item \textbf{Capitolato d'appalto C7}:\newline
      \url{https://www.math.unipd.it/~tullio/IS-1/2020/Progetto/C7.pdf}
\item \NdP{} \versNdP ;
\item \VE{} 2020-12-28; %incontro con l'azienda
\end{itemize}

\subsubsection{Riferimenti informativi}
\begin{itemize}
\item \textbf{Presentazione del capitolato:}\newline
      \url{https://www.math.unipd.it/~tullio/IS-1/2020/Progetto/C7.pdf}
\item \textbf{Materiale didattico del corso di Ingegneria del \ignore{Software}:}
    \begin{itemize}
    \item Analisi dei \ignore{requisiti}\newline
      \url{https://www.math.unipd.it/~tullio/IS-1/2020/Dispense/L07.pdf}
    \item Diagrammi dei Casi d'Uso\newline
      \url{https://www.math.unipd.it/~rcardin/swea/2021/Diagrammi\%20Use\%20Case_4x4.pdf}
    \end{itemize}
\item \textbf{Documentazione \ignore{Zextras} \ignore{Drive} \ignore{API}}

\end{itemize}