\section{Descrizione del prodotto}
\subsection{Caratteristiche del prodotto}
Lo scopo del capitolato C7 è la creazione di un algoritmo \glo{solido} ed efficiente in grado di garantire il salvataggio e la sincronizzazione dei cambiamenti presenti in cloud. È inoltre richiesta la creazione di una interfaccia grafica multipiattaforma. Il progetto deve dunque funzionare sui principali sistemi operativi desktop quali Windows, MacOS e Linux senza richiedere l'installazione manuale di ulteriori prodotti per il corretto funzionamento. 
\subsection{Obiettivo del prodotto}
L'obiettivo del progetto è la realizzazione di un modulo per la piattaforma \glo{Zimbra}, che abbia le caratteristiche sopra descritte, e che si appoggi per il suo funzionamento al modulo \glo{Zextras Drive}.
Nello specifico, lo scopo finale del prodotto è quello di fornire agli utenti un servizio che automatizzi il salvataggio e la sincronizzazione con il cloud.
\subsection{Caratteristiche degli utenti}
Il modulo \progetto{}, per natura stessa dei moduli di Zextras e dei prodotti di sincronizzazione desktop, è rivolto ad utenti professionisti che utilizzano \glo{Zimbra}. 
\subsection{Vincoli progettuali}
Il prodotto \glo{software} finale è soggetto a vincoli progettuali obbligatori ed opzionali così come specificato all'interno del capitolato C7.
Una tabella contenente tutti i vincoli progettuali imposti dal \glo{proponente} si può trovare nel capitolo \S{}\ref{req_vincolo}.