\section{Introduzione}
\subsection{Scopo del documento}

Lo scopo del  \PdQ{}  è quello di descrivere le strategie adottate dal gruppo \Gruppo{} utilizzate nelle fasi di controllo qualità e validazione per garantire la qualità del prodotto durante l'intera durata del progetto.  I contenuti iniziali verranno aggiornati o modificati andando avanti con la realizzazione del prodotto,  per questo il documento alla forma attuale è da considerarsi incompleto.  

\subsection{Scopo del prodotto}

L'obiettivo del prodotto \progetto{} proposto da \textit{Zextras} è quello di creare un algoritmo e un'applicazione Desktop multipiattaforma per la sincronizzazione e il salvataggio su cloud di file,  appoggiandosi al servizio Zextras Drive.  

\subsection{Glossario}

Al fine di evitare incomprensioni o ambiguità dovuti a termini non chiari,  è stato redatto un \G{} contenente tutte le terminologie utilizzate nel progetto che possono risultare ambigue o che si riferiscono ad argomenti tecnici o specifici.  Le parole presenti nel suddetto documento sono indicate tramite una \glo{} a pedice della prima ricorrenza della parola. 

\subsection{Riferimenti}
\subsubsection{Riferimenti normativi}
Si fa riferimento alle \NdP{} per quanto riguarda la presentazione di metriche d'interesse per il documento e gli strumenti a esse correlati. 
\subsubsection{Riferimenti informativi}
\begin{itemize}

\item Qualità di prodotto: \url{https://www.math.unipd.it/~tullio/IS-1/2019/Dispense/L12.pdf} ;

\item Qualità di processo: \url{https://www.math.unipd.it/~tullio/IS-1/2019/Dispense/L13.pdf} ;

\item Verifica e validazione: \url{https://www.math.unipd.it/~tullio/IS-1/2019/Dispense/L14.pdf} .

\end{itemize}
