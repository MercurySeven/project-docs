\section{Qualità del prodotto}
\subsection{Scopo}
Per garantire la qualità dei prodotti realizzati il gruppo \gruppo{} ha deciso di adottare lo standard ISO/IEC 9126.
Le metriche e gli obiettivi definiti di seguito dovranno essere integrate nel processo di testing di ogni sprint.
\subsection{Prodotti}
\subsubsection{Descrizione}
In questa sezione vengono definite le caratteristiche di cui i membri del gruppo \gruppo{} dovranno tenere conto durante la creazione dei prodotti.
\subsubsection{Qualità dei documenti}

\paragraph{Comprensibilità}
I documenti prodotti devono essere facilmente leggibili e accessibili a tutti i membri del progetto. \\

\textbf{Obiettivi}
\begin{itemize}
\item \textbf{Leggibilità}: i documenti devono essere leggibili e comprensibili;
\item \textbf{Correttezza ortografica}: i documenti non devono presentare errori ortografici.
\end{itemize}

\textbf{Metriche}
\begin{itemize}
\item MDC1 Indice di Gulpease;
\item MDC2 Correttezza ortografica.
\end{itemize}

\subsubsection{Qualità del software}

\paragraph{Funzionalità}

Il prodotto deve fornire tutte le funzionalità determinate nel documento di \AdR{}. \\

\textbf{Obiettivi}
\begin{itemize}
\item \textbf{Accuratezza}: le funzionalità del prodotto soddisfano le richieste del committente;
\item \textbf{Adeguatezza}: le funzionalità del prodotto hanno il comportamento aspettato;
\item \textbf{Sicurezza}: il prodotto garantisce la sicurezza dei dati sensibili gestiti.
\end{itemize}

\textbf{Metriche}
\begin{itemize}
\item MPR3 Soddisfazione requisiti obbligatori;
\item MPR4 Soddisfazione requisiti opzionali.
\end{itemize}

\paragraph{Affidabilità}

Durante l'utilizzo del prodotto non si devono riscontrare errori durante l'utilizzo delle funzionalità da parte dell'utente. Eventuali errori devono essere adeguatamente gestiti. \\

\textbf{Obiettivi}
\begin{itemize}
\item \textbf{Maturità}: il prodotto non deve avere malfunzionamenti;
\item \textbf{Tolleranza agli errori}: capacità del prodotto di gestire casi di errore.
\end{itemize}

\textbf{Metriche}
\begin{itemize}
\item MVR1 Percentuale di malfunzionamenti del programma in rapporto ai test eseguiti;
\item MCD1 Percentuale degli errori gestiti.
\end{itemize}

\paragraph{Usabilità}

Il prodotto deve essere intuitivo e pratico. \\

\textbf{Obiettivi}
\begin{itemize}
\item \textbf{Comprensibilità}: il prodotto deve essere quanto più intuitivo possibile per garantire che l'utente sia in grado di riconoscere e utilizzare le funzionalità offerte con uno sforzo e un investimento di tempo minimo;
\item \textbf{Apprendibilità}: presenza di documentazione per l'utente;
\item \textbf{Operabilità}: le funzionalità offerte dal prodotto sono in linea con le aspettative dell'utente;
\item \textbf{Attrattiva}: il prodotto deve essere utilizzabile facilmente e avere una presentazione attraente per l'utente.
\end{itemize}

\textbf{Metriche}
\begin{itemize}
\item MVR2 Comprensibilità delle funzionalità offerte;
\item MVR3 Facilità di apprendimento e di utilizzo del prodotto.
\end{itemize}

\subparagraph{Testing}
I dati per le metriche associate a questo particolare aspetto del software verranno raccolti tramite sessioni di \glo{QA} svolte da di membri del gruppo \gruppo{}. \\


\paragraph{Efficienza}

La quantità di risorse utilizzate deve essere ottimale. \\

\textbf{Obiettivi}
\begin{itemize}
\item \textbf{Complessità delle funzioni}: le funzioni codificate devono rimanere al di sotto di una specifica soglia di complessità.
\item \textbf{Utilizzo delle risorse}: il prodotto deve essere efficiente e consumare la quantità ottimale di risorse.
\end{itemize}

\textbf{Metriche}
\begin{itemize}
\item MCD2 Complessità ciclomatica;
\item MCD5 Code coverage.
\end{itemize}

\paragraph{Manutenibilità}

Il prodotto deve essere facilmente estendibile e adattabile. \\

\textbf{Obiettivi}
\begin{itemize}
\item \textbf{Stabilità}: la codifica del prodotto deve essere chiara e robusta così da facilitare l'aggiunta di funzionalità;
\item \textbf{Testabilità}: il prodotto deve essere nella condizione di poter essere testato in modo facile e veloce;
\item \textbf{Modificabilità}: la codifica del prodotto deve essere facilmente estensibile;
\item \textbf{Analizzabilità}: la struttura del prodotto deve facilitare la ricerca delle cause di eventuali malfunzionamenti.
\end{itemize}

\textbf{Metriche}
\begin{itemize}
\item MCD3 Variabili non utilizzate;
\item MCD4 Rapporto linee di codice e linee commentate.
\end{itemize}

\subsubsection{Tabella riassuntiva delle metriche}
\paragraph{Documenti}
Di seguito sono riportate le metriche in forma di tabella.
\begin{table}[H]
		\begin{center}
			\setlength{\aboverulesep}{0pt}
			\setlength{\belowrulesep}{0pt}
			\setlength{\extrarowheight}{.75ex}
			\rowcolors{2}{AzzurroGruppo!10}{white}
			\begin{tabular}{ c C{6cm} C{3cm} C{3cm} }
				\rowcolor{AzzurroGruppo!30} 
				\textbf{Metrica} & \textbf{Obiettivo} & \textbf{Valori accettati} & \textbf{Valori ottimali}  \\
				\toprule
				MDC1 & Indice di Gulpease & $ 50 < x < 100$ & $60 < x < 100$ giorni \\
				MDC2 & Correttezza ortografica & $ 95 \%$ privi di errori & $ 100\%$ privi di errori \\
				\bottomrule
			\end{tabular}
			\caption{Tabella delle metriche e degli obiettivi relativi alla documentazione}
		\end{center}
	\end{table}
	
\paragraph{Prodotto}
\begin{table}[H]
		\begin{center}
			\setlength{\aboverulesep}{0pt}
			\setlength{\belowrulesep}{0pt}
			\setlength{\extrarowheight}{.75ex}
			\rowcolors{2}{AzzurroGruppo!10}{white}
			\begin{tabular}{ c C{6cm} C{3cm} C{3cm} }
				\rowcolor{AzzurroGruppo!30} 
				\textbf{Metrica} & \textbf{Obiettivo} & \textbf{Valori accettati} & \textbf{Valori ottimali}  \\
				\toprule
				MPR3 & Soddisfazione requisiti obbligatori & $100 \%$ & $100 \%$ giorni \\
				MPR4 & Soddisfazione requisiti opzionali & $ 0 \%$ & $ 100\%$ \\
				MVR1 & Percentuale di malfunzionamenti del programma in rapporto ai test eseguiti & $ 0\%$ - $5\%$  & $0\%$ \\
				MCD1 & Percentuale degli errori gestiti & $ 95\%$ - $100\% $ & $100\%$ \\
				MVR2 & Comprensibilità delle funzionalità offerte & $ 80\%$ - $100\% $ & $100\%$ \\
				MVR3 & Facilità di apprendimento e di utilizzo del prodotto & $10$-$15$ minuti & $5$-$10$ minuti \\
				MCD2 & Complessità ciclomatica & $0$ - $30$ & $0$ - $30$ \\
				MCD3 & Variabili non utilizzate & $0$ & $0$ \\
				MCD4 & Rapporto linee di codice e linee commentate & $5$ - $3$ & $3$ - $1$\\
				MCD5 & Code coverage & $80\%$-$100\%$ & $100\%$ \\
				\bottomrule
			\end{tabular}
			\caption{Tabella delle metriche e degli obiettivi relativi al prodotto}
		\end{center}
	\end{table}