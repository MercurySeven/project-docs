\section{Qualità del processo}

\subsection{Scopo}
Per assicurarsi dell'efficienza dello sviluppo è importante definire delle metriche di qualità relative al processo.
Per il progetto è stato scelto il modello di sviluppo \glo{Agile}.
Le metriche scelte devono quindi essere generali per garantire una veloce iterazione in base al feedback dei membri del gruppo.
L'obbiettivo di questo approccio è quello di garantire la centralità del prodotto in ogni fase del progetto, subordinando quindi i processi al prodotto.

\subsection{Processi}

\subsubsection{PROC001 - Pianificazione di progetto}
Il macro-processo di pianificazione di progetto ha come scopo quello di definire dei piani di sviluppo per il raggiungimento degli obbiettivi richiesti dal committente, nel tempo disponibile, senza un consumo eccessivo di risorse.

\paragraph{Obiettivi}
Il gruppo deve quindi strutturare la pianificazione attorno ai seguenti elementi:
\begin{itemize}
\item \textbf{Timeline}: Assicurarsi che il progetto rispetti le scadenze imposte dal committente;
\item \textbf{Budget}: Assicurarsi che gli elementi e i tempi definiti nel piano rispettino le risorse disponibili.
\end{itemize}

\paragraph{Metriche}%
Per tenere traccia di questi elementi vengono utilizzate le seguenti metriche:
\begin{itemize}
\item MPRC001 Schedule Variance;
\item MPRC002 Budget Variance.
\end{itemize}

\paragraph{Tabella riassuntiva delle metriche}%
Di seguito sono riportate le metriche in forma di tabella.



