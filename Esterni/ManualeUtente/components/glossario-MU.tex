\appendix

\section{Glossario}
\label{sec:Glossario}
\subsubsection*{B}
\begin{itemize}
    \item \textbf{Bug}: Un problema che porta al malfunzionamento del software, tipicamente dovuto a un errore nella scrittura del codice sorgente di un programma.
\end{itemize}

\subsubsection*{C}
\begin{itemize}
    \item \textbf{Client}: Un qualunque componente software che accede ai servizi o alle risorse di un'altra componente detta server.
    \item \textbf{Cloud}: Indica un paradigma di erogazione di servizi offerti su richiesta da un fornitore a un cliente finale attraverso la rete internet, a partire da un insieme di risorse preesistenti, configurabili e disponibili in remoto sotto forma di architettura distribuita. 
    \item \textbf{Conflitto}: Si definisce conflitto tra due file se essi hanno lo stesso nome ed estensione risultando però diversi nel contenuto.
\end{itemize}
\subsubsection*{G}
\begin{itemize}
    \item \textbf{GitHub}: è un servizio di hosting per progetti software. Il nome "GitHub" deriva dal fatto che GitHub è una implementazione dello strumento di controllo versione distribuito Git.
\end{itemize}
\subsubsection*{H}
\begin{itemize}
    \item \textbf{Hardware}: Parte materiale di un sistema elettronico di elaborazione.
\end{itemize}

\subsubsection*{P}
\begin{itemize}
    \item \textbf{Python}: Linguaggio di programmazione orientato agli oggetti ad alto livello.
\end{itemize}

\subsubsection*{R}
\begin{itemize}
    \item \textbf{Repository}: Archivio digitale dove dati e informazioni sono raccolti, valorizzati e archiviati sulla base di metadati che ne permettono la rapida individuazione. Grazie alla sua peculiare architettura, un repository consente di gestire in modo ottimale anche grandi volumi di dati.
    \item \textbf{Requisiti minimi di sistema}: I requisiti sono le capacità minime sia hardware che software che il sistema dovrà avere per svolgere le funzioni.
\end{itemize}

\subsubsection*{S}
\begin{itemize}
    \item \textbf{Server}: Un componente che fornisce un servizio ad altre componenti chiamate client.
    \item \textbf{Software}: Insieme delle componenti immateriali di un sistema elettronico di elaborazione.
    \item \textbf{Solido}: In informatica, e in particolare in programmazione, l'acronimo SOLID si riferisce ai "primi cinque principi" dello sviluppo del software orientato agli oggetti descritti da Robert C. Martin in diverse pubblicazioni. I SOLID principles (Single responsibility, Open-closed, Liskov substitution, Interface segregation, Dependency inversion) sono intesi come linee guida per lo sviluppo di software leggibile, estendibile e mantenibile, in particolare nel contesto di pratiche di sviluppo agili.
    \item \textbf{SSD}: È l'acronimo di Soluzioni di Sincronizzazione Desktop. Il progetto didattico è stato presentato nel capitolato C7 del corso di Ingegneria del Software tenuto all'Università di Padova nel Corso di Laurea in Informatica dell'anno 2020/2021.
    \item \textbf{Sync engine}: È il software che opera nel computer, che confronta i file presenti nel computer e nel drive. Si occupa inoltre di decidere che operazioni fare con il server come il download o l'upload dei file.
\end{itemize}

\subsubsection*{Z}
\begin{itemize}
    \item \textbf{Zextras Drive}: Un nuovo componente di Zimbra che offre un sistema di archiviazione integrato con il WebClient di Zimbra. 
    \item \textbf{Zimbra}: Una suite di software collaborativi che include sia un server email che un web client.
\end{itemize}
\appendix