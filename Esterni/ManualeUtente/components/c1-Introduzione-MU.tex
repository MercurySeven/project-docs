\section{Introduzione}
\subsection{Scopo del documento}
Lo scopo di questo documento è illustrare tutte le funzionalità del \gloman{sync engine} richiesto dal capitolato \textit{C7}. L'utente finale, in questo modo, avrà a disposizione tutte le indicazioni per il corretto uso del \gloman{software}.

\subsection{Scopo del prodotto}
Lo scopo del capitolato \textit{C7} è la creazione di un algoritmo \gloman{solido} ed efficiente in grado di garantire il salvataggio e la sincronizzazione dei cambiamenti presenti in \gloman{cloud}. È inoltre richiesta la creazione di una interfaccia grafica multipiattaforma. Il progetto deve dunque funzionare sui principali sistemi operativi desktop quali Windows, macOS e Linux senza richiedere l'installazione manuale di ulteriori prodotti per il corretto funzionamento. 
\subsection{Obiettivo del prodotto}
L'obiettivo del progetto è la realizzazione di un modulo per la piattaforma \gloman{Zimbra}, che abbia le caratteristiche sopra descritte, e che si appoggi per il suo funzionamento al \gloman{cloud} \gloman{Zextras Drive}.
Nello specifico, lo scopo finale del prodotto è quello di fornire agli utenti un servizio che automatizzi il salvataggio e la sincronizzazione con il \gloman{cloud}.
\subsection{Glossario}
Nel documento sono presenti termini che possono presentare dei significati ambigui o poco chiari a seconda del contesto.
Per evitare incomprensioni il gruppo fornisce un glossario individuabile nell'appendice \S{}\ref{sec:Glossario} contenente i termini e la loro spiegazione.\newline{}
Nel documento corrente si è deciso di segnare tutte le parole presenti nel \G{} con una "G" a pedice di ogni parola.