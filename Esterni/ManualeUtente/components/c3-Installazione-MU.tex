\section{Installazione ed avvio di SSD}
Per installare ed avviare l'applicazione \gloman{SSD} bisogna effettuare passi diversi in base al sistema operativo. Le varie versioni possono essere scaricate dal link: \newline{}
\centerline{\url{https://github.com/MercurySeven/project-SSD}}

\subsection{Windows}
\begin{itemize}
\item \textbf{Scaricare:} cliccare sull'apposito link e scaricare il file  denominato \textit{NOME FILE WINDOWS};
\item \textbf{Avviare l'applicazione:} cliccare sull'applicazione appena scaricata.
\end{itemize}

\subsection{Linux}
\begin{itemize}
\item \textbf{Scaricare:} cliccare sull'apposito link e scaricare il file  denominato \textit{NOME FILE LINUX};
\item \textbf{Avviare l'applicazione:} cliccare sull'applicazione appena scaricata.
\end{itemize}

\subsection{MacOS}
\begin{itemize}
\item \textbf{Scaricare:} cliccare sull'apposito link e scaricare il file  denominato \textit{ssdMacOS.zip};
\item \textbf{Avviare l'applicazione:} decomprimere il file, aprire la cartella e cliccare il file \textit{"launch.command"}, premere \textit{consenti} a ogni finestra pop-up che si apre e apparirà la schermata di login (\S{}\ref{sec:autenticazione}).
\end{itemize}

