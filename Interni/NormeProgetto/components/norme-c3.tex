\section{Processi di supporto}


\subsection{Processo di documentazione}

\subsubsection{Scopo}

\subsubsection{Descrizione}

\subsubsection{Aspettative}

\subsubsection{Ciclo di vita di un documento}

\subsubsection{Template in formato \LaTeX}

\subsubsection{Documenti prodotti}

\subsubsection{Directory di un documento}

\subsubsection{Struttura dei documenti}

\paragraph{Attribuzione del nome}

\begin{itemize}

	\item La denominazione dei documenti formali sarà la seguente \newline
	\textbf{[NomeDocumento]\_v[X].[Y].[Z]}
	\begin{itemize}
		\item \textbf{NomeDocumento} è il nome ufficiale del documento, senza spazi tra le parole e con solamente le iniziali di ogni parola in maiuscolo (convensione PascalCase);
		\item \textbf{v[X].[Y].[Z]} con \textbf{v} che abbrevia la parola versione, \textbf{[X]} indica il numero incrementale di revisione del responsabile di progetto, \textbf{[Y]} indica il numero incrementale della verifica, \textbf{[Z]} indica il numero di modifiche apportate al documento, che riparte da 0 dopo ogni verifica ed approvazione.
	\end{itemize}
	\item La denominazione dei verbali sarà la seguente\newline
	\textbf{Verbale[Interno/Esterno]\_[YYYY]\_[MM]\_[DD]}\newline
	dove
	\begin{itemize}
		\item \textbf{Verbale} indica il tipo di documento;
		\item \textbf{Interno/Esterno]} indica se il verbale in questione è interno od esterno;
		\item \textbf{[YYYY]-[MM]-[DD]}  indica la data della riunione, \textbf{[YYYY]} indica l'anno, \textbf{[MM]} indica il mese, \textbf{[DD]} indica il giorno.
	\end{itemize}
		

\end{itemize}

\paragraph{Frontespizio}

La prima pagina di ogni documento sarà strutturata nel seguente modo:

\begin{itemize}

	\item \textbf{Logo del gruppo}: posto in basso al centro;
	\item \textbf{Titolo:} posto in alto al centro;
	\item \textbf{Nome del gruppo e del progetto:} posti subito sotto il titolo, al centro;
	\item \textbf{Recapito:} Indirizzo di posta elettronica del gruppo, posizionato sotto il nome del gruppo
	\item \textbf{Informazioni sul documento,} includono:
	
	\begin{itemize}
		\item \textbf{Versione:} indica la versione del documento;
		\item \textbf{Uso:} indica se il documento sarà ad uso interno od esterno;
		\item \textbf{Distribuzione:} indica i destinatari del documento;
		\item \textbf{Redazione:} nome e cognome degli incaricati della redazione del documento;
		\item \textbf{Verifica:} nome e cognome degli incaricati della verifica del documento;
		\item \textbf{Data di approvazione:}
		\item \textbf{Responsabile:}
		\item \textbf{Descrizione:} sintetica descrizione del contenuto del documento;
	\end{itemize}

\end{itemize}

\paragraph{Registro delle modifiche}

\paragraph{Indici}

\paragraph{Corpo del documento}

\paragraph{Verbali}

\subsubsection{Norme tipografiche}

\paragraph{Convenzioni}





\subsection{Processo di configurazione}

\subsection{Processo di verifica}

\subsection{Processo di validazione}