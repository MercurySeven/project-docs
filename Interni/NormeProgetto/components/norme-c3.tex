\section{Processi di supporto}


\subsection{Processo di documentazione}

\subsubsection{Scopo}

L'obiettivo proposto dalla sezione è quello di prendere nota delle norme che andranno a regolare la stesura della documentazione durante il ciclo di vita del software,  così da ottenere documenti formalmente concordi.

\subsubsection{Descrizione}

La sezione contiene le norme tramite le quali avviene una corretta stesura dei documenti,  comprendente verifica e approvazione.  I membri del gruppo sono tenuti ad attenersi alla suddetta sezione.

\subsubsection{Aspettative}

Le aspettative riguardanti il processo in questione sono:
\begin{itemize}
	\item l'identificazione di una struttura ben organizzata comune ai prodotti del processo in questione;
	\item la raccolta di norme da rispettare durante la stesura dei documenti.
\end{itemize}

\subsubsection{Ciclo di vita di un documento}

Ogni documento redatto passa per le seguenti fasi:
\begin{itemize}
	\item \textbf{Creazione} del documento tramite template;
	\item \textbf{Strutturazione} del documento tramite registro delle modifiche,  indice di contenuti e/o indice di figure e tabelle;
	\item \textbf{Stesura} del corpo del documento,  scritto da più membri del gruppo;
	\item \textbf{Revisione} del documento, gestita da un \textit{Verificatore} che,  regolarmente,  riesamina le sezioni del suddetto;
	\item \textbf{Approvazione} del documento, stabilita dal \textit{Responsabile di Progetto} quando la revisione risulta positiva; consente il rilascio del documento.
\end{itemize}

\subsubsection{Template in formato \LaTeX}

La stesura viene svolta utilizzando il linguaggio \LaTeX . La definizione di un template aiuta a rendere automatico l'adottare delle norme decise, oltre che a rendere più semplice la modifica delle suddette e la loro applicazione a tutti i documenti redatti.

\subsubsection{Documenti prodotti}
I documenti prodotti saranno di due tipi:
\begin{itemize}
	\item \textbf{Formali}: sono caratterizzati da storico delle versioni prodotte e da approvazione della versione definitiva da parte del \textit{Responsabile di Progetto}. Possono essere:
	\begin{itemize}
		\item \textbf{interni}: riguardanti le dinamiche del gruppo;
		\item \textbf{esterni}: di interesse ai committenti e al proponente, vengono consegnati nell'ultima versione approvata.
	\end{itemize} 
In particolare, i documenti formali prodotti saranno:
	\begin{itemize}
		\item \textbf{Norme di Progetto}: contiene tutte le regole stabilite dai membri alle quali attenersi durante la durata del progetto; \textit{documento interno};
	\item \textbf{Piano di Progetto}: mostra la pianificazione di tutte le attività previste, comprendente il preventivo delle spese e una previsione dell'impegno in ore per ogni membro del gruppo; \textit{documento esterno};
	\item \textbf{Piano di Qualifica}: descrive i criteri di valutazione della qualità impiegate dal gruppo; \textit{documento esterno};
	\item \textbf{Studio di Fattibilità}: analisi dei capitolati messi a disposizione, evidenziandone pregi e difetti. Contiene,  inoltre,  il capitolato scelto dal gruppo; \textit{documento interno};
	\item \textbf{Analisi dei requisiti}: descrive i requisiti che il prodotto dovrà possedere per essere in linea con le richieste dei committenti; \textit{documento esterno};
	\item \textbf{Glossario}: elenco di tutti i termini presenti nella documentazione che, secondo i membri, necessitano una definizione al fine di chiarirne il significato e rimuovere eventuali ambivalenze; \textit{documento esterno};
	\end{itemize}
\item \textbf{Informali}: sono documenti non soggetti a versionamento o che non sono ancora stati approvati dal \textit{Responsabile di Progetto}. \newline 
In particolare, i documenti informali prodotti saranno:
	\begin{itemize}
	
		\item \textbf{Verbali interni}: contengono le informazioni e le decisioni prese durante gli incontri tra i membri del gruppo; \textit{documento interno};
		\item \textbf{Verbali esterni}: comprendono le informazioni e i chiarimenti dati durante gli incontri tra i membri e un committente; \textit{documento esterno};
	\end{itemize}
\end{itemize}

\subsubsection{Directory di un documento}

Le directory prendono il nome dal documento contenuto, nella forma \textbf{NomeDelDocumento}.  Questa viene poi collocata,  a seconda del tipo di contenuto,  in una delle directory \textbf{DocumentiInterni} o \textbf{DocumentiEsterni}. 

\subsubsection{Struttura dei documenti}

\paragraph{Attribuzione del nome}
\begin{itemize}

	\item La denominazione dei documenti formali sarà la seguente \newline
	\textbf{[NomeDocumento]\_v[X].[Y].[Z]}
	\begin{itemize}
		\item \textbf{NomeDocumento} è il nome ufficiale del documento, senza spazi tra le parole e con solamente le iniziali di ogni parola in maiuscolo (convensione PascalCase);
		\item \textbf{v[X].[Y].[Z]} con \textbf{v} che abbrevia la parola versione, \textbf{[X]} indica il numero incrementale di revisione del responsabile di progetto, \textbf{[Y]} indica il numero incrementale della verifica, \textbf{[Z]} indica il numero di modifiche apportate al documento, che riparte da 0 dopo ogni verifica ed approvazione.
	\end{itemize}
	\item La denominazione dei verbali sarà la seguente\newline
	\textbf{Verbale[Interno/Esterno]\_[YYYY]\_[MM]\_[DD]}\newline
	dove
	\begin{itemize}
		\item \textbf{Verbale} indica il tipo di documento;
		\item \textbf{Interno/Esterno]} indica se il verbale in questione è interno od esterno;
		\item \textbf{[YYYY]-[MM]-[DD]}  indica la data della riunione, \textbf{[YYYY]} indica l'anno, \textbf{[MM]} indica il mese, \textbf{[DD]} indica il giorno.
	\end{itemize}
		

\end{itemize}

\paragraph{Frontespizio}    

La prima pagina di ogni documento sarà strutturata nel seguente modo:

\begin{itemize}

	\item \textbf{Logo del gruppo}: posto in basso al centro;
	\item \textbf{Titolo:} posto in alto al centro;
	\item \textbf{Nome del gruppo e del progetto:} posti subito sotto il titolo, al centro;
	\item \textbf{Recapito:} Indirizzo di posta elettronica del gruppo, posizionato sotto il nome del gruppo
	\item \textbf{Informazioni sul documento,} includono:
	
	\begin{itemize}
		\item \textbf{Versione:} indica la versione del documento;
		\item \textbf{Uso:} indica se il documento sarà ad uso interno od esterno;
		\item \textbf{Distribuzione:} indica i destinatari del documento;
		\item \textbf{Redazione:} nome e cognome degli incaricati della redazione del documento;
		\item \textbf{Verifica:} nome e cognome degli incaricati della verifica del documento;
		\item \textbf{Data di approvazione:}
		\item \textbf{Responsabile:}
		\item \textbf{Descrizione:} sintetica descrizione del contenuto del documento;
	\end{itemize}

\end{itemize}

\paragraph{Registro delle modifiche}      

Ogni documento presenta subito dopo il frontespizio il registro modifiche, esso è una tabella che traccia tutte le modifiche significative apportate al documento durante il suo ciclo di vita. Ogni voce della tabella riporta:

\begin{itemize}
	\item \textbf{Versione:} numero versione documento dopo la modifica;
	\item \textbf{Data:} data in cui è stata effettuata la modifica;
	\item \textbf{Autore:} nome e cogmome dell'autore della modifica;
	\item \textbf{Ruolo:} Ruolo dell'autore che ha apportato la modifica;
	\item \textbf{Descrizione:} sintentica descrizione delle modifiche apportate;
\end{itemize}

\paragraph{Indici}     

In ogni documento è presente un indice che fornisce una visione sintetica della struttura del documento. L'indice è strutturato in sezioni, sottosezioni e paragrafi aventi numeri progressivi seperati tra loro da un punto.\newline
Dove necessario,saranno presenti anche un indice delle illustrazioni ed uno delle tabelle presenti.

\paragraph{Corpo del documento}       

Ogni pagina del corpo del docomento è strutturata nel seguente modo:

\begin{itemize}

	\item L'intestazione, composta nel seguente modo:
	
	\begin{itemize}

		\item In alto a sinistra si trova una miniatura a colori del logo del gruppo
	
		\item In alto a destra viene indicato la sezione corrente
	
		\item Sotto i primi due elementi elencati è presente una 	linea nera separatrice

	\end{itemize}
		
	\item il contenuto della pagina si trova sotto l'intestazione
	
	\item Il piè di pagina, composto nel seguente modo:
	
	\begin{itemize}
	
		\item Sotto il contenuto della pagina è presenta una linea nera separatrice
	
		\item Sotto la linea a sinistra è presente il titolo del documento ed il nome del gruppo
		
		\item Sotto la linea a destra è presente il numero della pagina corrente e quello delle pagine totali
		
	\end{itemize}

\end{itemize}

\paragraph{Verbali}      

I verbali, sia interni che esterni, presentano una struttura fissa e sono soggetti alle stesse norme strutturali degli altri argomenti, ad eccezione del fatto che, essendo informali, non sono soggetti a versionamento.
La struttura sarà quindi la seguente:

\begin{itemize}

	\item \textbf{Canale di comunicazione:} Riporta il software utilizzato per effettuare l'incontro;
	\item \textbf{Data:} Riporta la data della riunione;
	\item \textbf{Ora inizio:} Riporta l'orario di inizio della riunione;
	\item \textbf{Ora fine:} Riporta l'orario di fine della riunione;
	\item \textbf{Segretario:} 
	\item \textbf{Partecipanti:} Contiene l'elenco dei presenti all'incontro;   
	\item \textbf{Ordine del giorno:} Contiene l'elenco degli argomenti trattati durante la riunione;
	\item \textbf{Resoconto:} Riporta l'esito della discussione dei singoli punti inseriti nell'ordine del giorno; per i verbali esterni, questa sezione è organizzata secondo il pattern domanda-risposta.
	\item \textbf{Riepilogo delle decisioni:}  Contiene una tabella che riassume tutte le decisioni avvenute durante la riunione, numerandole con un codice univoco del tipo: \newline 
	\centerline{\textbf{V[Interno/Esterno]\_[YYYY]\_[MM]\_[DD].[X]}}\newline con \textbf{X} indicante il numero progressivo della decisione, a partire da 1.

\end{itemize}


\subsubsection{Norme tipografiche}

\paragraph{Convenzioni di denominazione}   

I nomi delle cartelle e dei file relativi ai documenti adottano la convenzione \textit{Pascal Case}, ossia il nome del file presenta l'iniziale di ogni parola che lo compone in maiuscolo, senza nessun separatore; per file e cartelle legati alla struttura del documento i nomi non contengono caratteri maiuscoli.

\paragraph{Stili di testo}

Gli stili di testo adottati nei documenti sono: 

\begin{itemize}

	\item \textbf{Grassetto:} Viene utilizzato nei titoli delle sezioni e dei paragrafi, per enfatizzare parole, per indicare gli elementi di un elenco puntato;
	\item \textbf{Corsivo:} Viene applicato per i nomi propri (per la precisione i membri del gruppo, il proponente, i committenti), Viene inoltre utilizzato per citare il nome di un documento, privo della sua versione, e nei termini che fanno parte del glossario;
	\item \textbf{Monospace:} per gli snippet di codice;
	\item \textbf{Maiuscolo:} verrà usato negli acronimi e, nel caso di nomi di documenti, per le lettere iniziali delle parole che li compongono secondo la convenzione \textit{PascalCase};

\end{itemize}

\paragraph{Glossario}  

TODOOOOOOOOOOOOOOOOOOOOOOOOOOOOOOOOOOOOOOOOOOOOOOOOOOOOOOOOOOOOOOOOOOOOOO

\paragraph{Elenchi puntati}

Gli elenchi puntati al primo livello di annidazione vengono scanditi con • (pallino); al livello successivo di annidamento il simbolo usato è -- (doppio trattino); all'ultimo livello è .(punto fermo).\newline
Ogni voce contenuta all'interno degli elenchi puntati inizia con la lettera maiuscola e termina con ;(punto e virgola), tranne l'ultima voce, la quale termina con .(punto fermo).\newline
Le voci che corrispondono allo schema NomeElemento-descrizione presentano il primo termine in grassetto, il secondo invece nel formato normale. 

\paragraph{Formato di data e formato di ora}

Le date vengono indicato usato il formato stabilito dallo standard ISO 8601:\newline
\centerline{\textbf{[YYYY]\-[MM]\-[DD]}}
\newline
dove:

\begin{itemize}
	\item \textbf{[YYYY]} indica l'anno;
	\item \textbf{[MM]} indica il mese;
	\item \textbf{[GG]} indica il giorno.
\end{itemize}



\paragraph{Sigle}
\begin{itemize}
	\item Le sigle relative ai nomi dei documenti sono:
	\begin{itemize}

		\item \textbf{AdR}: \textit{Analisi dei Requisiti;}
	
		\item \textbf{SdF}: \textit{Studio di Fattibilità;}
	
		\item \textbf{NdP}: \textit{Norme di Progetto;}
	
		\item \textbf{PdP}: \textit{Piano di Progetto;}
	
		\item \textbf{PdQ}: \textit{Piano di Qualifica;}
	
		\item \textbf{G}: \textit{Glossario;}
	
		\item \textbf{VE}: \textit{Verbale Esterno;}
	
		\item \textbf{VI}: \textit{Verbale Interno.}
	
	\end{itemize}
	
	\item Le sigle relative ai ruoli di progetto sono:
	
	\begin{itemize}
		
		\item \textbf{RE}: \textit{Responsabile;}
		
		\item \textbf{AM}: \textit{Amministratore}
		
		\item \textbf{AN}: \textit{Analista}
		
		\item \textbf{PT}: \textit{Progettista;}
		
		\item \textbf{PT}: \textit{Programmatore;}
		
		\item \textbf{VE}: \textit{Verificatore.}
	
	\end{itemize}
	
	\item Le sigle relative alle revisioni di progetto sono:
	
	\begin{itemize}
	
		\item \textbf{RR}:\textit{Revisione dei requisiti;} 	
		
		\item \textbf{RP}:\textit{Revisione di progettazione;} 
		
		\item \textbf{RQ}:\textit{Revisione di qualifica;} 
		
		\item \textbf{RA}:\textit{Revisione di accettazione.} 
	
	\end{itemize}
	
\end{itemize}

\subsubsection{Elementi grafici}

Questa sezione definisce le norme per l'uso di elementi grafici quali immagini, tabelle e diagrammi.

\begin{itemize}

\item \textbf{Immagini:} Le figure saranno centrate ed avranno una opportuna didascalia;

\item \textbf{Grafici UML:} I grafici in linguaggio UML saranno inseriti nel documento come immagini;

\item \textbf{Tabelle:} Ogni tabella è contrassegnata da una didascalia e numerata in modo progressivo, a questa prassi fanno eccezione le tabelle del registro delle modifiche, che non presentano una didascalia.


\end{itemize}

\subsubsection{Metriche}

\subsubsection{Strumenti di stesura}

Gli strumenti usati per la stesura dei documenti sono il linguaggio \LaTeX, l'editor \textit{Texmaker} e \textit{filler}.

\paragraph{\LaTeX}

Per la stesura dei documenti è stato scelto \LaTeX, un linguaggio compilato che permette un facile versionamento, la produzione coerente, ordinatata, templetizzata e collaborativa dei documenti

\paragraph{Texmaker}

L'editor scelto per la stesura dei documenti è \textit{Texmaker}, adottato da tutti i membri del gruppo. \textit{Texmaker} è un editor intuitivo che permette una facile compilazione e visualizzazione dei documenti.\newline
https://www.xm1math.net/texmaker/

\subsection{Processo di configurazione}

\subsection{Processo di verifica}

\subsection{Processo di validazione}
