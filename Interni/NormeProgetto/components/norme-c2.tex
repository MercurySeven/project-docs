\section{Processi primari}

\subsection{Fornitura}

\subsubsection{Scopo}

La fornitura rappresenta ciò che il fornitore svolge al fine di capire e soddisfare appieno quello che viene richiesto dal proponente.  In particolare,  il fornitore svolge un'analisi approfondita,  alla quale consegue la stesura dello \textit{Studio di Fattibilità},  in modo da determinare criticità e rischi legati alla richiesta d'appalto.  Inoltre,  è necessario che venga definito un accordo sotto forma di contratto,  così da regolarizzare la corrispondenza col proponente e tutto ciò che è legato col prodotto finale,  tra cui consegna e manutenzione.  A questo punto,  si passa al determinare le procedure e le risorse necessarie,  sviluppandole nel \textit{Piano di Progetto}.  Quest'ultimo pone le basi per la realizzazione del prodotto, fino alla consegna finale.  
Il processo di fornitura è composto dalle seguenti fasi:
\begin{itemize}

\item avvio;

\item approntamento di risposte alle richieste;

\item contrattazione;

\item pianificazione;

\item esecuzione e controllo;

\item revisione e valutazione;

\item consegna e completamento.

\end{itemize}

\subsubsection{Descrizione}

Il documento include le norme che i membri del gruppo \textit{Mercury Seven } devono fare riferimento al fine di ottenere il ruolo di fornitori del proponente NOMEPROPONENTE e dei committenti \textit{Prof. Tullio Vardanega} e \textit{Prof.  Riccardo Cardin}.  I membri del gruppo sono tenuti ad attenersi a quanto scritto nella suddetta sezione durante tutte le fasi di sviluppo del prodotto NOMEPRODOTTO.

\subsubsection{Aspettative}

Durante lo svolgimento del progetto il gruppo intende mantenere un dialogo costante con il proponente,  in modo da instaurare un rapporto di collaborazione e ricevere un riscontro proficuo sul lavoro svolto.  In particolare,  si vuole:
\begin{itemize}

\item determinare gli aspetti chiave che il prodotto dovrà soddisfare;

\item definire vincoli e requisiti riguardanti i processi;

\item stimare costi e tempistiche di lavoro;

\item accordarsi sulla qualifica del prodotto.

\end{itemize}


\subsubsection{Studio di Fattibilità}

Il \textit{Responsabile di Progetto} ha il compito di organizzare riunioni tra i membri in modo da promuovere lo lo scambio di opinioni sui capitolati proposti.  Una volta che viene trovato un capitolato di interesse comune,  gli \textit{Analisti} provvedono alla stesura dello \textit{Studio di Fattibilità} che,  per ogni capitolato,  indica:

\begin{itemize}

\item \textbf{Informazioni generali}: vengono elencate le informazioni di base,  quali nome del progetto,  il proponente e il committente;

\item \textbf{Descrizione e finalità del capitolato}: sintesi del prodotto, comprendente le caratteristiche principali che vengono richieste e l'obiettivo da raggiungere;

\item \textbf{Tecnologie interessate}: elenco delle tecnologie richieste per lo svolgimento del prodotto;

\item \textbf{Aspetti positivi,  criticità e fattori di rischio}: esposizione delle considerazioni fatte dal gruppo riguardanti gli aspetti positivi e sulle criticità del capitolato in questione;

\item \textbf{Conclusioni}: esposizione delle ragioni per le quali il gruppo accetta o rifiuta il capitolato.

\end{itemize}

\subsubsection{Altra documentazione fornita}

In questa sezione vengono elencati i documenti che sono stati forniti all'azienda NOMEAZIENDA e ai committenti \textit{Prof. Tullio Vardanega} e \textit{Prof.  Riccardo Cardin}.  La suddetta documentazione risulta essenziale in modo da tracciare le attività  svolte dal gruppo.
I documenti forniti riguardano: 
\begin{itemize}

\item \textbf{Analisi}: contenuta nell' \textit{Analisi dei requisiti},  documento che esamina tutti i requisiti e i casi d'uso individuati;
\item \textbf{Pianificazione e consegna}: nel documento \textit{Piano di Progetto} viene presentato un preventivo riguardante tempistiche,  l' analisi dei rischi, la pianificazione delle attività e il consuntivo;
\item \textbf{Verifica,  validazione e qualità}: all'interno del \textit{Piano di Qualifica} vengono descritte le metodologie utilizzate per valutare e garantire la qualità del prodotto.

\end{itemize}

\subsubsection{Strumenti}

\paragraph{Gantt Project}

I diagrammi di Gantt sono uno strumento molto versatile per visualizzare e tracciare tempistiche e avanzamento del lavoro.  Sono utili al \textit{Responsabile di Progetto} per tenere traccia di tutte attività, tra le quali l'assegnazione di risorse,  la gestione del budget e l'analisi dei carichi di lavoro, e le dipendenze tra esse.  Per la realizzazione dei suddetti diagrammi, viene utilizzato \textit{Gantt Project}.

\subsection{Sviluppo}

\subsubsection{Scopo}

Il processo di Sviluppo ha come scopo il descrivere attività e compiti da svolgere,  in modo da realizzare il prodotto finale richiesto dal proponente.

\subsubsection{Descrizione}

Il processo di Sviluppo è caratterizzato da varie attività,  quali:
\begin{itemize}

\item \AdR;

\item Progettazione;

\item Codifica.

\end{itemize}

\subsubsection{Aspettative}

Le aspettative sono le seguenti:
\begin{itemize}

\item fissare gli obiettivi di sviluppo;

\item fissare i vincoli tecnologici;

\item fissare i vincoli di design;

\item realizzare un prodotto finale che,  oltre a superare i test,  soddisfi i requisiti e le richieste del proponente.

\end{itemize}

\subsubsection{Analisi dei Requisiti}

\paragraph{Scopo}

Gli \textit{Analisti} hanno il compito di redigere il documento di \AdR, individuando tutti i requisiti che il proponente richiede per la realizzazione del prodotto.  Lo scopo,  quindi,  è quello di:
\begin{itemize}

\item definire lo scopo del lavoro;

\item fornire ai \textit{Progettisti} riferimenti precisi e affidabili;

\item fissare le funzionalità e i requisiti concordati col cliente;

\item fornire una base per raffinamenti successivi al fine di garantire un miglioramento continuo del prodotto e del processo di sviluppo;

\item fornire ai \textit{Verificatori} riferimenti per l'attività di controllo dei test;

\item tracciare riferimenti sulla mole di lavoro in modo da avere una stima dei costi.

\end{itemize}

\paragraph{Descrizione}

I requisiti vengono ricavati da diverse fonti,  tra cui:
\begin{itemize}

\item \textbf{Capitolati d'Appalto}: grazie alla lettura approfondita del capitolato scelto vengono individuati dei requisiti;

\item \textbf{Casi d'uso}: requisiti estrapolati da uno o più casi d'uso; 

\item \textbf{Verbali}: i requisiti possono emergere sia dalle riunioni interne effettuate dagli \textit{Analisti},  sia da contatti o incontri con i rappresentanti dell'azienda proponente.

\end{itemize}

\paragraph{Aspettative}

L'obiettivo dell'attività di analisi è la creazione della documentazione formale contenente tutti i requisiti richiesti dal proponente.

\paragraph{Struttura}

La struttura che contraddistingue l'\AdR  è:
\begin{itemize}

\item \textbf{Descrizione}: contiene informazioni riguardanti il prodotto,  le componenti presenti all'interno del sistema,  la piattaforma d'esecuzione e i vincoli di progettazione individuati;

\item \textbf{Casi d'uso}: vengono identificati gli attori che interagiscono con le componenti del sistema e le interazioni tra sistema,  attori ed elementi esterni.  Approfondimento nel paragrafo §2.2.4.6;
\item \textbf{Requisiti}: rappresentati in una tabella che contiene:
 \begin{itemize}
 
 \item tracciamento dei requisiti,  suddivisi in obbligatori,  desiderabili e opzionali,  individuati nel sistema durante l'analisi; approfondimento nel paragrafo §2.2.4.5;
 
 \item tracciamento fonte-requisiti e requisiti-fonte.
 
 \end{itemize}

\end{itemize}

\paragraph{Classificazione dei requisiti}

//// DA FARE

\paragraph{Classificazione dei casi d'uso}

//// DA FARE

\paragraph{Metriche}

//// DA FARE

\subsubsection{Progettazione}

\paragraph{Scopo}

L'attività di progettazione definisce, in funzione dei requisiti specificati nel documento \AdR, le caratteristiche che il prodotto richiesto deve avere in modo di fornire la soluzione migliore che soddisfi i requisiti degli stakeholders.  Questa fase provvede a:
\begin{itemize}

\item garantire la qualità del prodotto applicando un approccio sistematico ai problemi;

\item suddividere compiti implementativi,  così da provvedere alla diminuzione della complessità del problema originale e rendere più semplice la codifica da parte dei \textit{Programmatori};

\item ottimizzare l'utilizzo delle risorse.

\end{itemize}

\paragraph{Descrizione}

Le parti principali sono due:
\begin{itemize}

\item \textbf{Technology baseline}: contiene le specifiche della progettazione ad alto livello del prodotto e delle sue componenti,  l'elenco dei diagrammi UML che saranno utilizzati per la realizzazione dell'architettura e i test di verifica;
\item \textbf{Product baseline}: rende ancora più dettagliata l'attività di progettazione,  integrando ciò che è riportato nella Technology baseline.  Inoltre,  definisce i test necessari alla verifica.

\end{itemize}

\paragraph{Aspettative}

Il compito di ogni \textit{Progettista} è quello di definire l'architettura logica del sistema.  Quest'ultima,  in particolare,  dovrà avere le seguenti caratteristiche:
\begin{itemize}

\item soddisfare i requisiti segnati nel documento \AdR;

\item adattarsi a ogni modifica effettuata nell' \AdR;

\item essere comprensibile,  modulare e robusta;

\item risultare affidabile,  quindi svolgere i propri compiti ogni qual volta venga richiesto ;

\item rispettare le caratteristiche di safety e security rispetto ai malfunzionamenti;

\item essere disponibile;

\item mostrare un utilizzo efficiente delle risorse necessarie;

\item garantire riusabilità;

\item presentare componenti semplici e coese nel raggiungimento degli obiettivi,  incapsulate e con un basso livello di accoppiamento.

\end{itemize}

\paragraph{Design Patterns}

////DA FARE

\paragraph{Diagrammi UML 2.0}

L'utilizzo dei diagrammi UML 2.0 rende le scelte progettuali adottate più comprensibili e riduce ambivalenze che possono crearsi.  Tra i possibili diagrammi utilizzabili,  quelli che verranno principalmente utilizzati sono:
\begin{itemize}

\item \textbf{Diagrammi delle attività}: descrivono tutte le operazioni di una certa attività,  anche fuori dal contesto software;

\item \textbf{Diagrammi delle classi}: mostrano attributi e metodi di classi e tipi;

\item \textbf{Diagrammi dei package}: mostrano raggruppamenti tra le classi;

\item \textbf{Diagrammi di sequenza}: descrivono sequenze di azioni usando scelte definite.

\end{itemize}

\paragraph{Test}

//// DA FARE


\subsubsection{Codifica}

\paragraph{Scopo}

La codifica ha come scopo quello di normare l'effettiva realizzazione del prodotto software richiesto.  In questa fase,  di fatto, si concretizza l'architettura di alto livello pensata dai \textit{Progettisti}.  I \textit{Programmatori} dovranno attenersi a queste norme durante la fase di programmazione e implementazione.

\paragraph{Descrizione}

La scrittura del codice dovrà rispettare quanto stabilito nella documentazione di prodotto.  Dovrà perseguire gli obiettivi di qualità definiti all'interno del documento \PdQ  per poter garantire una buona qualità del codice.

\paragraph{Aspettative}

l'Obiettivo dell'attività di codifica è la creazione di un prodotto software conforme alle richieste prefissate con il proponente.  L'uso di norme e convenzioni in questa fase è fondamentale per permettere la generazione di codice leggibile e uniforme,  agevolare le fasi di manutenzione,  verifica e validazione e migliorare la qualità del prodotto.


\subsubsection{Metriche}

//// DA FARE

\paragraph{•}

\subsubsection{Strumenti}

//// DA FARE