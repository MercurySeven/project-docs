\section{Processi primari}
\subsection{Fornitura}
\subsubsection{Scopo}
La fornitura è il processo che il fornitore utilizza al fine di capire e soddisfare appieno quello che viene richiesto dal proponente.  In particolare,  il fornitore svolge un'analisi approfondita,  alla quale segue la stesura dello \SdF{}, in modo da determinare criticità e rischi legati alla richiesta d'appalto. Inoltre, è necessario che venga definito un accordo sotto forma di contratto, così da formalizzare la corrispondenza con il proponente e tutto ciò che è legato col prodotto finale, tra cui consegna e manutenzione. Vengono poi determinate e formalizzate nel \PdP{} le risorse e le procedure necessarie. Quest'ultimo pone le basi per la realizzazione del prodotto, fino alla consegna finale.\newline 
Il processo di fornitura è composto dalle seguenti fasi:
\begin{itemize}
	\item avvio;
	\item approntamento di risposte alle richieste;
	\item contrattazione;
	\item pianificazione;
	\item esecuzione e controllo;
	\item revisione e valutazione;
	\item consegna e completamento.
\end{itemize}
\subsubsection{Descrizione}
Il documento include le norme a cui i membri del gruppo \Gruppo{} devono fare riferimento al fine di ottenere il ruolo di fornitori del proponente \proponente{} e dei committenti \textit{Prof. Tullio Vardanega} e \textit{Prof. Riccardo Cardin}. I membri del gruppo sono tenuti ad attenersi a quanto scritto nella suddetta sezione durante tutte le fasi di sviluppo del prodotto \progetto{}.
\subsubsection{Aspettative}
Durante lo svolgimento del progetto il gruppo intende mantenere un dialogo costante con il proponente, in modo da instaurare un rapporto di collaborazione e ricevere un riscontro proficuo sul lavoro svolto.  In particolare, si vuole:
\begin{itemize}
	\item determinare gli aspetti chiave che il prodotto dovrà soddisfare;
	\item definire vincoli e requisiti riguardanti i processi;
	\item stimare costi e tempistiche di lavoro;
	\item accordarsi sulla qualifica del prodotto.
\end{itemize}
\subsubsection{Studio di fattibilità}
Il \RdP{} ha il compito di organizzare riunioni tra i membri in modo da promuovere lo scambio di opinioni sui capitolati proposti. Una volta identificato un capitolato di interesse comune, gli \anas{} provvedono alla stesura dello \SdF{} che, per ogni capitolato, indica:
\begin{itemize}
	\item \textbf{Informazioni generali}: elenco delle informazioni di base, quali nome del progetto, il proponente e il committente;
	\item \textbf{Descrizione e finalità del capitolato}: sintesi del prodotto, comprendente le caratteristiche principali che vengono richieste e l'obiettivo da raggiungere;
	\item \textbf{Tecnologie interessate}: elenco delle tecnologie richieste per lo svolgimento del prodotto;
	\item \textbf{Aspetti positivi, criticità e fattori di rischio}: esposizione delle considerazioni fatte dal gruppo riguardanti gli aspetti positivi e sulle criticità del capitolato in questione;
	\item \textbf{Conclusioni}: esposizione delle ragioni per le quali il gruppo accetta o rifiuta il capitolato.
\end{itemize}
\subsubsection{Altra documentazione fornita}
In questa sezione vengono elencati i documenti che sono stati forniti all'azienda \proponente{} e ai committenti \textit{Prof. Tullio Vardanega} e \textit{Prof. Riccardo Cardin}. La suddetta documentazione risulta essenziale per il tracciamento delle attività svolte dal gruppo.
I documenti forniti riguardano: 
\begin{itemize}
	\item \textbf{Analisi}: contenuta nell' \AdR{}, documento che esamina tutti i requisiti e i casi d'uso individuati;
	\item \textbf{Pianificazione e consegna}: nel documento \PdP{} viene presentato un preventivo riguardante tempistiche, l'analisi dei rischi, la pianificazione delle attività e il consuntivo;
	\item \textbf{Verifica, validazione e qualità}: all'interno del \PdQ{} vengono descritte le metodologie utilizzate per valutare e garantire la qualità del prodotto.
\end{itemize}
\subsubsection{Strumenti}
\paragraph{Gantt project}
I diagrammi di Gantt sono uno strumento molto versatile per visualizzare e tracciare tempistiche e avanzamento del lavoro. Sono utili al \RdP{} per tenere traccia di tutte le attività, tra le quali l'assegnazione di risorse, la gestione del budget e l'analisi dei carichi di lavoro, e le dipendenze tra esse. Per la realizzazione dei suddetti diagrammi, viene utilizzato \textit{Gantt Project}.
\subsection{Sviluppo}
\subsubsection{Scopo}
Il processo di sviluppo ha come scopo il descrivere attività e compiti da svolgere, in modo da realizzare il prodotto finale richiesto dal proponente.
\subsubsection{Descrizione}
Il processo di sviluppo è caratterizzato da varie attività,  quali:
\begin{itemize}
	\item Analisi dei Requisiti;
	\item progettazione;
	\item codifica.
\end{itemize}
\subsubsection{Aspettative}
Le aspettative sono le seguenti:
\begin{itemize}
	\item fissare gli obiettivi di sviluppo;
	\item fissare i vincoli tecnologici;
	\item fissare i vincoli di design;
	\item realizzare un prodotto finale che,  oltre a superare i test,  soddisfi i requisiti e le richieste del proponente.
\end{itemize}
\subsubsection{Analisi dei requisiti}
\paragraph{Scopo}
Gli \anas{} hanno il compito di redigere il documento di \AdR{}, individuando tutti i requisiti che il proponente richiede per la realizzazione del prodotto.  Lo scopo,  quindi,  è quello di:
\begin{itemize}
	\item definire lo scopo del lavoro;
	\item fornire ai \progs{} riferimenti precisi e affidabili;
	\item fissare le funzionalità e i requisiti concordati col cliente;
	\item fornire una base per raffinamenti successivi al fine di garantire un miglioramento continuo del prodotto e del processo di sviluppo;
	\item fornire ai \vers{} riferimenti per l'attività di controllo dei test;
	\item tracciare riferimenti sulla mole di lavoro in modo da avere una stima dei costi.
\end{itemize}
\paragraph{Descrizione}
I requisiti vengono ricavati da diverse fonti,  tra cui:
\begin{itemize}
	\item \textbf{Capitolati d'Appalto}: grazie alla lettura approfondita del capitolato scelto vengono individuati dei requisiti;
	\item \textbf{Casi d'uso}: requisiti estrapolati da uno o più casi d'uso; 
	\item \textbf{Verbali}: i requisiti possono emergere sia dalle riunioni interne effettuate dagli \anas{}, sia da contatti o incontri con i rappresentanti dell'azienda proponente.
\end{itemize}
\paragraph{Aspettative}
L'obiettivo dell'attività di analisi è la creazione della documentazione formale contenente tutti i requisiti richiesti dal proponente.
\paragraph{Struttura}
La struttura che contraddistingue l'\AdR{} è:
\begin{itemize}
	\item \textbf{Descrizione}: contiene informazioni riguardanti il prodotto, le componenti presenti all'interno del sistema, la piattaforma d'esecuzione e i vincoli di progettazione individuati;
	item \textbf{Casi d'uso}: vengono identificati gli attori che interagiscono con le componenti del sistema e le interazioni tra sistema, attori ed elementi esterni. Approfondimento nel paragrafo §2.2.4.6;
	\item \textbf{Requisiti}: rappresentati in una tabella che contiene:
	\begin{itemize}
 		\item tracciamento dei requisiti, suddivisi in obbligatori, desiderabili e opzionali, individuati nel sistema durante l'analisi; approfondimento nel paragrafo §2.2.4.5;
		\item tracciamento fonte-requisiti e requisiti-fonte. 
	\end{itemize}
\end{itemize}
\paragraph{Classificazione dei requisiti}
La classificazione dei requisiti verrà effettuata mediante la seguente codifica:\newline
\centerline{\textbf{R[Tipo][Priorità][Numero]}}\newline
dove:
\begin{itemize}
	\item \textbf{R:} abbrevia "Requisito";
	\item Tipo: può assumere i seguenti valori:
	\begin{itemize}
		\item \textbf{F:} requisito funzionale, ossia la definizione di una particolare caratteristica che deve essere inclusa nel software;
		\item \textbf{Q:} requisito di qualità, relativi alle regole di qualità del software (efficienza ed efficacia);
		\item \textbf{V:} requisito di vincolo che rappresenta dei vincoli avanzati dal proponente.
	\end{itemize}
	\item \textbf{Priorità:}
	\begin{itemize}
		\item \textbf{O:} requisito obbligatorio;
		\item \textbf{D:} requisito desiderabile;
		\item \textbf{F:} requisito facoltativo.
	\end{itemize}
	\item Numero: codice identificativo dei requisiti.
\end{itemize}
\paragraph{Classificazione dei casi d'uso}
//// DA FARE
\paragraph{Metriche}
//// DA FARE
\subsubsection{Progettazione}
\paragraph{Scopo}
L'attività di progettazione definisce, in funzione dei requisiti specificati nel documento \AdR{}, le caratteristiche che il prodotto richiesto deve avere in modo da fornire la soluzione migliore che soddisfi i requisiti degli \glo{stakeholders}.  Questa fase provvede a:
\begin{itemize}
\item garantire la qualità del prodotto applicando un approccio sistematico ai problemi;
\item suddividere compiti implementativi, così da provvedere alla diminuzione della complessità del problema originale e rendere più semplice la codifica da parte dei \progrs{};
\item ottimizzare l'utilizzo delle risorse.
\end{itemize}
\paragraph{Descrizione}
Le parti principali sono due:
\begin{itemize}
\item \textbf{Technology baseline}: contiene le specifiche della progettazione ad alto livello del prodotto e delle sue componenti, l'elenco dei diagrammi UML che saranno utilizzati per la realizzazione dell'architettura e i test di verifica;
\item \textbf{Product baseline}: rende ancora più dettagliata l'attività di progettazione, integrando ciò che è riportato nella Technology baseline. Inoltre, definisce i test necessari alla verifica.
\end{itemize}
\paragraph{Aspettative}
Il compito di ogni \prog{} è quello di definire l'architettura logica del sistema. Quest'ultima, in particolare, dovrà avere le seguenti caratteristiche:
\begin{itemize}
\item soddisfare i requisiti segnati nel documento \AdR{};
\item adattarsi a ogni modifica effettuata nell' \AdR{};
\item essere comprensibile, modulare e robusta;
\item risultare affidabile, quindi svolgere i propri compiti ogni qual volta venga richiesto;
\item rispettare le caratteristiche di safety e security rispetto ai malfunzionamenti;
\item essere disponibile;
\item mostrare un utilizzo efficiente delle risorse necessarie;
\item garantire riusabilità;
\item presentare componenti semplici e coese nel raggiungimento degli obiettivi, incapsulate e con un basso livello di accoppiamento.
\end{itemize}
\paragraph{Design patterns}
Devono essere descritti i design pattern utilizzati per realizzare l'architettura. Ogni design pattern deve essere accompagnato da una descrizione e da un diagramma, che ne esponga il significato e la struttura;
\paragraph{Diagrammi UML 2.0}
L'utilizzo dei diagrammi UML 2.0 rende le scelte progettuali adottate più comprensibili e riduce ambivalenze che possono crearsi. Tra i possibili diagrammi utilizzabili, quelli che verranno principalmente utilizzati sono:
\begin{itemize}
\item \textbf{Diagrammi delle attività}: descrivono tutte le operazioni di una certa attività, anche fuori dal contesto software;
\item \textbf{Diagrammi delle classi}: mostrano attributi e metodi di classi e tipi;
\item \textbf{Diagrammi dei package}: mostrano raggruppamenti tra le classi;
\item \textbf{Diagrammi di sequenza}: descrivono sequenze di azioni usando scelte definite.
\end{itemize}
\paragraph{Test}
Ogni \prog{} avrà il compito di definire test opportuni, che dovranno essere accompagnati da eventuali classi utili ad individuare errori ed anomalie.
\subsubsection{Codifica}
La codifica ha come scopo quello di normare l'effettiva realizzazione del prodotto software richiesto. In questa fase, di fatto, si concretizza l'architettura di alto livello pensata dai \progs{} nel documento \PdQ{}. I \progrs{} dovranno attenersi a queste norme durante la fase di programmazione e implementazione. L'uso di norme e convenzioni in questa fase è fondamentale per permettere la generazione di codice leggibile e uniforme, agevolare le fasi di manutenzione, verifica e validazione e migliorare la qualità del prodotto.
\paragraph{Convenzioni per i nomi}
\begin{itemize}
	\item ogni nome deve essere univoco;
	\item ogni nome di variabile, metodo, funzione seguirà il modello \glo{Lower Camel Case};
	\item ogni nome di classe seguirà il modello \glo{Pascal Case}.
\end{itemize}
\newpage
\paragraph{Convenzioni sulle pratiche}
\
{
	\setlength{\freewidth}{\dimexpr\textwidth-0\tabcolsep}
	\renewcommand{\arraystretch}{1.5}
	\setlength{\aboverulesep}{0pt}
	\setlength{\belowrulesep}{0pt}
	\rowcolors{2}{AzzurroGruppo!10}{white}
	\begin{longtable}{L{.210\freewidth} L{.6\freewidth} L{.080\freewidth}}
		\toprule 
		\rowcolor{AzzurroGruppo!30}
		\textbf{Pratica} & \textbf{Descrizione norma} \\
		\toprule
		\endhead		
		Indentazione & Contiene tutte le regole stabilite dai membri alle quali attenersi durante l'intera durata del progetto. \\ 
		Spaziature & Contiene la pianificazione di tutte le attività previste, comprendente il preventivo delle spese e una previsione dell'impegno in ore per ogni membro del gruppo.  \\
		Commenti & Ogni commento deve essere separato con uno spazio dalla sua definizione \\ 
		Lunghezza della riga di codice & Ogni riga di codice non deve superare i 140 caratteri di lunghezza. \\
		Brevità dei metodi & Ogni blocco di codice non dovrà superare le 50 istruzioni. Se tuttavia questo risulta indispensabile può essere permesso, per questo la norma è consigliata e non obbligatoria.\\ 	
		Stringhe & Ogni stringa deve essere definita con singoli apici. \\
		Funzioni & Descrive i requisiti che il prodotto dovrà possedere per essere in linea con le richieste dei committenti.\\ 	
		Variabili & Non definire variabili che non verranno utilizzate.\\ 	
		Operatori & Ogni operatore dovrà essere preceduto e seguito da uno spazio.\\ 	
		Complessità ciclomatica & L'annidamento di cicli sarà evitato il più possibile. In questo modo le funzioni non saranno troppo onerose da eseguire e soprattutto godranno di una maggiore leggibilità. \\  			
		\bottomrule
		\hiderowcolors
		\caption{Descrizione norme sulle pratiche di codifica}
	\end{longtable}
}
\subsubsection{Metriche}
\paragraph{MCD02 Facilità di comprensione}
Indice che riporta la facilità con cui l'utente riesce a comprendere cosa fa il codice, rappresentata mediante il numero di linee di commento nel codice.
La formula adottata è:\newline
\centerline{\textbf{R=$\frac{numero\ linee\ commento}{numero\ linee\ codice}$}}
\textit{Il gruppo ritiene prematuro stabilire in questa fase altre metriche e valori soglia. La sezione sarà opportunamente ampliata.}
\subsubsection{Strumenti}
Di seguito vengono presentati gli strumenti che il gruppo ha deciso di adottare all'interno del progetto durante il processo di Sviluppo.
\paragraph{IDE}
\paragraph{StarUML}
Per la generazione di diagrammi UML il gruppo ha deciso di utilizzare StarUML in quanto offre molte agevolazioni che permettono una produzione veloce dei diagrammi e risulta uno strumento di semplice apprendimento e uso.