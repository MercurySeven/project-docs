\section{Processi primari}

\subsection{Fornitura}

\subsubsection{Scopo}

La fornitura rappresenta ciò che il fornitore svolge al fine di capire e soddisfare appieno quello che viene richiesto dal proponente.  In particolare,  il fornitore svolge un'analisi approfondita,  alla quale consegue la stesura dello \textit{Studio di Fattibilità},  in modo da determinare criticità e rischi legati alla richiesta d'appalto.  Inoltre,  è necessario che venga definito un accordo sotto forma di contratto,  così da regolarizzare la corrispondenza col proponente e tutto ciò che è legato col prodotto finale,  tra cui consegna e manutenzione.  \newline

(Il processo potrà allora essere avviato, stabilendo le procedure e le risorse necessarie che andranno a confluire nel \textit{Piano di Progetto}, ponendo cosí le basi per la realizzazione e consegna del prodotto finale.) \newline
COPIATO DALLE NORME DEI PROAPES,  DA SISTEMARE

\subsubsection{Descrizione}

\subsubsection{Aspettative}

\subsubsection{Studio di Fattibilità}

\subsubsection{Altra documentazione}

\subsubsection{Strumenti}

\subsection{Sviluppo}

\subsubsection{Scopo}

\subsubsection{Descrizione}

\subsubsection{Aspettative}

\subsubsection{Analisi dei Requisiti}

da aggiungere altre sezioni

\subsubsection{Progettazione}

da aggiungere altre sezioni

\subsubsection{Codifica}

\subsubsection{Metriche}

\subsubsection{Strumenti}