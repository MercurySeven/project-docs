\section{Processi primari}

\subsection{Fornitura}

\subsubsection{Scopo}

La fornitura rappresenta ciò che il fornitore svolge al fine di capire e soddisfare appieno quello che viene richiesto dal proponente.  In particolare,  il fornitore svolge un'analisi approfondita,  alla quale consegue la stesura dello \textit{Studio di Fattibilità},  in modo da determinare criticità e rischi legati alla richiesta d'appalto.  Inoltre,  è necessario che venga definito un accordo sotto forma di contratto,  così da regolarizzare la corrispondenza col proponente e tutto ciò che è legato col prodotto finale,  tra cui consegna e manutenzione.  A questo punto,  si passa al determinare le procedure e le risorse necessarie,  sviluppandole nel \textit{Piano di Progetto}.  Quest'ultimo pone le basi per la realizzazione del prodotto, fino alla consegna finale.  
Il processo di fornitura è composto dalle seguenti fasi:
\begin{itemize}

\item avvio;

\item approntamento di risposte alle richieste;

\item contrattazione;

\item pianificazione;

\item esecuzione e controllo;

\item revisione e valutazione;

\item consegna e completamento.

\end{itemize}

\subsubsection{Descrizione}

Il documento include le norme che i membri del gruppo \textit{Mercury Seven } devono fare riferimento al fine di ottenere il ruolo di fornitori del proponente NOMEPROPONENTE e dei committenti \textit{Prof. Tullio Vardanega} e \textit{Prof.  Riccardo Cardin}.  I membri del gruppo sono tenuti ad attenersi a quanto scritto nella suddetta sezione durante tutte le fasi di sviluppo del prodotto NOMEPRODOTTO.

\subsubsection{Aspettative}

Durante lo svolgimento del progetto il gruppo intende mantenere un dialogo costante con il proponente,  in modo da instaurare un rapporto di collaborazione e ricevere un riscontro proficuo sul lavoro svolto.  In particolare,  si vuole:
\begin{itemize}

\item determinare gli aspetti chiave che il prodotto dovrà soddisfare;

\item definire vincoli e requisiti riguardanti i processi;

\item stimare costi e tempistiche di lavoro;

\item accordarsi sulla qualifica del prodotto.

\end{itemize}


\subsubsection{Studio di Fattibilità}

Il \textit{Responsabile di Progetto} ha il compito di organizzare riunioni tra i membri in modo da promuovere lo lo scambio di opinioni sui capitolati proposti.  Una volta che viene trovato un capitolato di interesse comune,  gli \textit{Analisti} provvedono alla stesura dello \textit{Studio di Fattibilità} che,  per ogni capitolato,  indica:

\begin{itemize}

\item \textbf{Informazioni generali}: vengono elencate le informazioni di base,  quali nome del progetto,  il proponente e il committente;

\item \textbf{Descrizione e finalità del capitolato}: sintesi del prodotto, comprendente le caratteristiche principali che vengono richieste e l'obiettivo da raggiungere;

\item \textbf{Tecnologie interessate}: elenco delle tecnologie richieste per lo svolgimento del prodotto;

\item \textbf{Aspetti positivi,  criticità e fattori di rischio}: esposizione delle considerazioni fatte dal gruppo riguardanti gli aspetti positivi e sulle criticità del capitolato in questione;

\item \textbf{Conclusioni}: esposizione delle ragioni per le quali il gruppo accetta o rifiuta il capitolato.

\end{itemize}

\subsubsection{Altra documentazione fornita}

In questa sezione vengono elencati i documenti che sono stati forniti all'azienda NOMEAZIENDA e ai committenti \textit{Prof. Tullio Vardanega} e \textit{Prof.  Riccardo Cardin}.  La suddetta documentazione risulta essenziale in modo da tracciare le attività  svolte dal gruppo.
I documenti forniti riguardano: 
\begin{itemize}

\item \textbf{Analisi}: contenuta nell' \textit{Analisi dei requisiti},  documento che esamina tutti i requisiti e i casi d'uso individuati;
\item \textbf{Pianificazione e consegna}: nel documento \textit{Piano di Progetto} viene presentato un preventivo riguardante tempistiche,  l' analisi dei rischi, la pianificazione delle attività e il consuntivo;
\item \textbf{Verifica,  Validazione e qualità}: all'interno del \textit{Piano di Qualifica} vengono descritte le metodologie utilizzate per valutare e garantire la qualità del prodotto.

\end{itemize}

\subsubsection{Strumenti}

/////// DA SCRIVERE 

\subsection{Sviluppo}

\subsubsection{Scopo}

\subsubsection{Descrizione}

\subsubsection{Aspettative}

\subsubsection{Analisi dei Requisiti}

da aggiungere altre sezioni

\subsubsection{Progettazione}

da aggiungere altre sezioni

\subsubsection{Codifica}

\subsubsection{Metriche}

\subsubsection{Strumenti}