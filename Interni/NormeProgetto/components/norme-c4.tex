\section{Processi Organizzativi}
\subsection{Coordinamento}
\subsubsection{Scopo}
I processi di coordinamento definiscono gli strumenti e le modalità adottate dal gruppo per gestire la comunicazione sia interna che esterna.
\subsubsection{Coordinamento interno}
\paragraph{Descrizione}
In questa sezione vengono descritte le modalità e le norme adottate per organizzare in modo efficace la collaborazione all'interno del gruppo. 
Vengono in particolare esposte le modalità inerenti a:
\begin{itemize}
\item Riunioni interne
\item Comunicazione tra membri del gruppo
\end{itemize}
\paragraph{Organizzazione delle riunioni interne}
Gli incontri formali interni avvengono esclusivamente sulla piattaforma \glo{Zoom}.
Tali incontri devono attenersi al regolamento seguente:
\begin{itemize}
\item La partecipazione a questi incontri è limitata ai membri del gruppo.
\item Prima di ogni incontro deve essere redatto l'ordine del giorno dal \RdP{}
\item Prima di ogni incontro deve essere comunicata la data e l'ora della riunione sull'apposito canale \glo{Slack} del gruppo
\item A inizio riunione devono essere nominati due dei partecipanti per redigere il verbale
\end{itemize}
I partecipanti sono tenuti a informare per tempo della loro assenza nel caso non potessero partecipare all'incontro.
\paragraph{Organizzazione della comunicazione tra i membri del gruppo}
Il canale di comunicazione principale è un \glo{workspace} \glo{Slack}.
Il workspace è suddiviso nei seguenti canali accessibili a tutti:
\begin{itemize}
\item \textbf{annunci}: canale dedicato alla comunicazione dell'ordine del giorno delle riunioni e ad annunci di interesse generale
\item \textbf{chiamate}: canale dedicato alla comunicazione dei link alle chiamate zoom, siano esse ricorrenti o meno.
\item \textbf{date}: canale dedicato alla comunicazione di scadenze e date di interesse per il gruppo come ad esempio seminari o riunioni esterne
\item \textbf{documenti-vari}: canale dedicato alla condivisione e archiviazione di bozze non finalizzate di documenti redatti dal gruppo
\item \textbf{general}: canale dedicato a comunicazioni senza un topic specifico tra i membri del gruppo
\end{itemize}
Oltre a questi canali vi sono canali accessibili solo a membri del gruppo che stanno svolgendo ruoli specifici che vengono creati e resi accessibili all'occorrenza.
\subsubsection{Comunicazioni esterne}
\paragraph{Descrizione}
In questa sezione vengono descritte le modalità e le norme adottate per organizzare in modo efficace la comunicazione con soggetti esterni.
Vengono in particolare esposte le modalità e le norme inerenti a:
\begin{itemize}
\item Modalità di comunicazione con soggetti esterni
\item Incontri formali esterni
\end{itemize}
\paragraph{Modalità di comunicazione con soggetti esterni}
Tutte le comunicazioni con soggetti esterni devono avvenire tramite l'indirizzo di posta elettronica \emailgruppo{}. Nel caso non rientri nei compiti del proprio ruolo, prima di inviare un'email utilizzando l'indirizzo del gruppo è obbligatorio consultare il \RdP{}.
\paragraph{Incontri formali esterni}
Gli incontri formali esterni vengono svolti tramite glo{Zoom}. Come negli incontri formali interni vengono nominati due dei partecipanti appartenenti al gruppo per redigere il verbale dell'incontro. 

\subsubsection{Lavoro a distanza}
Il gruppo ha sviluppato le precedenti norme con l'obbiettivo di diminuire il più possibile l'impatto che le restrizioni causate dal virus \textit{SARS-CoV2} comportano sul lavoro. Il gruppo ha favorito soluzioni di lavoro da remoto e piattaforme di teleconferenza rispetto a meeting in persona.

\subsection{Pianificazione}
\subsubsection{Scopo}
Il processo di pianificazione ha come obbiettivo principale quello di definire le attività e allocare le risorse necessarie alla progressione del progetto. La gestione di questo processo è affidata al \RdP{}.
\subsubsection{Descrizione}
In questa sezione vengono descritti i seguenti aspetti del processo di pianificazione:
\begin{itemize}
\item i ruoli di progetto
\item il processo e il criterio di assegnazione dei compiti
\item il sistema di ticketing per la gestione degli obbiettivi a medio/corto termine
\end{itemize} 
\subsubsection{Ruoli di progetto}
Durante il progetto i membri del gruppo ricoprono i seguenti ruoli
\begin{itemize}
\item \RdP{}
\item \adm{}
\item \ana{}
\item \prog{}
\item \progr{}
\item \ver{}
\end{itemize}
Questi ruoli vengono ricoperti dai membri del gruppo a seconda di un calendario che assicura a tutti i membri del gruppo di ricoprire durante la durata del progetto ogni ruolo diverso.

\paragraph{\RdP{}}
Il \RdP{} è la figura centrale del progetto, esso ha il compito di coordinare il team e il ruolo di rappresentante del progetto durante le comunicazioni con l'esterno. I suoi compiti sono riassunti nei seguenti punti:
\begin{itemize}
\item Definizione delle scadenze interne
\item Allocazione delle risorse umane a seconda delle attività necessarie
\item Rappresentazione del gruppo e del progetto con soggetti esterni
\item Valutazione dei rischi e distribuzione di questi ultimi
\item Approvazione della documentazione
\end{itemize}

\paragraph{\adm{}}
L'\adm{} ha il compito di gestire e configurare adeguatamente le piattaforme utilizzate dal gruppo per lo svolgimento del progetto. Da lui dipendono l'affidabilità e l'efficacia dei mezzi scelti per lo svolgimento del progetto.
I suoi compiti sono riassunti nei seguenti punti:
\begin{itemize}
\item Amministrazione delle infrastrutture di supporto
\item Assistenza nella risoluzione di problematiche relative alla gestione dei processi
\item Gestione e archiviazione della documentazione di progetto
\item Salvaguardia dell'integrità del processo di correzione e verifica della documentazione
\item Manutenzione delle norme e delle procedure relative alle piattaforme coinvolte nei processi di lavoro
\end{itemize}

\paragraph{\ana{}}
L' \ana{} segue il progetto principalmente nelle fasi iniziali ed è fortemente coinvolto nella stesura dell' \AdR{}. Il suo ruolo è quello di analizzare i problemi posti dal progetto e chiarire le dipendenze e le ramificazioni di ogni attività necessaria alla consegna del prodotto.
I suoi compiti sono riassunti nei seguenti punti:
\begin{itemize}
\item Redigere l'\AdR{}
\item Studio e definizione dei problemi sollevati dallo sviluppo del progetto
\item Definizione di dipendenze tra differenti attività all'interno del progetto
\item Definizione dei potenziali casi d'uso e delle necessità dell'utente finale
\end{itemize}

\paragraph{\prog{}}
Il \prog{} segue lo sviluppo del progetto e, a partire dai requisiti, definisce le scelte tecniche necessarie per lo sviluppo del prodotto. 
I suoi compiti sono riassunti nei seguenti punti:
\begin{itemize}
\item Analisi tecnica dei problemi posti dall' \ana{}
\item Modellazione di soluzioni ai problemi posti dall' \ana{}
\item Realizzazione di soluzioni sostenibili e di facile manutenzione durante la fase di sviluppo
\end{itemize}

\paragraph{\progr{}}
Il \progr{} ha il compito di codificare i modelli realizzati dal \prog{}. Il codice prodotto dal \progr{} deve attenersi il più possibile alle specifiche elaborate dal \prog{} e documentare opportunamente il codice creato per aumentarne la manutenibilità.
I suoi compiti sono riassunti nei seguenti punti:
\begin{itemize}
\item Codifica dei modelli realizzati dal \prog{}
\item Documentazione del codice prodotto per facilitarne la manutenzione
\item Codifica delle componenti necessarie alla verifica e alla validazione del prodotto durante le varie fasi dello sviluppo
\item Stesura del \MU{} destinato ai manutentori
\end{itemize}

\paragraph{\ver{}}
Il \ver{} segue l'intero ciclo di vita del progetto. Egli si assicura che la qualità della documentazione prodotta aderisca alla norme stabilite.
I suoi compiti sono riassunti nei seguenti puti:
\begin{itemize}
\item Controllo e correzione della documentazione prodotta
\item Controllo dell'aderenza alle norme delle attività di processo
\item Descrizione, ove e quando necessario, delle procedure di verifica
\end{itemize}

\subsubsection{Assegnazione dei compiti}
Le attività necessarie al completamento di un progetto, per garantire l'efficienza e la flessibilità del processo di sviluppo, devono essere suddivise in compiti. Tali compiti possono essere svolti sequenzialmente o in parallelo a seconda delle dipendenze tra di loro. Per ottimizzare il processo di suddivisione dei compiti il gruppo utilizza il sistema di ticket offerto dalla piattaforma \glo{GitHub}.
Una volta individuate dall' \ana{} le attività vengono all'occorrenza suddivise in sotto attività e compiti dal \RdP{}. Tra i compiti del \RdP{} vi è quello di assegnare a ogni membro del gruppo compiti diversi per ottimizzare il progresso del progetto. Il \RdP{} considera i seguenti fattori quando si occupa di valutare l'idoneità di un membro allo svolgimento di un certo compito:
\begin{itemize}
\item Disponibilità
\item Competenze tecniche
\item Carico di lavoro attuale del candidato
\end{itemize}
Il \RdP{} dopo aver deciso a chi assegnare l'incarico procederà ad aprire una \glo{issue} su \glo{GitHub} dettagliando il compito da svolgere.
Il membro del gruppi a cui viene assegnato l'incarico dovrà svolgere il compito nei tempi definiti dal \RdP{} e , nel caso di contrattempi, notificarlo quanto prima possibile. Il membro assegnato alla issue è inoltre responsabile della chiusura di quest'ultima.

\subsection{Infrastruttura}
\subsubsection{Scopo}
L'infrastruttura progettuale comprende tutte le piattaforme e applicazioni utilizzate durante le varie fasi di:
\begin{itemize}
\item Analisi
\item Pianificazione
\item Sviluppo
\item Revisione
\end{itemize}
Il corretto funzionamento di queste piattaforme è fondamentale per la collaborazione tra i membri del gruppo e la configurazione e gestione di esse è delegata all' \adm{}.

\subsubsection{Piattaforme e applicazioni}
\paragraph{Descrizione}
In questa sezione vengono descritte le piattaforme e le applicazioni usate dal team per lo sviluppo del progetto.
\paragraph{\textit{\glo{Slack}}}
\textit{Slack} è una piattaforma di messaggistica orientata al coordinamento di team. Il gruppo ha deciso di adottarla per via della granularità con la quale è possibile gestire i canali e i ruoli di ogni membro del team. Questa granularità ci permette di stabilire in poco tempo dei processi per il monitoraggio e l'ottimizzazione del lavoro all'interno del team responsabilizzando ogni membro.
\subparagraph{Uso}
Questa piattaforma è utilizzata per l'intera durata del progetto e svolge un ruolo centrale nella coordinazione efficiente del team. 

\paragraph{\glo{GitHub}}
\textit{GitHub} è una piattaforma di sviluppo collaborativo basata su \glo{Git}. Il team ha deciso di optare per questa piattaforma per via della sua flessibilità e compatibilità con il modello di sviluppo remoto che il team si è ritrovato a dover adottare. Inoltre il sistema di monitoraggio del progetto tramite issue e milestone è stato integrato nei processi di sviluppo e revisione sia della documentazione che del prodotto.
\subparagraph{Uso}
Questa piattaforma è utilizzata principalmente durante la fase di Pianificazione, utilizzando il sistema di issue e milestone, e durante la fase di Sviluppo.

\paragraph{\glo{GMail}}
\textit{GMail} è il servizio di posta elettronica offerto da \glo{Google}. Viene utilizzato per gestire le interazioni con soggetti esterni al gruppo. La scelta è ricaduta su questo servizio per via della familiarità che ogni membro de gruppo ha con esso.
\subparagraph{Uso}
Questa piattaforma è utilizzata per l'intera durata del progetto ogniqualvolta è necessario interagire con soggetti esterni.

\subsection{Sviluppo e ottimizzazione dei processi}

