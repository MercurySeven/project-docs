\section{Capitolato C6}

\subsection{Informazioni generali}
\begin{itemize}
\item \textbf{Nome:} RGP: Realtime Gaming Platform
\item \textbf{Proponente:} zero12
\item \textbf{Committenti:} \textit{prof. Tullio Vardanega} e \textit{prof. Riccardo Cardin}
\end{itemize}

\subsection{Descrizione del capitolato}
Il capitolato richiede di sviluppare un gioco sia single player che multiplayer.
Il focus del progetto è però centrato nell'apprendere le tecnologie offerte da Amazon attraverso la loro piattaforma AWS.
In particolare, si chiede di usare AWS GameLift per la sincronizzazione del gioco tra più giocatori.
I giocatori non si sfideranno direttamente ma potranno vedere in real-time il progresso dell'avversario.

\subsection{Finalità del progetto}
Il capitolato chiede di sviluppare i seguenti punti:
\begin{itemize}
\item Analizzare le tecnologie AWS per capire quale si può adattare meglio ad un gioco con requisiti di realtime, raccogliendo le motivazioni che supportano la scelta di una tecnologia rispetto ad un'altra.
\item Implementazione della componente server.
\item Implementazione del gioco in una piattaforma mobile.
\end{itemize}

\subsection{Tecnologie interessate}
Le tecnologie consigliate sono:
\begin{itemize}
\item L'utilizzo del servizio AWS GameLift per gestire il multiplayer all'interno del gioco.
\item L'utilizzo di AWS DynamoDB per il salvataggio dei dati cloud attraverso un database NoSQL.
\item L'uso di NodeJS per creare il server che avrà il compito di fornire le API utili al gioco per chiedere informazioni sullo stato degli altri giocatori.
\item L'uso di tecnologie native per la progettazione del gioco, Swift per Apple e Kotlin per Android.
\end{itemize}
Un aspetto molto importante per il proponente è che l'architettura server deve essere scalabile.

\subsection{Aspetti positivi}
\begin{itemize}
\item L'utilizzo dei servizi Amazon AWS, che al giorno d'oggi sono sempre più diffusi.
\item Il mondo del videogame, soprattutto negli ultimi anni, sta avendo una crescita esponenziale, questo capitolato ci offre la possibilità di apprendere le tecnologie necessarie che servono per sviluppare un videogame multiplayer.
\end{itemize}

\subsection{Criticità e fattori di rischio}
\begin{itemize}
\item Sviluppare un gioco per iOS/iPadOS richiede per forza un Mac.
\item Se si vuole sviluppare per Android, bisogna sviluppare su Android Studio e appoggiarsi a librerie di cui nessun componente del gruppo conosce. 
Si è valutato inoltre di sviluppare su Unity, ma il linguaggio supportato è C\# e non tutti i componenti del gruppo lo conoscono.
\end{itemize}

\subsection{Valutazione finale}
Per le tecnologie proposte il gruppo \Gruppo{} ha deciso di scegliere questo capitolato come seconda scelta tra i capitolati proposti.