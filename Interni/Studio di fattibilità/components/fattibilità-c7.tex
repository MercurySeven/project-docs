\section{Capitolato C7}

\subsection{Informazioni generali}
\begin{itemize}
\item \textbf{Nome:} SSD: Soluzioni di sincronizzazione desktop
\item \textbf{Proponente:} ZEXTRAS
\item \textbf{Committenti:} prof. Tullio Vardanega e prof. Riccardo Cardin
\end{itemize}

\subsection{Descrizione del capitolato}
Il capitolato richiede lo sviluppo di un algoritmo e di un frontend per sincronizzare file tra server e client appoggiandosi sul sevizio Zextras Drive.
L'interfaccia deve essere multipiattaforma e il prodotto deve essere sviluppato aderendo al pattern MVC.

\subsection{Finalità del progetto}
Il capitolato pone i 3 seguenti obiettivi:
\begin{itemize}
\item \textbf{Sviluppo di un algoritmo di sincronizzazione solido ed efficiente}: deve garantire la sincronizzazione ottimale tra cloud e desktop utilizzando il quantitativo minimale di risorse.
\item \textbf{Sviluppo di un interfaccia multipiattaforma}: deve fornire l'interfaccia necessaria per interagire con i servizi messi a disposizione 
dall'algoritmo e essere utilizzabile almeno sui tre sistemi operativi più popolari (Linux, MacOS, Windows). 
La soluzione sviluppata non deve dipendere da installazioni di framework di terze parti da parte dell'utente finale.
\item \textbf{Integrazione con Zextras Drive}: sia l'algoritmo che l'interfaccia devono essere integrati con il servizio di gestione file Zextras Drive.
\end{itemize}
Oltre al raggiungimento degli obbiettivi precedentemente riportati le seguenti funzionalità dovrebbero essere aggiunte:
\begin{itemize}
\item Configurazione e autenticazione dell'utente.
\item Gestione dei file da sincronizzare o ignorare sia dal lato cloud che dal lato desktop e la possibilità di modificare in ogni momento la configurazione.
\item Sincronizzazione costante dei cambiamenti, sia locali che remoti.
\item Sistema di notifica dei cambiamenti.
\item Funzionalità presenti in altre soluzioni attualmente disponibili:
	\begin{itemize}
	\item Gestione delle condivisioni.
	\item Integrazione con il protocollo MAPI per permettere l'integrazione con applicazioni di IM.
	\end{itemize}
\end{itemize}

\subsection{Tecnologie interessate}
Le seguenti tecnologie possono essere impiegate nello sviluppo del progetto
\begin{itemize}
\item \textbf{Tecnologie consigliate}:
\begin{itemize}
	\item \textbf{Framework Qt}: framework consigliato per lo sviluppo di interfacce grafiche basato su C++.
	\item \textbf{Python}: linguaggio consigliato per lo sviluppo del modello. Molto flessibile. Supportato dalla maggior parte dei framework per lo sviluppo di interfacce grafiche. 
	Semplifica il processo di integrazione con le API di Zextras Drive.
	\end{itemize}
\newpage
\item \textbf{Tecnologie necessarie}:
	\begin{itemize}
	\item \textbf{Zimbra}: sistema collaborativo per la gestione della posta elettronica.
	\item \textbf{Zextras Drive}: servizio proprietario di gestione file in cloud con funzionalità di:
		\begin{itemize}
		\item Versionamento.
		\item Gestione degli accessi e della condivisione dei file.
		\item Modifica collaborativa di file documentali.
		\end{itemize}
	Il servizio espone inoltre delle API sviluppate utilizzando il framework GraphQL.
\end{itemize}
\item \textbf{Altre tecnologie}:
	\begin{itemize}
	\item \textbf{GraphQL}: linguaggio di query per API utilizzato dalle API di Zextras Drive.	
	\item \textbf{Apollo per GraphQL}: implementazione di GraphQL utilizzata dalle API di Zextras Drive.
	\end{itemize}
\end{itemize}

\subsection{Aspetti positivi}
\begin{itemize}
\item Il progetto offre l'opportunità di approfondire delle tecnologie fondamentali per la collaborazione a distanza, un aspetto che sarà sempre più rilevante negli ambienti lavorativi futuri.
\item Il team ha già familiarità con il framework Qt e Python.
\end{itemize}

\subsection{Criticità e fattori di rischio}
\begin{itemize}
\item Mancanza di documentazione dettagliata pubblicamente accessibile sulle API del prodotto Zextras Drive.
\item Rischio di doversi appoggiare eccessivamente sul committente per la corretta integrazione del prodotto sviluppato con il loro servizio.
\end{itemize}

\subsection{Valutazione finale}
Dopo alcuni incontri per discutere sui capitolati proposti, il gruppo \Gruppo{} ha deciso di scegliere questo capitolato valutandolo come il più interessante.