\section{Informazioni generali}

\begin{itemize}

    \item \textbf{Canale di comunicazione:} Zoom;

    \item \textbf{Data:} \DataMeeting{};

    \item \textbf{Ora inizio:} 16:00;

    \item \textbf{Ora fine:} 18:00;

    \item \textbf{Segretario:} \ACapoRedazione{};

    \item \textbf{Partecipanti}: 
        \begin{itemize}
            \item \Daniele{};
            \item \Davide{};
            \item \Francesco{};
            \item \Giosue{};
            \item \Lucrezia{};
            \item \Matteo{};
            \item \Tommaso{}.
        \end{itemize}
\end{itemize}

\section{Ordine del giorno}

\begin{itemize}
    \item\textbf{Pattern e architettura;}
    \item\textbf{Assegnazione diagrammi;}
    \item\textbf{Obiettivi di sprint;}
    \item\textbf{Richiesta colloquio con l'azienda;}
\end{itemize}
\newpage


\section{Resoconto}
\subsection{Pattern e architettura}
Si è svolta una discussione sui design pattern utilizzati ed eventualmente da aggiungere al codice. In particolare, si è deciso di utilizzare il singleton dei pattern creazionali,  il proxy dei pattern strutturali e lo strategy dei pattern comportamentali. Inoltre, si è consolidata l'aderenza all' MVC dei pattern architetturali, adattando il diagramma UML al codice. In particolare, le classi Subject e Observer diventano rispettivamente le nostre classi Signal e Slot. 

\subsection{Assegnazione diagrammi}
Dopo aver discusso dei pattern da implementare, si è deciso di iniziare ad assegnare ad alcuni membri del gruppo la creazione di bozze dei diagrammi UML riguardanti il codice, così da velocizzarne la creazione effettiva che avverrà prossimamente in una riunione con tutti i membri.

\subsection{Obiettivi di sprint}
Si sono discussi gli obiettivi degli sprint futuri e sì è deciso di assegnare a ogni membro del gruppo una parte delle attività da svolgere per raggiungerli. 

\subsection{Colloquio video con l'azienda}
Visti i miglioramenti apportati all'applicazione e i dubbi scaturiti da discussioni interne, il gruppo ha deciso all'unanimità di chiedere un colloquio con l'azienda, comunicandolo a quest'ultima tramite l'apposito canale su Slack.

\newpage

\section{Riepilogo delle decisioni \hfil}
{
    \setlength{\freewidth}{\dimexpr\textwidth-4\tabcolsep}
    \renewcommand{\arraystretch}{1.5}
    \setlength{\aboverulesep}{0pt}
    \setlength{\belowrulesep}{0pt}
    \rowcolors{2}{AzzurroGruppo!10}{white}
    \begin{longtable}{L{.3\freewidth} L{.7\freewidth}}
        \toprule
        \rowcolor{AzzurroGruppo!30}
        \textbf{Codice} & \textbf{Decisione}\\
        \toprule
        \endhead

        VI\_\DataMeeting{}.1 &  Si sono decisi i pattern da implementare nell'architettura del codice.\\
        VI\_\DataMeeting{}.2 &  Si è deciso di assegnare la stesura dei diagrammi delle classi ai membri del gruppo.\\
	   VI\_\DataMeeting{}.3 &  Si è deciso come organizzare il lavoro tra i membri del gruppo.\\
        VI\_\DataMeeting{}.4 &  Si è deciso di chiedere un colloquio video all'azienda.\\
        \bottomrule
        \hiderowcolors
    \end{longtable}
}