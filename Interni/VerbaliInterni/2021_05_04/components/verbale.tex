\section{Informazioni generali}

\begin{itemize}

    \item \textbf{Canale di comunicazione:} Zoom;

    \item \textbf{Data:} \DataMeeting{};

    \item \textbf{Ora inizio:} 21:00;

    \item \textbf{Ora fine:} 22:00;

    \item \textbf{Segretario:} \ACapoRedazione{};

    \item \textbf{Partecipanti}:
        \begin{itemize}
            \item \Daniele{};
            \item \Davide{};
            \item \Francesco{};
            \item \Giosue{};
            \item \Lucrezia{};
            \item \Matteo{};
            \item \Tommaso{}.
        \end{itemize}
\end{itemize}

\section{Ordine del giorno}

\begin{itemize}
    \item\textbf{Discussione presentazione demo tecnica;}
    \item\textbf{Correzioni da apportare ai manuali;}
    \item\textbf{Correzioni da apportare alle \textit{Norme di Progetto};}
    \item\textbf{Correzioni da apportare ai consuntivi.}
\end{itemize}
\newpage


\section{Resoconto}

\subsection{Discussione presentazione demo tecnica}
La presentazione ai committenti è risultata per la maggior parte soddisfacente tranne il passaggio alla demo e la demo stessa. Come il committente ha fatto notare, il passaggio alla demo risulta rompere il flusso della presentazione e la demo stessa risulta leggermente scarna nei contenuti tecnici che si sarebbero dovuti mostrare. \newline{} Per la presentazione finale questi aspetti dovranno dunque essere completamente migliorati. La presentazione dovrà naturalmente confluire nella demo e la demo dovrà esporre caratteri più tecnici.

\subsection{Correzioni da apportare ai manuali}
Il gruppo ha deciso di ristrutturare entrambi i manuali. Il \textit{Manuale Utente} presenta attualmente una forma troppo generica e dal punto di vista di uno sviluppatore e dovrà essere ristrutturato per poter presentare informazioni più utili per il pubblico a cui il documento è destinato. \newline{}
Il \textit{Manuale Sviluppatore} presenta anch'esso una forma troppo generica ed astratta, risultando poco utile per uno sviluppatore. Verrà quindi ristrutturato per poter offrire informazioni più a basso livello e di maggiore utilità.

\subsection{Correzioni da apportare alle \textit{Norme di Progetto}}
Comprese le critiche mosse nella struttura delle \textit{Norme di Progetto}, si è deciso di ristudiare il documento, uniformando le sezioni e rivedendo il contenuto di esse. Al momento il documento presenta sezioni completamente non uniformate e nelle quali i contenuti vengono guidati dai prodotti e non dalle loro attività.

\subsection{Correzioni da apportare ai consuntivi}
Le critiche mosse nei consuntivi presenti nel \textit{Piano di Progetto} hanno portato a realizzare una nuova sezione nel documento. Alla fine di ogni sprint si dovrà fare un resoconto degli obiettivi raggiunti, portando ad una migliore comprensione della situazione del progetto.



\newpage

\section{Riepilogo delle decisioni \hfil}
{
    \setlength{\freewidth}{\dimexpr\textwidth-4\tabcolsep}
    \renewcommand{\arraystretch}{1.5}
    \setlength{\aboverulesep}{0pt}
    \setlength{\belowrulesep}{0pt}
    \rowcolors{2}{AzzurroGruppo!10}{white}
    \begin{longtable}{L{.3\freewidth} L{.7\freewidth}}
        \toprule
        \rowcolor{AzzurroGruppo!30}
        \textbf{Codice} & \textbf{Decisione}\\
        \toprule
        \endhead

        VI\_\DataMeeting{}.1 & Realizzare la prossima presentazione collegando e approfondendo meglio la demo .\\
        VI\_\DataMeeting{}.2 & Ristrutturare entrambi i manuali.\\
        VI\_\DataMeeting{}.3 & Ristrutturare in funzione delle attività le \textit{Norme di Progetto}.\\
        VI\_\DataMeeting{}.4 & Aggiungere i consuntivi degli obiettivi per ogni sprint.\\
        \bottomrule
        \hiderowcolors
    \end{longtable}
}
