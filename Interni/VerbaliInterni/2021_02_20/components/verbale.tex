\section{Informazioni generali}

\begin{itemize}

	\item \textbf{Canale di comunicazione:} Zoom;
	
	\item \textbf{Data:} \DataMeeting{};
	
	\item \textbf{Ora inizio:} 21;
	
	\item \textbf{Ora fine:} 22;
	
	\item \textbf{Segretario:} Tommaso Poppi;
	
	\item \textbf{Partecipanti:}
	
		\begin{itemize}

			\item Daniele Giachetto;
			\item Davide Albiero;
			\item Francesco De Marchi;
			\item Giosuè Calgaro;
			\item Lucrezia Gradi;
			\item Tommaso Poppi.

		\end{itemize}

\end{itemize}



\section{Ordine del giorno}

\begin{itemize}

	\item\textbf{Server;}
	
	\item\textbf{Verifica del cambiamento di un file;}
	
	\item\textbf{Download e Upload;}
	
	\item\textbf{GUI.}

\end{itemize}

\newpage


\section{Resoconto}

Durante l'incontro il gruppo \gruppo{} ha deciso quali funzionalità e quali tecnologie mostrare per il PoC.

\subsection{Server}

Non avendo ancora a disposizione il server di Zextras, si è deciso di crearne uno apposito, in modo da permettere l'invio e la ricezione dei metadata tramite chiamate graphql. I metadata disponibili all'estrazione sono stati forniti dall'azienda ed elencati nelle API.

\subsection{Download e Upload}

La gestione del download e dell'upload viene gestita da un algoritmo che effettua chiamate al server tramite GraphQL. In particolare, per il download si ricevono i metadati dal server, mentre per l'upload si inviano i metadati del file locale al server.  L'intero algoritmo è scritto in Python.

\subsection{Watchdog e verifica}

Per funzionare correttamente, l'applicazione ha bisogno di un watchdog, una sentinella che controlla se ci sono modifiche ai file. Il controllo è impostato ogni tot e se ci sono modifiche si accende una spia. A questo punto, il programma deve verificare il tipo di azione da eseguire, cioè se fare il download o l'upload. Si è deciso quindi di far si che la verifica effettui un controllo del metadata della root directory e, se l'ultima modifica del file in locale è diversa dall'ultima modifica del file sul server allora è cambiato qualcosa, sta al programma decidere come agire. 

\subsection{GUI}

Si è deciso di realizzare l'interfaccia grafica dell'applicazione con Qt. Essa mostra la root directory con i file al suo interno e le varie impostazioni implementate. 






\newpage

\section{Riepilogo delle decisioni \hfil}
{
	\setlength{\freewidth}{\dimexpr\textwidth-4\tabcolsep}
	\renewcommand{\arraystretch}{1.5}
	\setlength{\aboverulesep}{0pt}
	\setlength{\belowrulesep}{0pt}
	\rowcolors{2}{AzzurroGruppo!10}{white}
	\begin{longtable}{L{.3\freewidth} L{.7\freewidth}}
		\toprule 
		\rowcolor{AzzurroGruppo!30}
		\textbf{Codice} & \textbf{Decisione}\\
		\toprule
		\endhead
		
		VI\_\DataMeeting{}.1 & Si è deciso di creare un server apposito  \\
		VI\_\DataMeeting{}.2 & Si sono decise le modalità di download e upload\\
		VI\_\DataMeeting{}.3 & Si è deciso come avviene la verifica dei cambiamenti in locale e sul server\\
		VI\_\DataMeeting{}.4 & Si è deciso cosa mostrare della GUI\\
		
		\bottomrule
		\hiderowcolors
	\end{longtable}
}