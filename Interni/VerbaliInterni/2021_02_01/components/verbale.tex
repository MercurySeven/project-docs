\section{Informazioni generali}

\begin{itemize}

	\item \textbf{Canale di comunicazione:} Zoom;
	
	\item \textbf{Data:} \DataMeeting{};
	
	\item \textbf{Ora inizio:} 21:30;
	
	\item \textbf{Ora fine:} 22:30;
	
	\item \textbf{Segretario:} Tommaso Poppi;
	
	\item \textbf{Partecipanti:}
	
		\begin{itemize}

			\item Daniele Giachetto;
			\item Davide Albiero;
			\item Francesco De Marchi;
			\item Giosuè Calgaro;
			\item Lucrezia Gradi;
			\item Tommaso Poppi.

		\end{itemize}

\end{itemize}



\section{Ordine del giorno}

\begin{itemize}
	\item\textbf{Discussione su correzioni fatte all' \AdR{};}
	\item\textbf{Discussione concept GUI;}
	\item\textbf{Discussione su approccio ai test;}
	\item\textbf{Cambiamento del codice di versionamento;}
	\item\textbf{Chiamata con azienda.}
\end{itemize}

\newpage


\section{Resoconto}

\subsection{Discussione su correzioni fatte all' \AdR{}}
Durante l'incontro sono state discusse le annotazioni fatte all'\AdR{} 	contenute nell'esito parziale della prima convocazione alla RR. In particolare, viene segnalato che i codici identificativi non sono stati assegnati ai casi d'uso individuati ma al diagramma stesso, questo rende o casi d'uso non tracciabili individualmente. Come scritto nell'esito, si è deciso di produrre una versione migliorativa dell'\AdR{} intervenendo sulle segnalazioni da consegnare entro la prossima consegna. Prima di ciò, si è deciso di chiedere un incontro col professor \Riccardo{} riguardo alle segnalazioni e ad altri particolari che il gruppo pensava di modificare. 

\subsection{Discussione concept GUI}
Dopo aver discusso della bozza dell'algoritmo e della GUI, si è deciso di approvare il concept dell'interfaccia da presentare al PoC. 

\subsection{Discussione su approccio ai test }
Dopo alcune modifiche effettuate al \PdQ{}, è stato deciso che lo sviluppo sarà test driven. Questo tipo di approccio è composto da due fasi che si alterneranno: la parte di code driven testing, dove il test verrà strutturato in modo da sapere già il risultato che si vuole ottenere, e la parte di refactoring, dove verrà migliorata la struttura del codice, scritto in modo da soddisfare il test,  per adeguarsi agli standard di codifica. Questo tipo di approccio assicura un'adeguata manutenibilità del codice e permette di raggiungere un livello qualitativo soddisfacente.

\subsection{Cambiamento del codice di versionamento}
Durante la chiamata è stato discusso il codice di versionamento. In particolare, nella revisione dei documenti è stato segnalato come il nostro metodo di versionamento considerasse uno scatto di versione a fronte di una qualsiasi azione sul prodotto, mentre sarebbe più compatibile con una buona produzione dei documenti far sì che lo scatto avvenga solo quando la modifica viene validata da un verificatore. Considerando il nostro codice di versionamento, X.Y.Z-a.b, si è deciso di effettuare lo scatto di versione della Z solo quando viene effettuata anche la verifica della parte modificata. 

\subsection{Chiamata con azienda}
Dopo un'attenta valutazione delle funzionalità e delle tecnologie da mostrare al PoC, si è deciso di contattare l'azienda e di richiedere un incontro al fine di consolidare i requisiti, di richiedere l'accesso al server aziendale per effettuare eventuali test e di chiedere consigli su cosa concentrarci durante il PoC.

\newpage

\section{Riepilogo delle decisioni \hfil}
{
	\setlength{\freewidth}{\dimexpr\textwidth-4\tabcolsep}
	\renewcommand{\arraystretch}{1.5}
	\setlength{\aboverulesep}{0pt}
	\setlength{\belowrulesep}{0pt}
	\rowcolors{2}{AzzurroGruppo!10}{white}
	\begin{longtable}{L{.3\freewidth} L{.7\freewidth}}
		\toprule
		\rowcolor{AzzurroGruppo!30}
		\textbf{Codice} & \textbf{Decisione}\\
		\toprule
		\endhead
		
		VI\_\DataMeeting{}.1 & Si è deciso di richiedere un incontro con il professor \Riccardo{} per discutere delle modifiche da effettuare sull'\AdR{}\\
		VI\_\DataMeeting{}.2 & Si è approvato il concept della GUI\\
		VI\_\DataMeeting{}.3 & Si è deciso di adottare l'approccio allo sviluppo test driven\\
		VI\_\DataMeeting{}.4 & Si è deciso di modificare lo scatto di versione nel codice di versionamento\\
		VI\_\DataMeeting{}.5 & Si è deciso di contattare l'azienda in preparazione al PoC\\
		
		\bottomrule
		\hiderowcolors
	\end{longtable}
}