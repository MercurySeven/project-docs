\section{Informazioni generali}

\begin{itemize}

	\item \textbf{Canale di comunicazione:} Zoom;
	
	\item \textbf{Data:} \DataMeeting{};
	
	\item \textbf{Ora inizio:} 14;
	
	\item \textbf{Ora fine:} 15;
	
	\item \textbf{Segretario:} Tommaso Poppi;
	
	\item \textbf{Partecipanti:}
	
		\begin{itemize}
		
			\item Daniele Giachetto;
			\item Davide Albiero;
			\item Francesco De Marchi;
			\item Giosuè Calgaro;
			\item Lucrezia Gradi;
			\item Tommaso Poppi.
				 
		\end{itemize}

\end{itemize}



\section{Ordine del giorno}

\begin{itemize}

	\item\textbf{Implementazione CI su GitHub;}
	
	\item\textbf{Discussione sulla presentazione di gruppo del 18-01-2021;}

	\item\textbf{Cambio Gantt sul \PdP{};}

	\item\textbf{Cambio del nome dei documenti;}
	
	\item\textbf{Distribuzione lavoro restante.}

\end{itemize}

\newpage


\section{Resoconto}

\subsection{Implementazione CI su GitHub}

Su GitHub è stato implementato uno script che permette di scegliere quali documenti compilare sul main, sia con che senza verbali. Inoltre, è stata sviluppata una CI per il \G{} che lo costruisce ogni volta che c'è un commit sul main. Dopo la spiegazione sul funzionamento, è stato deciso che il registro delle modifiche del \G{} viene aggiornato ogni volta che vengono aggiunte delle definizioni. 

\subsection{Discussione sulla presentazione di gruppo del 18-01-2021}

Il gruppo ha discusso sulla presentazione di gruppo che si terrà il 18-01-2021. Dovrà essere correlata di un PowerPoint dove vengono riassunti i punti salienti della documentazione fornita alla RR e ogni membro del gruppo dovrà presentarne una parte. Ciò che sarà presente nel PowerPoint, però, verrà comunicato dal prof. Vardanega dopo la consegna dei documenti, che avverrà il giorno 11-01-2020.

\subsection{Cambio Gantt sul \PdP{} }

Dopo aver discusso sul punto della situazione dei documenti, si è visto come i diagrammi di Gantt presenti sul \PdP{} considerassero il periodo di verifica solo nella parte finale. In realtà, il modo di lavorare del gruppo fa si che la verifica inizi pochi giorni dopo l'inizio della stesura, rendendola di fatto un'attività continua e in linea con la produzione dei documenti. Per questo, si è deciso di modificare i diagrammi di Gantt così da renderli più veritieri.

\subsection{Cambio del nome dei documenti}

Si è deciso, dopo la discussione sui documenti, di cambiare il nome dei documenti così da rendere il lavoro più uniforme.

\subsection{Distribuzione del lavoro restante}

Sono state assegnate le ultime modifiche e le verifiche restanti. In particolare:
\begin{itemize}

\item Stesura ultimi \textit{verbali}: Daniele e Lucrezia con verificatori Tommaso e Matteo;

\item Stesura parti mancanti \NdP{}: Davide e Francesco con verificatore Giosuè;

\item Verifica del \PdP{}: Lucrezia;

\item Verifica del \PdQ{}: Daniele;

\item Popolazione del \G{}: durante l'attività di verifica.



\end{itemize}


\newpage

\section{Riepilogo delle decisioni \hfil}
{
	\setlength{\freewidth}{\dimexpr\textwidth-4\tabcolsep}
	\renewcommand{\arraystretch}{1.5}
	\setlength{\aboverulesep}{0pt}
	\setlength{\belowrulesep}{0pt}
	\rowcolors{2}{AzzurroGruppo!10}{white}
	\begin{longtable}{L{.3\freewidth} L{.7\freewidth}}
		\toprule 
		\rowcolor{AzzurroGruppo!30}
		\textbf{Codice} & \textbf{Decisione}\\
		\toprule
		\endhead
		
		VI\_\DataMeeting{}.1 & Implementazione CI su GitHub. \\  
		VI\_\DataMeeting{}.2 & Introduzione alla Presentazione del 18-01-2021. \\
		VI\_\DataMeeting{}.3 & Modifica dei diagrammi di Gantt sul \PdP{} \\ 
		VI\_\DataMeeting{}.4 & Modifica nome dei documenti. \\
		VI\_\DataMeeting{}.5 & Assegnazione verifiche e stesure restanti. \\
		
		\bottomrule
		\hiderowcolors
	\end{longtable}
