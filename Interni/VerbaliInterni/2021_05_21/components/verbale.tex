\section{Informazioni generali}

\begin{itemize}

    \item \textbf{Canale di comunicazione:} Zoom;

    \item \textbf{Data:} \DataMeeting{};

    \item \textbf{Ora inizio:} 21:00;

    \item \textbf{Ora fine:} 22:00;

    \item \textbf{Segretario:} \ACapoRedazione{};

    \item \textbf{Partecipanti}:
        \begin{itemize}
            \item \Daniele{};
            \item \Davide{};
            \item \Francesco{};
            \item \Giosue{};
            \item \Lucrezia{};
            \item \Matteo{};
            \item \Tommaso{}.
        \end{itemize}
\end{itemize}

\section{Ordine del giorno}

\begin{itemize}
    \item\textbf{Esito incontro con il proponente;}
    \item\textbf{Situazione documenti;}
    \item\textbf{Inizio preparazione presentazione.}
\end{itemize}
\newpage


\section{Resoconto}

\subsection{Esito incontro con il proponente}
È stato discusso l'esito dell'incontro realizzato con il proponente. In generale il gruppo si sente soddisfatto della propria comunicazione con l'azienda e non sono state richieste ulteriori modifiche al prodotto software prima del Collaudo. In merito a quest'ultimo aspetto, il team si è accordato con l'azienda per una data nella quale effettuare la presentazione durante la \textit{Revisione di Accettazione}.

\subsection{Situazione documenti}
Il gruppo ha fatto una panoramica dei documenti da consegnare per la \textit{Revisione di Accettazione}. La situazione attuale sembra soddisfare il team, in quanto la maggioranza dei documenti appare ormai chiusa ed approvata. Il gruppo dovrà comunque ultimare le attività di verifica già in corso.

\subsection{Inizio preparazione presentazione}

Il gruppo ha discusso, anche se in maniera astratta, come strutturare la presentazione per la \textit{Revisione di Accettazione}. Si è ricordata l'importanza di una demo fluida e, allo stesso tempo tecnica. Per quest'ultimo punto si lavorerà per poter avere una transizione migliore tra le varie parti, intrecciandole tra di loro quando possibile.




\newpage

\section{Riepilogo delle decisioni \hfil}
{
    \setlength{\freewidth}{\dimexpr\textwidth-4\tabcolsep}
    \renewcommand{\arraystretch}{1.5}
    \setlength{\aboverulesep}{0pt}
    \setlength{\belowrulesep}{0pt}
    \rowcolors{2}{AzzurroGruppo!10}{white}
    \begin{longtable}{L{.3\freewidth} L{.7\freewidth}}
        \toprule
        \rowcolor{AzzurroGruppo!30}
        \textbf{Codice} & \textbf{Decisione}\\
        \toprule
        \endhead

        VI\_\DataMeeting{}.1 & Discusso esito incontro con il proponente.\\
        VI\_\DataMeeting{}.2 & Effettuata panoramica dei documenti e del lavoro restante.\\
        VI\_\DataMeeting{}.3 & Discusso come migliorare la presentazione della demo tecnica live.\\
        \bottomrule
        \hiderowcolors
    \end{longtable}
}
