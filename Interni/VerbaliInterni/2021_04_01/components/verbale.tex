\section{Informazioni generali}

\begin{itemize}

    \item \textbf{Canale di comunicazione:} Zoom;

    \item \textbf{Data:} \DataMeeting{};

    \item \textbf{Ora inizio:} 16:00;

    \item \textbf{Ora fine:} 17:00;

    \item \textbf{Segretario:} \ACapoRedazione{};

    \item \textbf{Partecipanti}:
        \begin{itemize}
            \item \Daniele{};
            \item \Davide{};
            \item \Francesco{};
            \item \Giosue{};
            \item \Lucrezia{};
            \item \Matteo{};
            \item \Tommaso{}.
        \end{itemize}
\end{itemize}

\section{Ordine del giorno}

\begin{itemize}
    \item\textbf{Email al professor Cardin;}
    \item\textbf{Colloquio video con l'azienda.}
\end{itemize}
\newpage


\section{Resoconto}

\subsection{Email al professor Cardin}
Dopo aver discusso dei risultati della RP, si è deciso di mandare una email al professor Cardin riguardo a dei dubbi su una correzione nel documento \AdR{}.  In particolare, ci è stato segnalato nel caso d'uso UC4 come lo stesso non possa essere presente come sotto-caso nel suo diagramma, il che ha destato dubbi sui membri del gruppo in quanto non è stato segnalato in altri casi d'uso che presentano la medesima dicitura.

\subsection{Pattern e architettura}
Si è svolta una discussione sui design pattern utilizzati ed eventualmente da aggiungere al codice. In particolare, si è deciso di utilizzare il singleton dei pattern creazionali,  il proxy dei pattern strutturali e lo strategy dei pattern comportamentali. Inoltre, si è consolidata l'aderenza all' MVC dei pattern architetturali, adattando il diagramma UML al codice. In particolare, le classi Subject e Observer diventano rispettivamente le nostre classi Signal e Slot. 



\newpage

\section{Riepilogo delle decisioni \hfil}
{
    \setlength{\freewidth}{\dimexpr\textwidth-4\tabcolsep}
    \renewcommand{\arraystretch}{1.5}
    \setlength{\aboverulesep}{0pt}
    \setlength{\belowrulesep}{0pt}
    \rowcolors{2}{AzzurroGruppo!10}{white}
    \begin{longtable}{L{.3\freewidth} L{.7\freewidth}}
        \toprule
        \rowcolor{AzzurroGruppo!30}
        \textbf{Codice} & \textbf{Decisione}\\
        \toprule
        \endhead

        VE\_\DataMeeting{}.1 &  Si è deciso di mandare una email al professor Cardin riguardante dubbi sulla correzione dell'\AdR{}.\\
        VE\_\DataMeeting{}.2 &  Si sono decisi i pattern da adottare nel codice.\\
        \bottomrule
        \hiderowcolors
    \end{longtable}
}
