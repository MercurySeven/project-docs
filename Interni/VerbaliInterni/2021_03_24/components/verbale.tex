\section{Informazioni generali}

\begin{itemize}

    \item \textbf{Canale di comunicazione:} Zoom;

    \item \textbf{Data:} \DataMeeting{};

    \item \textbf{Ora inizio:} 16:00;

    \item \textbf{Ora fine:} 18:00;

    \item \textbf{Segretario:} \ACapoRedazione{};

    \item \textbf{Partecipanti}: 
        \begin{itemize}
            \item \Daniele{};
            \item \Davide{};
            \item \Francesco{};
            \item \Giosue{};
            \item \Lucrezia{};
            \item \Matteo{};
            \item \Tommaso{}.
        \end{itemize}
\end{itemize}

\section{Ordine del giorno}

\begin{itemize}
    \item\textbf{Pattern e architettura;}
    \item\textbf{Versionamento;}
    \item\textbf{Cambio logica dell'algoritmo;}
    \item\textbf{Obiettivi di sprint.}
\end{itemize}
\newpage


\section{Resoconto}
\subsection{Versionamento}
TODO


\subsection{Cambio logica dell'algoritmo}
Fino a ora l'algoritmo si basava sul confronto tra due liste, una contenente gli elementi del server e l'altra contenente gli elementi della cartella in locale, faceva confronti tra le due e agiva di conseguenza. Questo tipo di approccio ha i suoi difetti, che si sono manifestati soprattutto nel momento in cui si è iniziato a introdurre cartelle annidate e l'eliminazione dei file dalla cartella. Dopo la chiamata con l'azienda e dopo aver ragionato sugli argomenti trattati si è deciso di cambiare radicalmente la logica, che si baserà non più su liste, bensì su alberi. I confronti, quindi, avverranno tra alberi, il che renderà meno laboriosa la gestione degli annidamenti e delle eliminazioni.

\subsection{Obiettivi di sprint}
Si sono discussi gli obiettivi degli sprint futuri e sì è deciso di assegnare a ogni membro del gruppo una parte delle attività da svolgere per raggiungerli. 

\subsection{Colloquio video con l'azienda}
Visti i miglioramenti apportati all'applicazione, e i dubbi scaturiti da discussioni interne, il gruppo ha deciso all'unanimità di chiedere un colloquio con l'azienda, comunicandolo a quest'ultima tramite l'apposito canale su Slack.

\newpage

\section{Riepilogo delle decisioni \hfil}
{
    \setlength{\freewidth}{\dimexpr\textwidth-4\tabcolsep}
    \renewcommand{\arraystretch}{1.5}
    \setlength{\aboverulesep}{0pt}
    \setlength{\belowrulesep}{0pt}
    \rowcolors{2}{AzzurroGruppo!10}{white}
    \begin{longtable}{L{.3\freewidth} L{.7\freewidth}}
        \toprule
        \rowcolor{AzzurroGruppo!30}
        \textbf{Codice} & \textbf{Decisione}\\
        \toprule
        \endhead

        VE\_\DataMeeting{}.1 &  -.\\
        VE\_\DataMeeting{}.2 &  Si è deciso di cambiare la logica dell'algoritmo, passando dal confronto di liste al confronto di alberi.\\

        VE\_\DataMeeting{}.3 &  Si è deciso come organizzare il lavoro tra i membri del gruppo.\\
        VE\_\DataMeeting{}.4 &  Si è deciso di chiedere un colloquio video all'azienda.\\
        \bottomrule
        \hiderowcolors
    \end{longtable}
}