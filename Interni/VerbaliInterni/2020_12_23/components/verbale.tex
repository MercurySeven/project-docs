\section{Informazioni generali}

\begin{itemize}

	\item \textbf{Canale di comunicazione:} Zoom;
	
	\item \textbf{Data:} \DataMeeting{};
	
	\item \textbf{Ora inizio:} 21;
	
	\item \textbf{Ora fine:} 22;
	
	\item \textbf{Segretario:} Matteo Pagotto;
	
	\item \textbf{Partecipanti:}
	
		\begin{itemize}
		
			\item Daniele Giachetto;
			\item Davide Albiero;
			\item Francesco De Marchi;
			\item Giosuè Calgaro;
			\item Lucrezia Gradi;
			\item Matteo Pagotto;
			\item Tommaso Poppi.
				 
		\end{itemize}

\end{itemize}

\section{Ordine del giorno}

\begin{itemize}

	\item\textbf{Punto della situazione sui documenti;}
	
	\item\textbf{Preparazione alla chiamata del 28-12-2020 con l'azienda.}

\end{itemize}

\newpage


\section{Resoconto}

\subsection{Punto della situazione sui documenti}

Il gruppo \gruppo{} ha svolto una breve verifica sulla situazione dei documenti, chiedendo lo stato di avanzamento ai membri interessati. Successivamente, sono state assegnate le stesure dei documenti restanti e le revisioni dei documenti fin'ora prodotti. In particolare,:
\begin{itemize}
\item Stesura del \PdQ{}: Francesco e Lucrezia;
\item Stesura dell'\AdR{}: Davide, Giosuè e Daniele;
\item Revisione delle \NdP{}: Davide e Francesco;
\item Revisione dello \SdF{}: Daniele;
\item Revisione del \PdP{}: Daniele e Lucrezia.
\end{itemize}
Nel mentre, Matteo e Tommaso continuano con la stesura del \PdP{}.


\subsection{Preparazione alla chiamata del 28-12-2020 con l'azienda}

Dopo la risposta dell'azienda all'email da noi inviata, nella quale viene fissato un meeting Zoom il 28-12-2020 alle 16:30, il gruppo ha discusso sulle domande da porre ai rappresentanti dell'azienda. Andando nel dettaglio, gli argomenti principali che vorremmo chiarire sono:
\begin{itemize}

\item Testing: chiarimento sul modo in cui verranno effettuati i test;

\item Requisiti: una più chiara esposizione dei requisiti obbligatori e opzionali richiesti;

\item Server: capire se il backend è compreso tra i requisiti obbligatori o se viene fornito dall'azienda;

\item Linguaggio: dopo una discussione tra i membri sui linguaggi di programmazione conosciuti e preferiti da ciascuno, il gruppo vorrebbe sapere perché scegliere Python, consigliato dall'azienda, e se è possibile decidere di usarne un'altro.

\end{itemize}


\newpage

\section{Riepilogo delle decisioni \hfil}
{
	\setlength{\freewidth}{\dimexpr\textwidth-4\tabcolsep}
	\renewcommand{\arraystretch}{1.5}
	\setlength{\aboverulesep}{0pt}
	\setlength{\belowrulesep}{0pt}
	\rowcolors{2}{AzzurroGruppo!10}{white}
	\begin{longtable}{L{.3\freewidth} L{.7\freewidth}}
		\toprule 
		\rowcolor{AzzurroGruppo!30}
		\textbf{Codice} & \textbf{Decisione}\\
		\toprule
		\endhead
		
		VI\_\DataMeeting{}.1 & Assegnazione stesura e revisione documenti. \\  
		VI\_\DataMeeting{}.2 & Chiarite le domande sul capitolato da fare all'incontro con l'azienda. \\
		
		\bottomrule
		\hiderowcolors
	\end{longtable}
