\section{Informazioni generali}

\begin{itemize}

	\item \textbf{Canale di comunicazione:} Zoom;
	
	\item \textbf{Data:} \DataMeeting{};
	
	\item \textbf{Ora inizio:} 21;
	
	\item \textbf{Ora fine:} 22;
	
	\item \textbf{Segretario:} Matteo Pagotto;
	
	\item \textbf{Partecipanti:}
	
		\begin{itemize}
		
			\item Daniele Giachetto;
			\item Davide Albiero;
			\item Francesco De Marchi;
			\item Giosuè Calgaro;
			\item Lucrezia Gradi;
			\item Matteo Pagotto;
			\item Tommaso Poppi.
				 
		\end{itemize}

\end{itemize}

\section{Ordine del giorno}

\begin{itemize}

	
	
	\item\textbf{Discussione avanzamento dei documenti;}
	
	\item\textbf{Dibattito tra modello agile e incrementale;}
	
	\item\textbf{Analisi dei requisiti opzionali;}

	\item\textbf{Assegnazione dei compiti rimasti.}

\end{itemize}

\newpage




\section{Resoconto}

\subsection{Discussione avanzamento dei documenti}

Ogni membro del gruppo ha esposto il lavoro svolto durante la stesura dei documenti agli altri, mostrando modifiche ed eventuali dubbi sul contenuto. 

\subsection{Dibattito tra modello agile e incrementale}

Durante l'esposizione della situazione del \PdP{} sono sorti dubbi sull'utilizzo della metodologia agile scelta. In particolare, chi si è occupato della stesura riteneva che il modello incrementale si adattasse più al lavoro svolto dal gruppo. Altri, invece, ritenevano che l'agile seguisse meglio l'idea per la quale la documentazione dovesse essere snellita e resa più leggera. Inoltre, il modo di lavorare del gruppo, anche se coinvolge caratteristiche del metodo incrementale, ne include anche dell'agile, in particolare la flessibilità. Alla fine, si è visto come l'agile rispecchi al meglio il modo di lavorare che il gruppo ha attualmente e che vuole mantenere durante il progetto.

\subsection{Analisi requisiti opzionali}

Durante l'esposizione dell'\AdR{} sono sorti dei dubbi sui requisiti opzionali, non chiariti dall'azienda. I membri del gruppo hanno pensato quindi di aggiungerne alcuni prendendo spunto dalla presentazione del capitolato, quali la gestione delle notifiche, la gestione delle condivisioni e del menù contestuale. 

\subsection{Assegnazione dei compiti rimasti}

Dopo l'esposizione dei capitolati, sono state segnate delle date entro le quali finire, o quasi, la stesura di tre documenti. In particolare, il \PdP{} entro il 5 gennaio, il \PdQ{} e l'\AdR{} entro il 7 gennaio. 


\newpage

\section{Riepilogo delle decisioni \hfil}
{
	\setlength{\freewidth}{\dimexpr\textwidth-4\tabcolsep}
	\renewcommand{\arraystretch}{1.5}
	\setlength{\aboverulesep}{0pt}
	\setlength{\belowrulesep}{0pt}
	\rowcolors{2}{AzzurroGruppo!10}{white}
	\begin{longtable}{L{.3\freewidth} L{.7\freewidth}}
		\toprule 
		\rowcolor{AzzurroGruppo!30}
		\textbf{Codice} & \textbf{Decisione}\\
		\toprule
		\endhead
		
		VI\_\DataMeeting{}.1 & Esposizione del lavoro svolto. \\  
		VI\_\DataMeeting{}.2 & Conferma della metodologia Agile. \\
		VI\_\DataMeeting{}.3 & Aggiunti dei requisiti opzionali. \\ 
		VI\_\DataMeeting{}.4 & Assegnazione compiti rimasti. \\
		
		\bottomrule
		\hiderowcolors
	\end{longtable}
