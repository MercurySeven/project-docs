\section{Capitolato C5}

\subsection{Informazioni generali}
\begin{itemize}
\item \textbf{Nome:} PORTACS;
\item \textbf{Proponente:} San Marco Informatica;
\item \textbf{Committenti:} \committenti{}.
\end{itemize}

\subsection{Descrizione del capitolato}
Il capitolato vuole realizzare un sistema-real-time dove vengono gestite più unità che hanno un punto di partenza e più punti di interesse.
Il server centrale dovrà indicare ad ogni unità la prossima "mossa" da fare in base ai prossimi punti di interesse e le posizioni delle altre unità. 
Inoltre, è richiesto che ci sia una mappa in real-time che consenta la visualizzazione di tutte le unità e dei loro movimenti all'interno della struttura.

\subsection{Finalità del progetto}
L'azienda consiglia di suddividere il prodotto nel seguente modo:
\begin{itemize}
\item il motore di calcolo che si occupa solamente di trovare il migliore percorso da far percorrere alla singola unità, in base a diversi algoritmi di ricerca operativa per l'ottimizzazione del percorso;
\item il server centrale, che gestisce la comunicazione con le unità e riceve dal motore di calcolo il percorso da far percorrere all'unità;
\item il visualizzatore/monitor real-time che dovrà fornire all'amministratore la disposizione in tempo reale di ogni unità che attualmente sta svolgendo un compito;
\item l'unità di lavoro dovrà ricevere dal server centrale le direzioni su dove spostarsi e informare il server su possibili ostacoli o imprevisti.
\end{itemize}

\subsection{Tecnologie interessate}
Il capitolato offre libera scelta su come implementare l'applicazione, il gruppo aveva pensato di usare le seguenti tecnologie:
\begin{itemize}
\item per il motore di calcolo si pensava di usare C++, per la sua velocità di esecuzione;
\item per il server che manda i comandi alle unità si era scelto Java;
\item le unità mobili anche loro usavano Java così da facilitare la connessione, attraverso socket, al server;
\item la mappa con l'aggiornamento real-time era stata pensata di realizzarla con ReactJs.
\end{itemize}

\subsection{Aspetti positivi}
\begin{itemize}
\item sicuramente il capitolato è molto interessante, inoltre offre la possibilità ad ogni componente del gruppo di svilupparsi in un'area di progetto specifica;
\item è un problema molto attuale in quanto la guida autonoma sarà una delle principali tecnologie nei prossimi anni.
\end{itemize}

\subsection{Criticità e fattori di rischio}
\begin{itemize}
\item come gruppo crediamo che ci sia molto lavoro sullo studio dell'invio dati e su come temporizzare le fasi di ricezione e invio dei dati;
\item la presentazione è risultata dispersiva, dando poca considerazione a dettagli che il gruppo riteneva necessari.  
\end{itemize}

\subsection{Valutazione finale}
Dalla presentazione effettuata dall'azienda il 2020-12-04, sono nati dei dubbi riguardo la scalabilità che l'applicazione deve avere. Questo ha portato ad una riflessione del gruppo sulla complessità del progetto, che ha deciso di non prendere in considerazione questo capitolato.