\section{Capitolato C2}

\subsection{Informazioni generali}
\begin{itemize}
\item \textbf{Nome:} EmporioLambda
\item \textbf{Proponente:} Red Babel
\item \textbf{Committenti:} \committenti{}
\end{itemize}

\subsection{Descrizione del capitolato}
Il capitolato richiede la realizzazione di un e-commerce.

\subsection{Finalità del progetto}
La finalità del progetto è realizzare un e-commerce in cui, lato server, viene utilizzato AWS. Le componenti che si dovranno sviluppare sono:
\begin{itemize}
\item Home page;
\item Product Listing pages;
\item Product Description pages;
\item Shopping Cart;
\item Checkout;
\item Account.
\end{itemize}

\subsection{Tecnologie interessate}
Il capitolato specifica le tecnologie da utilizzare:
\begin{itemize}
\item next.js (typescript) lato front-end;
\item Per back-end Amazon web service;
\item Per il monitoraggio amazon cloud watch;
\item Per la parte relativa al pagamento è richiesto di includere software di terze parti.
\end{itemize}

\subsection{Aspetti positivi}
\begin{itemize}
\item La gestione del server e delle strutture a micro servizi rende più facile la parte di back-end;
\item L'integrazione all'interno del progetto di un servizio di terze parti per la gestione dei pagamenti.
\end{itemize}

\subsection{Criticità e fattori di rischio}
\begin{itemize}
\item Il catalogo necessita di adattarsi ai tipi di prodotti e a ciò che è interessante poter visualizzare per questi.
\end{itemize}

\subsection{Valutazione finale}
Il gruppo non trova l'idea innovativa e abbastanza stimolante, inoltre il carico di lavoro potrebbe essere eccessivo.
