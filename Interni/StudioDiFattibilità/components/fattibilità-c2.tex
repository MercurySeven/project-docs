\section{Capitolato C2}

\subsection{Informazioni generali}
\begin{itemize}
\item \textbf{Nome:} \textit{EmporioLambda}:  piattaforma di e-commerce in stile Serverless;
\item \textbf{\glo{Proponente}:} Red Babel;
\item \textbf{Committenti:} \textit{Prof. \Tullio{}} e \textit{Prof. \Riccardo{}}.
\end{itemize}

\subsection{Descrizione del capitolato}
Il capitolato propone la realizzazione di un e-commerce. Il progetto dovrà essere sviluppato seguendo la filosofia dei micro servizi. Ogni servizio che compone l'e-commerce dovrà essere indipendente dagli altri, ciò consentirà di scalare il servizio molto più facilmente garantendo maggior stabilità all'e-commerce.

\subsection{Finalità del progetto}
La finalità del progetto è realizzare un e-commerce in cui, lato server, viene utilizzato \glo{AWS}. Le componenti che si dovranno sviluppare sono:
\begin{itemize}
\item home page;
\item product listing pages;
\item product description pages;
\item shopping cart;
\item checkout;
\item account.
\end{itemize}

\subsection{Tecnologie interessate}
Il capitolato specifica le tecnologie da utilizzare:
\begin{itemize}
\item Next.js (TypeScript) lato front-end;
\item Amazon Web Service, per la parte di back-end;
\item Amazon CloudWatch, per il monitoraggio;
\item per la parte relativa al pagamento è richiesto di includere \glo{software} di terze parti.
\end{itemize}

\subsection{Aspetti positivi}
\begin{itemize}
\item la gestione del server e delle strutture a micro servizi rende più facile la parte di back-end;
\item l'integrazione all'interno del progetto di un servizio di terze parti per la gestione dei pagamenti.
\end{itemize}

\subsection{Criticità e fattori di rischio}
\begin{itemize}
\item il catalogo necessita di adattarsi ai tipi di prodotti e a ciò che è interessante poter visualizzare per questi.
\end{itemize}

\subsection{Valutazione finale}
Il gruppo non trova l'idea innovativa e abbastanza stimolante, inoltre il carico di lavoro potrebbe essere eccessivo.
