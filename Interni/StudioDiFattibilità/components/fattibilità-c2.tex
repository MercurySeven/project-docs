\section{Capitolato C2}

\subsection{Informazioni generali}
\begin{itemize}
\item \textbf{Nome:} EmporioLambda;
\item \textbf{Proponente:} Red Babel;
\item \textbf{Committenti:} \committenti{}.
\end{itemize}

\subsection{Descrizione del capitolato}
Il capitolato propone la realizzazione di un e-commerce, queste piattaforme sono uniche dal punto di vista del percorso che effettua l'utente: comincia dal website/app entrando nella home, passando attraverso i vari prodotti, arrivando alla lista degli oggetti scelti, ed infine alla pagina di pagamento. 

\subsection{Finalità del progetto}
La finalità del progetto è realizzare un e-commerce in cui, lato server, viene utilizzato AWS. Le componenti che si dovranno sviluppare sono:
\begin{itemize}
\item home page;
\item product listing pages;
\item product description pages;
\item shopping cart;
\item checkout;
\item account.
\end{itemize}

\subsection{Tecnologie interessate}
Il capitolato specifica le tecnologie da utilizzare:
\begin{itemize}
\item Next.js (TypeScript) lato front-end;
\item Amazon Web Service, per la parte di back-end;
\item Amazon CloudWatch, per il monitoraggio;
\item per la parte relativa al pagamento è richiesto di includere software di terze parti.
\end{itemize}

\subsection{Aspetti positivi}
\begin{itemize}
\item La gestione del server e delle strutture a micro servizi rende più facile la parte di back-end;
\item L'integrazione all'interno del progetto di un servizio di terze parti per la gestione dei pagamenti.
\end{itemize}

\subsection{Criticità e fattori di rischio}
\begin{itemize}
\item Il catalogo necessita di adattarsi ai tipi di prodotti e a ciò che è interessante poter visualizzare per questi.
\end{itemize}

\subsection{Valutazione finale}
Il gruppo non trova l'idea innovativa e abbastanza stimolante, inoltre il carico di lavoro potrebbe essere eccessivo.
