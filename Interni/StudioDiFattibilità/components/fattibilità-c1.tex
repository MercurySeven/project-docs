\section{Capitolato C1}

\subsection{Informazioni generali}
\begin{itemize}
\item \textbf{Nome:} BlockCOVID
\item \textbf{Proponente:} Imola Informatica
\item \textbf{Committenti:} \committenti{}
\end{itemize}

\subsection{Descrizione del capitolato}
Il capitolato vuole realizzare un sistema che permetta di tutelare la salute e la sicurezza dei lavoratori dal rischio contagio e garantire la salubrità dell'ambiente di lavoro. E' chiesto dunque di realizzare un apparato che soddisfi:
\begin{itemize}
\item il tracciamento immutabile e certificato delle presenze in tempo reale nelle postazioni di lavoro di un laboratorio informatico, contrassegnate tramite dei tag RFID;
\item il tracciamento immutabile e certificato della pulizia delle postazioni.
\end{itemize}
\subsection{Finalità del progetto}
Il prodotto e' composto dalle seguenti parti:
\begin{itemize}
\item Il server centrale, gestisce le varie stanze e postazioni, deve poter:
\begin{itemize}
\item sapere in ogni momento se la postazione è occupata, prenotata oppure da pulire (condizioni per cui
lo studente/dipendente non può utilizzarla);
\item controllare quali postazioni sono prenotate (da chi) e bloccare le prenotazioni per una determinata stanza;
\item prevedere una tracciatura autenticata e tutti i cambiamenti di stato relativi alla pulizia della
postazione, nonché le informazioni su chi ha igienizzato la postazione, devono essere salvate su
memoria immutabile e certificata;
\item Deve essere possibile prenotare una postazione con granularità di 1 ora;
\item Il motore di calcolo che si occupa solamente di trovare il migliore percorso da far percorrere alla singola unità, in base a diversi algoritmi di ricerca operativa per l'ottimizzazione del percorso.
\end{itemize}
\item GUI lato amministratore:
\begin{itemize}
\item Definisce stanze e postazioni;
\item Crea le utenze ai dipendenti;
\item Inserisce i tag RFID e li associa alle rispettive postazioni.
\end{itemize}
\item GUI lato utente, deve permettere:
\begin{itemize}
\item Recupero lista delle postazioni libere;
\item Prenotazione di una postazione;
\item Tracciamento in tempo reale tramite tag RFID;
\item Pulizia di una postazione;
\item Storico delle postazioni occupate;
\item Storico delle postazioni igienizzate.
\end{itemize}
\end{itemize}

\subsection{Tecnologie interessate}
Il capitolato offre libera scelta su come implementare l'applicazione, anche se consigliano caldamente alcuni linguaggi, in modo da poter usufruire al meglio dei referenti aziendali:
\begin{itemize}
\item Java (versione 8 o superiori), Python o nodejs per lo sviluppo del server back-end;
\item protocolli asincroni per le comunicazioni app mobile-server;
\item Un sistema blockchain per salvare con opponibilità a terzi i dati di sanificazione;
\item IAAS Kubernetes o di un PAAS, Openshift o Rancher, per il rilascio delle componenti del server e la
gestione della scalabilità orizzontale.
\end{itemize}

\subsection{Aspetti positivi}
\begin{itemize}
\item Chiarezza e completezza del capitolato (sono già presenti i casi d'uso principali con relativo risparmio di tempo);
\item Stimolante per interfacciarsi con altre tecnologie; 
\item Non vi sono algoritmi complessi da sviluppare.
\end{itemize}

\subsection{Criticità e fattori di rischio}
\begin{itemize}
\item Struttura blockchain sconosciuta ai componenti del gruppo;
\item Elevato carico di lavoro, bisogna sviluppare sia app che server;
\item Comunicazioni tra app e server hanno necessità di encryption.
\end{itemize}

\subsection{Valutazione finale}
Le criticità trovate non sono insormontabili, ma dato il grande interesse di altri gruppi per questo capitolato si è deciso di non considerarlo.
