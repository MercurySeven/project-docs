\section{Capitolato C1}

\subsection{Informazioni generali}
\begin{itemize}
\item \textbf{Nome:} \textit{BlockCOVID}: supporto digitale al contrasto della pandemia;
\item \textbf{Proponente:} Imola Informatica;
\item \textbf{Committenti:} \textit{Prof. \Tullio{}} e \textit{Prof. \Riccardo{}}.
\end{itemize}

\subsection{Descrizione del capitolato}
Il capitolato vuole realizzare un sistema che permetta di tutelare la salute e la sicurezza dei lavoratori dal rischio contagio e garantire la sanificazione dell'ambiente di lavoro. È chiesto dunque di realizzare un apparato che soddisfi:
\begin{itemize}
\item il tracciamento immutabile e certificato delle presenze in tempo reale nelle postazioni di lavoro di un laboratorio informatico, contrassegnate tramite dei tag \glo{RFID};
\item il tracciamento immutabile e certificato della pulizia delle postazioni.
\end{itemize}
\subsection{Finalità del progetto}
Il prodotto è composto dalle seguenti parti:
\begin{itemize}
\item il server centrale, che gestisce le varie stanze e postazioni, deve:
\begin{itemize}
\item sapere in ogni momento se la postazione è occupata, prenotata oppure da pulire (condizioni per cui
lo studente/dipendente non può utilizzarla);
\item controllare quali postazioni sono prenotate (da chi) e bloccare le prenotazioni per una determinata stanza;
\item prevedere una tracciatura autenticata e tutti i cambiamenti di stato relativi alla pulizia della
postazione, nonché le informazioni su chi ha igienizzato la postazione, devono essere salvate su
memoria immutabile e certificata;
\item garantire di poter prenotare una postazione con granularità di 1 ora.
\end{itemize}
\item GUI utente, deve permettere:
\begin{itemize}
\item il recupero della lista con le postazioni libere;
\item di prenotare una postazione;
\item di segnare come pulita la postazione;
\item la visone dello storico delle postazioni occupate dall'utente;
\item la visone dello storico delle postazioni igienizzate dall'utente.
\end{itemize}
\end{itemize}

\subsection{Tecnologie interessate}
Il proponente offre libera scelta su come implementare l'applicazione, anche se consiglia di usare specifici linguaggi, in modo da poter usufruire al meglio dei referenti aziendali:
\begin{itemize}
\item Java (versione 8 o superiori), Python o Node.js per lo sviluppo del server back-end;
\item protocolli asincroni per le comunicazioni app mobile-server;
\item un sistema blockchain per salvare immutabilmente i dati di sanificazione;
\item IAAS Kubernetes o di un PAAS, Openshift o Rancher, per il rilascio delle componenti del server e la
gestione della scalabilità orizzontale.
\end{itemize}

\subsection{Aspetti positivi}
\begin{itemize}
\item chiarezza e completezza del capitolato (sono già presenti i casi d'uso principali con conseguente risparmio di tempo);
\item stimolante per interfacciarsi con altre tecnologie; 
\item non vi sono algoritmi complessi da sviluppare.
\end{itemize}

\subsection{Criticità e fattori di rischio}
\begin{itemize}
\item la struttura blockchain è sconosciuta ai componenti del gruppo;
\item le comunicazioni tra app e server hanno necessità di essere cifrate.
\end{itemize}

\subsection{Valutazione finale}
Le criticità trovate non sono insormontabili, ma dato il grande interesse di altri gruppi per questo capitolato si è deciso di non considerarlo.
