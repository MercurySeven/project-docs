\section{Capitolato C4}

\subsection{Informazioni generali}
\begin{itemize}
\item \textbf{Nome:} \textit{HD Viz}:  visualizzazione di dati multidimensionali;
\item \textbf{Proponente:} Zucchetti;
\item \textbf{Committenti:} \textit{Prof. \Tullio{}} e \textit{Prof. \Riccardo{}}.
\end{itemize}

\subsection{Descrizione del capitolato}
L'obbiettivo del capitolato è quello di progettare, attraverso la libreria D3.js, 4 tipi di grafici. Questi grafici devono far risaltare i possibili errori di inserimento da parte degli utenti.
Il capitolato porta l'esempio del controllo dei cedolini riguardanti gli stipendi di un dipendente. I cedolini richiedono la compilazione manuale da parte di una persona e spesso contengono errori di digitazione. I grafici richiesti dall'applicazione devono essere in grado di evidenziare valori al di fuori della norma, in modo che si possano avviare dei controlli e correggere l'errore.

\subsection{Finalità del progetto}
L'azienda propone di seguire il seguente flusso di operazioni:
\begin{enumerate}
\item L'utente effettua una query al database oppure legge i dati da un file \glo{CSV};
\item L'applicazione \textit{"HD Viz"} deve dare la possibilità di usare almeno le seguenti visualizzazioni:
    \begin{itemize}
    \item Scatter plot Matrix (fino ad un massimo di cinque dimensioni;
    \item Force Field;
    \item Heat Map (ordinando i punti nel grafico per evidenziare i "cluster" presenti);
    \item Proiezione Lineare Multi Asse.
    \end{itemize}
\end{enumerate}

Inoltre, vengono proposti i seguenti requisiti opzionali:
\begin{itemize}
\item utilizzo di altri grafici adatti alla visualizzazione di dati con più di tre dimensioni;
\item utilizzo di funzioni di calcolo della distanza diverse dalla distanza Euclidea in tutte le funzioni che dipendono da essa;
\item utilizzo di funzioni di "forza" diverse da quelle previste in automatico dal grafico "force based" di D3;
\item analisi automatiche per evidenziare situazioni di particolare interesse;
\item algoritmi di preparazione del dato per la visualizzazione.
\end{itemize}

\subsection{Tecnologie interessate}
Le tecnologie raccomandate sono:
\begin{itemize}
\item HTML/CSS/JavaScript per sviluppare la parte di front-end, utile all'utente per la visualizzazione dei vari grafici;
\item la libreria JavaScript D3.js per la creazione dei grafici da visualizzare all'utente;
\item un database SQL/NoSQL per la memorizzazione dei dati nel caso l'utente decidesse di leggere i dati da un database;
\item viene consigliato l’utilizzo di Java con server Tomcat o JavaScript con server Node.js, per la parte di back-end.
\end{itemize}

\subsection{Aspetti positivi}
\begin{itemize}
\item le tecnologie richieste per lo sviluppo sono di facile apprendimento e utilizzo;
\item la libreria JavaScript D3.js è molto ben documentata, su internet ci sono molte risorse utili per l'implementazione dei grafici richiesti.
\end{itemize}

\subsection{Criticità e fattori di rischio}
\begin{itemize}
\item un possibile problema è il fatto che lavorando con dati multidimensionali è difficile valutare la corretta visualizzazione nel grafico;
\item il capitolato non ha attirato la maggior parte dei membri del gruppo.
\end{itemize}

\subsection{Valutazione finale}
Nonostante la facile realizzazione del progetto, il gruppo ha deciso di orientarsi verso capitolati di altri proponenti.