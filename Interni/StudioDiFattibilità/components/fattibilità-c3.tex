\section{Capitolato C3}

\subsection{Informazioni generali}
\begin{itemize}
\item \textbf{Nome:} Gathering Detection Platform;
\item \textbf{Proponente:} Sync Lab;
\item \textbf{Committenti:} \committenti{}.
\end{itemize}

\subsection{Descrizione del capitolato}
Il capitolato vuole realizzare un sistema che permetta non solo di rappresentare mediante visualizzazione grafica zone potenzialmente a rischio di assembramento, ma addirittura cercare di prevenirle. E' quindi richiesto lo sviluppo di software per acquisire, monitorare, utilizzare e correlare tra loro tutti i dati e le informazioni generate dai sistemi e dai dispositivi installati ed operativi in specifiche zone, con l'intento di identificare i possibili eventi che concorrono all'insorgere di variazioni di flussi di utenti. 

\subsection{Finalità del progetto}
La finalità del progetto è di:
\begin{itemize}
\item realizzare dei software ‘contapersone’ che, date le immagini/stream delle videocamere installate sui mezzi misurino quante persone ci sono a bordo;
\item realizzare dei simulatori di altre sorgenti dati sia dei dati storici/in monitoraggio che dati previsionali
(ad esempio valutando nei prossimi 10 minuti il dato che arriverà un autobus con 20 persone presenti
in una specifica fermata lungo il suo tragitto);
\item avere la capacità di acquisire continuamente nel tempo e in modalità a bassa latenza delle informazioni
raccolte dai suddetti sistemi e dispositivi;
\item elaborare in tempo reale i dati acquisiti;
\item identificare, a partire dai dati acquisiti ed elaborati, eventi che nel tempo hanno portato all'insorgere di alterazioni significative del flusso di utenti;
\item prevedere l'insorgere, nel futuro, di variazioni significative nei flussi di persone in un luogo specifico.
\end{itemize}

\subsection{Tecnologie interessate}
Il capitolato non impone obblighi specifici per le tecnologie da utilizzare, ma ne consiglia comunque alcune:
\begin{itemize}
\item utilizzo di Java e Angular per lo sviluppo delle parti di back-end e di front-end della componente Web
Application del sistema;
\item per la gestione delle mappe (heatmap ecc.) il framework Leaflet;
\item utilizzo di protocolli asincroni per le comunicazioni tra le diverse componenti;
\item utilizzo del pattern Publisher/Subscriber, e adozione del protocollo MQTT (‘MQ Telemetry Transport
or Message Queue Telemetry Transport’), caratterizzato per essere open, di facile implementazione
e ampia diffusione in applicazioni M2M (MachineToMachine) e IoT (Internet of Things).
\end{itemize}

\subsection{Aspetti positivi}
\begin{itemize}
\item Il gruppo trova le tecnologie proposte innovative ed interessanti;
\item L'azienda assicura l'affiancamento di professionisti per il corretto sviluppo del progetto.
\end{itemize}

\subsection{Criticità e fattori di rischio}
\begin{itemize}
\item Non è spiegato come implementare l'intelligenza artificiale per il riconoscimento delle facce;
\item Il gruppo trova che l'elaborazione dei dati e correlazione degli stessi sia molto difficile da realizzare;
\item Nessun componente del gruppo ha conoscenza delle tecnologie proposte;
\item La fase di test potrebbe essere complessa;
\item In generale la proposta del capitolato pare complessa.
\end{itemize}

\subsection{Valutazione finale}
Le criticità trovate, insieme alla complessità con cui appare il progetto hanno spinto i componenti del gruppo a non optare per tale capitolato.
