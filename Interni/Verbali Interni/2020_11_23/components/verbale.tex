\section{Informazioni generali}

\begin{itemize}

	\item \textbf{Canale di comunicazione:} Zoom;
	
	\item \textbf{Data:} \DataMeeting{};
	
	\item \textbf{Ora inizio:} 21;
	
	\item \textbf{Ora fine:} 22;
	
	\item \textbf{Segretario:} Daniele Giachetto;
	
	\item \textbf{Partecipanti:}
	
		\begin{itemize}
		
			\item Daniele Giachetto;
			\item Davide Albiero;
			\item Francesco De Marchi;
			\item Giosuè Calgaro;
			\item Lucrezia Gradi;
			\item Matteo Pagotto;
			\item Tommaso Poppi.
				 
		\end{itemize}

\end{itemize}

\section{Ordine del giorno}

\begin{itemize}

	\item\textbf{Discussione e Realizzazione template \LaTeX;}

	\item\textbf{Impostazione repository GitHub;}

	\item\textbf{Organizzazione milestones;}

	\item\textbf{Spartizione carico studio di fattibilità.}

\end{itemize}

\newpage


\section{Resoconto}

\subsection{Discussione e Realizzazione template \LaTeX}

È stato creato un template per i documenti che verranno generati con \LaTeX. È stata inoltre fornita ai membri con lacune nell'uso del linguaggio una breve spiegazione delle basi di \LaTeX.

\subsection{Impostazione repository GitHub}

È stato deciso di creare due repository remote per il versionamento, una dedicata alla elaborazione e mantenimento dei documenti ed una per il futuro codice. Si è inoltre deciso di utilizzare il workflow feature branch per la repository dei documenti mentre il workflow gitflow per la repository del codice.

\subsection{Organizzazione milestones}

\begin{itemize}

	\item \textbf{2021-01-11:} Consegna documentazione;
	\item \textbf{2021-01-04:} Completamento documenti(salvo verbali);
	\item \textbf{2020-12-21:} Verifica documenti redatti;
	\item \textbf{2020-12-18:} Stesura di lettera di presentazione, studio di fattibilità, norme di progetto e piano di progetto.
	
\end{itemize}

\subsection{Spartizione carico studio di fattibilità}

Si è deciso di approcciare lo Studio di Fattibilità con la seguente modalità: 
\begin{itemize}
	\item Spartizione della esposizione dei capitolati tra i singoli membri;
	\item Studio di ogni membro di tutti i capitolati;
	\item Esposizione dei capitolati e discussione criticità ed elementi positivi;
	\item Stesura del documento Studio di Fattibilità.
\end{itemize}
La divisione dei capitolati dunque risulta la seguente:

\begin{itemize}

	\item C1 - Tommaso;
	\item C2 - Lucrezia;
	\item C3 - Daniele;
	\item C4 - Matteo;
	\item C5 - Giosuè;
	\item C6 - Davide;
	\item C7 - Francesco.
\end{itemize}

\newpage

\section{Riepilogo delle decisioni \hfil}
{
	\setlength{\freewidth}{\dimexpr\textwidth-4\tabcolsep}
	\renewcommand{\arraystretch}{1.5}
	\setlength{\aboverulesep}{0pt}
	\setlength{\belowrulesep}{0pt}
	\rowcolors{2}{AzzurroGruppo!10}{white}
	\begin{longtable}{L{.3\freewidth} L{.7\freewidth}}
		\toprule 
		\rowcolor{AzzurroGruppo!30}
		\textbf{Codice} & \textbf{Decisione}\\
		\toprule
		\endhead
		
		VI\_\DataMeeting{}.1 & Realizzato il template \LaTeX. \\  
		VI\_\DataMeeting{}.2 & Realizzate le repository GitHub e scelto il loro workflow. \\ 
		VI\_\DataMeeting{}.3 & Approvate le milestones interne. \\
		VI\_\DataMeeting{}.4 & Scelto l'approccio per realizzare lo studio di fattibilità. \\  		
		
		\bottomrule
		\hiderowcolors
	\end{longtable}
