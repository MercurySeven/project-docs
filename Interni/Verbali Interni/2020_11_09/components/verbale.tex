\section{Informazioni generali}

\begin{itemize}

	\item \textbf{Canale di comunicazione:} Zoom;
	
	\item \textbf{Data:} \DataMeeting{};
	
	\item \textbf{Ora inizio:} 21;
	
	\item \textbf{Ora fine:} 22;
	
	\item \textbf{Segretario:} Francesco De Marchi;
	
	\item \textbf{Partecipanti:}
	
		\begin{itemize}
		
			\item Daniele Giachetto;
			\item Davide Albiero;
			\item Francesco De Marchi;
			\item Giosuè Calgaro;
			\item Lucrezia Gradi;
			\item Matteo Pagotto;
			\item Tommaso Poppi.
				 
		\end{itemize}

\end{itemize}

\section{Ordine del giorno}

\begin{itemize}

	\item\textbf{ Presentazione membri del gruppo;}

	\item\textbf{ Votazione su canali di comunicazione da utilizzare;}

	\item\textbf{ Impostazione dei canali di comunicazione;}

	\item\textbf{ Votazione sulla registrazione delle chiamate;}

	\item\textbf{ Confronto sugli orari e sulla frequenza delle riunioni.}


\end{itemize}
\newpage

\section{Resoconto}

\subsection{Presentazione membri del gruppo}

I membri del gruppo si sono presentati, permettendo quindi di capire gli esami arretrati, le esperienze lavorative e scolastiche ed inoltre gli impegni di ogni membro.

\subsection{Votazione su canali di comunicazione da utilizzare}

Dopo una discussione nella quale sono stati confrontati diversi software, sono stati scelti Slack per la comunicazione testuale e Zoom per la comunicazione audio e video.

\subsection{Impostazione dei canali di comunicazione}

È stato creato un canale Slack ed è stato opportunamente impostato in vari sotto canali per permettere una migliore organizzazione ed una più chiara comunicazione tra membri.

\subsection{Votazione sulla registrazione delle chiamate}

Sotto proposta di Daniele è stato deciso all'unanimità di registrare le chiamate, questo perché la registrazione ed il salvataggio delle chiamate risultano un peso trascurabile mentre i benefici che esse possono portare sono diversi.

\subsection{Confronto sugli orari e sulla frequenza delle riunioni}

Creazione di un documento con gli impegni di prima necessità di ogni singolo membro del gruppo in modo da poter decidere quando fare le chiamate. Il gruppo si riunirà con cadenza settimanale, di lunedì, in modo da poter monitorare strettamente il progresso delle attività.


\section{Riepilogo delle decisioni \hfil}
{
	\setlength{\freewidth}{\dimexpr\textwidth-4\tabcolsep}
	\renewcommand{\arraystretch}{1.5}
	\setlength{\aboverulesep}{0pt}
	\setlength{\belowrulesep}{0pt}
	\rowcolors{2}{AzzurroGruppo!10}{white}
	\begin{longtable}{L{.3\freewidth} L{.7\freewidth}}
		\toprule 
		\rowcolor{AzzurroGruppo!30}
		\textbf{Codice} & \textbf{Decisione}\\
		\toprule
		\endhead
		
		VI\_\DataMeeting{}.1 & Presentazione membri \\  
		VI\_\DataMeeting{}.2 & \textit{Zoom} e \textit{Slack} scelti come mezzi di comunicazione \\ 
		VI\_\DataMeeting{}.3 & Creazione meeting ricorrente Zoom e creazione di vari canali su Slack \\
		VI\_\DataMeeting{}.4 & Approvata la registrazione degli incontri \\  
		VI\_\DataMeeting{}.5 & Riunioni del gruppo a cadenza settimanale \\
		
		
		\bottomrule
		\hiderowcolors
	\end{longtable}
}
