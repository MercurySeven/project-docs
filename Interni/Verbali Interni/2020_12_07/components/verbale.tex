\section{Informazioni generali}

\begin{itemize}

	\item \textbf{Canale di comunicazione:} Zoom;
	
	\item \textbf{Data:} \DataMeeting{};
	
	\item \textbf{Ora inizio:} 21;
	
	\item \textbf{Ora fine:} 23;
	
	\item \textbf{Segretario:} Daniele Giachetto;
	
	\item \textbf{Partecipanti:}
	
		\begin{itemize}
		
			\item Daniele Giachetto;
			\item Davide Albiero;
			\item Francesco De Marchi;
			\item Giosuè Calgaro;
			\item Lucrezia Gradi;
			\item Matteo Pagotto;
			\item Tommaso Poppi.
				 
		\end{itemize}

\end{itemize}

\section{Ordine del giorno}

\begin{itemize}

	\item\textbf{Spiegazione struttura dei repository GitHub;}

	\item\textbf{Raccolta delle domande da fare all'azienda per il capitolato 7;}

	\item\textbf{Visualizzazione del documento \SdF{};}
	
	\item\textbf{Assegnazione del documento \PdP{}.}

\end{itemize}

\newpage


\section{Resoconto}

\subsection{Spiegazione struttura dei repository GitHub}

È stato spiegato ad ogni membro del gruppo come verrà gestito il repository GitHub riguardante la documentazione. Sono state mostrate le pratiche che ogni componente del gruppo dovrà seguire per caricare i propri documenti. Il modo con cui verrà effettuato il merge nel ramo main sarà attraverso Pull Requests che verranno approvate dal \RdP.

\subsection{Raccolta delle domande da fare all'azienda per il capitolato 7}

È stata scritta una mail contenente tutte le domande che ogni membro del gruppo voleva fare all'azienda riguardante il capitolato \textit{\NomeProgetto}.
La mail verrà spedita il giorno 2020-12-09.

\subsection{Visualizzazione del documento \SdF{}}

È stato mostrato il documento riguardante allo \SdF\ scritto da Davide e Giosuè. Daniele si è proposto di verificare lui stesso il documento una volta creata la Pull Request su GitHub.

\subsection{Assegnazione del documento \PdP{}}

Dopo aver svolto individualmente lo studio del \PdP, Tommaso e Matteo si sono proposti di cominciare la stesura del documento.

\newpage

\section{Riepilogo delle decisioni \hfil}
{
	\setlength{\freewidth}{\dimexpr\textwidth-4\tabcolsep}
	\renewcommand{\arraystretch}{1.5}
	\setlength{\aboverulesep}{0pt}
	\setlength{\belowrulesep}{0pt}
	\rowcolors{2}{AzzurroGruppo!10}{white}
	\begin{longtable}{L{.3\freewidth} L{.7\freewidth}}
		\toprule 
		\rowcolor{AzzurroGruppo!30}
		\textbf{Codice} & \textbf{Decisione}\\
		\toprule
		\endhead
		
		VI\_\DataMeeting{}.1 & Fornite nozioni base sullo strumento GitHub. \\  
		VI\_\DataMeeting{}.2 & Inviata la mail contenente delle domande all'azienda. \\ 
		VI\_\DataMeeting{}.3 & Daniele verificherà il documento \SdF. \\
		VI\_\DataMeeting{}.4 & Assegnata la stesura del documento \PdP. \\  		
		
		\bottomrule
		\hiderowcolors
	\end{longtable}
