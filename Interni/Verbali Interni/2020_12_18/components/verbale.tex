\section{Informazioni generali}

\begin{itemize}

	\item \textbf{Canale di comunicazione:} Zoom;
	
	\item \textbf{Data:} \DataMeeting{};
	
	\item \textbf{Ora inizio:} 21;
	
	\item \textbf{Ora fine:} 22;
	
	\item \textbf{Segretario:} Matteo Pagotto;
	
	\item \textbf{Partecipanti:}
	
		\begin{itemize}
		
			\item Daniele Giachetto;
			\item Davide Albiero;
			\item Francesco De Marchi;
			\item Giosuè Calgaro;
			\item Lucrezia Gradi;
			\item Matteo Pagotto;
			\item Tommaso Poppi.
				 
		\end{itemize}

\end{itemize}

\section{Ordine del giorno}

\begin{itemize}

	\item\textbf{Richiesta videoconferenza con l'azienda proponente;}
	
	\item\textbf{Discussione punti esposti nella chiamata del 2020-12-17 con il prof. Tullio Vardanega;}

	\item\textbf{Discussione organigramma;}

	\item\textbf{Script Glossario.}

\end{itemize}

\newpage


\section{Resoconto}

\subsection{Richiesta videoconferenza con l'azienda proponente}

È stata fatta richiesta via email di un incontro con l'azienda proponente per discutere meglio i requisiti del capitolato \textit{SSD: Soluzioni di sincronizzazione desktop} e la documentazione fornitaci.  
La mail verrà spedita il giorno 2020-12-21.

\subsection{Discussione chiamata del 2020-12-17}

Sono stati discussi i punti rilevanti emersi durante la chiamata Zoom del 2020-12-17 con il prof. Tullio Vardanega:
\begin{itemize}
	\item \textbf{scadenze}: le scadenze diventano date minime di consegna oltre le quali ogni gruppo, previa richiesta, è libero di scegliere le proprie 
	\item \textbf{documenti}: è stato fissato un numero massimo di pagine per ogni documento e mantenedo i documenti allineati con l'effettivo stato di avanzamento del progetto. 
\end{itemize}

\subsection{Discussione organigramma}

Dopo una discussione sulla durata del ruolo che ricopre ogni membro del gruppo, è stato deciso di effettuarne il cambio ogni 3 settimane.
Questo per poter favorire un buon numero di riassegnazioni e al tempo stesso permettere al singolo individuò di acquisire famigliarità con il ruolo assegnatogli.

\subsection{Script Glossario}

È stato esposto ai membri del gruppo il funzionamento dello script creato da Davide e Francesco per la generazione e la gestione automatica del glossario.

\newpage

\section{Riepilogo delle decisioni \hfil}
{
	\setlength{\freewidth}{\dimexpr\textwidth-4\tabcolsep}
	\renewcommand{\arraystretch}{1.5}
	\setlength{\aboverulesep}{0pt}
	\setlength{\belowrulesep}{0pt}
	\rowcolors{2}{AzzurroGruppo!10}{white}
	\begin{longtable}{L{.3\freewidth} L{.7\freewidth}}
		\toprule 
		\rowcolor{AzzurroGruppo!30}
		\textbf{Codice} & \textbf{Decisione}\\
		\toprule
		\endhead
		
		VI\_\DataMeeting{}.1 & Inviata la mail all'azienda. \\  
		VI\_\DataMeeting{}.2 & Ogni membro è aggiornato sulle nuove linee guida del progetto di Ingegneria del Software. \\
		VI\_\DataMeeting{}.3 & Ogni membro sono aggiornati sulla nuova durata di rotazione dei ruoli. \\ 
		VI\_\DataMeeting{}.4 & Ogni membro sanno utilizzare il nuovo script per il glossario. \\
		
		\bottomrule
		\hiderowcolors
	\end{longtable}
