\section{Infomazioni generali}

\begin{itemize}

	\item \textbf{Canale di comunicazione:} Zoom;
	
	\item \textbf{Data:} \DataMeeting{};
	
	\item \textbf{Ora inizio:} 21;
	
	\item \textbf{Ora fine:} 22;
	
	\item \textbf{Segretario:} Lucrezia Gradi;
	
	\item \textbf{Partecipanti:}
	
		\begin{itemize}
		
			\item Daniele Giachetto;
			\item Davide Albiero;
			\item Francesco De Marchi;
			\item Giosuè Calgaro;
			\item Lucrezia Gradi;
			\item Matteo Pagotto;
			\item Tommaso Poppi.
				 
		\end{itemize}

\end{itemize}

\section{Ordine del giorno}

\begin{itemize}

	\item\textbf{Nome del gruppo;}

	\item\textbf{Logo del gruppo;}

	\item\textbf{Creazione mail del gruppo;}

	\item\textbf{Votazione su strumenti di versionamento e scrittura collaborativa;}
	
	\item\textbf{Analisi dei documenti per la Revisione dei Requisiti.}


\end{itemize}

\newpage

\section{Resoconto}

\subsection{Nome del gruppo}

Dopo una attenta riflessione individuale nella quale ogni membro ha pensato a dei nomi per il gruppo, essi sono stati discussi insieme e tra i molti nomi proposti,il gruppo ha scelto all’unanimità “\textit{\Gruppo{}}”.

\subsection{Logo del gruppo}

È stato approvato il logo del gruppo disegnato da Lucrezia, esso presenta due versioni possibili versioni, una che presenta il testo "\Gruppo{}" adatta alle intestazioni dei documenti mentre l'altra senza nessun testo usata per le miniature

\subsection{Creazione mail del gruppo}

Il gruppo, all'unanimità, ha approvato "\textit{\Mail{}}" come email ufficiale.

\subsection{Votazione su strumenti di versionamento e per stesura documenti}

È stato deciso di utilizzare Git come sistema di versionamento del progetto e GitHub come piattaforma di versionamento. Per la stesura dei documenti vi era molte alternative ma si è deciso di optare per l'utilizzo di \LaTeX


\subsection{Analisi dei documenti per la Revisione dei Requisiti}

Il gruppo ha analizzato i diversi documenti ed ha infine deciso di concentrarsi sullo studio di fattibilità poichè avrebbe richiesto una maggiore interazione di tutti i membri.

\section{Riepilogo delle decisioni \hfil}
{
	\setlength{\freewidth}{\dimexpr\textwidth-4\tabcolsep}
	\renewcommand{\arraystretch}{1.5}
	\setlength{\aboverulesep}{0pt}
	\setlength{\belowrulesep}{0pt}
	\rowcolors{2}{AzzurroGruppo!10}{white}
	\begin{longtable}{L{.3\freewidth} L{.7\freewidth}}
		\toprule 
		\rowcolor{AzzurroGruppo!30}
		\textbf{Codice} & \textbf{Decisione}\\
		\toprule
		\endhead
		
		VI\_\DataMeeting{}.1 & \textit{\Gruppo{}} scelto come nome del gruppo. \\  
		VI\_\DataMeeting{}.2 & Approvato il logo del gruppo. \\ 
		VI\_\DataMeeting{}.3 & \textit{\Mail{}} scelta come email del gruppo. \\
		VI\_\DataMeeting{}.4 & Approvato \textit{GitHub} come sistema di versionamento e \LaTeX per la stesura della documentazione. \\  
		VI\_\DataMeeting{}.5 & Rilevati i punti critici dei documenti da realizzare. \\
		
		
		\bottomrule
		\hiderowcolors
	\end{longtable}
}
