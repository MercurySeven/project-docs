\section{Processi primari}
\subsection{Processo di fornitura}

Il processo di fornitura viene svolto al fine di comprendere al meglio la richiesta di appalto del \glo{proponente}: viene svolta un'analisi approfondita che ne stabilisce criticità e punti di forza, i quali verranno spiegati nello \SdF{}. Per sancire il legame tra \glo{proponente} e fornitore è necessaria la definizione di un contratto, che formalizza le fasi della realizzazione del prodotto finale, tra cui consegna e manutenzione. Per instaurare un rapporto di collaborazione proficuo, il gruppo intende mantenere un dialogo costante col \glo{proponente} in modo da accordarsi sugli aspetti che il prodotto dovrà soddisfare, definendo vincoli e \glo{requisiti} e stimandone costi e tempistiche.
Le fasi che compongono il suddetto processo sono: 
\begin{itemize}
	\item avvio;
	\item approntamento di risposte alle richieste;
	\item contrattazione;
	\item pianificazione;
	\item esecuzione e controllo;
	\item revisione e valutazione;
	\item consegna e completamento.
\end{itemize}

Il documento include le norme a cui i membri del gruppo \Gruppo{} devono fare riferimento al fine di ottenere il ruolo di fornitori del \glo{proponente} \proponente{} e dei committenti \textit{Prof. Tullio Vardanega} e \textit{Prof. Riccardo Cardin}. Ognuno è tenuto ad attenersi a quanto scritto nella suddetta sezione durante tutte le fasi di sviluppo del prodotto \progetto{}.

\subsubsection{Studio di fattibilità}
Lo \SdF{} è un documento che descrive le motivazioni che hanno spinto il gruppo \gruppo{} a scegliere un capitolato tra quelli proposti e viene redatto dopo un'attenta analisi svolta dai membri.
Per ogni capitolato sono indicate:
\begin{itemize}
	\item \textbf{Informazioni generali}: elenco delle informazioni di base, quali nome del progetto, il \glo{proponente} e il committente;
	\item \textbf{Descrizione e finalità}: sintesi del prodotto, comprendente le caratteristiche principali che vengono richieste e l'obiettivo da raggiungere;
	\item \textbf{Tecnologie interessate}: elenco delle tecnologie richieste per lo svolgimento del prodotto;
	\item \textbf{Aspetti positivi, criticità e fattori di rischio}: considerazioni fatte dal gruppo riguardanti gli aspetti positivi e sulle criticità del capitolato;
	\item \textbf{Conclusioni}: ragioni per le quali il gruppo accetta o rifiuta il capitolato.
\end{itemize}

\subsubsection{Altra documentazione fornita}

Il gruppo fornisce all'azienda \proponente{} e ai committenti \textit{Prof. Tullio Vardanega} e \textit{Prof. Riccardo Cardin} i seguenti documenti, essenziali per tracciare le attività svolte:
\begin{itemize}
	\item \textbf{\AdR{}}: documento che contiene l'analisi approfondita del capitolato scelto, comprendente tutti i \glo{requisiti} e i \glo{casi d'uso} individuati;
	\item \textbf{\PdP{}}: documento nel quale viene pianificato nel dettaglio il modo di lavorare del gruppo, contenente un preventivo riguardante tempistiche, l'analisi dei rischi, la pianificazione delle attività e il consuntivo;
	\item \textbf{\PdQ{}}: documento dove vengono stabilite e descritte le modalità di \glo{verifica} e \glo{validazione}, in modo da garantire la qualità del prodotto.
\end{itemize}

\subsubsection{Strumenti}

\setlength{\freewidth}{\dimexpr\textwidth-0\tabcolsep}
	\renewcommand{\arraystretch}{1.5}
	\setlength{\aboverulesep}{0pt}
	\setlength{\belowrulesep}{0pt}
	\rowcolors{2}{AzzurroGruppo!10}{white}
	\begin{longtable}{L{.210\freewidth} L{.6\freewidth} L{.080\freewidth}}
		\toprule 
		\rowcolor{AzzurroGruppo!30}
		\textbf{Nome} & \textbf{Descrizione} \\
		\toprule
		\endhead		
		\textbf{Gantt Project} & Strumento utilizzato per la realizzazione di  diagrammi di Gantt. Questi sono utili per tenere traccia delle attività coinvolte nella realizzazione del prodotto e delle relazioni tra loro, permettendo la visualizzazione delle tempistiche di avanzamento del lavoro in maniera immediata. \\
		\bottomrule
		\hiderowcolors
		\caption{Strumenti utilizzati nel processo di fornitura}
	\end{longtable}


\subsection{Processo di sviluppo}
Durante il processo di sviluppo vengono descritte tutte le attività e i compiti che devono essere svolti in modo da realizzare al meglio il prodotto richiesto dal \glo{proponente}: una volta definiti i requisiti di sviluppo, i vincoli tecnologici e i vincoli di design ci si aspetta di realizzare un prodotto finale che, oltre a superare i \glo{test}, soddisfi i \glo{requisiti} e le richieste del \glo{proponente}. Le attività coinvolte riguardano l'Analisi dei \ignore{Requisiti}, la progettazione e la codifica.
\subsubsection{Analisi dei requisiti}
L' \AdR{} è un documento formale, redatto dagli \anas{}, dove vengono individuati tutti i \glo{requisiti} che il \glo{proponente} richiede per la realizzazione del prodotto. Stenderlo in maniera efficace è molto importante in quanto coinvolto in diverse fasi della realizzazione del prodotto: oltre che definire funzionalità e \glo{requisiti} individuati e concordati col cliente, fornisce ai \progs{} riferimenti precisi e affidabili, ai \vers{} riferimenti per l'attività di controllo dei \glo{test} ed è la base dalla quale partire per eventuali raffinamenti successivi, garantendo un continuo miglioramento del prodotto. 
Ogni requisito può essere ricavato da diverse fonti: 
\begin{itemize}
	\item \textbf{Capitolati d'Appalto}: grazie alla lettura approfondita del capitolato scelto vengono individuati dei \glo{requisiti};
	\item \textbf{\glo{Casi d'uso}}: \glo{requisiti} estrapolati da uno o più \glo{casi d'uso}; 
	\item \textbf{Verbali}: i \glo{requisiti} possono emergere sia dalle riunioni interne effettuate dagli \anas{}, sia da contatti o incontri con i rappresentanti dell'azienda \glo{proponente}.
\end{itemize}
\paragraph*{Struttura} 
La struttura che contraddistingue l'\AdR{} è:
\begin{itemize}
	\item \textbf{Descrizione}: contiene informazioni riguardanti il prodotto, le componenti presenti all'interno del \glo{sistema}, la piattaforma d'esecuzione e i vincoli di progettazione individuati;
	\item \textbf{\glo{Casi d'uso}}: vengono identificati gli attori che interagiscono con le componenti del \glo{sistema} e le interazioni tra \glo{sistema}, attori ed elementi esterni. Approfondimento nel paragrafo \S{}\ref{Class_uso};
	\item \textbf{\glo{Requisiti}}: rappresentati in una tabella che contiene:
	\begin{itemize}
 		\item tracciamento dei \glo{requisiti}, suddivisi in obbligatori, desiderabili e opzionali, individuati nel \glo{sistema} durante l'analisi; approfondimento nel paragrafo \S{}\ref{Class_req};
		\item tracciamento fonte-\glo{requisiti} e \glo{requisiti}-fonte. 
	\end{itemize}
\end{itemize} 
\paragraph*{Classificazione dei requisiti} 
\label{Class_req}
La classificazione dei \glo{requisiti} verrà effettuata mediante la seguente codifica:\newline \newline
\centerline{\textbf{R[Tipo][Priorità][Numero]}}

\setlength{\freewidth}{\dimexpr\textwidth-0\tabcolsep}
	\renewcommand{\arraystretch}{1.5}
	\setlength{\aboverulesep}{0pt}
	\setlength{\belowrulesep}{0pt}
	\rowcolors{2}{AzzurroGruppo!10}{white}
	\begin{longtable}{L{.15\freewidth} L{.6\freewidth} L{.080\freewidth}}
		\toprule 
		\rowcolor{AzzurroGruppo!30}
		\textbf{Nome} & \textbf{Descrizione} \\
		\toprule
		\endhead		
		\textbf{R} & abbrevia "Requisito" \\
		\textbf{Tipo} & può essere: \begin{itemize}
		\item \textbf{F:} requisito funzionale, ossia la definizione di una particolare caratteristica che deve essere inclusa nel \glo{software};
		\item \textbf{Q:} requisito di qualità, relativi alle regole di qualità del \glo{software} (efficienza ed efficacia);
		\item \textbf{V:} requisito di vincolo che rappresenta dei vincoli avanzati dal \glo{proponente}.
	\end{itemize}  \\
		\textbf{Priorità} & \begin{itemize}
		\item \textbf{O:} requisito obbligatorio;
		\item \textbf{D:} requisito desiderabile;
		\item \textbf{F:} requisito facoltativo.
	\end{itemize} \\
		\textbf{Numero } & codice identificativo \\
		\bottomrule
		\hiderowcolors
		\caption{Descrizione elementi che classificano i requisiti}
	\end{longtable}


\paragraph*{Classificazione dei casi d'uso} 
\label{Class_uso}
La convenzione prescelta per la rappresentazione dei \glo{casi d'uso} è la seguente: \newline \newline
\centerline{\textbf{UC[CodiceBase](.[CodiceSottoCaso])*}}

\setlength{\freewidth}{\dimexpr\textwidth-0\tabcolsep}
	\renewcommand{\arraystretch}{1.5}
	\setlength{\aboverulesep}{0pt}
	\setlength{\belowrulesep}{0pt}
	\rowcolors{2}{AzzurroGruppo!10}{white}
	\begin{longtable}{L{.2\freewidth} L{.6\freewidth} L{.080\freewidth}}
		\toprule 
		\rowcolor{AzzurroGruppo!30}
		\textbf{Nome} & \textbf{Descrizione} \\
		\toprule
		\endhead		
		\textbf{UC} & acronimo di "Use Case" \\
		\textbf{CodiceBase} & identificativo del caso d'uso generico \\
		\textbf{CodiceSottoCaso} & identificativo opzionale per gli eventuali sotto casi del caso d'uso \\
		\bottomrule
		\hiderowcolors
		\caption{Descrizione elementi che classificano i casi d'uso}
	\end{longtable}

Ogni caso d'uso possiede inoltre la seguente struttura:
\begin{itemize}
	\item \textbf{Identificativo:} codice del caso d'uso prescelto secondo quanto stabilito dalla convenzione enunciata sopra;
	\item \textbf{Nome:} stringa testuale relativa al titolo del caso d'uso posta immediatamente dopo all'identificativo;
	\item \textbf{Descrizione grafica:} realizzata impiegando il formalismo dell'UML 2.0;
	\item \textbf{Attori:} un attore è tutto ciò che è esterno al \glo{sistema} e con il quale esso interagisce. Possono essere individuate due tipologie di attori:
	\begin{itemize}
		\item \textbf{Attore Primario:} interagisce con il \glo{sistema} per il raggiungimento di un obiettivo specifico;
		\item \textbf{Attore Secondario:} entità esterna al \glo{sistema} che aiuta l'attore primario nel raggiungimento del suo scopo.
	\end{itemize}
	\item \textbf{Precondizione:} specifica le condizioni del \glo{sistema} prima del verificarsi degli eventi del caso d'uso;
	\item \textbf{Postcondizione:} specifica le condizioni del \glo{sistema} dopo il verificarsi degli eventi del caso d'uso;
	\item \textbf{Scenario principale:} rappresentazione attraverso elenco numerato del flusso degli eventi;
	\item \textbf{Estensioni:} opzionali, impiegati per modellare scenari alternativi. Al verificarsi di una determinata condizione, il caso d'uso ad essa collegata viene interrotto.
\end{itemize}
\paragraph*{Qualità dei requisiti} 

\setlength{\freewidth}{\dimexpr\textwidth-0\tabcolsep}
	\renewcommand{\arraystretch}{1.5}
	\setlength{\aboverulesep}{0pt}
	\setlength{\belowrulesep}{0pt}
	\rowcolors{2}{AzzurroGruppo!10}{white}
	\begin{longtable}{L{.15\freewidth} L{.6\freewidth} L{.080\freewidth}}
		\toprule 
		\rowcolor{AzzurroGruppo!30}
		\textbf{QUalità} & \textbf{Descrizione} \\
		\toprule
		\endhead		
		\textbf{Completezza} &  descrizione specifica e dettagliata di ogni funzionalità richiesta dal prodotto \glo{software} e del suo comportamento in risposta agli input dati\\
		\textbf{Consistenza} &  non devono crearsi situazioni di contraddizione tra \glo{requisiti}\\
		\textbf{Correttezza} &  il requisito deve risultare veramente necessario e richiesto dagli utenti finali\\
		\textbf{Univocità} & ogni requisito deve essere contraddistinto da un codice formale univoco \\
		\textbf{Modificabilità} &  il requisito deve poter cambiare nel tempo in modo facile, completo, consistente e in modo da mantenere la struttura e lo stile\\
		\textbf{Non ambiguità} &  il requisito deve avere una sola interpretazione\\
		\textbf{Priorità} &  il requisito ha un identificatore che ne esprime l'importanza\\
		\textbf{Verificabilità} &  si deve poter verificare che l'applicazione realizzi il requisito individuato\\
		\textbf{Tracciabilità} &  il requisito ha un'origine chiara, un nome e un numero in modo da poterlo referenziare in futuro\\
		\bottomrule
		\hiderowcolors
		\caption{Elementi che contribuiscono alla qualità dei requisiti}
	\end{longtable}

\subsubsection{Progettazione}
\paragraph{Scopo}
L'attività di progettazione definisce, in funzione dei \glo{requisiti} specificati nel documento \AdR{}, le caratteristiche che il prodotto richiesto deve avere in modo da fornire la soluzione migliore che soddisfi i \glo{requisiti} degli \glo{stakeholder}. Questa fase provvede a:
\begin{itemize}
\item garantire la qualità del prodotto applicando un approccio sistematico ai problemi;
\item suddividere compiti implementativi, così da provvedere alla diminuzione della complessità del problema originale e rendere più semplice la codifica da parte dei \progrs{};
\item ottimizzare l'utilizzo delle risorse.
\end{itemize}
\paragraph{Descrizione}
Le parti principali sono due:
\begin{itemize}
\item \textbf{Technology baseline}: contiene le specifiche della progettazione ad alto livello del prodotto e delle sue componenti, l'elenco dei diagrammi UML che saranno utilizzati per la realizzazione dell'architettura e i \glo{test} di \ignore{verifica};
\item \textbf{Product baseline}: rende ancora più dettagliata l'attività di progettazione, integrando ciò che è riportato nella Technology baseline. Inoltre, definisce i \glo{test} necessari alla \glo{verifica}.
\end{itemize}
\paragraph{Aspettative}
Il compito di ogni \prog{} è quello di definire l'architettura logica del \glo{sistema}. Quest'ultima, in particolare, dovrà avere le seguenti caratteristiche:
\begin{itemize}
\item soddisfare i \glo{requisiti} segnati nel documento \AdR{};
\item adattarsi a ogni modifica effettuata nell'\AdR{};
\item essere comprensibile, modulare e robusta;
\item risultare affidabile, quindi svolgere i propri compiti ogni qual volta venga richiesto;
\item rispettare le caratteristiche di safety e security rispetto ai malfunzionamenti;
\item essere disponibile;
\item mostrare un utilizzo efficiente delle risorse necessarie;
\item garantire riusabilità;
\item presentare componenti semplici e coese nel raggiungimento degli obiettivi, incapsulate e con un basso livello di accoppiamento.
\end{itemize}
\paragraph{Design patterns}
Devono essere descritti i design pattern utilizzati per realizzare l'architettura. Ogni design pattern deve essere accompagnato da una descrizione e da un diagramma, che ne esponga il significato e la struttura.
\paragraph{Diagrammi UML 2.0}

{
\setlength{\freewidth}{\dimexpr\textwidth-0\tabcolsep}
	\renewcommand{\arraystretch}{1.5}
	\setlength{\aboverulesep}{0pt}
	\setlength{\belowrulesep}{0pt}
	\rowcolors{2}{AzzurroGruppo!10}{white}
	\begin{longtable}{L{.210\freewidth} L{.6\freewidth} L{.080\freewidth}}
		\toprule 
		\rowcolor{AzzurroGruppo!30}
		\textbf{Tipo} & \textbf{Descrizione} \\
		\toprule
		\endhead		
		\textbf{Diagrammi delle attività} & mostrano tutte le operazioni di una certa attività, anche fuori dal contesto. \\
		\textbf{Diagrammi delle classi} & mostrano gli attributi e i metodi di classi e tipi.\\
		\textbf{DIagrammi dei package} & mostrano i raggruppamenti tra le classi. \\
		\textbf{Diagrammi di sequenza} & mostrano sequenze di azioni usando scelte definite.. \\
		\bottomrule
		\hiderowcolors
		\caption{Descrizione dei diagrammi UML 2.0}
	\end{longtable}
}
\paragraph{Test}
Ogni \prog{} avrà il compito di definire \glo{test} opportuni, che dovranno essere accompagnati da eventuali classi utili ad individuare errori ed anomalie.
\subsubsection{Codifica}
La codifica ha lo scopo di normare l'effettiva realizzazione del prodotto \glo{software} richiesto. In questa fase, di fatto, si concretizza l'architettura di alto livello pensata dai \progs{} nel documento \PdQ{}. I \progrs{} dovranno attenersi a queste norme durante la fase di programmazione e implementazione. L'uso di norme e convenzioni in questa fase è fondamentale per permettere la generazione di codice leggibile e uniforme, agevolare le fasi di manutenzione, \glo{verifica} e \glo{validazione} e migliorare la qualità del prodotto.
\paragraph{Convenzioni per i nomi}
\begin{itemize}
	\item ogni nome deve essere univoco;
	\item ogni nome di variabile, metodo, funzione seguirà il modello \glo{Lower Camel Case};
	\item ogni nome di classe seguirà il modello \glo{PascalCase}.
\end{itemize}
\newpage
\paragraph{Convenzioni sulle pratiche}
\
{
	\setlength{\freewidth}{\dimexpr\textwidth-0\tabcolsep}
	\renewcommand{\arraystretch}{1.5}
	\setlength{\aboverulesep}{0pt}
	\setlength{\belowrulesep}{0pt}
	\rowcolors{2}{AzzurroGruppo!10}{white}
	\begin{longtable}{L{.210\freewidth} L{.6\freewidth} L{.080\freewidth}}
		\toprule 
		\rowcolor{AzzurroGruppo!30}
		\textbf{Pratica} & \textbf{Descrizione norma} \\
		\toprule
		\endhead		
		\textbf{Indentazione} & Il codice dovrà essere indentato usando 4 spazi. \\ 
		\textbf{Spaziature} & Dovrà essere presente una spaziatura dopo l'apertura di una parentesi graffa.  \\
		\textbf{Commenti} & Ogni commento deve essere separato con uno spazio dalla sua definizione \\ 
		\textbf{Lunghezza della riga di codice} & Ogni riga di codice non deve superare i 140 caratteri di lunghezza. \\
		\textbf{Brevità dei metodi} & Ogni blocco di codice non dovrà superare le 50 istruzioni. Se tuttavia questo risulta indispensabile può essere permesso, per questo la norma è consigliata e non obbligatoria.\\ 	
		\textbf{Stringhe} & Ogni stringa deve essere definita con singoli apici. \\
		\textbf{Funzioni} & Bisogna evitare quando possibile l'utilizzo di funzioni ricorsive.\\ 	
		\textbf{Variabili} & Non definire variabili che non verranno utilizzate.\\ 	
		\textbf{Operatori} & Ogni operatore dovrà essere preceduto e seguito da uno spazio.\\ 	
		\textbf{Complesità ciclomatica} & L'annidamento di cicli sarà evitato il più possibile. In questo modo le funzioni non saranno troppo onerose da eseguire e soprattutto godranno di una maggiore leggibilità. \\  			
		\bottomrule
		\hiderowcolors
		\caption{Descrizione delle norme delle pratiche di codifica}
	\end{longtable}
}
\subsubsection{Metriche}
\textit{Il gruppo si riserva la possibilità di ampliare opportunamente la sezione in futuro.}
{
	\setlength{\freewidth}{\dimexpr\textwidth-0\tabcolsep}
	\renewcommand{\arraystretch}{1.5}
	\setlength{\aboverulesep}{0pt}
	\setlength{\belowrulesep}{0pt}
	\rowcolors{2}{AzzurroGruppo!10}{white}
	\begin{longtable}{L{.1\freewidth} L{.15\freewidth} L{.4\freewidth} L{.2\freewidth}}
		\toprule 
		\rowcolor{AzzurroGruppo!30}
		\textbf{Codice} & \textbf{Nome} & \textbf{Descrizione metrica} & \textbf{Formula}\\
		\toprule
		\endhead		
		\textbf{MCD03 }& Variabili non utilizzate & Il codice non deve avere variabili inutilizzate, esse andrebbero ad intaccare le performance e la facilità di manutenzione del codice & - \\
		\textbf{MCD04} & Facilità di comprensione & Indice che riporta la facilità con cui l'utente riesce a comprendere cosa fa il codice, rappresentata mediante il numero di linee di commento nel codice & \small{R=$\frac{n^\circ\ linee\ commento}{n^\circ\ linee\ codice}$}  \\
		\textbf{MCD05} & Code coverage & Metrica che riporta la quantità di codice coperta da \glo{test}. Più questo valore aumenta e più il codice può ricevere implementazioni ed il cliente utilizzarlo senza incorrere in problemi & - \\
		\textbf{MCD06} & Lunghezza della riga di codice & Per facilitare la lettura e la riduzione della complessità di una singola riga di codice non sarà possibile avere righe di codice più lunghe di quanto stabilito dalla metrica & - \\
		\textbf{MCD07} & Brevità dei metodi & I metodi devono generalmente essere brevi per poter assicurare che la loro complessità non risulti eccessiva & - \\
		\bottomrule
		\hiderowcolors
		\caption{Descrizione delle metriche}\\
	\end{longtable}
}



\subsubsection{Strumenti}
Di seguito vengono presentati gli strumenti che il gruppo ha deciso di adottare all'interno del progetto durante il processo di Sviluppo.
\paragraph{StarUML}
Per la generazione di diagrammi UML il gruppo ha deciso di utilizzare StarUML in quanto offre molte agevolazioni che permettono una produzione veloce dei diagrammi e risulta uno strumento di semplice apprendimento e uso.