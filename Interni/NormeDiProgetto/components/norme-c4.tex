\section{Processi Organizzativi}
\subsection{Gestione delle infrastrutture}

Il processo ha come scopo quello di mantenere l'integrità di tutti i componenti della configurazione, cioè tutte le piattaforme e applicazioni utilizzate durante le varie fasi di:
\begin{itemize}
\item Analisi;
\item Pianificazione;
\item Sviluppo;
\item Revisione.
\end{itemize}
Il corretto funzionamento di queste piattaforme è fondamentale per la collaborazione tra i membri del gruppo e la configurazione e gestione di esse è delegata all' \adm{}. Inoltre, vengono descritte le modalità con cui il gruppo ha gestito le comunicazioni tramite le suddette componenti.

\subsubsection{Lavoro a distanza}
Il gruppo ha sviluppato le seguenti norme con l'obbiettivo di diminuire il più possibile l'impatto che le restrizioni causate dal virus \textit{SARS-CoV2} comportano sul lavoro, per questo sono state favorite soluzioni di lavoro da remoto e piattaforme di teleconferenza rispetto a meeting in persona.

\subsubsection{Infrastrutture di comunicazione}

\setlength{\freewidth}{\dimexpr\textwidth-0\tabcolsep}
	\renewcommand{\arraystretch}{1.5}
	\setlength{\aboverulesep}{0pt}
	\setlength{\belowrulesep}{0pt}
	\rowcolors{2}{AzzurroGruppo!10}{white}
	\begin{longtable}{L{.210\freewidth} L{.6\freewidth} L{.080\freewidth}}
		\rowcolor{AzzurroGruppo!30}
		\textbf{Nome} & \textbf{Descrizione} \\
		\toprule
		\endhead
		\textbf{\ignore{GitHub}} & Per gestire la pubblicazione di questo progetto è stata creata un \glo{repository} su \glo{GitHub} dove risiederanno tutti i documenti e il codice approvati \\
		\textbf{Gmail} & Account che rappresenta l’intero gruppo \Mail{}; utilizzato per interagire con gli \glo{stakeholder} e i committenti, per creare l’account su \glo{GitHub} ed il corrispondente \glo{repository} \\
		\textbf{\ignore{Slack}} & Utilizzato per le comunicazioni interne ed esterne \\
		\textbf{Telegram} & Utilizzato per le comunicazioni più rapide e informali tra i membri del
gruppo \\
		\textbf{\ignore{Zoom}} & Utilizzato per le comunicazioni sia interne che esterne, in particolare per le video chiamate\\
		\bottomrule
		\hiderowcolors
		\caption{Descrizione strumenti organizzativi}
	\end{longtable}


\subsubsection{Coordinamento interno }
In questa sezione vengono descritte le modalità e le norme adottate per organizzare in modo efficace la collaborazione all'interno del gruppo. In particolare, vengono trattate le metodologie inerenti alle riunioni interne e alle comunicazioni tra i membri.
\paragraph*{Organizzazione delle riunioni interne}
Gli incontri formali interni avvengono esclusivamente sulla piattaforma \glo{Zoom}.
Tali incontri devono attenersi al regolamento seguente:
\begin{itemize}
\item La partecipazione a questi incontri è limitata ai membri del gruppo;
\item Prima di ogni incontro deve essere comunicata la data e l'ora della riunione sull'apposito canale \glo{Slack} del gruppo;
\item Prima di ogni incontro deve essere redatto l'ordine del giorno dal \RdP{};
\item A inizio riunione devono essere nominati due dei partecipanti per redigere il verbale.
\end{itemize}
I partecipanti sono tenuti a informare per tempo della loro assenza nel caso non potessero partecipare all'incontro.
\paragraph*{Organizzazione della comunicazione tra i membri }
Il canale di comunicazione principale è un \glo{workspace} \glo{Slack}.
Il \glo{workspace} è suddiviso nei seguenti canali:

\setlength{\freewidth}{\dimexpr\textwidth-0\tabcolsep}
	\renewcommand{\arraystretch}{1.5}
	\setlength{\aboverulesep}{0pt}
	\setlength{\belowrulesep}{0pt}
	\rowcolors{2}{AzzurroGruppo!10}{white}
	\begin{longtable}{L{.210\freewidth} L{.6\freewidth} L{.080\freewidth}}
		\rowcolor{AzzurroGruppo!30}
		\textbf{Nome} & \textbf{Descrizione} \\
		\toprule
		\endhead		
		\textbf{Annunci} & Canale dedicato alla comunicazione dell'ordine del giorno delle riunioni e ad annunci di interesse generale \\
		\textbf{Chiamate} & Canale dedicato alla comunicazione dei link alle chiamate \glo{zoom}, siano esse ricorrenti o meno \\
		 \textbf{Date} & Canale dedicato alla comunicazione di scadenze e date di interesse per il gruppo come ad esempio seminari o riunioni esterne \\
		 \textbf{Dev} & Canale dedicato per comunicazioni sul codice e sul suo sviluppo \\
		\textbf{Documenti-vari} & Canale dedicato alla condivisione e archiviazione di bozze non finalizzate di documenti redatti dal gruppo \\
		 \textbf{General} & Canale dedicato a comunicazioni senza un topic specifico tra i membri del gruppo \\
		 \textbf{Github-alert} & Canale che comunica il successo o fallimento della \glo{CI} del progetto \\
		 \textbf{Comunicazioni-zextras} & Canale utilizzato per avere una comunicazione diretta con l'azienda. \\
		\bottomrule
		\hiderowcolors
		\caption{Descrizione canali su Slack}
	\end{longtable}
Oltre ai canali sopra citati, ce ne sono di accessibili solo a membri del gruppo che stanno svolgendo ruoli specifici e che vengono creati e resi disponibili all'occorrenza.
\subsubsection{Comunicazioni esterne}
In questa sezione vengono descritte le modalità e le norme adottate per organizzare in modo efficace la comunicazione con soggetti esterni.
\paragraph*{Comunicazione con soggetti esterni}
Tutte le comunicazioni con soggetti esterni devono avvenire tramite l'indirizzo di posta elettronica \url{\Mail}. Nel caso non rientri nei compiti del proprio ruolo, prima di inviare un'email utilizzando l'indirizzo del gruppo è obbligatorio consultare il \RdP{}.
\paragraph*{Incontri formali esterni}
Gli incontri formali esterni vengono svolti tramite \glo{Zoom}. Come negli incontri formali interni vengono nominati due dei partecipanti appartenenti al gruppo per redigere il verbale dell'incontro. 



\subsection{Formazione del personale}

Il processo di formazione del personale viene gestito personalmente dal \RdP{} e ha l'obiettivo principale di fornire risorse umane adeguate e di mantenere le loro competenze consistenti durante la progressione del progetto. In particolare, vengono descritti i ruoli di progetto, il criterio di assegnazione dei compiti e il \glo{sistema} di ticketing per la gestione degli obbiettivi a medio/corto termine.
\subsubsection{Ruoli di progetto}
		{
\setlength{\freewidth}{\dimexpr\textwidth-0\tabcolsep}
	\renewcommand{\arraystretch}{1.5}
	\setlength{\aboverulesep}{0pt}
	\setlength{\belowrulesep}{0pt}
	\rowcolors{2}{AzzurroGruppo!10}{white}
	\begin{center}
	\begin{longtable}{L{.2\freewidth} L{.4\freewidth} L{.3\freewidth}}
		\rowcolor{AzzurroGruppo!30}
		\textbf{Nome} & \textbf{Descrizione} & \textbf{Compiti}\\
		\toprule
		\endhead		
		\multirow{5}*\textbf{Responsabile di Progetto} & \multirow{5}*\textbf{Il \RdP{} è la figura centrale del progetto, esso ha il compito di coordinare il \glo{team} e il ruolo di rappresentante del progetto durante le comunicazioni con l'esterno} & Definizione delle scadenze interne; \\
		\cline{3-3}
		& & Allocazione delle risorse umane a seconda delle attività necessarie;\\
		\cline{3-3}
		& & Rappresentazione del gruppo e del progetto con soggetti esterni; \\
		\cline{3-3}
		& & Valutazione dei rischi e distribuzione di questi ultimi; \\
		\cline{3-3}
		& & Approvazione della documentazione. \\
		\multirow{5}*\textbf{Amministratore} & \multirow{5}*\textbf{L'\adm{} ha il compito di gestire e configurare adeguatamente le piattaforme utilizzate dal gruppo per lo svolgimento del progetto. Da lui dipendono l'affidabilità e l'efficacia dei mezzi scelti per lo svolgimento del progetto} & Amministrazione delle infrastrutture di supporto; \\
		\cline{3-3}
		& & Assistenza nella risoluzione di problematiche relative alla gestione dei processi; \\
		\cline{3-3}
		& & Gestione e archiviazione della documentazione di progetto; \\
		\cline{3-3} 
		& & Salvaguardia dell'integrità del processo di correzione e \glo{verifica} della documentazione; \\
		\cline{3-3}
		& & Manutenzione delle norme e delle procedure relative alle piattaforme coinvolte nei processi di lavoro. \\
		\multirow{5}*\textbf{Analista} & \multirow{5}*\textbf{L' \ana{} segue il progetto principalmente nelle fasi iniziali ed è fortemente coinvolto nella stesura dell' \AdR{}. Il suo ruolo è quello di analizzare i problemi posti dal progetto e chiarire le dipendenze e le ramificazioni di ogni attività necessaria alla consegna del prodotto} & Redigere l'\AdR{}; \\
		\cline{3-3}
		& & Studio e definizione dei problemi sollevati dallo sviluppo del progetto; \\
		\cline{3-3}
		& & Definizione di dipendenze tra differenti attività all'interno del progetto; \\
		\cline{3-3}
		& & Definizione dei potenziali \glo{casi d'uso} e delle necessità dell'utente finale.\\
		\multirow{5}*\textbf{Progettista} & \multirow{5}*\textbf{Il \prog{} segue lo sviluppo del progetto e, a partire dai \glo{requisiti}, definisce le scelte tecniche necessarie per lo sviluppo del prodotto} & Analisi tecnica dei problemi posti dall' \ana{}; \\
		\cline{3-3}
		& & Modellazione di soluzioni ai problemi posti dall' \ana{}; \\
		\cline{3-3}
		& & Realizzazione di soluzioni sostenibili e di facile manutenzione durante la fase di sviluppo. \\
		\multirow{5}*\textbf{Programmatore} & \multirow{5}*\textbf{Il \progr{} ha il compito di codificare i modelli realizzati dal \prog{}. Il codice prodotto dal \progr{} deve attenersi il più possibile alle specifiche elaborate dal \prog{} e documentare opportunamente il codice creato per aumentarne la manutenibilità} & Codifica dei modelli realizzati dal \prog{}; \\
		\cline{3-3}
		& & Documentazione del codice prodotto per facilitarne la manutenzione; \\
		\cline{3-3}
		& & Codifica delle componenti necessarie alla \glo{verifica} e alla \glo{validazione} del prodotto durante le varie fasi dello sviluppo; \\
		\cline{3-3}
		& & Stesura del \MU{} destinato ai manutentori. \\
		\multirow{5}*\textbf{Verificatore} & \multirow{5}*\textbf{Il \ver{} segue l'intero ciclo di vita del progetto. Egli si assicura che la qualità della documentazione prodotta aderisca alla norme stabilite} & Controllo e correzione della documentazione prodotta; \\
		\cline{3-3}
		& & Controllo dell'aderenza alle norme delle attività di processo; \\
		\cline{3-3}
		& & Descrizione, ove e quando necessario, delle procedure di \glo{verifica}. \\

		\bottomrule
		\hiderowcolors
		\caption{Ruoli di progetto}
	\end{longtable}
	\end{center}
}
	


\subsubsection{Assegnazione dei compiti}
Per garantire l'efficienza e la flessibilità del processo di sviluppo, le attività necessarie al completamento di un progetto devono essere suddivise in compiti, che possono essere svolti sequenzialmente o in parallelo a seconda delle dipendenze tra di loro. Per ottimizzare il processo di suddivisione dei compiti il gruppo utilizza il \glo{sistema} di ticket offerto dalla piattaforma \glo{GitHub}.
Una volta individuate dall'\ana{}, le attività vengono all'occorrenza suddivise in sotto attività e compiti dal \RdP{}. Quando deve assegnare a ogni membro del gruppo compiti diversi in modo da ottimizzare il progresso del progetto e, il \RdP{} deve valutarne l'idoneità, tenendo conto di disponibilità, competenze tecniche e carico di lavoro attuale del membro in esame.  Dopo aver deciso a chi assegnare l'incarico,  procederà ad aprire una \glo{issue} su \glo{GitHub} dettagliando il compito da svolgere. A questo punto, il membro assegnato all'incarico dovrà svolgere il compito nei tempi definiti dal \RdP{} e , nel caso di contrattempi, notificarlo quanto prima possibile. Il membro assegnato alla \glo{issue} è inoltre responsabile della chiusura di quest'ultima.


