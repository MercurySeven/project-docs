\section{Introduzione}
\subsection{Scopo del documento}
Le \NdP{} hanno l'obbiettivo di stabilire e normare i processi utili allo sviluppo del prodotto.
Il presente documento deve essere visionato da ogni membro del gruppo, il quale dovrà attenersi alle norme qui descritte durante l'intera durata del progetto.
Le norme e i processi descritti in questo documento sono soggetti a modifiche durante l'evoluzione del progetto.
Questo documento viene frequentemente aggiornato per assicurarsi che le norme descritte rispecchino la realtà operativa del gruppo rendendo il documento un importante risorsa.
Ogni cambiamento e aggiornamento del documento deve essere discusso con il \RdP{} prima di essere inserito all'interno del documento.
\subsection{Scopo del prodotto}
Il capitolato chiede di sviluppare un algoritmo e una applicazione per la sincronizzazione file tra server e client appoggiandosi sul sevizio \glo{Zextras Drive}.
Il programma client deve essere multipiattaforma e il prodotto deve essere sviluppato aderendo al pattern \glo{MVC}.
\subsection{Glossario}
All'interno dei documenti possono essere presenti termini sconosciuti o ambigui a seconda del contesto. Al fine di poter garantire una lettura senza l'insorgere di incomprensioni o di interpretazioni differenti, viene fornito un glossario nel file \G{} contenente i suddetti termini e la loro spiegazione. Nella documentazione tali parole vengono indicate con una "G" a pedice ad ogni occorrenza.
\subsection{Riferimenti}
\subsubsection*{Riferimenti normativi}
\begin{itemize}
	\item \textbf{Capitolato d'Appalto C7:} \url{https://www.math.unipd.it/~tullio/IS-1/2020/Progetto/C7.pdf};
	\item \textbf{Regolamento del progetto didattico:} \url{https://www.math.unipd.it/~tullio/IS-1/2020/Dispense/P1.pdf}.
\end{itemize}
\subsubsection*{Riferimenti informativi}
\begin{itemize}
	\item \textbf{\ignore{Software} Engineering, Ian Sommerville, 10th Edition;}
	\item \textbf{ISO 8601:} \url{https://it.wikipedia.org/wiki/ISO_8601};
	\item \textbf{Slide del corso di Ingegneria del \ignore{Software}:}
		\begin{itemize}
			\item \textbf{Introduzione all’Ingegneria del Software:} \url{https://www.math.unipd.it/~tullio/IS-1/2020/Dispense/L01.pdf};
			\item \textbf{Il ciclo di vita del software:} \url {https://www.math.unipd.it/~tullio/IS-1/2020/Dispense/L05.pdf};
			\item \textbf{Strumenti di collaborazione:} \url{https://www.math.unipd.it/~tullio/IS-1//2010/Approfondimenti/A12.pdf}.
		\end{itemize}
\end{itemize}
