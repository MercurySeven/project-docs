\section{Processi di supporto}
\subsection{Processo di documentazione}
\subsubsection{Scopo}
L'obiettivo della sezione è quello di normare le attività di stesura e aggiornamento della documentazione svolte durante il ciclo di vita del \glo{software}, così da ottenere documenti formalmente concordi.
\subsubsection{Descrizione}
La sezione contiene le norme tramite le quali avviene la corretta stesura dei documenti, comprendente \glo{verifica} e approvazione. I membri del gruppo sono tenuti ad attenersi alla suddetta sezione.
\subsubsection{Aspettative}
Le aspettative riguardanti il processo in questione sono:
\begin{itemize}
	\item l'identificazione di una struttura ben organizzata comune ai prodotti del processo in questione;
	\item la raccolta di norme da rispettare durante la stesura dei documenti.
\end{itemize}
\subsubsection{Ciclo di vita di un documento}
Ogni documento redatto passa per le seguenti fasi:
\begin{itemize}
	\item \textbf{Creazione} del documento tramite template;
	\item \textbf{Strutturazione} del documento tramite registro delle modifiche, indice di contenuti e/o indice delle figure e delle tabelle;
	\item \textbf{Stesura} del corpo del documento, scritto da più membri del gruppo;
	\item \textbf{Revisione} del documento, gestita da un \textit{Verificatore} che, regolarmente, riesamina le sezioni del suddetto;
	\item \textbf{Approvazione} del documento, stabilita dal \textit{Responsabile di Progetto} quando la revisione risulta positiva; consente il rilascio del documento. 
\end{itemize}
\subsubsection{Documenti prodotti \hfil}

I documenti prodotti saranno di due tipi:
\begin{itemize}
	\item \textbf{Formali}: sono caratterizzati da storico delle versioni prodotte e da approvazione della versione definitiva da parte del \textit{Responsabile di Progetto}. Possono essere:
	\begin{itemize}
		\item \textbf{interni}: riguardanti le dinamiche del gruppo;
		\item \textbf{esterni}: di interesse ai committenti e al \glo{proponente}, vengono consegnati nell'ultima versione approvata.
	\end{itemize} 
In particolare, i documenti formali prodotti saranno:
\newpage
{
	\setlength{\freewidth}{\dimexpr\textwidth-1\tabcolsep}
	\renewcommand{\arraystretch}{1.5}
	\setlength{\aboverulesep}{0pt}
	\setlength{\belowrulesep}{0pt}
	\rowcolors{2}{AzzurroGruppo!10}{white}
	\begin{longtable}{L{.210\freewidth} L{.6\freewidth} L{.080\freewidth}}
		\toprule 
		\rowcolor{AzzurroGruppo!30}
		\textbf{Nome} & \textbf{Descrizione} & \textbf{Uso}\\
		\toprule
		\endhead		
		\NdP{} & Contiene tutte le regole stabilite dai membri alle quali attenersi durante l'intera durata del progetto. & Interno \\ 
		\PdP{} & Contiene la pianificazione di tutte le attività previste, comprendente il preventivo delle spese e una previsione dell'impegno in ore per ogni membro del gruppo. & Esterno \\
		\PdQ{} & Descrive i criteri di valutazione della qualità impiegate dal gruppo & Esterno \\ 
		\SdF{} & Contiene l'analisi dei capitolati messi a disposizione, evidenziandone pregi e difetti. Contiene, inoltre, il capitolato scelto dal gruppo & Interno \\
		\AdR{} & Descrive i \glo{requisiti} che il prodotto dovrà possedere per essere in linea con le richieste dei committenti. & Esterno \\ 	
		\G{} & Elenco di tutti i termini presenti nella documentazione che, secondo i membri, necessitano una definizione al fine di chiarirne il significato e rimuovere eventuali ambivalenze. & Esterno \\  			
		\bottomrule
		\hiderowcolors
		\caption{Nome, Descrizione ed uso dei documenti formali prodotti}
	\end{longtable}
}

\item \textbf{Informali}: sono documenti non soggetti a versionamento o che non sono ancora stati approvati dal \RdP{}. \newline 
In particolare, i documenti informali prodotti saranno:
{
	\setlength{\freewidth}{\dimexpr\textwidth-1\tabcolsep}
	\renewcommand{\arraystretch}{1.5}
	\setlength{\aboverulesep}{0pt}
	\setlength{\belowrulesep}{0pt}
	\rowcolors{2}{AzzurroGruppo!10}{white}
	\begin{longtable}{L{.210\freewidth} L{.6\freewidth} L{.080\freewidth}}
		\toprule 
		\rowcolor{AzzurroGruppo!30}
		\textbf{Nome} & \textbf{Descrizione} & \textbf{Uso}\\
		\toprule
		\endhead		
		Verbali interni & Contengono le informazioni e le decisioni prese durante gli incontri tra i membri del gruppo. & Interno \\ 
		Verbali esterni & Comprendono le informazioni ed i chiarimenti ricevuti durante gli incontri tra i membri ed il committente. & Esterno \\
		\bottomrule
		\hiderowcolors
		\caption{Nome, Descrizione ed uso dei documenti informali prodotti}
	\end{longtable}
}
\end{itemize}
\subsubsection{Directory di un documento}
Le directory prendono il nome dal documento contenuto, nella forma \textbf{NomeDelDocumento}. Questa viene poi collocata, a seconda del tipo di contenuto, in una delle directory \textbf{Interni} o \textbf{Esterni}. 
\subsubsection{Struttura dei documenti}
\paragraph{Attribuzione del nome}
\begin{itemize}
	\item La denominazione dei documenti formali sarà la seguente \newline
	\textbf{[NomeDocumento]\_v[X].[Y].[Z]-[a].[b]}
	\begin{itemize}
		\item \textbf{NomeDocumento} è il nome ufficiale del documento, senza spazi tra le parole e con solamente le iniziali di ogni parola in maiuscolo (convenzione \glo{PascalCase});
		\item \textbf{v[X].[Y].[Z]} con \textbf{v} che abbrevia la parola versione, \textbf{[X]} indica il numero incrementale di revisione del responsabile di progetto, \textbf{[Y]} indica il numero incrementale della \glo{verifica}, \textbf{[Z]} indica il numero di modifiche apportate al documento, che riparte da 0 dopo ogni \glo{verifica} ed approvazione;
		\item \textbf{[a].[b]} con \textbf{A} che indica una versione completa e funzionante del prodotto: le componenti \glo{software}
implementano tutti i \glo{requisiti} obbligatori, i \glo{test} sono stati tutti superati, le \glo{metriche} di qualità soddisfatte e sono presenti versioni approvate di tutti i documenti richiesti per il prodotto finale. Viene incrementato quando il prodotto \glo{software} non è più retrocompatibile; la numerazione parte da 0 e non viene annullata; \textbf{B} indica un incremento nello sviluppo totale del prodotto: la numerazione incrementa ad ogni \glo{sprint}, partendo da 1 e non venendo mai annullata.
	\end{itemize}
	\item La denominazione dei verbali sarà la seguente\newline
	\textbf{Verbale[Interno/Esterno]\_[YYYY]\_[MM]\_[DD]}\newline
	dove
	\begin{itemize}
		\item \textbf{Verbale} indica il tipo di documento;
		\item \textbf{Interno/Esterno]} indica se il verbale in questione è interno od esterno;
		\item \textbf{[YYYY]-[MM]-[DD]} indica la data della riunione, \textbf{[YYYY]} indica l'anno, \textbf{[MM]} indica il mese, \textbf{[DD]} indica il giorno.
	\end{itemize}
\end{itemize}
\paragraph{Frontespizio}    
La prima pagina di ogni documento sarà strutturata nel seguente modo:
\begin{itemize}
	\item \textbf{Logo del gruppo}: posto in basso al centro;
	\item \textbf{Titolo:} posto in alto al centro;
	\item \textbf{Nome del gruppo e del progetto:} posti subito sotto il titolo, al centro;
	\item \textbf{Recapito:} Indirizzo di posta elettronica del gruppo, posizionato sotto il nome del gruppo;
	\item \textbf{Informazioni sul documento,} includono:
	\begin{itemize}
		\item \textbf{Versione:} indica la versione del documento;
		\item \textbf{Uso:} indica se il documento sarà ad uso interno od esterno;
		\item \textbf{Distribuzione:} indica i destinatari del documento;
		\item \textbf{Approvazione:} nome e cognome dell'incaricato alla approvazione del documento;
		\item \textbf{Redazione:} nome e cognome degli incaricati della redazione del documento;
		\item \textbf{\glo{Verifica}:} nome e cognome degli incaricati della \glo{verifica} del documento;
		\item \textbf{Descrizione:} sintetica descrizione del contenuto del documento.
	\end{itemize}
\end{itemize}
\paragraph{Registro delle modifiche}      
Ogni documento presenta subito dopo il frontespizio il registro modifiche, esso è una tabella che traccia tutte le modifiche significative apportate al documento durante il suo ciclo di vita. Ogni voce della tabella riporta:
\begin{itemize}
	\item \textbf{Versione:} numero versione documento dopo la modifica;
	\item \textbf{Data:} data in cui è stata effettuata la modifica;
	\item \textbf{Autore:} nome e cognome dell'autore della modifica;
	\item \textbf{Ruolo:} ruolo dell'autore che ha apportato la modifica;
	\item \textbf{Descrizione:} sintetica descrizione delle modifiche apportate.
\end{itemize}
\paragraph{Corpo del documento}       
Ogni pagina del corpo del documento è strutturata nel seguente modo:
\begin{itemize}
	\item L'intestazione, composta nel seguente modo:
	\begin{itemize}
		\item In alto a sinistra si trova una miniatura a colori del logo del gruppo;	
		\item In alto a destra viene indicato la sezione corrente;
		\item Sotto i primi due elementi elencati è presente una 	linea nera separatrice;
	\end{itemize}		
	\item il contenuto della pagina si trova sotto l'intestazione;	
	\item Il piè di pagina, composto nel seguente modo:	
	\begin{itemize}	
		\item Sotto il contenuto della pagina è presenta una linea nera separatrice;
		\item Sotto la linea a sinistra è presente il titolo del documento ed il nome del gruppo;
		\item Sotto la linea a destra è presente il numero della pagina corrente e quello delle pagine totali.		
	\end{itemize}
\end{itemize}
\paragraph{Verbali}      
I verbali, sia interni che esterni, presentano una struttura fissa e sono soggetti alle stesse norme strutturali degli altri argomenti, ad eccezione del fatto che, essendo informali, non sono soggetti a versionamento.
La struttura sarà quindi la seguente:
\begin{itemize}
	\item \textbf{Canale di comunicazione:} riporta il \glo{software} utilizzato per effettuare l'incontro;
	\item \textbf{Data:} riporta la data della riunione;
	\item \textbf{Ora inizio:} riporta l'orario di inizio della riunione;
	\item \textbf{Ora fine:} riporta l'orario di fine della riunione;
	\item \textbf{Segretario:} nome e cognome dell'incaricato alla stesura del verbale;
	\item \textbf{Partecipanti:} contiene l'elenco dei presenti all'incontro;   
	\item \textbf{Ordine del giorno:} contiene l'elenco degli argomenti trattati durante la riunione;
	\item \textbf{Resoconto:} riporta l'esito della discussione dei singoli punti inseriti nell'ordine del giorno; per i verbali esterni, questa sezione è organizzata secondo il pattern domanda-risposta;
	\item \textbf{Riepilogo delle decisioni:} contiene una tabella che riassume tutte le decisioni avvenute durante la riunione, numerandole con un codice univoco del tipo: \newline 
	\centerline{\textbf{V[Interno/Esterno]\_[YYYY]\_[MM]\_[DD].[X]}}\newline con \textbf{X} indicante il numero progressivo della decisione, a partire da 1.
\end{itemize}
\subsubsection{Norme tipografiche}
\paragraph{Convenzioni di denominazione}   
I nomi delle cartelle e dei file relativi ai documenti adottano la convenzione \glo{PascalCase}, ossia il nome del file presenta l'iniziale di ogni parola che lo compone in maiuscolo, senza nessun separatore. Per file e cartelle legati alla struttura del documento i nomi non contengono caratteri maiuscoli.
\paragraph{Stili di testo}
Gli stili di testo adottati nei documenti sono: 
\begin{itemize}
	\item \textbf{Grassetto:} viene utilizzato nei titoli delle sezioni e dei paragrafi, per enfatizzare parole, per indicare gli elementi di un elenco puntato;
	\item \textbf{Corsivo:} viene applicato per i nomi propri (per la precisione i membri del gruppo, il \glo{proponente}, i committenti), per citare il nome di un documento e per i termini che fanno parte del glossario;
	\item \textbf{Monospace:} per gli snippet di codice;
	\item \textbf{Maiuscolo:} verrà usato negli acronimi e, nel caso di nomi di documenti, per le lettere iniziali delle parole che li compongono secondo la convenzione \textit{\glo{PascalCase}}.
\end{itemize}
\paragraph{Glossario}  
Al fine di evitare ambiguità fra i termini è stato realizzato un documento denominato \textit{Glossario}. I termini il cui significato può risultare ambiguo saranno contrassegnati con una \textbf{G} a pedice (e.g: \texorpdfstring{glossario\textsubscript{\textbf{G}}})) ad ogni loro occorrenza e saranno riportati in ordine alfabetico con le loro rispettive definizioni.
\paragraph{Elenchi puntati}
Gli elenchi puntati al primo livello di annidazione vengono scanditi con • (pallino); al livello successivo di annidamento il simbolo usato è -- (doppio trattino); all'ultimo livello è .(punto fermo).\newline
Ogni voce contenuta all'interno degli elenchi puntati inizia con la lettera maiuscola e termina con ;(punto e virgola), tranne l'ultima voce, la quale termina con .(punto fermo).\newline
Le voci che corrispondono allo schema NomeElemento-descrizione presentano il primo termine in grassetto, il secondo invece nel formato normale. 
\paragraph{Formato di data e formato di ora}
Le date vengono indicato usato il formato stabilito dallo standard ISO 8601:\newline
\centerline{\textbf{[YYYY]\-[MM]\-[DD]}}
\newline
dove:
\begin{itemize}
	\item \textbf{[YYYY]} indica l'anno;
	\item \textbf{[MM]} indica il mese;
	\item \textbf{[GG]} indica il giorno.
\end{itemize}
\paragraph{Sigle}
\begin{itemize}
	\item Le sigle relative ai nomi dei documenti sono:
	\begin{itemize}
		\item \textbf{AdR}: \textit{Analisi dei \ignore{Requisiti};}	
		\item \textbf{SdF}: \textit{Studio di Fattibilità;}	
		\item \textbf{NdP}: \textit{Norme di Progetto;}	
		\item \textbf{PdP}: \textit{Piano di Progetto;}	
		\item \textbf{PdQ}: \textit{Piano di Qualifica;}
		\item \textbf{G}: \textit{Glossario;}
		\item \textbf{VE}: \textit{Verbale Esterno;}	
		\item \textbf{VI}: \textit{Verbale Interno.}
	\end{itemize}	
	\item Le sigle relative ai ruoli di progetto sono:	
	\begin{itemize}		
		\item \textbf{RE}: \textit{Responsabile;}		
		\item \textbf{AM}: \textit{Amministratore;}		
		\item \textbf{AN}: \textit{Analista;}		
		\item \textbf{PT}: \textit{Progettista;}		
		\item \textbf{PT}: \textit{Programmatore;}	
		\item \textbf{VE}: \textit{Verificatore.}	
	\end{itemize}	
	\item Le sigle relative alle revisioni di progetto sono:	
	\begin{itemize}	
		\item \textbf{RR}:\textit{Revisione dei \ignore{requisiti};} 			
		\item \textbf{RP}:\textit{Revisione di progettazione;} 		
		\item \textbf{RQ}:\textit{Revisione di qualifica;} 		
		\item \textbf{RA}:\textit{Revisione di accettazione.} 	
	\end{itemize}	
\end{itemize}
\subsubsection{Elementi grafici}
Questa sezione definisce le norme per l'uso di elementi grafici quali immagini, tabelle e diagrammi.
\begin{itemize}
\item \textbf{Immagini:} le figure saranno centrate ed avranno una opportuna didascalia;
\item \textbf{Grafici UML:} i grafici in linguaggio UML saranno inseriti nel documento come immagini;
\item \textbf{Tabelle:} ogni tabella è contrassegnata da una didascalia e numerata in modo progressivo, a questa prassi fanno eccezione le tabelle del registro delle modifiche, che non presentano una didascalia.
\end{itemize}
\subsubsection{Metriche}
\paragraph{MDC01 Indice Gulpease}
Indice che riporta il grado di leggibilità di un testo redatto in lingua italiana:
La formula adottata è:\newline
\centerline{\textbf{$Ig= 89 + \frac{300 * (n\ frasi) - 10 * (n\ lettere)}{n\ parole}$}}
\paragraph{MDC02 Correttezza ortografica}
Indice che riporta la correttezza ortografica del testo. Questo valore deve essere il più elevato possibile come indicato nel \PdQ{}.

\subsubsection{Strumenti di stesura}
Gli strumenti usati per la stesura dei documenti sono il linguaggio \LaTeX e l'editor \textit{Texmaker}.
\paragraph{\LaTeX}
Per la stesura dei documenti è stato scelto \LaTeX, un linguaggio compilato che permette un facile versionamento, la produzione coerente, ordinata, templetizzata e collaborativa dei documenti.
\paragraph{Texmaker}
L'editor scelto per la stesura dei documenti è \textit{Texmaker}, adottato da tutti i membri del gruppo. \textit{Texmaker} è un editor intuitivo che permette una facile compilazione e visualizzazione dei documenti.\newline
\url{(https://www.xm1math.net/texmaker/)}
\subsection{Processo di configurazione}
\subsubsection{Scopo}
Lo scopo del processo è gestire in modo ordinato e sistematico la produzione di documenti e codice, oltre che a renderli facilmente reperibili. Un elemento soggetto a configurazione ha una collocazione, una denominazione ed uno stato definiti; per ogni oggetto configurato sono garantiti la modifica normata ed il versionamento.
\subsubsection{Descrizione}
Il processo è composto dalle seguenti attività:
\begin{enumerate}
	\item implementazione del processo;
	\item identificazione della configurazione;
	\item controllo della configurazione;
	\item resoconto dello stato della configurazione (gestione delle modifiche);
	\item valutazione della configurazione;
	\item gestione del rilascio e consegna.
\end{enumerate}
\subsubsection{Aspettative}
Le attese, per l’implementazione di questo processo, sono:
\begin{itemize}
	\item rendere sistematica la produzione di codice e documentazione;
	\item unificare ed uniformare lo stato degli strumenti usati durante lo svolgimento di tutto
il progetto;
	\item classificare i prodotti dei vari processi implementati.
\end{itemize}
\subsubsection{Versionamento}
\paragraph{Codice di versionamento}
\label{cod-versionamento}
La storia di un prodotto realizzato dal gruppo, che questo sia un documento o codice, deve poter essere sempre ricostruibile attraverso le sue versioni. Il codice di una versione è nel formato\newline
\centerline{\textbf{[X].[Y].[Z]-[a].[b]}}\newline
\paragraph{Tecnologie adottate}
Per la gestione delle versioni è stato scelto di utilizzare il \glo{sistema} di versionamento distribuito \textit{\glo{Git}} con due \glo{repository} remoti presenti su \textit{\glo{GitHub}}.
\paragraph{Repository}
Il gruppo \textit{\Gruppo{}} ha scelto di creare due \glo{repository} per il progetto:
\begin{itemize}
	\item \textbf{project-SSD} per il versionamento del codice realizzato(\url{https://github.com/MercurySeven/project-SSD});
	\item \textbf{\repoDoc{}} per il versionamento dei documenti redatti (\url{https://github.com/MercurySeven/project-docs}).
\end{itemize}
Il gruppo ha deciso di optare per una separazione logica del codice e della documentazione per poter avere una struttura ed una organizzazione migliore. I due repository sono comunque entrambi facilmente reperibili nella pagina dell'organizzazione (\url{https://github.com/MercurySeven}).
\paragraph{Struttura del repository}
Per entrambi i \glo{repository}, esiste una versione \textbf{locale} sul computer di ogni membro ed una versione \textbf{remota} su \textit{\glo{GitHub}}.\newline
Il \glo{repository} \textbf{project-SSD} risulta avere una struttura ancora indefinita data l'assenza di codice.\newline
Il \glo{repository} \textbf{\repoDoc{}} è invece strutturato nel seguente modo:
\begin{itemize}
	\item \textbf{Tutorial:} cartella contenente un esempio di documento \LaTeX che elenca anche le basi del linguaggio e del suo utilizzo; 
	\item \textbf{Stile:} cartella contenente i file del template in linguaggio \LaTeX usato per produrre tutti i documenti;
	\item Una cartella per ogni documento necessario per la \textit{Revisione dei \ignore{Requisiti}}.
\end{itemize}
Si è optato per questa soluzione per poter avere una documentazione in continua evoluzione, senza categorizzarla e bloccarla negli stati delle revisioni tramite sottocartelle contenenti le versioni delle documentazioni durante una specifica revisione.
\paragraph{Tipi di file}
Nelle singole cartelle dei documenti sono inclusi:
\begin{itemize}
\item file \textit{README.md};
\item file \textit{.tex} per i sorgenti in \LaTeX; 
\item file \textit{.png} per le immagini da inserire nei documenti.
\end{itemize}
Il file \textit{.gitignore} contiene al suo interno un riferimento a tutti i file che non verranno versionati; il file è posizionato al livello più esterno del \glo{repository}.
\paragraph{Norme di branching}
\label{NormeBranching}
Il \glo{repository} inerente alla documentazione sarà composto da diversi \glo{branch}:
\begin{itemize}
	\item \textbf{Main:} \glo{branch} principale che viene aggiornato solamente quando un documento arriva in fase di rilascio, essa avviene quando incrementa la \textbf{[X]} di una versione del documento (\S{}\ref{cod-versionamento});
	\item un \glo{branch} per ogni documento, usato per le modifiche minori effettuate tra i vari rilasci. Una volta che un documento ha raggiunto una maturità sufficiente per un rilascio, verrà eseguito una operazione di \glo{merge} con il \textbf{main}.
\end{itemize}
\paragraph{Modifiche ai repository}
Tutti i membri del gruppo possono apportare modifiche ai file elaborati, salvo quelli presenti nel ramo \textbf{master}. Per apportare cambiamenti a questo ramo, come già citato in precedenza (\S{}\ref{NormeBranching}), è necessaria una \textit{pull request}, con conseguente approvazione di \textbf{almeno} un altro elemento del gruppo.\newline
Per modifiche minori ai file ogni singolo membro può operare con libero arbitrio, mentre per modifiche più sostanziose deve essere contattato il \RdP{} per il file in questione.\newline
Ogni modifica deve poter essere giustificata davanti a tutto il \glo{team}.
\subsection{Gestione della qualità}
\subsubsection{Scopo}
Lo scopo del processo di gestione della qualità è garantire che il prodotto finale rispetti i \glo{requisiti} di qualità individuati dagli \glo{stakeholder}, che le esigenze espresse dal \glo{proponente} siano soddisfatte e che la qualità sia nel complesso soddisfacente. Si mira inoltre ad un maggior controllo di prodotti e processi.
\subsubsection{Descrizione}
Al fine di formalizzare i metodi e gli standard adottati per assicurare la qualità della documentazione e dei prodotti è redatto il \PdQ{}.
Il documento affronta i seguenti punti:
\begin{itemize}
\item Strategie di sviluppo e standard adottati dal gruppo;
\item Obiettivi e \glo{metriche} relative ai processi adottati;
\item Obiettivi e \glo{metriche} relative ai prodotti sviluppati.
\end{itemize}
Il \PdQ{} oltre a definire i \glo{requisiti} qualitativi e le \glo{metriche} usate per tracciarli ha lo scopo di tenere traccia di tali \glo{metriche} tramite aggiornamenti periodici al documento.
Il documento è strutturato in modo tale da distinguere la parte riguardante la documentazione e quella riguardante il \glo{software}.
\subsubsection{Aspettative}
Le aspettative di questo processo sono:
\begin{itemize}
	\item conseguimento della qualità nel prodotto, secondo le richieste del \textit{\glo{proponente}};
	\item prova oggettiva della qualità del prodotto;
	\item conseguimento della qualità nell'organizzazione delle attività del gruppo e dei processi;
	\item raggiungimento della piena soddisfazione del \glo{proponente}.
\end{itemize}
\subsubsection{Controllo di qualità}
Il gruppo \textit{\Gruppo{}} ritiene fondamentale stabilire delle regole che tutti i membri vadano a rispettare per il pieno raggiungimento di un \glo{sistema} di qualità predisposto al miglioramento continuo.
Le attività che ciascun membro deve svolgere sono:
\begin{itemize}
	\item comprensione degli obiettivi da raggiungere;
	\item individuazione di errori e discrepanze;
	\item produzione di una stima del valori di ogni task;
	\item produzione di una stima della complessità di ogni task;
	\item impiego delle conoscenze di ciascun membro del gruppo;
	\item produzione di risultati concreti nel tempo stimato;
	\item miglioramento continuo della propria formazione;
	\item continua comunicazione con il gruppo;
	\item rispetto totale degli standard e delle normative presenti all'interno del suddetto documento.
\end{itemize} 
\subsubsection{Denominazione delle metriche}
Per la denominazione delle \glo{metriche} il gruppo ha deciso di adottare il formato\newline \centerline{\textbf{M[Processo/Attività][Numero]}}\newline
dove
\begin{itemize}
	\item \textbf{M} indica che ci si sta riferendo ad una metrica;
	\item \textbf{Processo/Attività}: specifica il processo o l'attività tra:
	\begin{itemize}
		\item AR: \AdR{};
		\item PR: \emph{Progettazione};
		\item CD: \emph{Codifica};
		\item DC: \emph{Documentazione};
		\item VR: \emph{\glo{Verifica}};
		\item VL: \emph{\glo{Validazione}}.
	\end{itemize}
	\item \textbf{Numero}: definisce il codice univoco della metrica.
\end{itemize}
\subsubsection{Istanziazione di un processo}
Con il perseguimento della qualità in mente, il gruppo \textit{\Gruppo{}} ha deciso di adottare le seguenti regole al momento della definizione di ogni processo, in modo da curare la qualità sin dall'istanziazione dei processi:
\begin{itemize}
	\item un singolo processo deve essere più atomico possibile;
	\item l'obiettivo di un processo non può entrare in conflitto con quello di altri processi;
	\item l'affidamento di risorse umane, temporali e materiali ai singoli processi deve tenere conto tutti gli altri processi e deve puntare ad essere più efficiente possibile;
	\item ogni processo deve essere pianificato correttamente tramite una opportuna analisi dei rischi;
	\item ogni processo al termine della pianificazione deve avere una durata ben definita.
\end{itemize}
\subsection{Processo di verifica}
\subsubsection{Scopo}
Il processo di \glo{verifica} si occupa di determinare se i prodotti di una data attività sono corretti, coesi e completi e di individuare eventuali errori introdotti nella fase di sviluppo. Sono soggetti a \glo{verifica} sia il \glo{software} che la documentazione.
\subsubsection{Descrizione}
Il processo di \glo{verifica} si compone di due attività che permettono di verificare che il prodotto sia conforme alle aspettative di analisi (\S{}\ref{Analisi}) e di \glo{test} (\S{}\ref{Test}). Esso viene applicato al \glo{software} ed ai documenti.
\subsubsection{Aspettative}
Il processo di \glo{verifica} si occupa di prendere ciò che è stato prodotto e di restituirlo in uno stato conforme alle aspettative. Per assicurare il corretto funzionamento di tale processo devono essere rispettate alcune regole:
\begin{itemize}
	\item seguire procedure definite;
	\item seguire criteri chiari ed affidabili;
	\item i prodotti devono passare attraverso fasi successive opportunamente verificate;
	\item il prodotto deve trovarsi in uno stato stabile dopo la \glo{verifica};
	\item alla fine del processo deve essere possibile procedere alla fase di \glo{validazione}.
\end{itemize}
\subsubsection{Attività}
\paragraph{Analisi}
\label{Analisi}
Il processo di analisi si suddivide in Analisi statica ed Analisi dinamica.
\paragraph{Analisi statica}
L'analisi statica permette di effettuare controlli su documenti e codice, verificando così che la corretta applicazione della correttezza (intesa come assenza di errori e difetti). Vi sono due metodologie per applicarla:
\begin{itemize}
	\item \textbf{Walkthrough:} tecnica di \glo{verifica} consistente nella lettura da parte del \glo{team} dell'intero documento o codice in cerca di anomalie. Verrà applicata principalmente nella fase iniziale del progetto, quando il \glo{team} ancora non conosce i possibili errori che si possono incontrare. Questa tecnica risulta molto onerosa in termini di efficienza ed efficacia.
	\item \textbf{Inspection:} tecnica di \glo{verifica} attraverso il quale il \textit{Verificatore} eseguirà una lettura mirata e strutturata del documento o del codice nei punti in cui si sa già che possano essere presenti degli errori. Risulta meno dispendiosa in termini di tempo ma richiede una buona conoscenza di fondo.
\end{itemize}
\paragraph{Analisi dinamica}
L'analisi dinamica prevede l'esecuzione del prodotto \glo{software} e la sua analisi tramite l'utilizzo di \glo{test} che verificano se il prodotto funziona o se vi sono presenti anomalie.
\paragraph{Test}
\label{Test}
L'attività di testing è la base dell'analisi dinamica. I \glo{test} permettono di individuare tutti i possibili errori che possono essere stati commessi e tutti i casi limite che possono risultare problematici. Per ogni \glo{test} devono essere definiti i seguenti parametri:
\begin{itemize}
	\item \textbf{ambiente:} \glo{sistema} hardware e \glo{software} sulla quale viene eseguito il \glo{test};
	\item \textbf{stato iniziale:} stato iniziale dal quale parte il \glo{test};
	\item \textbf{input:} tutti i dati ammessi in ingresso;
	\item \textbf{output:} tutti i dati attesi in uscita;
	\item \textbf{istruzioni aggiuntive:} istruzioni opzionali che permettono di impostare come il \glo{test} deve essere eseguito e come i risultati devono essere interpretati.
\end{itemize}
Dei \glo{test} ben progettati e scritti devono:
\begin{itemize}
	\item essere ripetibili;
	\item definire correttamente ed identificare i parametri già elencati;
	\item avvertire la presenza di effetti indesiderati nel codice;
	\item fornire informazioni sui risultati e sull'esecuzione stessa tramite file di log.
\end{itemize}
Le tipologie di \glo{test} utilizzate sono le seguenti.
\begin{itemize}
	\item \glo{Test di \ignore{unità}};
	\item \glo{Test di integrazione};
	\item \glo{Test di \ignore{sistema}};
	\item \glo{Test di regressione};
	\item \glo{Test di accettazione}.
\end{itemize}
\paragraph{Codice identificativo dei test}
Ogni \glo{test} viene descritto con:
\begin{itemize}
	\item codice identificativo;
	\item descrizione;
	\item stato, che può essere:
	\begin{itemize}
		\item implementato;
		\item non implementato;
		\item non eseguito;
		\item superato;
		\item non superato.
	\end{itemize}
\end{itemize}
Il codice identificativo presenta la seguente struttura:\newline
\centerline{\textbf{T[Tipo][Id]}}
\newline
Dove:
\begin{itemize}
	\item \textbf{T:} indica il "\glo{Test}";
	\item \textbf{Tipo:} il tipo di \glo{test}, che può essere:
	\begin{itemize}
		\item \textbf{U}: \glo{test di \ignore{unità}};
		\item \textbf{I}: \glo{test di integrazione};
		\item \textbf{S}: \glo{test di \ignore{sistema}};
		\item \textbf{R}: \glo{test di regressione};
		\item \textbf{A}: \glo{test di accettazione}.
	\end{itemize}
	\item \textbf{Id:} rappresenta un codice identificativo.
\end{itemize}

\subsubsection{Metriche}
Il gruppo ritiene prematuro stabilire in questa fase delle \glo{metriche} per i \glo{test} in questione. La sezione sarà opportunamente integrata.
\subsubsection{Verifica ortografica}
Per la \glo{verifica} ortografica dei documenti scritti in \LaTeX si utilizzerà \glo{GNU aspell} ed il correttore automatico fornito da \glo{TexMaker}.
\subsection{Processo di validazione}
\subsubsection{Scopo}
Il processo di \glo{validazione} stabilisce se il prodotto è in grado di soddisfare il compito per il quale è stato creato e se rispetta i \glo{requisiti} e le aspettative del committente.
\subsubsection{Descrizione}
Il processo di \glo{validazione} prende in esame il prodotto ottenuto dalla fase di \glo{Verifica} e lo restituisce garantendo che esso sia conforme ai \glo{requisiti} ed ai bisogni del committente.
Il \RdP{} avrà la responsabilità di controllare i risultati e decidere se:
\begin{itemize}
	\item accettare ed approvare il prodotto;
	\item rigettare il documento, rendendo necessario un nuovo processo di \glo{verifica} con nuove indicazioni.
\end{itemize}
\subsubsection{Aspettative}
Devono essere rispettati i seguenti punti:
\begin{itemize}
	\item identificazione degli oggetti da validare;
	\item elaborazione di una strategia per rendere le procedure di \glo{verifica} riutilizzabili;
	\item valutazione dei risultati rispetto alle attese.
\end{itemize}

